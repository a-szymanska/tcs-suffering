Jeśli \( w \) jest wielomianem nierozkładalnym stopnia \( n \), to elementy 
elementy \( \integer_p/(w) \) to wielomiany stopnia co najwyżej \( n \) nad \( \integer_p \).
\begin{itemize}
    \item Dodawanie - \( \bigO(n) \) \\
    Dodajemy po współrzędnych modulo \( p \).
    \item Mnożenie - \( \bigO(n^2) \) lub lepiej \\
    Wykonujemy zwykłe mnożenie z wynikiem modulo \( w \). Można przyspieszyć tę operację algorytmem Karatsuby, Tooma-Cooka lub Sch\"onhage–Strassena - \( \bigO(n\log n) \).
    \item Dzielenie - \( \bigO(n^2) \) \\
    Obliczamy odwrotność rozszerzonym algorytmem Euklidesa, korzystając z tego, że wszystkie wielomiany w \( \integer_p/(w) \) są względnie pierwsze z \( w \). Następnie mnożymy przez odwrotność.
\end{itemize}
