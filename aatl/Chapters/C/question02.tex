\begin{theorem}
Dla ciała skończonego \( \mathbb{F}_q \) istnieje taki generator \( g \), że \( \mathbb{F}^{*}_q = \set{1, g, g^2, \text{\dots}} \).
\end{theorem}
\begin{proof}
    Oznaczmy \( n = \abs{\mathbb{F}^{*}_q} = q-1 \). Definiujemy zbiory:
    \[
        A_d = \set{a \in \mathbb{F}_q \mid a^d = 1}
    \]
    \[
        B_d = \set{a \in \mathbb{F}_q \mid a^d = 1, a^{d'} \neq 1 \text{ dla } d' < d}
    \]
    czyli \( B_d \) zawiera elementy, których rząd jest równy \( d \). Widać, że \( B_d \subseteq A_d \).

    Z twierdzenia Lagrange'a wynika, że \( B_d \) może być niepusty tylko dla \( d \) będącego dzielnikiem \( n \), a także \( \sum_{d \mid n} \abs{B_d} = n \), bo każdy element należy do któregoś \( B_d \). Celem jest pokazanie, że \( B_n \neq \emptyset \), czyli istnieje element rzędu \( n \).

    Jeśli dla pewnego \( d \) zachodzi \( B_d \neq \emptyset \), to \( \abs{A_d} \geq d \), bo jeśli pewne \( a^d = 1 \), to również \( (a^j)^d = 1 \) dla \( 0 \leq j < d \). Z drugiej strony, \( \abs{A_d} \leq d \), bo wielomian \( X^d-1 \) może mieć co najwyżej \( d \) pierwiastków. Zatem jeśli istnieje element rzędu \( d \), to \( \abs{A_d} = d \) oraz \( \abs{B_d} = \varphi(d) \).
    Na ćwiczeniach udowodniliśmy, że jeśli istnieje jakiś generator, to można znaleźć \( \varphi(d) \) generatorów.
    Podsumowując, dla każdego \( d \) zachodzi jedna z dwóch możliwości:
    \begin{itemize}
        \item \( \abs{B_d} = 0 \)
        \item \( \abs{B_d} = \varphi(d) \)
    \end{itemize}
    Wynika stąd, że
    \[
        n = \sum_{d \mid n} \abs{B_d} \leq \sum_{d \mid n} \varphi(d), 
    \]
    przy czym jeśli chociaż jedno \( d \) ma \( B_d = \emptyset \), to nierówność jest ostra.
    
    Wiemy jednak, że \( \sum_{d \mid n} \varphi(d) = n \), ponieważ możemy rozważyć wszystkie \( n \) ułamków właściwych postaci \( \frac{k}{n} \) dla \( 1 \leq k \leq n \).
    Po skróceniu mają postać \( \frac{k'}{d} \), gdzie \( d \mid n \) oraz \( k' \) jest względnie pierwsze z \( d \), więc dokładnie \( \phi(d) \) ułamków ma taki mianownik. To dowodzi równości.

    Zatem nierówność nie może być ostra i zachodzi \( B_d \neq \emptyset \) dla każdego \( d \mid n \), w szczególności istnieje element rzędu \( n \).
\end{proof}