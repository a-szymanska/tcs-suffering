\begin{theorem}
Dla ciała skończonego \( \mathbb{F}_q \) istnieje taki generator \( g \), że \( \mathbb{F}^{*}_q = \{1, g, g^2, \text{\dots}\} \).
\end{theorem}
\begin{proof}
    Oznaczmy \( n = \abs{\mathbb{F}^{*}_q} = q-1 \). Definiujemy zbiory \( A_d = \{a \in \mathbb{F}_q \mid a^d = 1\} \) oraz \( B_d = \{a \in \mathbb{F}_q \mid a^d = 1, a^{d'} \neq 1 \text{ dla } d' < d\} \), czyli elementy, których rząd jest równy \( d \). Widać, że \( B_d \subseteq A_d \). Z twierdzenia Lagrange'a wynika, że \( B_d \) może być niepusty tylko dla \( d \) będącego dzielnikiem \( n \), a także \( \sum_{d \mid n}\; \abs{B_d} = n \), bo każdy element należy do któregoś \( B_d \). Celem jest pokazanie, że \( B_n \neq \emptyset \), czyli istnieje element rzędu \( n \). \\
    Jeśli dla pewnego \( d \) zachodzi \( B_d \neq \emptyset \), to \( \abs{A_d} \geq d \), bo jeśli pewne \( a^d = 1 \), to również \( (a^j)^d = 1 \) dla \( 0 \leq j < d \). Z drugiej strony, \( \abs{A_d} \leq d \), bo wielomian \( X^d-1 \) może mieć co najwyżej \( d \) pierwiastków. Zatem jeśli istnieje element rzędu \( d \), to \( \abs{A_d} = d \) oraz \( \abs{B_d} = \varphi(d) \). Na ćwiczeniach udowodniliśmy, że jeśli istnieje jakiś generator, to można znaleźć \( \varphi(d) \) generatorów.
    Podsumowując, dla każdego \( d \) zachodzi jedna z dwóch możliwości:
    \begin{itemize}
        \item \( \abs{B_d} = 0 \)
        \item \( \abs{B_d} = \varphi(d) \)
    \end{itemize}
    Wynika stąd, że
    \[
        n = \sum_{d \mid n}\; \abs{B_d} \geq \sum_{d \mid n}\; \varphi(d), 
    \]
    przy czym jeśli chociaż jedno \( d \) ma \( B_d \textcolor{red}{=} \emptyset \), to nierówność jest ostra. Wiemy jednak, że \( \sum_{d \mid n}\; \varphi(d) = n \), więc nierówność nie może być ostra. Zatem dla każdego \( d \mid n \) zachodzi \( B_d \neq \emptyset \), w szczególności istnieje element rzędu n.
\end{proof}