Dla zadanej liczby \( A \) chcemy znaleźć jej rozkład na czynniki pierwsze, umiejąc wyznaczyć rząd dowolnego elementu z \( \integer_A^* \).
Pomysł jest następujący:
\begin{greyframe}
    \begin{enumerate}
        \item Wylosuj \( a \in \integer_A^* \) o parzystym rzędzie \( r \).
        \item Jeśli \( d = \gcd(A, \ a^{\frac{r}{2}} - 1) \) jest nietrywialnym dzielnikiem, zwróć \(d\).
        \item Jeśli \( d' = \gcd(A, \ a^{\frac{r}{2}} + 1) \) jest nietrywialnym dzielnikiem, zwróć \(d'\).
        \item Wpp powtórz losowanie.
    \end{enumerate}
\end{greyframe}
Ponieważ \( A = (a^{\frac{r}{2}} + 1) (a^{\frac{r}{2}} - 1) \), to z dużym prawdopodobieństem \( d \) jest nietrywialnym dzielnikiem \( A \).

Jak trudno jest wylosować odpowiednie \( a \)?
\begin{theorem}
    Niech \( N \) będzie nieparzystą liczbą złożoną, nie potęgą liczby pierwszej. Jeśli \( r \) jest rzędem \( a \in \set{0, \ldots, N-1} \), to prawdopodobieństwo,
    że \( r \) jest nieparzyste lub \( a^{\frac{r}{2}} = -1 \) nie przekracza \( \frac{1}{2} \).
\end{theorem}
\begin{proof}
    Niech \( N = p_1^{\alpha_1} \cdot \ldots \cdot p_s^{\alpha_s} \), gdzie \(s \geq 2\) i \( 2 \nmid p_i \).
    Z Chińskiego Twierdzenia o Resztach wynika, że losowanie \(a\) jest równoważne niezależnemu losowaniu s reszt \(a_i\) modulo \(p_i^{\alpha_i}\).
    Oznaczmy przez \(r_i\) rząd \(a_i\) modulo \(p_i^{\alpha_i}\) oraz niech \(r = \text{lcm}(r_1, \ldots, r_s)\).

    Żeby \(r\) było nieparzyste, wszystkie \(r_1, \ldots, r_s\) muszą być nieparzyste. Inaczej byłoby takie~\( i \), dla którego \(2r_i \mid r\), więc też \(a^{r_i} \mid a^{r/2}\).
    Zatem \(a^{r/2} = 1 \pmod{p_i^{\alpha_i}}\), co jest sprzeczne z~tym, że \(a^{r/2} = -1 \mod N\). Żeby \(a^{r/2} = -1 \pmod{N}\), musi być spełnione \(v(r_1) = \ldots = v(r_s)\), gdzie \(v(r_i)\) oznacza liczbę dwójek w rozkładzie \(r_i\).
    Jest to też warunek konieczny, żeby \(r\) było nieparzyste.
    
\newpage
    Jakie jest prawdopodobieństwo, że tak będzie?
    
    Korzystamy z tego, że \(G_i = \integer_{p_i^{\alpha_i}}^*\) jest grupą cykliczną. Niech \( G_i \) ma generator \(g\) i rząd \\ \(t = \varphi(p_i^{\alpha_i} - 1)\). Zbiory \(G_i' = \{g^1, g^3, g^5, \ldots\}\) i~\(G_i'' = \{1, g^2, g^4, \ldots\}\) są równoliczne, niezależne i \(G_i = G_i' \cup G_i''\).
    Rząd każdego elementu z \(G_i'\) ma tyle samo dwójek w rozkładzie co \(t\), a rząd elementu z \(G_i''\) ma ich mniej. Wynika z tego, że losując element z \(G_i\), mamy szansę co najmniej \(\frac{1}{2}\) nie trafić w element rzędu \( r \) taki, że \(v(r) = q\) dla ustalonego \(q\).
    
    Prawdopodobieństwo, że dla wylosowanych \(a_1, \ldots, a_s\) zajdzie \(v(r_1) = \ldots = v(r_s)\), wynosi co najwyżej \(\frac{1}{2^{s-1}} \leq \frac{1}{2}\).
\end{proof}
