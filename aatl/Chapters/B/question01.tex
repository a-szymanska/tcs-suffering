Algorytm mnożenia Karatsuby opiera się o technikę, dziel i zwyciężaj.
Chcemy pomnożyć \(n\)-cyfrowe liczby A oraz B zapisane w systemie binarnym. Zakładamy, że $n$ jest potęgą dwójki (zawsze można dodać zera wiodące). \\
Zapisujemy liczby w postaci:
\[
    A = A_1 \cdot K + A_0,
\]
\[
    B = B_1 \cdot K + B_0,
\]
gdzie \( K = 2^{\frac{n}{2}} \). Zauważmy teraz, że
\[
    A \cdot B = A_1 B_1 \cdot K^2 + (A_0 B_1 + A_1 B_0) \cdot K + A_0 B_0
\]
Mnożenie przez \( K \) to po prostu przesunięcie bitowe, więc możemy wykonać je w czasie \( \bigO(n) \), podobnie dodawanie. Korzystając z obserwacji, że:
\[
    A_0 B_1 + A_1 B_0 = (A_0 + A_1) \cdot (B_0 + B_1) - A_0 B_0 - A_1 B_1
\]
pozostają nam do wykonania 3 rekurencyjne mnożenia: \( A_0 B_0 \), \( A_1 B_1 \) i \( (A_0 + A_1) \cdot (B_0 + B_1) \). W ten sposób otrzymujemy \( T(n) = 3 \cdot T(n / 2) + \bigO(n) \), więc z Uniwersalnego Twierdzenia o~Rekurencji złożoność algorytmu Karatsuby wynosi \( \bigO(n^{\log_2 3}) \approx \bigO(n^{1.59}) \).
