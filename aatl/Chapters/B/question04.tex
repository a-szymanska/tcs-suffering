\subsection{Ideał}
Jeżeli \( R \) jest pierścieniem, to \( I \subseteq R \) jest ideałem wtedy i tylko wtedy, gdy:
\begin{itemize}
    \item \( x, y \in I \implies x + y \in I \) (zamknięcie na sumę)
    \item \( x \in I, y \in R \implies y\cdot x \in I \) (własność wciągania -- silniejsza niż zamkniętość na mnożenie)
\end{itemize}

Definicję ideału można uogólnić na więcej elementów -- ideał \( I \subseteq R \) generowany przez elementy \( (a_1, \ldots, a_s) \) to:
\[
    \set{ x_1 a_1 + \ldots + x_s a_s : a_1, \ldots, a_s \in I, x_1, \ldots, x_s \in R}
\]
zbiór sum wszystkich wielokrotności elementów \( a_1, \ldots, a_s \).

Ideałem głównym nazywamy ideał generowany przez jeden element.
Przykładowo zbiór liczb całkowitych jest pierścieniem ideałów głównych, czyli każdy jego ideał jest ideałem głównym.

\subsection{Pierścień ilorazowy}
Definiujemy relację równoważności \( R \) następująco:
\[
    x \sim y \iff x - y \in I
\]
Pierścień ilorazowy to zbiór klas abstrakcji \( R/I = \set{[x]_{\sim} : x \in R} = \set{x + I : x \in R} \). \\
Mniej formalnie mówimy po prostu o operacjach modulo ideał \( I \).

Przykładem pierścienia ilorazowego jest pierścień \( \integer \) z ideałem głównym \( n \integer = \set{n \cdot x : x \in \integer} \), dla pewnego \( n \).
Wtedy pierścień ilorazowy \( \integer / n\integer \) to pierścień liczb całkowitych z działaniami modulo \( n \).
\begin{theorem}
Pierścień \( \integer_p[X]/(W) \) jest ciałem wtw, gdy \( W \) jest nierozkładalny, czyli nie daje się przedstawić jako iloczyn wielomianów stopnia co najmniej 1.
\end{theorem}
\begin{proof}
    \textit{Pierścień \( \mathbb{Z}_p[X]/(W) \) jest ciałem \( \implies \) \( W \) jest nierozkładalny} \\
    Załóżmy nie wprost, że \( \mathbb{Z}_p[X]/W \) jest ciałem oraz \( W \) jest rozkładalny, czyli można zapisać \linebreak \( W = g_1 \cdot g_2 \), gdzie \( 1 < \deg(g_1), \deg(g_2) < \deg(W) \).
    Wynika z tego, że \( g_1 \cdot g_2 = 0 \), czyli są dzielnikami zera w tym ciele. Nie mają więc odwrotności względem mnożenia, co daje sprzeczność.

    \textit{Pierścień \( \mathbb{Z}_p[X]/(W) \) jest ciałem \( \impliedby \) \( W \) jest nierozkładalny} \\
    Wiemy, że \( \integer_p[X]/W \) jest pierścieniem. Wystarczy pokazać, że każdy wielomian \( g \) stopnia mniejszego niż \( W \) ma odwrotność.
    Skoro \( W \) jest nierozkładalny, to \( \gcd(g, W) = 1 \). Możemy więc zastosować rozszerzony algorytm Eulidesa do znajdowania odwrotności \( g \), która istnieje dla każdego \( g \neq 0 \), zatem \( \integer_p[X]/W \) jest ciałem.
\end{proof}