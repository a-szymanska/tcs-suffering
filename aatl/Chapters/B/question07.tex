\textbf{Digital Signature Algorithm (DSA)} służy do generowania podpisu cyfrowego -- mając wiadomość \( M \) oraz klucz prywatny chcemy wygenerować podpis, czyli obliczyć wartość pewnej funkcji na \( M \), którą może sprawdzić każdy, kto dysponuje kluczem publicznym.
Bez klucza prywatnego powinno być trudno wygenerować podpis przechodzący pomyślnie weryfikację. \\

\noindent
Algorytm składa się z następujących etapów:
\begin{greyframe}
    Przygotowanie:
    \begin{enumerate}
        \item Wybieramy duże liczby pierwsze \( p \) i \( q \) takie, że \( q \mid (p-1) \). Zaleca się \( p \) o długości 2048 lub 3072 bitów i \( q \) o długości 256 bitów.
        \item Znajdujemy element \( \zeta \), który jest równy \( q \) modulo \( p \). Możemy wylosować \( \xi \), po czym ustalić
        \[
            \zeta = \xi^{\frac{p-1}{q}}
        \]
        Wtedy otrzymujemy
        \[
            \zeta^q = \xi^{p-1} = 1 \mod p,
        \]
        czyli rząd \( \zeta \) rzeczywiście jest równy \( q \), bo \( q \) jest liczbą pierwszą, więc rząd nie może być mniejszy.
    \end{enumerate}
\end{greyframe}
Wartości \( p \), \( q \) oraz \( \zeta \) są publiczne, wszyscy mogą ich używać (dokładnie tych samych).

\begin{greyframe}
    Generowanie klucza:
    \begin{enumerate}
        \item Losujemy \( x \) i obliczamy \( y = \zeta^x \mod p \).
        \item Wartość \( x \) jest kluczem prywatnym, a wartość \( y \) kluczem publicznym.
    \end{enumerate}
\end{greyframe}
Odzyskanie klucza prywatnego wymaga rozwiązania logarytmu dyskretnego w~grupie \( \integer^{*}_p \).

\begin{greyframe}
    Podpisywanie wiadomości:
    \begin{enumerate}
        \item Obliczamy hash \( H \) wiadomości \( M \), np. za pomocą \texttt{SHA-2}.
        \item Losujemy \( k \) z \( \set{2, \ldots, q-1} \).
        \item Podpisem jest para \( (r, s) \), gdzie:
        \[
            r = (\zeta^k \mod p) \mod q
        \]
        \[
            s = \frac{H + x \cdot r}{k} \mod q
        \]
    \end{enumerate}
\end{greyframe}

\begin{greyframe}
    Weryfikacja podpisu:
    \begin{enumerate}
        \item Obliczamy
        \[
            \alpha = \frac{H}{s} \mod q,
        \]
        \[
            \beta = \frac{r}{s} \mod q,
        \]
        \[
            \gamma = (\zeta^{\alpha} \cdot y^{\beta} \mod p) \mod q
        \]
        \item Jeśli \( \gamma = r \), to podpis się zgadza.
    \end{enumerate}
\end{greyframe}

Algorym działa poprawnie.
\begin{proof}
    Definiujemy \( w = (H + x\cdot r)^{-1} \mod q \), a wtedy \( \alpha = H \cdot k \cdot w \mod q \) oraz \( \beta = r \cdot k \cdot w \mod q \).
    Element \( \zeta \) jest dobrany tak, że ma rząd \( q \) modulo \( p \), czyli dla dowolnego \( t \) zachodzi 
    \[
        \zeta^t \mod p = \zeta^{t \!\mod q} \mod p
    \]
    Wynika z tego, że
    \[
        \gamma = (\zeta^{\alpha} \cdot y^{\beta} \mod p) \mod q = (\zeta^{H \cdot k \cdot w \mod q} \cdot (\zeta^{x})^{r \cdot k \cdot w} \mod p) \mod q  = \zeta^{k \cdot w \cdot (H + x\cdot r) \mod q} \mod p \mod q
    \]
    Z definicji \( w \) jest odwrotnością \( (H + x\cdot r) \) modulo \( q \), zachodzi więc \( \gamma = (\zeta^k \mod p) \mod q \), co (ponownie z definicji) jest równe \( r \).
\end{proof}