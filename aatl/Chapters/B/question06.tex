Celem jest rozłożyć liczbę \( n \) na czynniki pierwsze. Zakładamy, że \( n \) jest nieparzysta (wpp można wydzielić największą potęgę 2) i nie jest liczbą pierwszą.
Podsumowując, \( n = p \cdot q \), gdzie \( p, \ q \) są nieparzyste. Bez straty ogólności przyjmujemy, że \( p > q \). Definiujemy \( a = \frac{p+q}{2} \) oraz \( b = \frac{p-q}{2} \). Wtedy:
\[
    a^2 - b^2 = (a+b) \cdot (a-b) = p \cdot q = n
\]
Zatem jeśli znajdziemy \( a \) takie, że \( a^2 - n = b^2 \), gdzie \( b^2 \) jest dowolnym kwadratem liczby naturalnej, to możemy łatwo wyliczyć \( p \) oraz \( q \).

\begin{greyframe}
    Algorytm faktoryzacji Fermata:
    \begin{enumerate}
        \item Przypisz \( a = \ceil{\sqrt{n}} \).
        \item Jeśli istnieje \( b \in \natural \) takie, że \( b^2 = a^2 - n \) (co sprawdzamy wyszukiwaniem binarnym), to zwróć dzielniki \( p = a+b, \ q = a-b \).
        \item Wpp \( a = a + 1\), powtórz procedurę.
    \end{enumerate}
\end{greyframe}

\textbf{Złożoność:} \\
Algorytm znajduje rozkład \( n \) na dwa czynniki po około \( a - \sqrt{n} \) krokach, bo trzeba sprawdzić po kolei przypadki dla \( \sqrt{n}, \dots, a \), każdy w \( \Theta(1) \).
Całkowita liczba operacji jest równa:
\[
    a - \sqrt{n} = \frac{a^2 - n}{a + \sqrt{n}} = \frac{b^2}{a + \sqrt{n}} \leq \frac{b^2}{\sqrt{n}} = \frac{(p-q)^2}{4\sqrt{n}}
\]
Pesymistyczna złożoność algorytmu to \( \bigO(n) \). Jednak przy małej różnicy \( \abs{p - q} \) algorytm działa szybko, a dla  \( \abs{p-q} \leq \sqrt[4]{n} \) nawet w czasie stałym.

Wniosek jest taki, żeby do RSA nie używać liczb pierwszych położonych blisko siebie.