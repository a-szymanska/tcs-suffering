Algorytm Tonellego-Shanksa służy do obliczania pierwiastka dyskretnego w czasie wielomianowym randomizowanym.

\begin{lemma}\label{A11:lemma1}
    Grupa cykliczna \( G \) o liczbie elementów \(n = 2m \), zawiera dokładnie \( m \) kwadratów, a każdy kwadrat ma dokładnie dwa pierwiastki.
\end{lemma}
\begin{proof}
    Zauważmy, że każdy element grupy \( G \) należy do zbioru \( \{g^0, g^1, \dots, g^{2m-1}\} \), czyli traktujemy potęgi generatora modulo \( 2m \). Do rozważenia są dwa przypadki:
    \begin{enumerate}
        \item \( a \) jest parzystą potęgą generatora: \\
        Wtedy piewiastkami \( a = g^{2k} \) są \( g^k \) oraz \( g^{k+m} \), ponieważ
        \[
            (g^k)^2 = g^{2k} = a
        \]
        \[
            (g^{k+m})^2 = g^{2k+2m} = g^{2k} = a
        \]
        Element \( a \) nie ma też żadnego innego pierwiastka.
        \item \( a \) jest nieparzystą potęgą generatora: \\
        Dla dowodu nie wprost, załóżmy że istnieje \( b = g^j \), które jest pierwiastkiem \( a = g^{2k+1} \). Wtedy
        \[
            b^2 = g^{2j} = g^{2k + 1},
        \]
        czyli \( 2j = 2k + 1 \pmod{2m} \), co prowadzi do sprzeczności.
    \end{enumerate}
    Zatem tylko parzyste potęgi generatora są kwadratami.
\end{proof}

\newpage
\begin{lemma}\label{A11:lemma2}
    Niech \( G \) będzie grupą cykliczną o \( n = 2m \) elementach. Element \( a \) jest kwadratem w \( G \) wtedy i tylko wtedy, gdy \( a^m = 1 \).
\end{lemma}
\begin{proof}
    \textit{Jeśli \( a \) jest kwadratem, to \( a^m = 1\).} \\
    Z lematu~\ref{A11:lemma1} wynika, że jeśli \( a \) jest kwadratem, to musi być postaci \( a = g^{2k} \). Zatem
    \[
        a^m = \pars{g^{2k}}^m = g^{2m \cdot k} = g^0 = 1
    \]
    \textit{Jeśli \( a \) nie jest kwadratem, to \( a^m \neq 1\).} \\
    Z lematu~\ref{A11:lemma1} wynika, że \( a \) jest postaci \( a = g^{2k+1} \) dla pewnego nieparzystego \( k \). Zatem
    \[
        g^{(2k+1)m} = g^{2m \cdot k} \cdot g^m = g^m = \xi = -1,
    \]
    ponieważ rząd \( g \) jest równy \( 2m \).
\end{proof}

Tak przygotowani możemy przejść do algorytmu: Mając grupę \( G \) o liczności \( \abs{G} = n = 2m \) oraz dane na wejściu \( a \in G \), celem jest znaleźć \( x \) takie, że \( x^2 = a \) w grupie \( G \).
\begin{greyframe}
    Algorytm Tonellego-Shanksa:
    \begin{enumerate}
        \item Wylosuj \( z \) takie, że \( z^m \neq 1 \).
        \item Przypisz \( q = m, \ t = 2m \).
        \item Dopóki \( 2 \mid q \):
        \begin{enumerate}
            \item Zaktualizuj \( q = q / 2, \ t = t / 2 \).
            \item Jeśli \( a^q \cdot z^t \neq 1 \), to zaktualizuj \( t = t + m \).
        \end{enumerate}
        \item Zwróć \( a^{\frac{q+1}{2}} \cdot z^{\frac{t}{2}} \).
    \end{enumerate}
\end{greyframe}
Algorytm zachowuje niezmienniki:
\begin{itemize}
    \item \( a^q \cdot z^t = 1 \)
    \item Jeśli \( 2^k \mid q \), to \( 2^{k+1} \mid t \).
\end{itemize}
Początkowo \(a^q = z^t = 1 \) oraz \( t = 2q \).
Z pierwszego niezmiennika wiadomo, że \( a^{\frac{q}{2}} \cdot z^{\frac{t}{2}} = \pm 1 \), ponieważ kwadrat tej liczby jest jedynką.
Jeśli \( a^{\frac{q}{2}} \cdot z^{\frac{t}{2}} = 1 \), to możemy podzielić \( q, \ t \) przez 2 i~niezmiennik pozostaje zachowany.
Jeśli \( a^{\frac{q}{2}} \cdot z^{\frac{t}{2}} = \xi \), to
\[
    a^{\frac{q}{2}} \cdot z^{\frac{t}{2} + m} = (a^{\frac{q}{2}} \cdot z^{\frac{t}{2}}) \cdot z^m = \xi \cdot \xi = 1
\]
Korzystamy z tego, że \( m \) niezmiennie jest wielokrotnością \( 2q \).

Ostateczny wynik to:
\[
    x^2 = a^{q+1} \cdot z^t = a \cdot (a^q \cdot z^t) = a
\]

Algorytm wykonuje \( \bigO(\log n) \) iteracji, bo po tylu dzieleniach \( q \) staje się nieparzyste, więc jest to algorytm wielomianowy.
Z lematu~\ref{A11:lemma1} wynika, że przy losowaniu \( z \) znajdziemy niekwadrat z prawdopodobieństwem \( \frac{1}{2} \), czyli w oczekiwaniu po stałej liczbie kroków będzie można przejść do kroku 2.
Nie potrafimy zdeterminizować tego kroku, ponieważ gdyby dało się znaleźć niekwadrat bez losowania, czyli policzyć pierwiastek dyskretny deterministycznie, to również można by deterministycznie rozłożyć liczbę na czynniki pierwsze (patrz: pytanie~\ref{A:question09}).