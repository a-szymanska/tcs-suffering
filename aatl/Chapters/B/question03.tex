\begin{theorem}[Chińskie Twierdzenie o Resztach]
    Dla parami względnie pierwszych liczb całkowitych \( m_1, \dots m_k \geq 1 \), \( M = m_1 \cdot \ldots \cdot m_k \) oraz liczb \( a_1, \ldots, a_k \), gdzie \( 0 \leq a_i \leq m_i \), układ kongruencji:
    \begin{center}
        \( x = a_1 \pmod{m_1} \) \\
        \( \dots \) \\
        \( x = a_k \pmod{m_k} \)
    \end{center}
    ma dokładnie jedno rozwiązanie \( x < M \).
\end{theorem}
\begin{proof}
    Trzeba udowodnić, że rozwiązanie istnieje oraz że wszystkie rozwiązania przystają do siebie modulo \( M \). \\
    \textbf{Istnienie}\footnote{Dowód został zaczerpnięty z \href{https://en.wikipedia.org/wiki/Chinese_remainder_theorem#Proof}{,,Chinese remainder theorem'', Wikipedia}.} \\
    Korzystamy z twierdzenia B\'ezout, z którego wynika, że dla względnie pierwszych liczb całkowitych \( x \), \( y \) istnieją takie liczby całkowite \( a \) i \( b \), że \( ax + by = 1 \). \\
    Definiujemy \( M_i = \frac{M}{m_i} \), wtedy \( M_i \) oraz \( m_i \) są względnie pierwsze oraz istnieją takie liczby \( B_i, \ b_i \), że \( B_iM_i + b_im_i = 1 \).
    Możemy zatem skonstruować rozwiązanie:
    \[
        x = \sum_{i=1}^{k} a_iB_iM_i
    \]
    Rozwiązanie rzeczywiście spełnia układ kongruencji, ponieważ:
    \[
        a_iB_iM_i = 0 \pmod{m_j} \ \text{dla } i \neq j
    \]
    \[
        a_iB_iM_i = a_i(1 - b_im_i) = a_i \pmod{m_i}
    \]

    \textbf{Jedyność} \\
    Załóżmy, że istnieją dwa rozwiązania \( x \) oraz \( y \), które są różne modulo \( M \). Wiemy, że zachodzi:
    \begin{center}
        \( x = a_i \pmod{m_i} \) \\
        \( y = a_i \pmod{m_i} \)
    \end{center}
    Zatem otrzymujemy \( x-y = 0 \pmod{m_i} \). Skoro \( m_i \) są parami względnie pierwsze, to zachodzi \(  x-y = 0 \pmod{M} \).
    Ponieważ \( x, y < M \), to musi zachodzić \( x = y \).
\end{proof}