\section{Zestaw 3 - Liczby pierwsze, pierścienie}
\subsection{Zadanie 1}
$\textbf{Treść:}$  \\
\\

Niech $S$ będzie pierścieniem:\\ \\
1. Czy zbiór dzielników zera w $S$ musi być ideałem? Udowodnić lub podać kontrprzykład.\\
2. Czy zbiór nilpotentów $S$ musi być ideałem? Udowodnić lub podać kontrprzykład.
\\
\\
$\textbf{Rozwiązanie:}$ \\
\\
1. Zbiór dzielników zera w $S$ jest ideałem. \\ \\
Nie jest. Przykład Pierścień $\mathbb{R}\times\mathbb{R}$ z mnożeniem po współżędnych.\\
Weźmy sobie element (1,0) oraz (0,1), po wymnożeniu otrzymamy (0,0) czyli nasze 0 w pierścieniu, natomiast gdy dodamy owe liczby po współżędnych otrzymamy (1,1) co nie może być dzielnikiem 0, gdyż jest elementem neutalnym względem mnożenia. \\ \\
2. Zbiór nilpotentów jest ideałem. \\ \\
Tak, pokażemy sobie to: \\ 
1. $0 \in I$ \\ \\
2. $a,b \in I => a + b \in I$ ( zamkniętość na dodawanie ) \\
Weźmy sobie takie $r,s$ że $a^r = 0, b^s = 0$.\\
Pokażemy sobie że $(a+b)^{r+s} = 0$.\\
Rozpiszmy sobie $(a+b)^{r+S}$ ze wzoru dwumanowego ( z czego dwumiany nie będa nas obchodziły wiec oznaczymy sobie je kolejno przez $c_{0},...,c_{r+s}$. A więć : \\
$(a+b)^{r+s} = \sum_{i=0}^{r+s} c_{i}\cdot a^{i}b^{r+s-i}$
zauważmy że dla $i \leq r$ wyraz $r+s-i \ge s$ co oznacza że $b^{r+s-i} = 0$ czyli czynniki dla $i \leq r$ zerują się. Nastomiast gdy $i > r$ czynnik $a^{i} = 0$ czyli czynniki również się wyzerują co oznacza że otrzymamy 0. Czyli $(a+b)^{r+s} = 0 => (a+b) \in I$.
\\ \\
3.$I$ posiada właściowść wciągania ( po wymnożeniu z każdym elementem z pierścienia element ten będzie w ideale ).
\\
Wiemy że dla każego $a \in I$ istnieje sobie $n$ takie że $a^{n} = 0$, teraz wymnażając go z jakimkolwiek elementes $s$ z pierścienia mamy element $a\cdot s$, a jak podniesiemy to do potęgi $n$ to otrzymamy $0$. $(ab)^n = a^{n}b^{n} = 0\cdot b^{n} = 0$ Czyli jest on ideałem ( my zajmujemy się pierścieniami przemiennymi wiec jest to ok ).



\subsection{Zadanie 2}
$\textbf{Treść:}$ \\ \\
Pokaż, że pierścień $S$ jest ciałem wtedy i tylko wtedy, kiedy jego jedynymi
ideałami są (0) (ideał z jednym elementem 0) oraz cały $S$.
\\
\\
 $\textbf{Rozwiązanie:}$ \\
 \\
 1) Pierścień $S$ jest ciałem \\
 2) Pierścień $S$ ma tylko ideały $(0)$ oraz cały $S$ \\
 \\
 1 => 2 \\ \\
 $(0)$ ma na pewno ( trywialny ideał ) \\
 Weźmy sobie nie wprost że ma jakiś ideał różny od $(0)$ oraz mniejszy od całego $S$. Wiemy że skoro $S$ jest ciałem to dla każdego elementu ma element odwrotny. Wieny też że istnieje $x$ w naszym ideale więc z faktu że mamy $x^{-1}$ to znaczy że $x^{-1}x$ należy do ideału ( z definicji ) co oznacza że $1$ należy do ideału. Skoro 1 należy do ideału to możemy przeiterować się po każdym elemencie z $S$ i wymożyć go z $1$, która musi być w tym ideale co oznacza że każdy element $S$ będzie w ideale ale założyliśmy że owy ideał nie być całym $S$ SPRZECZNOŚĆ \\ 
 \\
 2 => 1 \\ \\
Do tego faktu musimy pokazać że każdy element ma element odwrotny względem mnożenia. Pokażemy sobie że dla dowolnego $a$ znajdziemy dla niego element odwrotny. \\ \\
Weżmy sobie zbiór w którym narazie jest tylko element $a$ ( nie będzie to jeszcze ideał ale nie ma problemu za chwile z niego zrobimy ideał ). Teraz wiemy że jakiegoś elementu jeszcze nie osiągneliśmy. A więc weźmy sobie ideał w postaci dla każdego $i \in S$ dodajmy do naszego zbioru $a\cdot i$. Skoro wiemy że dla każdego elementu będzie on w ideale to z definicji jest to ideałem ( spełnia własność wciągania do ideału ) a to oznacza, że $1$ jest w ideale ponieważ nasz ideał musi być całym $S$ z założenia bo $S$ ma tylko 2 ideały ( albo siebie albo $(0)$ ), A jeżeli $1$ jest w ideale oraz jedyne liczby w ideale to $i\cdot a$ to oznacza że jakiś element $i$ po wymnożeniu z $a$ dał $1$ co oznacza że jest jego odwrotnością. Taką procedurę możemy powtórzyć dla każdego $a \in S$ co będzie oznaczało że każdy element ma element odwrotny. co należało pokazać

\subsection{Zadanie 3}
$\textbf{Treść:}$  \\ \\
Pokaż, że test Millera-Rabina działa również wtedy, kiedy testowana liczba
$n$ jest potęgą liczby pierwszej (tzn. odpowiada \textit{złożona} z prawdopodobieństwem > $\frac{1}{2}$
bez potrzeby osobnego sprawdzania tego przypadku).
\\
\\
$\textbf{Rozwiązanie:}$ \\
 Korzystam z ćwiczenia 5. Wiemy że liczby Charmichaela są bez kwadratowe co znaczyczy że nie dzieli ich żaden kwadrat liczby natruralnej większej od 1. Co oznacza że liczba w postaci $p^k$, gdzie $p$ pierwsza i $k > 1$ nie może być liczbą Charmichala gdyż $p^2 | p^k$ dla $k \ge 2$. Wynika z tego że istnieje $x$ taki że nie spełnia on testu fermata dla $p^k$. Oznaczmy sobie dla ułatwienia $n = p^k$. Weźmy sobie więc podgupę G = \{$b \in  \mathbb{Z}_{n} : b^{n-1} = 1$ mod $n$ \} $\subset \mathbb{Z}_{n}^{*}$. Wiemy że G jest podgrupą oraz istnieje $x$, który nie należy do G. Z Lagrange'a rząd podgrupy dzieli rząd grupy czyli oraz z wcześniej wiemy że G nie jest rozmiaru $n$ co oznacza $|G| \leq \frac{n}{2}$. Czyli pradowpodobieństwo że wybierzemy $x$ nie spełniającego testu fermata (którego wykonujemy w końcowym kroku testu millera-rabina) wynosi conajwyżej $\frac{1}{2}$


\subsection{Zadanie 4}
$\textbf{Treść:}$  \\ \\
 Pokaż, że jeśli dla pewnego $n$ liczby $6n + 1, 12n + 1 i 18n + 1$ są pierwsze,
to $m = (6n + 1)(12n + 1)(18n + 1)$ jest liczbą Carmichaela
\\
\\
$\textbf{Rozwiązanie:}$ \\
\\
$m = (6n + 1)(12n + 1)(18n + 1) = pqr $(oznaczmy sobie kolejne czynniki przez takie litery).$m = 1296 n^3 + 396 n^2 + 36 n + 1$ a wiec $p-1 | m-1$ oraz $q-1 | m-1$ oraz $r-1 | m-1$ ( jak przepałujemy to to dostajemy bo wyciągamy $6n,12n,18n$ kolejno przed nawias). A więc wiemy że dla każdego $a$ względnie pierwszego z $p,q,r$, $a^{m-1} = 1$ mod $p$ oraz $a^{m-1} = 1$ mod $q$ oraz $a^{m-1} = 1$ mod $r$ z czego z własności kongruencji dostajemy, że $a^{m-1} = 1$ mod $pqr$ = $a^{m-1} = 1$ mod $m$ co jest definicją liczby Charmichalea
\\ \\


\subsection{Zadanie 5}
$\textbf{Treść:}$ \\ \\
Pokaż, że liczba Carmichaela musi być bezkwadratowa (niepodzielna przez
żaden kwadrat liczby naturalnej większej od $1$). \\ \\
$\textbf{Rozwiązanie:}$ \\
\\
Dowód nie wprost: \\
Zakładamy sobie że nie jest bez kwadratowa. Czyli n ( liczba Charmichela ) możemy zapisać jako n = $p^k \cdot l$, gdzie $\gcd(p,l) = 1$ oraz $k \ge 2$. Weźmy sobie $a = p^{k-1}$. Z małego twierdzenia fermata wiemy że $n | a^n - a$ z czego wynika że $p^k | a^n - a$ oraz z definicji a wiemy że $p^k | a^n$ (gdyż $a = p^{k-1}$ oraz $n \ge 2$ co oznacza że $n\cdot (k-1) \ge k )$ z czego również mamy że $p^k | a$ co oznacza żę $p^k | p^{k-1}$ SPRZECZNOŚĆ gdyż $p^k > p^{k-1}$.
