  % Żeby nie było syfu to kolejne sekcje dodajemy do chapters/
% A potem includujemy za pomocą \input{chapters/...}

% Używamy \( \) i \[ \] zamiast dolarów -- tak jak się robi w LaTeXu



\documentclass[12pt, a4paper, openany]{book} % TODO: add polish as a global option when babel is updated and `polish.ldf` becomes available

% Please, let's familiarize ourselves with notatki.sty and tcs.sty so that we don't reinvent the wheel
\usepackage{notatki}
\fancyhead[L]{\textbf{\textit{TCS}}}

\begin{document}

% Front page and table of contents
\frontmatter

\begin{titlepage} 
    \begin{center}
         \begin{figure}[H]
            \centering
            \includegraphics[scale=1]{img/Frog_(PSF).png}
           
        \end{figure}
        
        \Huge
        \textbf{\textsc{Egzamin licencjacki}}
        
        \vspace{0.5cm}
        \Large
        \textsc{Odpowiedzi na pytania} \\ 
        \line(1,0){330}

        \normalsize
        \textit{,,Nigdy nie rozumiałem, po co ludzie czytają treści z obiektówki''}
        \vspace{1cm}

        \textit{\textsc{Autorzy}}\\
        \vspace{5mm}
  
        \textbf{\textsc{Dziurawy Ponton} \\ \textsc{Załatany Ponton} \\ \textsc{Puchaty Pompon} \\ \textsc{Zatopiony Ponton} \\ \textsc{Tonący Ponton}\\ \textsc{V} \\ \textsc{Nijaki Ponton} \\ \textsc{Pontus Euxinus} \\ \textsc{Pontón Baitonio}} 
 
        \vfill

        Kraków \\
        Anno Domini 2024
    \end{center}
\end{titlepage}

\tableofcontents

\input{license}
Zawartość rozdziału \ref{sysopy} podchodzi z notatek zadedykowanych domenie publicznej (CC0 1.0); nie jest ona napisana przez wylistowanych (anonimowych) autorów tego opracowania.

% Actual content
\mainmatter

% ROK 1, SEMESTR ZIMOWY
%Jednym z zadań SZBD jest zapewnianie tego, żeby zmiany dokonane przez widoczną lub zatwierdzoną transakcję były trwałe, a efekty transakcji wycofanej lub przerwanyj przez awarię nie.

Jest parę typów awarii:
\begin{itemize}
    \item awaria systemu: awaria sprzętu, błędy oprogramowania, uszkodzenie dysku
    \item błędy logiczne (np. naruszenie ograniczeń integralnościowych)
    \item błędy systemowe (np. zakleszczenia)
\end{itemize}
Do poradzenia sobie z pierwszym rodzajem wystarczy replikacja bazy i rozproszenie danych. W przypadku pozostałych dwóch trzeba odtworzyć bazę po awarii.
Ważne jest, żeby zminimalizować czas potrzebny do tego, ponieważ w trakcie odtwarzania bazy serwer musi być wyłączony dla użytkowników.
Służące do tego algorytmy składają się z dwóch części:
\begin{enumerate}
    \item zapisywanie informacji potrzebnych do odtworzenia stanu bazy na bieżąco podczas zwykłych operacji
    \item czynności po awarii, prowadzące do przywrócenia spójności, atomowości i trwałości
\end{enumerate}

\subsection*{Strategie SZBD}
Brudne strony, czyli zmieniane po ostatnim zapisie na dysku, mogą pozostawać w buforze długo po zatwierdzeniu transakcji. Jeśli nastąpi awaria, to naniesione zmiany są tracone.
Żeby zadbać o poprawny stan bazy można wykonać operacje:
\begin{itemize}
    \item UNDO, wycofanie zmian - zapewnia atomowość
    \item REDO, powtórzenie transakcji - zapewnia trwałość.
\end{itemize}
Podjęte czynności zależą od strategii SZBD: \\
Czy niezatwierdzona transakcja może nadpisać wartość w pamięci trwałej? \\
\begin{itemize}
    \item strategia steal: TAK
    \item strategia no-steal: NIE
\end{itemize}
Czy wszystkie zmiany muszą być zapisane w pamięci trwałej przed zatwierdzeniem transakcji?
\begin{itemize}
    \item strategia force: TAK
    \item strategia no-force: NIE
\end{itemize}

Strategia \textbf{no-steal + force} pozwala uniknąć wykonywania sekwencji UNDO i REDO. Jednak jest mało efektywna i istnieje ryzyko samozakleszczenia transakcji, jeśli braknie miejsca w buforze.
Częściej wykorzystywana jest strategia \textbf{steal + no-force} opakowana przez \textbf{protokół WAL (write-ahead log)}.
Oprócz plików z danymi przechowywany jest dziennik z logami (dziennik), w którym zapisywane są wszystkie zmiany przed wprowadzeniem ich na dysku.
Zawiera on informacje wystarczające odzyskania zawartości bazy za pomocą operacji UNDO i REDO.
Żeby dzienniki nie rozrastały się w nieskończoność, tworzy się punkty kontrolne. W przypadku awarii powtarzane są tylko transakcja zatwierdzona po ostatnim punkcie kontrolnym, a niezatwierdzone są wycofywane.

\subsection{Algorytm ARIES}
Algorytm ARIES (Algorithm for Recovery and Isolation Exploiting Semantics) służy do odtwarzania transakcji, korzystając z dziennika WAL i blokad.
Składa się z trzech faz:
\begin{enumerate}
    \item Analiza: rekonstrukcja Tabeli Brudnych Stron (DPT) i Tabeli Transakcji (TT), wyznaczenie pierwszego wpisu o zmianach tworzących brudną stronę
    \item Faza REDO: powtórzenie wszystkich operacji od momentu powstania brudnej strony, przywrócenie stanu sprzed awarii
    \item Faza UNDO: wycofanie efektów niezatwierdzonych transakcji, dodanie wpisów kompensacyjnych do dziennika
\end{enumerate}
%Jednym z zadań SZBD jest zapewnianie tego, żeby zmiany dokonane przez widoczną lub zatwierdzoną transakcję były trwałe, a efekty transakcji wycofanej lub przerwanyj przez awarię nie.

Jest parę typów awarii:
\begin{itemize}
    \item awaria systemu: awaria sprzętu, błędy oprogramowania, uszkodzenie dysku
    \item błędy logiczne (np. naruszenie ograniczeń integralnościowych)
    \item błędy systemowe (np. zakleszczenia)
\end{itemize}
Do poradzenia sobie z pierwszym rodzajem wystarczy replikacja bazy i rozproszenie danych. W przypadku pozostałych dwóch trzeba odtworzyć bazę po awarii.
Ważne jest, żeby zminimalizować czas potrzebny do tego, ponieważ w trakcie odtwarzania bazy serwer musi być wyłączony dla użytkowników.
Służące do tego algorytmy składają się z dwóch części:
\begin{enumerate}
    \item zapisywanie informacji potrzebnych do odtworzenia stanu bazy na bieżąco podczas zwykłych operacji
    \item czynności po awarii, prowadzące do przywrócenia spójności, atomowości i trwałości
\end{enumerate}

\subsection*{Strategie SZBD}
Brudne strony, czyli zmieniane po ostatnim zapisie na dysku, mogą pozostawać w buforze długo po zatwierdzeniu transakcji. Jeśli nastąpi awaria, to naniesione zmiany są tracone.
Żeby zadbać o poprawny stan bazy można wykonać operacje:
\begin{itemize}
    \item UNDO, wycofanie zmian - zapewnia atomowość
    \item REDO, powtórzenie transakcji - zapewnia trwałość.
\end{itemize}
Podjęte czynności zależą od strategii SZBD: \\
Czy niezatwierdzona transakcja może nadpisać wartość w pamięci trwałej? \\
\begin{itemize}
    \item strategia steal: TAK
    \item strategia no-steal: NIE
\end{itemize}
Czy wszystkie zmiany muszą być zapisane w pamięci trwałej przed zatwierdzeniem transakcji?
\begin{itemize}
    \item strategia force: TAK
    \item strategia no-force: NIE
\end{itemize}

Strategia \textbf{no-steal + force} pozwala uniknąć wykonywania sekwencji UNDO i REDO. Jednak jest mało efektywna i istnieje ryzyko samozakleszczenia transakcji, jeśli braknie miejsca w buforze.
Częściej wykorzystywana jest strategia \textbf{steal + no-force} opakowana przez \textbf{protokół WAL (write-ahead log)}.
Oprócz plików z danymi przechowywany jest dziennik z logami (dziennik), w którym zapisywane są wszystkie zmiany przed wprowadzeniem ich na dysku.
Zawiera on informacje wystarczające odzyskania zawartości bazy za pomocą operacji UNDO i REDO.
Żeby dzienniki nie rozrastały się w nieskończoność, tworzy się punkty kontrolne. W przypadku awarii powtarzane są tylko transakcja zatwierdzona po ostatnim punkcie kontrolnym, a niezatwierdzone są wycofywane.

\subsection{Algorytm ARIES}
Algorytm ARIES (Algorithm for Recovery and Isolation Exploiting Semantics) służy do odtwarzania transakcji, korzystając z dziennika WAL i blokad.
Składa się z trzech faz:
\begin{enumerate}
    \item Analiza: rekonstrukcja Tabeli Brudnych Stron (DPT) i Tabeli Transakcji (TT), wyznaczenie pierwszego wpisu o zmianach tworzących brudną stronę
    \item Faza REDO: powtórzenie wszystkich operacji od momentu powstania brudnej strony, przywrócenie stanu sprzed awarii
    \item Faza UNDO: wycofanie efektów niezatwierdzonych transakcji, dodanie wpisów kompensacyjnych do dziennika
\end{enumerate}
%Jednym z zadań SZBD jest zapewnianie tego, żeby zmiany dokonane przez widoczną lub zatwierdzoną transakcję były trwałe, a efekty transakcji wycofanej lub przerwanyj przez awarię nie.

Jest parę typów awarii:
\begin{itemize}
    \item awaria systemu: awaria sprzętu, błędy oprogramowania, uszkodzenie dysku
    \item błędy logiczne (np. naruszenie ograniczeń integralnościowych)
    \item błędy systemowe (np. zakleszczenia)
\end{itemize}
Do poradzenia sobie z pierwszym rodzajem wystarczy replikacja bazy i rozproszenie danych. W przypadku pozostałych dwóch trzeba odtworzyć bazę po awarii.
Ważne jest, żeby zminimalizować czas potrzebny do tego, ponieważ w trakcie odtwarzania bazy serwer musi być wyłączony dla użytkowników.
Służące do tego algorytmy składają się z dwóch części:
\begin{enumerate}
    \item zapisywanie informacji potrzebnych do odtworzenia stanu bazy na bieżąco podczas zwykłych operacji
    \item czynności po awarii, prowadzące do przywrócenia spójności, atomowości i trwałości
\end{enumerate}

\subsection*{Strategie SZBD}
Brudne strony, czyli zmieniane po ostatnim zapisie na dysku, mogą pozostawać w buforze długo po zatwierdzeniu transakcji. Jeśli nastąpi awaria, to naniesione zmiany są tracone.
Żeby zadbać o poprawny stan bazy można wykonać operacje:
\begin{itemize}
    \item UNDO, wycofanie zmian - zapewnia atomowość
    \item REDO, powtórzenie transakcji - zapewnia trwałość.
\end{itemize}
Podjęte czynności zależą od strategii SZBD: \\
Czy niezatwierdzona transakcja może nadpisać wartość w pamięci trwałej? \\
\begin{itemize}
    \item strategia steal: TAK
    \item strategia no-steal: NIE
\end{itemize}
Czy wszystkie zmiany muszą być zapisane w pamięci trwałej przed zatwierdzeniem transakcji?
\begin{itemize}
    \item strategia force: TAK
    \item strategia no-force: NIE
\end{itemize}

Strategia \textbf{no-steal + force} pozwala uniknąć wykonywania sekwencji UNDO i REDO. Jednak jest mało efektywna i istnieje ryzyko samozakleszczenia transakcji, jeśli braknie miejsca w buforze.
Częściej wykorzystywana jest strategia \textbf{steal + no-force} opakowana przez \textbf{protokół WAL (write-ahead log)}.
Oprócz plików z danymi przechowywany jest dziennik z logami (dziennik), w którym zapisywane są wszystkie zmiany przed wprowadzeniem ich na dysku.
Zawiera on informacje wystarczające odzyskania zawartości bazy za pomocą operacji UNDO i REDO.
Żeby dzienniki nie rozrastały się w nieskończoność, tworzy się punkty kontrolne. W przypadku awarii powtarzane są tylko transakcja zatwierdzona po ostatnim punkcie kontrolnym, a niezatwierdzone są wycofywane.

\subsection{Algorytm ARIES}
Algorytm ARIES (Algorithm for Recovery and Isolation Exploiting Semantics) służy do odtwarzania transakcji, korzystając z dziennika WAL i blokad.
Składa się z trzech faz:
\begin{enumerate}
    \item Analiza: rekonstrukcja Tabeli Brudnych Stron (DPT) i Tabeli Transakcji (TT), wyznaczenie pierwszego wpisu o zmianach tworzących brudną stronę
    \item Faza REDO: powtórzenie wszystkich operacji od momentu powstania brudnej strony, przywrócenie stanu sprzed awarii
    \item Faza UNDO: wycofanie efektów niezatwierdzonych transakcji, dodanie wpisów kompensacyjnych do dziennika
\end{enumerate}
%Jednym z zadań SZBD jest zapewnianie tego, żeby zmiany dokonane przez widoczną lub zatwierdzoną transakcję były trwałe, a efekty transakcji wycofanej lub przerwanyj przez awarię nie.

Jest parę typów awarii:
\begin{itemize}
    \item awaria systemu: awaria sprzętu, błędy oprogramowania, uszkodzenie dysku
    \item błędy logiczne (np. naruszenie ograniczeń integralnościowych)
    \item błędy systemowe (np. zakleszczenia)
\end{itemize}
Do poradzenia sobie z pierwszym rodzajem wystarczy replikacja bazy i rozproszenie danych. W przypadku pozostałych dwóch trzeba odtworzyć bazę po awarii.
Ważne jest, żeby zminimalizować czas potrzebny do tego, ponieważ w trakcie odtwarzania bazy serwer musi być wyłączony dla użytkowników.
Służące do tego algorytmy składają się z dwóch części:
\begin{enumerate}
    \item zapisywanie informacji potrzebnych do odtworzenia stanu bazy na bieżąco podczas zwykłych operacji
    \item czynności po awarii, prowadzące do przywrócenia spójności, atomowości i trwałości
\end{enumerate}

\subsection*{Strategie SZBD}
Brudne strony, czyli zmieniane po ostatnim zapisie na dysku, mogą pozostawać w buforze długo po zatwierdzeniu transakcji. Jeśli nastąpi awaria, to naniesione zmiany są tracone.
Żeby zadbać o poprawny stan bazy można wykonać operacje:
\begin{itemize}
    \item UNDO, wycofanie zmian - zapewnia atomowość
    \item REDO, powtórzenie transakcji - zapewnia trwałość.
\end{itemize}
Podjęte czynności zależą od strategii SZBD: \\
Czy niezatwierdzona transakcja może nadpisać wartość w pamięci trwałej? \\
\begin{itemize}
    \item strategia steal: TAK
    \item strategia no-steal: NIE
\end{itemize}
Czy wszystkie zmiany muszą być zapisane w pamięci trwałej przed zatwierdzeniem transakcji?
\begin{itemize}
    \item strategia force: TAK
    \item strategia no-force: NIE
\end{itemize}

Strategia \textbf{no-steal + force} pozwala uniknąć wykonywania sekwencji UNDO i REDO. Jednak jest mało efektywna i istnieje ryzyko samozakleszczenia transakcji, jeśli braknie miejsca w buforze.
Częściej wykorzystywana jest strategia \textbf{steal + no-force} opakowana przez \textbf{protokół WAL (write-ahead log)}.
Oprócz plików z danymi przechowywany jest dziennik z logami (dziennik), w którym zapisywane są wszystkie zmiany przed wprowadzeniem ich na dysku.
Zawiera on informacje wystarczające odzyskania zawartości bazy za pomocą operacji UNDO i REDO.
Żeby dzienniki nie rozrastały się w nieskończoność, tworzy się punkty kontrolne. W przypadku awarii powtarzane są tylko transakcja zatwierdzona po ostatnim punkcie kontrolnym, a niezatwierdzone są wycofywane.

\subsection{Algorytm ARIES}
Algorytm ARIES (Algorithm for Recovery and Isolation Exploiting Semantics) służy do odtwarzania transakcji, korzystając z dziennika WAL i blokad.
Składa się z trzech faz:
\begin{enumerate}
    \item Analiza: rekonstrukcja Tabeli Brudnych Stron (DPT) i Tabeli Transakcji (TT), wyznaczenie pierwszego wpisu o zmianach tworzących brudną stronę
    \item Faza REDO: powtórzenie wszystkich operacji od momentu powstania brudnej strony, przywrócenie stanu sprzed awarii
    \item Faza UNDO: wycofanie efektów niezatwierdzonych transakcji, dodanie wpisów kompensacyjnych do dziennika
\end{enumerate}

% ROK 1, SEMESTR LETNI
%Jednym z zadań SZBD jest zapewnianie tego, żeby zmiany dokonane przez widoczną lub zatwierdzoną transakcję były trwałe, a efekty transakcji wycofanej lub przerwanyj przez awarię nie.

Jest parę typów awarii:
\begin{itemize}
    \item awaria systemu: awaria sprzętu, błędy oprogramowania, uszkodzenie dysku
    \item błędy logiczne (np. naruszenie ograniczeń integralnościowych)
    \item błędy systemowe (np. zakleszczenia)
\end{itemize}
Do poradzenia sobie z pierwszym rodzajem wystarczy replikacja bazy i rozproszenie danych. W przypadku pozostałych dwóch trzeba odtworzyć bazę po awarii.
Ważne jest, żeby zminimalizować czas potrzebny do tego, ponieważ w trakcie odtwarzania bazy serwer musi być wyłączony dla użytkowników.
Służące do tego algorytmy składają się z dwóch części:
\begin{enumerate}
    \item zapisywanie informacji potrzebnych do odtworzenia stanu bazy na bieżąco podczas zwykłych operacji
    \item czynności po awarii, prowadzące do przywrócenia spójności, atomowości i trwałości
\end{enumerate}

\subsection*{Strategie SZBD}
Brudne strony, czyli zmieniane po ostatnim zapisie na dysku, mogą pozostawać w buforze długo po zatwierdzeniu transakcji. Jeśli nastąpi awaria, to naniesione zmiany są tracone.
Żeby zadbać o poprawny stan bazy można wykonać operacje:
\begin{itemize}
    \item UNDO, wycofanie zmian - zapewnia atomowość
    \item REDO, powtórzenie transakcji - zapewnia trwałość.
\end{itemize}
Podjęte czynności zależą od strategii SZBD: \\
Czy niezatwierdzona transakcja może nadpisać wartość w pamięci trwałej? \\
\begin{itemize}
    \item strategia steal: TAK
    \item strategia no-steal: NIE
\end{itemize}
Czy wszystkie zmiany muszą być zapisane w pamięci trwałej przed zatwierdzeniem transakcji?
\begin{itemize}
    \item strategia force: TAK
    \item strategia no-force: NIE
\end{itemize}

Strategia \textbf{no-steal + force} pozwala uniknąć wykonywania sekwencji UNDO i REDO. Jednak jest mało efektywna i istnieje ryzyko samozakleszczenia transakcji, jeśli braknie miejsca w buforze.
Częściej wykorzystywana jest strategia \textbf{steal + no-force} opakowana przez \textbf{protokół WAL (write-ahead log)}.
Oprócz plików z danymi przechowywany jest dziennik z logami (dziennik), w którym zapisywane są wszystkie zmiany przed wprowadzeniem ich na dysku.
Zawiera on informacje wystarczające odzyskania zawartości bazy za pomocą operacji UNDO i REDO.
Żeby dzienniki nie rozrastały się w nieskończoność, tworzy się punkty kontrolne. W przypadku awarii powtarzane są tylko transakcja zatwierdzona po ostatnim punkcie kontrolnym, a niezatwierdzone są wycofywane.

\subsection{Algorytm ARIES}
Algorytm ARIES (Algorithm for Recovery and Isolation Exploiting Semantics) służy do odtwarzania transakcji, korzystając z dziennika WAL i blokad.
Składa się z trzech faz:
\begin{enumerate}
    \item Analiza: rekonstrukcja Tabeli Brudnych Stron (DPT) i Tabeli Transakcji (TT), wyznaczenie pierwszego wpisu o zmianach tworzących brudną stronę
    \item Faza REDO: powtórzenie wszystkich operacji od momentu powstania brudnej strony, przywrócenie stanu sprzed awarii
    \item Faza UNDO: wycofanie efektów niezatwierdzonych transakcji, dodanie wpisów kompensacyjnych do dziennika
\end{enumerate}
%Jednym z zadań SZBD jest zapewnianie tego, żeby zmiany dokonane przez widoczną lub zatwierdzoną transakcję były trwałe, a efekty transakcji wycofanej lub przerwanyj przez awarię nie.

Jest parę typów awarii:
\begin{itemize}
    \item awaria systemu: awaria sprzętu, błędy oprogramowania, uszkodzenie dysku
    \item błędy logiczne (np. naruszenie ograniczeń integralnościowych)
    \item błędy systemowe (np. zakleszczenia)
\end{itemize}
Do poradzenia sobie z pierwszym rodzajem wystarczy replikacja bazy i rozproszenie danych. W przypadku pozostałych dwóch trzeba odtworzyć bazę po awarii.
Ważne jest, żeby zminimalizować czas potrzebny do tego, ponieważ w trakcie odtwarzania bazy serwer musi być wyłączony dla użytkowników.
Służące do tego algorytmy składają się z dwóch części:
\begin{enumerate}
    \item zapisywanie informacji potrzebnych do odtworzenia stanu bazy na bieżąco podczas zwykłych operacji
    \item czynności po awarii, prowadzące do przywrócenia spójności, atomowości i trwałości
\end{enumerate}

\subsection*{Strategie SZBD}
Brudne strony, czyli zmieniane po ostatnim zapisie na dysku, mogą pozostawać w buforze długo po zatwierdzeniu transakcji. Jeśli nastąpi awaria, to naniesione zmiany są tracone.
Żeby zadbać o poprawny stan bazy można wykonać operacje:
\begin{itemize}
    \item UNDO, wycofanie zmian - zapewnia atomowość
    \item REDO, powtórzenie transakcji - zapewnia trwałość.
\end{itemize}
Podjęte czynności zależą od strategii SZBD: \\
Czy niezatwierdzona transakcja może nadpisać wartość w pamięci trwałej? \\
\begin{itemize}
    \item strategia steal: TAK
    \item strategia no-steal: NIE
\end{itemize}
Czy wszystkie zmiany muszą być zapisane w pamięci trwałej przed zatwierdzeniem transakcji?
\begin{itemize}
    \item strategia force: TAK
    \item strategia no-force: NIE
\end{itemize}

Strategia \textbf{no-steal + force} pozwala uniknąć wykonywania sekwencji UNDO i REDO. Jednak jest mało efektywna i istnieje ryzyko samozakleszczenia transakcji, jeśli braknie miejsca w buforze.
Częściej wykorzystywana jest strategia \textbf{steal + no-force} opakowana przez \textbf{protokół WAL (write-ahead log)}.
Oprócz plików z danymi przechowywany jest dziennik z logami (dziennik), w którym zapisywane są wszystkie zmiany przed wprowadzeniem ich na dysku.
Zawiera on informacje wystarczające odzyskania zawartości bazy za pomocą operacji UNDO i REDO.
Żeby dzienniki nie rozrastały się w nieskończoność, tworzy się punkty kontrolne. W przypadku awarii powtarzane są tylko transakcja zatwierdzona po ostatnim punkcie kontrolnym, a niezatwierdzone są wycofywane.

\subsection{Algorytm ARIES}
Algorytm ARIES (Algorithm for Recovery and Isolation Exploiting Semantics) służy do odtwarzania transakcji, korzystając z dziennika WAL i blokad.
Składa się z trzech faz:
\begin{enumerate}
    \item Analiza: rekonstrukcja Tabeli Brudnych Stron (DPT) i Tabeli Transakcji (TT), wyznaczenie pierwszego wpisu o zmianach tworzących brudną stronę
    \item Faza REDO: powtórzenie wszystkich operacji od momentu powstania brudnej strony, przywrócenie stanu sprzed awarii
    \item Faza UNDO: wycofanie efektów niezatwierdzonych transakcji, dodanie wpisów kompensacyjnych do dziennika
\end{enumerate}
%Jednym z zadań SZBD jest zapewnianie tego, żeby zmiany dokonane przez widoczną lub zatwierdzoną transakcję były trwałe, a efekty transakcji wycofanej lub przerwanyj przez awarię nie.

Jest parę typów awarii:
\begin{itemize}
    \item awaria systemu: awaria sprzętu, błędy oprogramowania, uszkodzenie dysku
    \item błędy logiczne (np. naruszenie ograniczeń integralnościowych)
    \item błędy systemowe (np. zakleszczenia)
\end{itemize}
Do poradzenia sobie z pierwszym rodzajem wystarczy replikacja bazy i rozproszenie danych. W przypadku pozostałych dwóch trzeba odtworzyć bazę po awarii.
Ważne jest, żeby zminimalizować czas potrzebny do tego, ponieważ w trakcie odtwarzania bazy serwer musi być wyłączony dla użytkowników.
Służące do tego algorytmy składają się z dwóch części:
\begin{enumerate}
    \item zapisywanie informacji potrzebnych do odtworzenia stanu bazy na bieżąco podczas zwykłych operacji
    \item czynności po awarii, prowadzące do przywrócenia spójności, atomowości i trwałości
\end{enumerate}

\subsection*{Strategie SZBD}
Brudne strony, czyli zmieniane po ostatnim zapisie na dysku, mogą pozostawać w buforze długo po zatwierdzeniu transakcji. Jeśli nastąpi awaria, to naniesione zmiany są tracone.
Żeby zadbać o poprawny stan bazy można wykonać operacje:
\begin{itemize}
    \item UNDO, wycofanie zmian - zapewnia atomowość
    \item REDO, powtórzenie transakcji - zapewnia trwałość.
\end{itemize}
Podjęte czynności zależą od strategii SZBD: \\
Czy niezatwierdzona transakcja może nadpisać wartość w pamięci trwałej? \\
\begin{itemize}
    \item strategia steal: TAK
    \item strategia no-steal: NIE
\end{itemize}
Czy wszystkie zmiany muszą być zapisane w pamięci trwałej przed zatwierdzeniem transakcji?
\begin{itemize}
    \item strategia force: TAK
    \item strategia no-force: NIE
\end{itemize}

Strategia \textbf{no-steal + force} pozwala uniknąć wykonywania sekwencji UNDO i REDO. Jednak jest mało efektywna i istnieje ryzyko samozakleszczenia transakcji, jeśli braknie miejsca w buforze.
Częściej wykorzystywana jest strategia \textbf{steal + no-force} opakowana przez \textbf{protokół WAL (write-ahead log)}.
Oprócz plików z danymi przechowywany jest dziennik z logami (dziennik), w którym zapisywane są wszystkie zmiany przed wprowadzeniem ich na dysku.
Zawiera on informacje wystarczające odzyskania zawartości bazy za pomocą operacji UNDO i REDO.
Żeby dzienniki nie rozrastały się w nieskończoność, tworzy się punkty kontrolne. W przypadku awarii powtarzane są tylko transakcja zatwierdzona po ostatnim punkcie kontrolnym, a niezatwierdzone są wycofywane.

\subsection{Algorytm ARIES}
Algorytm ARIES (Algorithm for Recovery and Isolation Exploiting Semantics) służy do odtwarzania transakcji, korzystając z dziennika WAL i blokad.
Składa się z trzech faz:
\begin{enumerate}
    \item Analiza: rekonstrukcja Tabeli Brudnych Stron (DPT) i Tabeli Transakcji (TT), wyznaczenie pierwszego wpisu o zmianach tworzących brudną stronę
    \item Faza REDO: powtórzenie wszystkich operacji od momentu powstania brudnej strony, przywrócenie stanu sprzed awarii
    \item Faza UNDO: wycofanie efektów niezatwierdzonych transakcji, dodanie wpisów kompensacyjnych do dziennika
\end{enumerate}
%Jednym z zadań SZBD jest zapewnianie tego, żeby zmiany dokonane przez widoczną lub zatwierdzoną transakcję były trwałe, a efekty transakcji wycofanej lub przerwanyj przez awarię nie.

Jest parę typów awarii:
\begin{itemize}
    \item awaria systemu: awaria sprzętu, błędy oprogramowania, uszkodzenie dysku
    \item błędy logiczne (np. naruszenie ograniczeń integralnościowych)
    \item błędy systemowe (np. zakleszczenia)
\end{itemize}
Do poradzenia sobie z pierwszym rodzajem wystarczy replikacja bazy i rozproszenie danych. W przypadku pozostałych dwóch trzeba odtworzyć bazę po awarii.
Ważne jest, żeby zminimalizować czas potrzebny do tego, ponieważ w trakcie odtwarzania bazy serwer musi być wyłączony dla użytkowników.
Służące do tego algorytmy składają się z dwóch części:
\begin{enumerate}
    \item zapisywanie informacji potrzebnych do odtworzenia stanu bazy na bieżąco podczas zwykłych operacji
    \item czynności po awarii, prowadzące do przywrócenia spójności, atomowości i trwałości
\end{enumerate}

\subsection*{Strategie SZBD}
Brudne strony, czyli zmieniane po ostatnim zapisie na dysku, mogą pozostawać w buforze długo po zatwierdzeniu transakcji. Jeśli nastąpi awaria, to naniesione zmiany są tracone.
Żeby zadbać o poprawny stan bazy można wykonać operacje:
\begin{itemize}
    \item UNDO, wycofanie zmian - zapewnia atomowość
    \item REDO, powtórzenie transakcji - zapewnia trwałość.
\end{itemize}
Podjęte czynności zależą od strategii SZBD: \\
Czy niezatwierdzona transakcja może nadpisać wartość w pamięci trwałej? \\
\begin{itemize}
    \item strategia steal: TAK
    \item strategia no-steal: NIE
\end{itemize}
Czy wszystkie zmiany muszą być zapisane w pamięci trwałej przed zatwierdzeniem transakcji?
\begin{itemize}
    \item strategia force: TAK
    \item strategia no-force: NIE
\end{itemize}

Strategia \textbf{no-steal + force} pozwala uniknąć wykonywania sekwencji UNDO i REDO. Jednak jest mało efektywna i istnieje ryzyko samozakleszczenia transakcji, jeśli braknie miejsca w buforze.
Częściej wykorzystywana jest strategia \textbf{steal + no-force} opakowana przez \textbf{protokół WAL (write-ahead log)}.
Oprócz plików z danymi przechowywany jest dziennik z logami (dziennik), w którym zapisywane są wszystkie zmiany przed wprowadzeniem ich na dysku.
Zawiera on informacje wystarczające odzyskania zawartości bazy za pomocą operacji UNDO i REDO.
Żeby dzienniki nie rozrastały się w nieskończoność, tworzy się punkty kontrolne. W przypadku awarii powtarzane są tylko transakcja zatwierdzona po ostatnim punkcie kontrolnym, a niezatwierdzone są wycofywane.

\subsection{Algorytm ARIES}
Algorytm ARIES (Algorithm for Recovery and Isolation Exploiting Semantics) służy do odtwarzania transakcji, korzystając z dziennika WAL i blokad.
Składa się z trzech faz:
\begin{enumerate}
    \item Analiza: rekonstrukcja Tabeli Brudnych Stron (DPT) i Tabeli Transakcji (TT), wyznaczenie pierwszego wpisu o zmianach tworzących brudną stronę
    \item Faza REDO: powtórzenie wszystkich operacji od momentu powstania brudnej strony, przywrócenie stanu sprzed awarii
    \item Faza UNDO: wycofanie efektów niezatwierdzonych transakcji, dodanie wpisów kompensacyjnych do dziennika
\end{enumerate}

% ROK 2, SEMESTR ZIMOWY

%Jednym z zadań SZBD jest zapewnianie tego, żeby zmiany dokonane przez widoczną lub zatwierdzoną transakcję były trwałe, a efekty transakcji wycofanej lub przerwanyj przez awarię nie.

Jest parę typów awarii:
\begin{itemize}
    \item awaria systemu: awaria sprzętu, błędy oprogramowania, uszkodzenie dysku
    \item błędy logiczne (np. naruszenie ograniczeń integralnościowych)
    \item błędy systemowe (np. zakleszczenia)
\end{itemize}
Do poradzenia sobie z pierwszym rodzajem wystarczy replikacja bazy i rozproszenie danych. W przypadku pozostałych dwóch trzeba odtworzyć bazę po awarii.
Ważne jest, żeby zminimalizować czas potrzebny do tego, ponieważ w trakcie odtwarzania bazy serwer musi być wyłączony dla użytkowników.
Służące do tego algorytmy składają się z dwóch części:
\begin{enumerate}
    \item zapisywanie informacji potrzebnych do odtworzenia stanu bazy na bieżąco podczas zwykłych operacji
    \item czynności po awarii, prowadzące do przywrócenia spójności, atomowości i trwałości
\end{enumerate}

\subsection*{Strategie SZBD}
Brudne strony, czyli zmieniane po ostatnim zapisie na dysku, mogą pozostawać w buforze długo po zatwierdzeniu transakcji. Jeśli nastąpi awaria, to naniesione zmiany są tracone.
Żeby zadbać o poprawny stan bazy można wykonać operacje:
\begin{itemize}
    \item UNDO, wycofanie zmian - zapewnia atomowość
    \item REDO, powtórzenie transakcji - zapewnia trwałość.
\end{itemize}
Podjęte czynności zależą od strategii SZBD: \\
Czy niezatwierdzona transakcja może nadpisać wartość w pamięci trwałej? \\
\begin{itemize}
    \item strategia steal: TAK
    \item strategia no-steal: NIE
\end{itemize}
Czy wszystkie zmiany muszą być zapisane w pamięci trwałej przed zatwierdzeniem transakcji?
\begin{itemize}
    \item strategia force: TAK
    \item strategia no-force: NIE
\end{itemize}

Strategia \textbf{no-steal + force} pozwala uniknąć wykonywania sekwencji UNDO i REDO. Jednak jest mało efektywna i istnieje ryzyko samozakleszczenia transakcji, jeśli braknie miejsca w buforze.
Częściej wykorzystywana jest strategia \textbf{steal + no-force} opakowana przez \textbf{protokół WAL (write-ahead log)}.
Oprócz plików z danymi przechowywany jest dziennik z logami (dziennik), w którym zapisywane są wszystkie zmiany przed wprowadzeniem ich na dysku.
Zawiera on informacje wystarczające odzyskania zawartości bazy za pomocą operacji UNDO i REDO.
Żeby dzienniki nie rozrastały się w nieskończoność, tworzy się punkty kontrolne. W przypadku awarii powtarzane są tylko transakcja zatwierdzona po ostatnim punkcie kontrolnym, a niezatwierdzone są wycofywane.

\subsection{Algorytm ARIES}
Algorytm ARIES (Algorithm for Recovery and Isolation Exploiting Semantics) służy do odtwarzania transakcji, korzystając z dziennika WAL i blokad.
Składa się z trzech faz:
\begin{enumerate}
    \item Analiza: rekonstrukcja Tabeli Brudnych Stron (DPT) i Tabeli Transakcji (TT), wyznaczenie pierwszego wpisu o zmianach tworzących brudną stronę
    \item Faza REDO: powtórzenie wszystkich operacji od momentu powstania brudnej strony, przywrócenie stanu sprzed awarii
    \item Faza UNDO: wycofanie efektów niezatwierdzonych transakcji, dodanie wpisów kompensacyjnych do dziennika
\end{enumerate}
%Jednym z zadań SZBD jest zapewnianie tego, żeby zmiany dokonane przez widoczną lub zatwierdzoną transakcję były trwałe, a efekty transakcji wycofanej lub przerwanyj przez awarię nie.

Jest parę typów awarii:
\begin{itemize}
    \item awaria systemu: awaria sprzętu, błędy oprogramowania, uszkodzenie dysku
    \item błędy logiczne (np. naruszenie ograniczeń integralnościowych)
    \item błędy systemowe (np. zakleszczenia)
\end{itemize}
Do poradzenia sobie z pierwszym rodzajem wystarczy replikacja bazy i rozproszenie danych. W przypadku pozostałych dwóch trzeba odtworzyć bazę po awarii.
Ważne jest, żeby zminimalizować czas potrzebny do tego, ponieważ w trakcie odtwarzania bazy serwer musi być wyłączony dla użytkowników.
Służące do tego algorytmy składają się z dwóch części:
\begin{enumerate}
    \item zapisywanie informacji potrzebnych do odtworzenia stanu bazy na bieżąco podczas zwykłych operacji
    \item czynności po awarii, prowadzące do przywrócenia spójności, atomowości i trwałości
\end{enumerate}

\subsection*{Strategie SZBD}
Brudne strony, czyli zmieniane po ostatnim zapisie na dysku, mogą pozostawać w buforze długo po zatwierdzeniu transakcji. Jeśli nastąpi awaria, to naniesione zmiany są tracone.
Żeby zadbać o poprawny stan bazy można wykonać operacje:
\begin{itemize}
    \item UNDO, wycofanie zmian - zapewnia atomowość
    \item REDO, powtórzenie transakcji - zapewnia trwałość.
\end{itemize}
Podjęte czynności zależą od strategii SZBD: \\
Czy niezatwierdzona transakcja może nadpisać wartość w pamięci trwałej? \\
\begin{itemize}
    \item strategia steal: TAK
    \item strategia no-steal: NIE
\end{itemize}
Czy wszystkie zmiany muszą być zapisane w pamięci trwałej przed zatwierdzeniem transakcji?
\begin{itemize}
    \item strategia force: TAK
    \item strategia no-force: NIE
\end{itemize}

Strategia \textbf{no-steal + force} pozwala uniknąć wykonywania sekwencji UNDO i REDO. Jednak jest mało efektywna i istnieje ryzyko samozakleszczenia transakcji, jeśli braknie miejsca w buforze.
Częściej wykorzystywana jest strategia \textbf{steal + no-force} opakowana przez \textbf{protokół WAL (write-ahead log)}.
Oprócz plików z danymi przechowywany jest dziennik z logami (dziennik), w którym zapisywane są wszystkie zmiany przed wprowadzeniem ich na dysku.
Zawiera on informacje wystarczające odzyskania zawartości bazy za pomocą operacji UNDO i REDO.
Żeby dzienniki nie rozrastały się w nieskończoność, tworzy się punkty kontrolne. W przypadku awarii powtarzane są tylko transakcja zatwierdzona po ostatnim punkcie kontrolnym, a niezatwierdzone są wycofywane.

\subsection{Algorytm ARIES}
Algorytm ARIES (Algorithm for Recovery and Isolation Exploiting Semantics) służy do odtwarzania transakcji, korzystając z dziennika WAL i blokad.
Składa się z trzech faz:
\begin{enumerate}
    \item Analiza: rekonstrukcja Tabeli Brudnych Stron (DPT) i Tabeli Transakcji (TT), wyznaczenie pierwszego wpisu o zmianach tworzących brudną stronę
    \item Faza REDO: powtórzenie wszystkich operacji od momentu powstania brudnej strony, przywrócenie stanu sprzed awarii
    \item Faza UNDO: wycofanie efektów niezatwierdzonych transakcji, dodanie wpisów kompensacyjnych do dziennika
\end{enumerate}
%Jednym z zadań SZBD jest zapewnianie tego, żeby zmiany dokonane przez widoczną lub zatwierdzoną transakcję były trwałe, a efekty transakcji wycofanej lub przerwanyj przez awarię nie.

Jest parę typów awarii:
\begin{itemize}
    \item awaria systemu: awaria sprzętu, błędy oprogramowania, uszkodzenie dysku
    \item błędy logiczne (np. naruszenie ograniczeń integralnościowych)
    \item błędy systemowe (np. zakleszczenia)
\end{itemize}
Do poradzenia sobie z pierwszym rodzajem wystarczy replikacja bazy i rozproszenie danych. W przypadku pozostałych dwóch trzeba odtworzyć bazę po awarii.
Ważne jest, żeby zminimalizować czas potrzebny do tego, ponieważ w trakcie odtwarzania bazy serwer musi być wyłączony dla użytkowników.
Służące do tego algorytmy składają się z dwóch części:
\begin{enumerate}
    \item zapisywanie informacji potrzebnych do odtworzenia stanu bazy na bieżąco podczas zwykłych operacji
    \item czynności po awarii, prowadzące do przywrócenia spójności, atomowości i trwałości
\end{enumerate}

\subsection*{Strategie SZBD}
Brudne strony, czyli zmieniane po ostatnim zapisie na dysku, mogą pozostawać w buforze długo po zatwierdzeniu transakcji. Jeśli nastąpi awaria, to naniesione zmiany są tracone.
Żeby zadbać o poprawny stan bazy można wykonać operacje:
\begin{itemize}
    \item UNDO, wycofanie zmian - zapewnia atomowość
    \item REDO, powtórzenie transakcji - zapewnia trwałość.
\end{itemize}
Podjęte czynności zależą od strategii SZBD: \\
Czy niezatwierdzona transakcja może nadpisać wartość w pamięci trwałej? \\
\begin{itemize}
    \item strategia steal: TAK
    \item strategia no-steal: NIE
\end{itemize}
Czy wszystkie zmiany muszą być zapisane w pamięci trwałej przed zatwierdzeniem transakcji?
\begin{itemize}
    \item strategia force: TAK
    \item strategia no-force: NIE
\end{itemize}

Strategia \textbf{no-steal + force} pozwala uniknąć wykonywania sekwencji UNDO i REDO. Jednak jest mało efektywna i istnieje ryzyko samozakleszczenia transakcji, jeśli braknie miejsca w buforze.
Częściej wykorzystywana jest strategia \textbf{steal + no-force} opakowana przez \textbf{protokół WAL (write-ahead log)}.
Oprócz plików z danymi przechowywany jest dziennik z logami (dziennik), w którym zapisywane są wszystkie zmiany przed wprowadzeniem ich na dysku.
Zawiera on informacje wystarczające odzyskania zawartości bazy za pomocą operacji UNDO i REDO.
Żeby dzienniki nie rozrastały się w nieskończoność, tworzy się punkty kontrolne. W przypadku awarii powtarzane są tylko transakcja zatwierdzona po ostatnim punkcie kontrolnym, a niezatwierdzone są wycofywane.

\subsection{Algorytm ARIES}
Algorytm ARIES (Algorithm for Recovery and Isolation Exploiting Semantics) służy do odtwarzania transakcji, korzystając z dziennika WAL i blokad.
Składa się z trzech faz:
\begin{enumerate}
    \item Analiza: rekonstrukcja Tabeli Brudnych Stron (DPT) i Tabeli Transakcji (TT), wyznaczenie pierwszego wpisu o zmianach tworzących brudną stronę
    \item Faza REDO: powtórzenie wszystkich operacji od momentu powstania brudnej strony, przywrócenie stanu sprzed awarii
    \item Faza UNDO: wycofanie efektów niezatwierdzonych transakcji, dodanie wpisów kompensacyjnych do dziennika
\end{enumerate}
Jednym z zadań SZBD jest zapewnianie tego, żeby zmiany dokonane przez widoczną lub zatwierdzoną transakcję były trwałe, a efekty transakcji wycofanej lub przerwanyj przez awarię nie.

Jest parę typów awarii:
\begin{itemize}
    \item awaria systemu: awaria sprzętu, błędy oprogramowania, uszkodzenie dysku
    \item błędy logiczne (np. naruszenie ograniczeń integralnościowych)
    \item błędy systemowe (np. zakleszczenia)
\end{itemize}
Do poradzenia sobie z pierwszym rodzajem wystarczy replikacja bazy i rozproszenie danych. W przypadku pozostałych dwóch trzeba odtworzyć bazę po awarii.
Ważne jest, żeby zminimalizować czas potrzebny do tego, ponieważ w trakcie odtwarzania bazy serwer musi być wyłączony dla użytkowników.
Służące do tego algorytmy składają się z dwóch części:
\begin{enumerate}
    \item zapisywanie informacji potrzebnych do odtworzenia stanu bazy na bieżąco podczas zwykłych operacji
    \item czynności po awarii, prowadzące do przywrócenia spójności, atomowości i trwałości
\end{enumerate}

\subsection*{Strategie SZBD}
Brudne strony, czyli zmieniane po ostatnim zapisie na dysku, mogą pozostawać w buforze długo po zatwierdzeniu transakcji. Jeśli nastąpi awaria, to naniesione zmiany są tracone.
Żeby zadbać o poprawny stan bazy można wykonać operacje:
\begin{itemize}
    \item UNDO, wycofanie zmian - zapewnia atomowość
    \item REDO, powtórzenie transakcji - zapewnia trwałość.
\end{itemize}
Podjęte czynności zależą od strategii SZBD: \\
Czy niezatwierdzona transakcja może nadpisać wartość w pamięci trwałej? \\
\begin{itemize}
    \item strategia steal: TAK
    \item strategia no-steal: NIE
\end{itemize}
Czy wszystkie zmiany muszą być zapisane w pamięci trwałej przed zatwierdzeniem transakcji?
\begin{itemize}
    \item strategia force: TAK
    \item strategia no-force: NIE
\end{itemize}

Strategia \textbf{no-steal + force} pozwala uniknąć wykonywania sekwencji UNDO i REDO. Jednak jest mało efektywna i istnieje ryzyko samozakleszczenia transakcji, jeśli braknie miejsca w buforze.
Częściej wykorzystywana jest strategia \textbf{steal + no-force} opakowana przez \textbf{protokół WAL (write-ahead log)}.
Oprócz plików z danymi przechowywany jest dziennik z logami (dziennik), w którym zapisywane są wszystkie zmiany przed wprowadzeniem ich na dysku.
Zawiera on informacje wystarczające odzyskania zawartości bazy za pomocą operacji UNDO i REDO.
Żeby dzienniki nie rozrastały się w nieskończoność, tworzy się punkty kontrolne. W przypadku awarii powtarzane są tylko transakcja zatwierdzona po ostatnim punkcie kontrolnym, a niezatwierdzone są wycofywane.

\subsection{Algorytm ARIES}
Algorytm ARIES (Algorithm for Recovery and Isolation Exploiting Semantics) służy do odtwarzania transakcji, korzystając z dziennika WAL i blokad.
Składa się z trzech faz:
\begin{enumerate}
    \item Analiza: rekonstrukcja Tabeli Brudnych Stron (DPT) i Tabeli Transakcji (TT), wyznaczenie pierwszego wpisu o zmianach tworzących brudną stronę
    \item Faza REDO: powtórzenie wszystkich operacji od momentu powstania brudnej strony, przywrócenie stanu sprzed awarii
    \item Faza UNDO: wycofanie efektów niezatwierdzonych transakcji, dodanie wpisów kompensacyjnych do dziennika
\end{enumerate}

% ROK 2, SEMESTR LETNI
%Jednym z zadań SZBD jest zapewnianie tego, żeby zmiany dokonane przez widoczną lub zatwierdzoną transakcję były trwałe, a efekty transakcji wycofanej lub przerwanyj przez awarię nie.

Jest parę typów awarii:
\begin{itemize}
    \item awaria systemu: awaria sprzętu, błędy oprogramowania, uszkodzenie dysku
    \item błędy logiczne (np. naruszenie ograniczeń integralnościowych)
    \item błędy systemowe (np. zakleszczenia)
\end{itemize}
Do poradzenia sobie z pierwszym rodzajem wystarczy replikacja bazy i rozproszenie danych. W przypadku pozostałych dwóch trzeba odtworzyć bazę po awarii.
Ważne jest, żeby zminimalizować czas potrzebny do tego, ponieważ w trakcie odtwarzania bazy serwer musi być wyłączony dla użytkowników.
Służące do tego algorytmy składają się z dwóch części:
\begin{enumerate}
    \item zapisywanie informacji potrzebnych do odtworzenia stanu bazy na bieżąco podczas zwykłych operacji
    \item czynności po awarii, prowadzące do przywrócenia spójności, atomowości i trwałości
\end{enumerate}

\subsection*{Strategie SZBD}
Brudne strony, czyli zmieniane po ostatnim zapisie na dysku, mogą pozostawać w buforze długo po zatwierdzeniu transakcji. Jeśli nastąpi awaria, to naniesione zmiany są tracone.
Żeby zadbać o poprawny stan bazy można wykonać operacje:
\begin{itemize}
    \item UNDO, wycofanie zmian - zapewnia atomowość
    \item REDO, powtórzenie transakcji - zapewnia trwałość.
\end{itemize}
Podjęte czynności zależą od strategii SZBD: \\
Czy niezatwierdzona transakcja może nadpisać wartość w pamięci trwałej? \\
\begin{itemize}
    \item strategia steal: TAK
    \item strategia no-steal: NIE
\end{itemize}
Czy wszystkie zmiany muszą być zapisane w pamięci trwałej przed zatwierdzeniem transakcji?
\begin{itemize}
    \item strategia force: TAK
    \item strategia no-force: NIE
\end{itemize}

Strategia \textbf{no-steal + force} pozwala uniknąć wykonywania sekwencji UNDO i REDO. Jednak jest mało efektywna i istnieje ryzyko samozakleszczenia transakcji, jeśli braknie miejsca w buforze.
Częściej wykorzystywana jest strategia \textbf{steal + no-force} opakowana przez \textbf{protokół WAL (write-ahead log)}.
Oprócz plików z danymi przechowywany jest dziennik z logami (dziennik), w którym zapisywane są wszystkie zmiany przed wprowadzeniem ich na dysku.
Zawiera on informacje wystarczające odzyskania zawartości bazy za pomocą operacji UNDO i REDO.
Żeby dzienniki nie rozrastały się w nieskończoność, tworzy się punkty kontrolne. W przypadku awarii powtarzane są tylko transakcja zatwierdzona po ostatnim punkcie kontrolnym, a niezatwierdzone są wycofywane.

\subsection{Algorytm ARIES}
Algorytm ARIES (Algorithm for Recovery and Isolation Exploiting Semantics) służy do odtwarzania transakcji, korzystając z dziennika WAL i blokad.
Składa się z trzech faz:
\begin{enumerate}
    \item Analiza: rekonstrukcja Tabeli Brudnych Stron (DPT) i Tabeli Transakcji (TT), wyznaczenie pierwszego wpisu o zmianach tworzących brudną stronę
    \item Faza REDO: powtórzenie wszystkich operacji od momentu powstania brudnej strony, przywrócenie stanu sprzed awarii
    \item Faza UNDO: wycofanie efektów niezatwierdzonych transakcji, dodanie wpisów kompensacyjnych do dziennika
\end{enumerate}
%Jednym z zadań SZBD jest zapewnianie tego, żeby zmiany dokonane przez widoczną lub zatwierdzoną transakcję były trwałe, a efekty transakcji wycofanej lub przerwanyj przez awarię nie.

Jest parę typów awarii:
\begin{itemize}
    \item awaria systemu: awaria sprzętu, błędy oprogramowania, uszkodzenie dysku
    \item błędy logiczne (np. naruszenie ograniczeń integralnościowych)
    \item błędy systemowe (np. zakleszczenia)
\end{itemize}
Do poradzenia sobie z pierwszym rodzajem wystarczy replikacja bazy i rozproszenie danych. W przypadku pozostałych dwóch trzeba odtworzyć bazę po awarii.
Ważne jest, żeby zminimalizować czas potrzebny do tego, ponieważ w trakcie odtwarzania bazy serwer musi być wyłączony dla użytkowników.
Służące do tego algorytmy składają się z dwóch części:
\begin{enumerate}
    \item zapisywanie informacji potrzebnych do odtworzenia stanu bazy na bieżąco podczas zwykłych operacji
    \item czynności po awarii, prowadzące do przywrócenia spójności, atomowości i trwałości
\end{enumerate}

\subsection*{Strategie SZBD}
Brudne strony, czyli zmieniane po ostatnim zapisie na dysku, mogą pozostawać w buforze długo po zatwierdzeniu transakcji. Jeśli nastąpi awaria, to naniesione zmiany są tracone.
Żeby zadbać o poprawny stan bazy można wykonać operacje:
\begin{itemize}
    \item UNDO, wycofanie zmian - zapewnia atomowość
    \item REDO, powtórzenie transakcji - zapewnia trwałość.
\end{itemize}
Podjęte czynności zależą od strategii SZBD: \\
Czy niezatwierdzona transakcja może nadpisać wartość w pamięci trwałej? \\
\begin{itemize}
    \item strategia steal: TAK
    \item strategia no-steal: NIE
\end{itemize}
Czy wszystkie zmiany muszą być zapisane w pamięci trwałej przed zatwierdzeniem transakcji?
\begin{itemize}
    \item strategia force: TAK
    \item strategia no-force: NIE
\end{itemize}

Strategia \textbf{no-steal + force} pozwala uniknąć wykonywania sekwencji UNDO i REDO. Jednak jest mało efektywna i istnieje ryzyko samozakleszczenia transakcji, jeśli braknie miejsca w buforze.
Częściej wykorzystywana jest strategia \textbf{steal + no-force} opakowana przez \textbf{protokół WAL (write-ahead log)}.
Oprócz plików z danymi przechowywany jest dziennik z logami (dziennik), w którym zapisywane są wszystkie zmiany przed wprowadzeniem ich na dysku.
Zawiera on informacje wystarczające odzyskania zawartości bazy za pomocą operacji UNDO i REDO.
Żeby dzienniki nie rozrastały się w nieskończoność, tworzy się punkty kontrolne. W przypadku awarii powtarzane są tylko transakcja zatwierdzona po ostatnim punkcie kontrolnym, a niezatwierdzone są wycofywane.

\subsection{Algorytm ARIES}
Algorytm ARIES (Algorithm for Recovery and Isolation Exploiting Semantics) służy do odtwarzania transakcji, korzystając z dziennika WAL i blokad.
Składa się z trzech faz:
\begin{enumerate}
    \item Analiza: rekonstrukcja Tabeli Brudnych Stron (DPT) i Tabeli Transakcji (TT), wyznaczenie pierwszego wpisu o zmianach tworzących brudną stronę
    \item Faza REDO: powtórzenie wszystkich operacji od momentu powstania brudnej strony, przywrócenie stanu sprzed awarii
    \item Faza UNDO: wycofanie efektów niezatwierdzonych transakcji, dodanie wpisów kompensacyjnych do dziennika
\end{enumerate}
%Jednym z zadań SZBD jest zapewnianie tego, żeby zmiany dokonane przez widoczną lub zatwierdzoną transakcję były trwałe, a efekty transakcji wycofanej lub przerwanyj przez awarię nie.

Jest parę typów awarii:
\begin{itemize}
    \item awaria systemu: awaria sprzętu, błędy oprogramowania, uszkodzenie dysku
    \item błędy logiczne (np. naruszenie ograniczeń integralnościowych)
    \item błędy systemowe (np. zakleszczenia)
\end{itemize}
Do poradzenia sobie z pierwszym rodzajem wystarczy replikacja bazy i rozproszenie danych. W przypadku pozostałych dwóch trzeba odtworzyć bazę po awarii.
Ważne jest, żeby zminimalizować czas potrzebny do tego, ponieważ w trakcie odtwarzania bazy serwer musi być wyłączony dla użytkowników.
Służące do tego algorytmy składają się z dwóch części:
\begin{enumerate}
    \item zapisywanie informacji potrzebnych do odtworzenia stanu bazy na bieżąco podczas zwykłych operacji
    \item czynności po awarii, prowadzące do przywrócenia spójności, atomowości i trwałości
\end{enumerate}

\subsection*{Strategie SZBD}
Brudne strony, czyli zmieniane po ostatnim zapisie na dysku, mogą pozostawać w buforze długo po zatwierdzeniu transakcji. Jeśli nastąpi awaria, to naniesione zmiany są tracone.
Żeby zadbać o poprawny stan bazy można wykonać operacje:
\begin{itemize}
    \item UNDO, wycofanie zmian - zapewnia atomowość
    \item REDO, powtórzenie transakcji - zapewnia trwałość.
\end{itemize}
Podjęte czynności zależą od strategii SZBD: \\
Czy niezatwierdzona transakcja może nadpisać wartość w pamięci trwałej? \\
\begin{itemize}
    \item strategia steal: TAK
    \item strategia no-steal: NIE
\end{itemize}
Czy wszystkie zmiany muszą być zapisane w pamięci trwałej przed zatwierdzeniem transakcji?
\begin{itemize}
    \item strategia force: TAK
    \item strategia no-force: NIE
\end{itemize}

Strategia \textbf{no-steal + force} pozwala uniknąć wykonywania sekwencji UNDO i REDO. Jednak jest mało efektywna i istnieje ryzyko samozakleszczenia transakcji, jeśli braknie miejsca w buforze.
Częściej wykorzystywana jest strategia \textbf{steal + no-force} opakowana przez \textbf{protokół WAL (write-ahead log)}.
Oprócz plików z danymi przechowywany jest dziennik z logami (dziennik), w którym zapisywane są wszystkie zmiany przed wprowadzeniem ich na dysku.
Zawiera on informacje wystarczające odzyskania zawartości bazy za pomocą operacji UNDO i REDO.
Żeby dzienniki nie rozrastały się w nieskończoność, tworzy się punkty kontrolne. W przypadku awarii powtarzane są tylko transakcja zatwierdzona po ostatnim punkcie kontrolnym, a niezatwierdzone są wycofywane.

\subsection{Algorytm ARIES}
Algorytm ARIES (Algorithm for Recovery and Isolation Exploiting Semantics) służy do odtwarzania transakcji, korzystając z dziennika WAL i blokad.
Składa się z trzech faz:
\begin{enumerate}
    \item Analiza: rekonstrukcja Tabeli Brudnych Stron (DPT) i Tabeli Transakcji (TT), wyznaczenie pierwszego wpisu o zmianach tworzących brudną stronę
    \item Faza REDO: powtórzenie wszystkich operacji od momentu powstania brudnej strony, przywrócenie stanu sprzed awarii
    \item Faza UNDO: wycofanie efektów niezatwierdzonych transakcji, dodanie wpisów kompensacyjnych do dziennika
\end{enumerate}

% ROK 3, SEMESTR ZIMOWY
%Jednym z zadań SZBD jest zapewnianie tego, żeby zmiany dokonane przez widoczną lub zatwierdzoną transakcję były trwałe, a efekty transakcji wycofanej lub przerwanyj przez awarię nie.

Jest parę typów awarii:
\begin{itemize}
    \item awaria systemu: awaria sprzętu, błędy oprogramowania, uszkodzenie dysku
    \item błędy logiczne (np. naruszenie ograniczeń integralnościowych)
    \item błędy systemowe (np. zakleszczenia)
\end{itemize}
Do poradzenia sobie z pierwszym rodzajem wystarczy replikacja bazy i rozproszenie danych. W przypadku pozostałych dwóch trzeba odtworzyć bazę po awarii.
Ważne jest, żeby zminimalizować czas potrzebny do tego, ponieważ w trakcie odtwarzania bazy serwer musi być wyłączony dla użytkowników.
Służące do tego algorytmy składają się z dwóch części:
\begin{enumerate}
    \item zapisywanie informacji potrzebnych do odtworzenia stanu bazy na bieżąco podczas zwykłych operacji
    \item czynności po awarii, prowadzące do przywrócenia spójności, atomowości i trwałości
\end{enumerate}

\subsection*{Strategie SZBD}
Brudne strony, czyli zmieniane po ostatnim zapisie na dysku, mogą pozostawać w buforze długo po zatwierdzeniu transakcji. Jeśli nastąpi awaria, to naniesione zmiany są tracone.
Żeby zadbać o poprawny stan bazy można wykonać operacje:
\begin{itemize}
    \item UNDO, wycofanie zmian - zapewnia atomowość
    \item REDO, powtórzenie transakcji - zapewnia trwałość.
\end{itemize}
Podjęte czynności zależą od strategii SZBD: \\
Czy niezatwierdzona transakcja może nadpisać wartość w pamięci trwałej? \\
\begin{itemize}
    \item strategia steal: TAK
    \item strategia no-steal: NIE
\end{itemize}
Czy wszystkie zmiany muszą być zapisane w pamięci trwałej przed zatwierdzeniem transakcji?
\begin{itemize}
    \item strategia force: TAK
    \item strategia no-force: NIE
\end{itemize}

Strategia \textbf{no-steal + force} pozwala uniknąć wykonywania sekwencji UNDO i REDO. Jednak jest mało efektywna i istnieje ryzyko samozakleszczenia transakcji, jeśli braknie miejsca w buforze.
Częściej wykorzystywana jest strategia \textbf{steal + no-force} opakowana przez \textbf{protokół WAL (write-ahead log)}.
Oprócz plików z danymi przechowywany jest dziennik z logami (dziennik), w którym zapisywane są wszystkie zmiany przed wprowadzeniem ich na dysku.
Zawiera on informacje wystarczające odzyskania zawartości bazy za pomocą operacji UNDO i REDO.
Żeby dzienniki nie rozrastały się w nieskończoność, tworzy się punkty kontrolne. W przypadku awarii powtarzane są tylko transakcja zatwierdzona po ostatnim punkcie kontrolnym, a niezatwierdzone są wycofywane.

\subsection{Algorytm ARIES}
Algorytm ARIES (Algorithm for Recovery and Isolation Exploiting Semantics) służy do odtwarzania transakcji, korzystając z dziennika WAL i blokad.
Składa się z trzech faz:
\begin{enumerate}
    \item Analiza: rekonstrukcja Tabeli Brudnych Stron (DPT) i Tabeli Transakcji (TT), wyznaczenie pierwszego wpisu o zmianach tworzących brudną stronę
    \item Faza REDO: powtórzenie wszystkich operacji od momentu powstania brudnej strony, przywrócenie stanu sprzed awarii
    \item Faza UNDO: wycofanie efektów niezatwierdzonych transakcji, dodanie wpisów kompensacyjnych do dziennika
\end{enumerate}


\end{document}
