\subsection*{Lista}
To zbiór elementów z liniowym porządkiem i możliwością wykonania na nich operacji ADD, DEL, FIND, RETRIEVE.

Implementacje:
\begin{itemize}
    \item \textbf{Tablica} - pozycją na liście jest indeks \\
        - cykliczna lub nie
    \item \textbf{Lista wskaźnikowa} - pozycją na liście jest wskaźnik \\
        - pojeynczo wiązana (jednokierunkowa) lub podwójnie wiązana (dwukierunkowa), z głową i ogonem lub bez, cykliczna lub nie
    \item \textbf{Lista kursorowa} - pozycją na liście jest indeks w tablicy elementów
\end{itemize}

\subsection*{Stos}
To struktura LIFO (Last In, First Out) z operacjami: PUSH, POP, TOP, EMPTY, CLEAR.

Implementacje:
\begin{itemize}
    \item \textbf{Tablica} z indeksem szczytu,
    \item \textbf{Lista dynamiczna} - wskaźnikowa lub kursorowa,
    \item \textbf{Lista ADT}: PUSH \( \to \) ADD\_FIRST, POP \( \to \) DEL\_FIRST.
\end{itemize}

\subsection*{Kolejka}
Struktura FIFO (First In, First Out) z operacjami: ENQUEUE, DEQUEUE, FRONT, EMPTY, CLEAR.

Implementacje:
\begin{itemize}
    \item \textbf{Tablica cykliczna} - wykorzystuje indeksy początku i końca.
    \item \textbf{Lista wskaźnikowa} - dodawanie na koniec, usuwanie z początku.
    \item \textbf{ADT List}: ENQUEUE \( \to \) ADD\_END, DEQUEUE \( \to \) DEL\_FIRST.
\end{itemize}

\subsection*{Kolejka dwustronna}
Struktura, w której możliwe jest dodawanie i usuwanie elementów na obu końcach. Operacje: PUSH\_FRONT, PUSH\_BACK, POP\_FRONT, POP\_BACK, FRONT, BACK.

Implementacje:
\begin{itemize}
    \item lista dwukierunkowa,
    \item lista cykliczna (z wartownikiem lub bez),
    \item tablica (mniej efektywna przy wielu operacjach z przodu).
\end{itemize}

Ogólnie tablice zapewniają szybki dostęp do elementów, ale mają kosztowne operacje dodawania/usuwania. Listy wskaźnikowe umożliwiają efektywne wstawianie/usuwanie, ale dostęp bezpośredni jest trudniejszy. 