\subsection{System o podstawie \( p \)}

Liczbę zapisaną w systemie pozycyjnym o podstawie \( p \) można przedstawić jako wielomian:

\[
    (b_n, b_{n-1}, \dots, b_0)_p = b_n \cdot p^n + b_{n-1} \cdot p^{n-1} + \ldots + b_1 \cdot p^1 + b_0 \cdot p^0
\]

Żeby skonwertować liczbę do systemu dziesiętnego i z powrotem można wykorzystać schemat Hornera:

\[
    W(p) = ((((b_n \cdot p + b_{n-1}) \cdot p + b_{n-2}) \cdot p + \ldots + b_2) \cdot p + b_1) \cdot p + b_0
\]

\subsection{Reprezentacja liczb całkowitych ze znakiem}
\subsubsection*{Znak-moduł prosty (ZMP)}
Po prostu dodajemy bit znaku (0 - liczba dodatnia, 1 - liczba ujemna) jako pierwszy bit. Pozostałe bity kodują wartość bezwzględną liczby.

\textbf{Wady:} Dwie reprezentacje zera, skomplikowane operacje arytmetyczne.

\subsubsection*{Kod uzupełnieniowy do 1 (UZP1)}
\begin{itemize}
    \item Dla liczb dodatnich - bez zmian (pierwszy bit 0)
    \item Dla liczb ujemnych - zamiana każdego bitu na przeciwny
\end{itemize}

\textbf{Wady:} Dwie reprezentacje zera, skomplikowane operacje.

\subsubsection*{Kod uzupełnieniowy do 2 (UZP2)}
To najczęściej stosowany sposób kodowania.
\begin{itemize}
    \item Dla liczb dodatnich: bez zmian
    \item Dla liczb ujemnych: odjęcie 1 od \( \abs{x} \), zamiana bitów na przeciwne, dodanie 1
\end{itemize}
Przykład:
\[
    \text{UZP2}(17) = 0001 \ 0001
\]
\[
    \text{UZP2}(-17) = 1110 \ 1111
\]

\textbf{Zalety:} Jedna reprezentacja zera, poprawne działanie operacji arytmetycznych.

\subsubsection*{Kod nadmiarowy (NDM)}
Kodowanie za pomocą stałego przesunięcia \( ndm \):
\[
    \text{NDM}(x) = x + ndm, \ \text{gdzie } ndm = 2^{k-1}
\]
Wszystkie liczby są przesunięte o \( ndm \), a ich kod binarny mieści się na \( k \) bitach.

\textbf{Zalety:} Jako jedyne z omawianych zachowuje kolejność liczb, czyli \( x < y \implies \text{NDM}(x) < \text{NDM}(y) \).

\subsection{Reprezentacja liczb rzeczywistych}
Ułamki w formacie komputerowym można czytać w ten sposób:
\[
    (d_n, d_{n-1}, \dots, d_1, 0)_p = d_n \cdot \left(\frac{1}{p}\right)^n + d_{n-1} \cdot \left(\frac{1}{p}\right)^{n-1} + \ldots + d_1 \cdot \left(\frac{1}{p}\right)^1
\]
Przykład:
\[
    (101, 110011)_2 = 5\frac{110011}{1000000} = 5 + \frac{1}{2} + \frac{1}{4} + \frac{1}{32} + \frac{1}{64} = 5\frac{16+4+2+1}{64} = 5\frac{51}{64}
\]

\subsubsection*{Zapis stałopozycyjny}
\begin{itemize}
    \item Liczby zapisane z ustaloną liczbą cyfr przed i po przecinku.
    \item Część całkowita: np. UZP2, część ułamkowa: bez kodowania.
\end{itemize}
Wada: nieefektywne wykorzystanie pamięci, ograniczony zakres.

\subsection*{Zapis zmiennoprzecinkowy}
Liczby są zapisywane w postaci
\[
    x = m \cdot p^c,
\]
gdzie \( m \) to mantysa, \( c \) to cecha, \( p \) to podstawa (zwyczajowo 2).
Ponieważ mantysa jest postaci liczba całkowite + ułamek, to w rejestrze zapisywany jest tylko ułamek. Najczęściej cecha kodowana jest kodem nadmiarowym, a mantysa jako znak-moduł.

Kodowanie:
\begin{enumerate}
    \item Zapis liczby w postaci binarnej
    \item Normalizacja mantysy: \( 1,\ldots \cdot 2^c \)
    \item Zakodowanie cechy: \( c + \text{ndm} \)
    \item Zakodowanie zaokrąglonej mantysy
\end{enumerate}

\subsubsection*{Specjalne wartości}
\begin{itemize}
    \item \( \pm 0 \): \( c = 0,\; m = 0 \)
    \item \( \pm\infty \): \( c = 255,\; m = 0 \)
    \item NaN: \( c = 255,\; m \neq 0 \)
    \item Liczby bliskie zeru: \( c = 0, m \neq 0 \)
\end{itemize}