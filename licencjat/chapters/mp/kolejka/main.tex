\subsection{Kopiec (heap)}
Kopiec jest drzewem, w którym każdy wierzchołek \( v \) spełnia warunek key(parent(i)) \( \geq \) key(i), czyli na dowolnej ścieżce w dół drzewa wartości maleją.
Omówiony zostanie max-kopiec, w wersji min-kopiec nierówność jest odwrotna.

W przypadku kopca binarnego każdy wierzchołek wewnętrzny ma dwa następniki, a poziom liści jest wypełniony od lewej strony.
Wysokość kopca binarnego to \( h = \floor{\lg n} \).

\subsubsection*{Implementacja}
Strukturę kopca można zrealizować przy pomocy zwykłej tablicy A o rozmiarze równym zaokrągleniu \( n \) w górę do najbliższej potęgi 2.
\begin{itemize}
    \item A[1] - korzeń
    \item A[2i] - lewy następnik A[i]
    \item A[2i+1] - prawy następnik A[i]
    \item A[i/\!/2] - poprzednik A[i]
\end{itemize}

Jeśli wartość A[k] zwiększy się, trzeba uaktualnić wierzchołki na ścieżce od A[k] do korzenia.
Wykonuje się to, trochę jak przy sortowaniu przez wstawianie, przez \textbf{przesiewanie w górę (sift-up)}:
\begin{verbatim}
siftUp(A, k):
    j = k;
    i = k//2;
    tmp = A[k];
    while j > 1 and A[i] < tmp do
        A[j] = A[i];
        j = i;
        i = i//2;
    A[j] = tmp;
\end{verbatim}

Jeśli wartość A[k] zmniejszyła się, trzeba uaktualnić wierzchołki w poddrzewie A[k], do czego służy \textbf{przesiewanie w dół}:
\begin{verbatim}
siftDown(A, k, n):
    tmp = A[k];
    while k <= n//2 do
        j = 2*k;
        if j < n and A[j] < A[j+1] then
            j++;
        if tmp >= A[j] then
            break;
        A[k] = A[j];
        k = j;
    A[k] = tmp;
\end{verbatim}
Złożoność tych operacji wynosi \( \Theta(\lg n) \).


\subsection{Kolejka priorytetowa}
Kolejka priorytetowa obsługuje trzy rodzaje operacji:
Operacje kolejki priorytetowej na kopcu:
\begin{itemize}
    \item insert(x): wstaw element x do kolejki Q
    \begin{verbatim}
    insert(x):
        A[++n] = x;
        siftUp(A, n);
    \end{verbatim}
    Złożoność: \( \Theta(\lg n) \)
    \item max(Q): podaj wartość największego elementu w kolejce Q
    \begin{verbatim}
    max():
        return A[1];
    \end{verbatim}
    Złożoność: \( \Theta(1) \)
    \item deleteMax(Q): usuń największy element z kolejki
    \begin{verbatim}
    deleteMax():
        A[1] = A[n];
        siftDown(A, 1, --n);
    \end{verbatim}
    Złożoność: \( \Theta(\lg n) \)
\end{itemize}

\subsection{Sortowanie metodą kopca (heapsort)}
Mając do dyspozycji kolejkę priorytetową, żeby posortować ciąg \( n \) obiektów, wystarczy włożyć każdy z nich do kolejki, po czym \( n \)-krotnie wykonać operację deleteMax.
Elementy opuszczają kolejkę w porządku malejącym. Złożoność takiego sortowania to \( \Theta(n \lg n) \). Wadą tej metody jest jednak, że potrzebuje dodatkowej tablicy na przechowywanie kolejki (kopca).

Ulepszeniem sortowania z wykorzystaniem kopca jest Heapsort (wersja Williamsa):
\begin{verbatim}
heapsort(A, n):
    // Konstrukcja kopca
    for k = n//2 downto 1 do
        siftDown(A, k, n);

    // Usuwanie maksimum
    while n > 1 do
        swap(A[1], A[n]);
        n--;
        siftDown(A, 1, n);
\end{verbatim}

Pozostaje jeden istotny mankament, mianowicie średni czas jest prawie równy pesymistycznemu. Heapsort przegrywa więc z quicksortem dla dużych danych i również jest niestabilny numerycznie
Rozwiązaniem jest wersja Floyda algorytmu, która zmniejsza średnią liczbę porównań około dwukrotnie.
W fazie usuwania maksimum korzysta ona ze zmodyfikowanej wersji przesiewania. Najpierw przesuwa w górę całą ścieżkę od korzenia do liścia (większy z następników jest przesuwany w miejsce
rodzica). Wracając od liścia w górę elementy mniejsze od przesiewanego są z powrotem przesuwane w dół.
W tej wersji średnia liczba porównań wynosi \( \lg n + c \), a pesymistyczna, tak jak w wersji Williamsa \( 2\lg n \).

\begin{verbatim}
siftDownFloyd(A, k, n):
    tmp = A[k];
    i = k;
    while i <= n//2 do
        j = 2*i;
        if j < n and A[j] < A[j+1] then
            j++;
        A[i] = A[j];
        i = j;
    while i > k and A[i//2] < tmp do
        A[i] = A[i//2];
        i = i//2;
    A[i] = tmp;
\end{verbatim}

Zalety sortowania metodą kopca:
\begin{itemize}
    \item pesymistyczna złożoność \( \Theta(n \lg n) \),
    \item brak pamięci roboczej (również niejawnej na stos rekurencji)
    \item prostota implementacji
\end{itemize}