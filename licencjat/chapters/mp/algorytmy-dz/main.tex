Algorytmy ,,Dziel i Zwyciężaj'' opierają się na strategii bitewnej podziel - pokonaj - połącz. Oznacza to, że oryginalne zadanie zostaje
podzielone na mniejsze i potencjalnie łatwiejsze do rozwiązania podzadania, których wyniki otrzymane rekurencyjnie są scalane w ostateczny. 
Zadania o rozmiarze poniżej pewnej ustalonej stałej są rozwiązywane zwykłą metodą. Podział podzadań niekiedy wymaga dodatkowych obliczeń.

\subsection{Złożoność algorytmów D-Z}
\begin{theorem}[o rekurencji uniwersalnej (Master Theorem)]
    Załóżmy, że podział generuje \( a \geq 1 \) podzadań o rozmiarze \( \floor{\frac{n}{b}} \) lub \( \ceil{\frac{n}{b}} \), gdzie \( b \geq 1 \)
    oraz złożoność algorytmu jest wyrażona równaniem
    \[ T(n) = a \cdot T(n/b) +\Theta(n^p \log^qn), n > c, \]
    \[ T(n) = \Theta(1), n \leq c, \]
    gdzie \( p,\; q \geq 0 \) oraz \( \log_b a = d \). Wówczas:
    \begin{enumerate}
        \item jeśli \( d > p \), to \( T(n) = \Theta(n^d) \),
        \item jeśli \( d = p \), to \( T(n) = \Theta(n^d \log^{q+1}n) \),
        \item jeśli \( d < p \), to \( T(n) = \Theta(n^p \log^qn) \).
    \end{enumerate}
\end{theorem}


\subsection{Algorytm mnożenia macierzy Strassena}
Dane są macierze kwadratowe \( A, B \in \real^{n \times n},\; n = 2^k \). Celem jest obliczyć \( C = A \cdot B \).
Możemy potraktować każdą macierz jako macierz \( 2 \times 2 \), czyli każdy z 4 elementów jest podmacierzą kwadratową rozmiaru \( \frac{n}{2} \times \frac{n}{2} \).
Oznaczmy podmacierze \( A \) przez \( A_{1,1}, A_{1,2}, A_{2,1}, A_{2,2} \), analogicznie dla \( B \) i \( C \). Standardowy algorytm wykonuje 8 mnożeń:
\[ C_{i,j} = A_{i,i,} \cdot B_{i,j} + A_{i,j} \cdot B_{j,j} \]
Złożoność podstawowego algorytmu jest więc równa \( T(n) = 8 \cdot T(n/2) + b \cdot n^2 = \Theta(n^3) \).

Poprawka Strassena polega na zmniejszenie liczby mnożeń do 7, poprzez sprytne obliczenia.
% \begin{center}
%     \( p_1 = a_{1,1} \cdot (b_{1,2}-b_{2,2}) \) \\
%     \( p_2 = (a_{1,1}+a_{1,2}) \cdot b_{2,2} \) \\
%     \( p_3 = (a_{2,1}+a_{2,2}) \cdot b_{1,1} \) \\
%     \( p_4 = a_{2,2} \cdot (b_{2,1}-b_{1,1}) \) \\
%     \( p_5 = (a_{1,1}+a_{2,2}) \cdot (b_{1,1}+b_{2,2}) \) \\
%     \( p_6 = (a_{1,2}-a_{2,2}) \cdot (b_{2,1}+b_{2,2}) \) \\
%     \( p_7 = (a_{1,1}-a_{2,1}) \cdot (b_{1,1}+b_{1,2}) \)
% \end{center}
% \begin{center}
%     \( c_{1,1} = p_5+p_4-p_2+p_6 \) \\
%     \( c_{1,2} = p_1+p_2 \) \\
%     \( c_{2,1} = p_3+p_4 \) \\
%     \( c_{2,2} = p_5+p_1-p_3-p_7 \)
% \end{center}
% (Powyższe wzory są dla zobrazowania, na czym polega trik. Ich znajomość nie jest wymagana na egzaminie!)
Kluczowy jest fakt, że metoda ta nie korzysta z przemienności mnożenia liczb, dzięki czemu może być wykorzystana do mnożenia podmacierzy.

Algorytm Strassena wygląda następująco:
\begin{verbatim}
function strassen_multiply(A11, ..., B11, ...):
    // scalanie wyniku mnożenia podmacierzy za pomocą wzorów Strassena
    // z rekurencyjnem wywołaniem multiply przy mnożeniu

function multiply(A, B):
    if n == 2 then
        return A * B
    strassen_multiply(split(A), split(B))
\end{verbatim}

Algorytm jest opisany równaniem 
\[ T(n) = 7 \cdot T(n/2) + \Theta(n^2), n > 2, \]
więc jego złożoność jest równa \( T(n) = \Theta(n^{\lg 7}) \approx \Theta(n^{2.808}) \).