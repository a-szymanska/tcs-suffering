Algorytm przeglądu grafu wszerz (BFS) przechodzi po wierzchołkach, zaczynając od ustalonego wierzchołka początkowego \( u \) i odwiedza je w kolejności niemalejącej odległości od \( u \).

Pełny algorytm BFS wygląda następująco:
\begin{verbatim}
// G - graf, u - wierzchołek początkowy
// c[v] = WHITE - kolor v (WHITE - nieodwiedzony, GRAY - w kolejce, BLACK - odwiedzony)
// d[v] = INF - odległość v od u
// p[v] = NULL - poprzednik v w drzewie rozpinającym
// Q - kolejka
c[u] = GRAY, d[s] = 0, p[s] = NULL
Q.enqueue(s)
while !Q.empty() do
    u = Q.dequeue()
    for each v in Adj[u] do
        if c[v] == WHITE then
            c[v] = GRAY
            d[v] = d[u] + 1
            p[v] = u
            Q.enqueue(v)
    c[u] = BLACK
\end{verbatim}
Dla każdego wierzchołka \( v \), obliczona odległość d[\(v\)] rzeczywiście jest odległością od \( v \) do \( u \), co można udowodnić indukcyjnie.

\subsection{Złożoność}
Ponieważ każdy wierzchołek v jest wstawiany do kolejki i usuwany z niej dokładnie jeden raz, to pętla \texttt{while} wykona się \( n \) razy.
Również każda krawędź wychodząca z wierzchołka jest badana dokładnie raz, więc pętla \texttt{for each} w sumie wykona \( 2m \) iteracji (\( m \) dla grafu skierowanego) - tyle ile wynosi suma długości wszystkich list sąsiedztwa.
Całkowita złożoność jest zatem równa \( \Theta(n + m) \).

\subsection{Zastosowania}
\begin{itemize}
    \item Wyznaczanie odległości wierzchołków od wierzchołka startowego
    \item Wyznaczanie najkrótszej ścieżki łączącej dwa wierzchołki
    \item Wyznaczanie spójnej składowej wierzchołka
    \item Wyznaczanie drzewa rozpinającego: \\
    \( T = (V, \{(p[u], u), u \in V, u \neq s\}) \) rozpina graf \( G=(V, E) \).
    \item Sprawdzanie dwudzielności grafu:
    \begin{verbatim}
        isBipartite(V, E, d):
            for each (u,v) in E do
                if (d[u] == d[v]) return false
            return true       
    \end{verbatim}
\end{itemize}