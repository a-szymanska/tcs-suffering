Oba algorytmy są przykładami zastosowania metody ,,Dziel i Zwyciężaj''.

\subsection{Sortowanie przez scalanie (mergesort)}
Algorytm wygląda następująco:
\begin{verbatim}
merge(A, l, m, r, B):
    static int B[n];
    for (i = l; i <= r; i++)
        B[i] = A[i];
    i = l; j = m+1; k = l;
    while (i <= m and j <= r)
        if (B[i] <= B[j])
            A[k++] = B[i++];
        else
            A[k++] = B[j++];
    while (i <= m)
        A[k++]=B[i++];
    while (j <= r)
        A[k++]=B[j++];

mergesort(A, l, r) // sortowanie A[l..r]
    if (l == r)
        return;
    m = (l+r) // 2 ;
    mergesort(a, l, m);
    mergesort(a, m+1, r);
    merge(A, l, m, r, B);
\end{verbatim}

\subsubsection*{Złożoność}
Algorytm można opisać równaniem:
\[ T(n) = T( \floor{n/2}) + T(\ceil{n/2}), \]
zatem złożoność całego algorytmu jest równa \( \Theta(n \log n) \).

\subsubsection*{Zalety i wady}
+ złożoność pesymistyczna równa średniej \\
+ stabilność numeryczna \\
- pamięć robocza wielkości \( \Theta(n) \) \\
- złożoność czasowa przy posortowanych danych wejściowych również \( \Theta(n) \)

\subsection{Sortowanie szybkie (quicksort)}
Na każdym poziomie rekurencji, celem jest takie uporządkowanie elementów tablicy \( A \), żeby dla pewnego \( p \), \( l \leq p < r \)
zachodziło \( A[i] \leq A[j] \), dla dowolnych \( i = l, \dots, p, j = p+1, \dots, r \).

W podziale w wesji Hoare'a, dla ustalonego \( p \in A[l \dots r] \), idąc wskaźnikiem \ ( i \) od lewej, szukamy \( A[i] > p \) oraz idąc wskaźnikiem \( j \) od prawej, szukamy \( A[j] < p \).
Jeśli znalezione \( i < j \), to zamieniamy \( A[i] \) z \( A[j] \) i kontynuujemy przesuwanie wskaźników. Jeśli \( i \geq j \), to ustalamy \( p \) gdzieś w pobliżu \( i,\; j \) tak, żeby powyższy warunek był spełniony.

Procedura z podziałem Hoare'a wygląda następujęco:
\begin{verbatim}
getPivot(l, r)
    return (r+l) / 2;

partition(A, l, r)
    pivot_idx = getPivot(l, r);
    pivot_val = A[pivot_idx];
    swap(A[pivot_idx], A[r]);
    j = l;
    for i=l to r-1 do
        if A[i] < pivot_val then
            swap(A[i], A[j]);
            j++;
    swap(A[j], A[r]);
    return j;

quicksort(A, l, r)
    if l < r then
        i = partition(A, l, r);
        quicksort(A, l, i-1);
        quicksort(A, i+1, r);
\end{verbatim}
Mogą wystąpić przypadki brzegowe - piwot może być największym lub najmniejszym elementem. Dobrze jest więc dodać wartowników na początku i końcu tablicy.
Można też zaimplementować podział w wersji Lomuto:
\begin{verbatim}
partition(A, l, r)
    p = A[R];
    i = l-1;
    for (j = l; j < r; j++)
        if (A[j] <= p)
            i++;
            swap(A[i], A[j]);
    swap(A[i+1], A[r]);
    return i+1;
\end{verbatim}
Ta wersja jest łatwiejsza do implementacji i nie wymaga wartowników, wykonuje jednak do trzech razy więcej zamian i jest szczególnie nieefektywna, gdy jakaś wartość powtarza się wielokrotnie w tablicy.
Rozwiązaniem może być zastosowanie potrójnego podziału Lomuto, gdzie wartości równe piwotowi zostają zgrupowane w środkowej części tablicy, a rekurencja jest wywoływana tylko na lewej i prawej podtablicy.

Dobre praktyki:
\begin{itemize}
    \item Losowanie piwota - uniemożliwia zadanie złośliwych danych
    \item Wybór piwota jako mediany z \( A[l], A[(l+r)/2], A[r] \) - poprawia działanie dla posortowanych ciągów, skonstruowanie złośliwego przypadku nadal jest możliwe
    \item Pozostawienie małych podzadań nieposortowanych, a po zakończeniu procedury \texttt{quicksort} posortowanie całej tablicy przez wstawianie (ostatni krok jest \( \Theta(n) \))
\end{itemize}

\subsubsection*{Złożoność}
W przypadku oczekiwanym (dla losowych danych), algorytm można opisać równaniem:
\[ T(n) = 2 \cdot T(n/2) + n+1, \]
zatem średnia złożoność algorytmu jest równa \( \Theta(n \log n) \).
Jednak w przypadku pesymistycznym złożoność algorytmu jest równa:
\[ T(n) = T(n-1) + n+1 = \frac{1}{2} \cdot n^2 + \frac{3}{2} \cdot n \]


\subsubsection*{Zalety i wady}
+ duża efektywność w oczekiwanym przypadku \\
+ w miejscu - brak jawnej pamięci roboczej \\
- niestabilność numeryczna \\
- pamięć robocza pesymistycznie wielkości \( \Theta(n) \) - niejawna, stos rekursji w pesymistycznym przypadku wymaga \( \Theta(n) \) pamięci

Z ostatnim punktem można sobie radzić, sortując mniejszą podtablicę rekurencyjnie, a większą iteracyjnie.