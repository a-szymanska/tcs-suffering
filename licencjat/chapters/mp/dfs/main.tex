Algorytm przeglądu grafu w głąb (DFS) przechodzi po wierzchołkach, odwiedzając wszystkie na pewnej ścieżce i wycofując się dopiero, kiedy kolejne przejście jest niemożliwe. Koncepcyjnie przypomina metodę nawrotów (backtracking).

Pełny algorytm DFS wygląda następująco:
\begin{verbatim}
// G - graf, u - wierzchołek początkowy
// c[v] = WHITE - kolor v
//    (WHITE - nieodwiedzony, GRAY - w kolejce, BLACK - odwiedzony)
// p[v] = NULL - poprzednik v w drzewie rozpinającym
// t_in[v] - czas wejścia do v (kolejność w preorder)
// t_out[v] - czas wyjścia z v (kolejność w postorder)
// time = 0 - zmienna globalna

DFS(G):
    for each u in V(G) do
        if c[u] = WHITE then
        DFS_visit(u);

DFS_visit(u):
    c[u] = GRAY
    t_in[u] = ++time
    for each v in Adj[u] do
        if c[v] = WHITE then
            p[v] = u
            DFS_visit(v)
    c[u] = BLACK
    t_out[u] = ++time
\end{verbatim}

Krawędź (u, v) w drzewie DFS może być sklasyfikowana jako: \\
drzewowa (T) - do następnika w drzewie: \texttt{c(v) = WHITE} \\
powrotna, wsteczna (W) - do przodka w drzewie: \texttt{c(v) = GRAY} \\
wzdłużna (Z) - do potomka w drzewie: \texttt{c(v) = BLACK, t\_in(u) < t\_in(v)} \\
poprzeczna (P) - do starszego poddrzewa: \texttt{c(v) = BLACK, t\_in(u) > t\_in(v)} \\
W przypadku grafu nieskierowanego występują tylko krawędzie drzewowe i powrotne.

\begin{figure}[H]
    \centering
    \begin{subfigure}
        \centering
        \resizebox{0.45\textwidth}{!}{
            \begin{circuitikz}
    \tikzstyle{every node}=[font=\LARGE]
    \draw  (2.5,10.75) circle (0cm);
    \draw  (2.5,10.75) circle (2cm) node {\LARGE a} ;
    \draw  (8.75,10.75) circle (2cm) node {\LARGE b} ;
    \draw  (15,10.75) circle (2cm) node {\LARGE e} ;
    \draw  (21.25,10.75) circle (2cm) node {\LARGE f} ;
    \draw  (21.25,4.5) circle (2cm) node {\LARGE h} ;
    \draw  (15,4.5) circle (2cm) node {\LARGE g} ;
    \draw  (8.75,4.5) circle (2cm) node {\LARGE d} ;
    \draw  (2.5,4.5) circle (2cm) node {\Huge c} ;
    \draw [->, >=Stealth] (4.5,10.75) -- (6.75,10.75);
    \draw [->, >=Stealth] (10.75,10.75) -- (13,10.75);
    \draw [->, >=Stealth] (2.5,8.75) -- (2.5,6.5);
    \draw [->, >=Stealth] (8.75,8.75) -- (8.75,6.5);
    \draw [->, >=Stealth] (19.25,4.5) -- (17,4.5);
    \draw [->, >=Stealth] (13,4.5) -- (10.75,4.5);
    \draw [->, >=Stealth] (6.75,4.5) -- (4.5,4.5);
    \draw [->, >=Stealth] (4,6) -- (7.25,9.25);
    \draw [->, >=Stealth] (13.5,9.25) -- (10.25,6);
    \draw [->, >=Stealth] (19.75,9.25) -- (16.5,6);
    \draw [->, >=Stealth] (21.5,8.75) -- (21.5,6.5);
    \draw [->, >=Stealth] (21,6.5) -- (21,8.75);
\end{circuitikz}
        }
    \end{subfigure}%
    ~
    \begin{subfigure}
        \centering
        \resizebox{0.45\textwidth}{!}{
            \begin{circuitikz}
    \tikzstyle{every node}=[font=\LARGE]
    \draw  (2.5,10.75) circle (0cm);
    \draw  (2.5,10.75) circle (2cm) node {\LARGE a} ;
    \draw  (2.5,4.5) circle (2cm) node {\Huge b} ;
    \draw  (2.5,-1.75) circle (2cm) node {\LARGE d} ;
    \draw  (2.5,-8) circle (2cm) node {\LARGE c} ;
    \draw  (8.75,-1.75) circle (2cm) node {\LARGE e} ;
    \draw  (15,4.5) circle (2cm) node {\LARGE g} ;
    \draw  (21.25,4.5) circle (2cm) node {\LARGE h} ;
    \draw  (15,10.75) circle (2cm);
    \draw  (15,10.75) circle (2cm) node {\LARGE f} ;
    \draw [->, >=Stealth] (2.5,8.75) -- (2.5,6.5)node[pos=0.5, fill=white]{T};
    \draw [->, >=Stealth] (2.5,2.5) -- (2.5,0.25)node[pos=0.5, fill=white]{T};
    \draw [->, >=Stealth] (2.5,-3.75) -- (2.5,-6)node[pos=0.5, fill=white]{T};
    \draw [->, >=Stealth] (4,3.25) -- (7.75,0)node[pos=0.5, fill=white]{T};
    \draw [->, >=Stealth] (15,8.75) -- (15,6.5)node[pos=0.5, fill=white]{T};
    \draw [->, >=Stealth] (16.75,9.75) -- (20.25,6.25)node[pos=0.5, fill=white]{T};
    \draw [->, >=Stealth, dashed] (6.75,-1.75) -- (4.5,-1.75)node[pos=0.5, fill=white]{P};
    \draw [->, >=Stealth, dashed] (14.25,2.5) .. controls (13.5,-1.75) and (14.5,-9) .. (4,-3)node[pos=0.5, fill=white]{P};
    \draw [->, >=Stealth, dashed] (0.5,-8) .. controls (-0.5,-3.25) and (-1,-2.25) .. (0.5,4.25)node[pos=0.5, fill=white]{W};
    \draw [->, >=Stealth, dashed] (19.25,4.5) -- (17,4.5)node[pos=0.5, fill=white]{P};
    \draw [->, >=Stealth, dashed] (22.75,5.75) .. controls (20.5,9) and (19.75,9.5) .. (17,10.75)node[pos=0.5, fill=white]{W};
    \draw [->, >=Stealth, dashed] (0.5,10.75) .. controls (-3.25,2.5) and (-4,0.25) .. (0.25,-8)node[pos=0.5, fill=white]{Z};
\end{circuitikz}
        }
    \end{subfigure}
    \caption{Graf \( G \) i odpowiadające mu drzewo przeglądu DFS dla list sąsiedztwa posortowanych alfabetycznie}
    \label{fig:num_bits}
\end{figure}

\subsection{Złożoność}
Ponieważ odwiedzane wierzchołki dostają kolor, funkcja \texttt{DFS\_visit} jest wywołana dokładnie jeden raz dla każdego wierzchołka (w części głównej lub jako wywołanie rekurencyjne).
Pętla \texttt{for each} przeglądająca sąsiadów odwiedzanego wierzchołka ma koszt \( \Theta(\abs{Adj(u)}) \), więc suma iteracji tej pętli dla wszystkich wierzchołków jest równa \( \Theta(m) \).
Zatem całkowita złożoność DFS to \( \Theta(n + m) \).

\subsection{Zastosowania}
\subsubsection{Silnie spójne składowe}
W grafie skierowanym \( G=(V,E) \) silnie spójna składowa to maksymalny w sensie inkluzji podgraf indukowany,
w którym dla każdej pary wierzchołków istnieje ścieżka z pierwszego do drugiego, czyli dowolne dwa wierzchołki są w obrębie jednej silnie spójnej składowej wtw leżą na cyklu.

Algorytm wyznaczania silnie spójnych składowych wygląda następująco:
\begin{verbatim}
t_out = DFS(G)
E' = {(u, v): (v, u) in E}
G' = (V, E')
DFS(G') // główne wywołania DFS_visit(u) po malejących t_out[u]
\end{verbatim}
Każde drzewo lasu wygenerowanego w wywołaniu \texttt{DFS(GT)} jest oddzielną silnie spójną składową.
\begin{proof}
    Dowód indukcyjny po kolejnych generowanych drzewach:
    Jako przypadek bazowy służy brak drzew. Załóżmy, że wygenerowane drzewa \( T_1, \dots, T_{k-1} \) odpowiadają silnie spójnym składowym \( C_1, \dots, C_{k-1} \) w \( G \).
    Z faktu, że dwa wierzchołki są w jednej silnie spójnej składowej wtw leżą na cyklu, wprost wynika, że spójne składowe w \( G \) oraz \( G^T \) są takie same.
    Z korzenia \( T_k \) wychodzą więc białe ścieżki do wszystkich wierzchołków \( C_k \), zatem \( C_k \subseteq T_k \).
    Składowe \( C_1, \dots, C_{k-1} \) również mogą być osiągalne z korzenia \( T_k \), ale w tym momencie przeglądu ścieżki prowadzące do nich są czarne, więc nie zostaną one uwzględnione w  \( T_k \).
    W \( T_k \) nie wystąpi też żaden wierzchołek z \( C_{k+1}, \dots \) - dlaczego? Jeśli \( C_{k+i}, i > 0 \), jest składową różną od \( C_1, \dots, C_k \), to \( \text{t\_out}[C_{k+i}] < \text{t\_out}[C_k] \).
    Gdyby istniała krawędź \( (u, v) \) z \( C_{k+i} \) do \( C_k \), to \( \text{t\_out}[C_{k+i}] > \text{t\_out}[C_k] \), czyli nie ma takiego połączenia. Nie może też być krawędzi z \( C_k \) do \( C_{k+i} \).
    Wynika z tego, że \( C_k = T_k \), co dowodzi kroku indukcyjnego.
\end{proof}

Złożoność algorytmu jest równa \( \Theta(n + m) \).

\subsubsection{Cykl Eulera}
Cykl Eulera to cykl w grafie przechodzący każdą krawędzia dokładnie raz. Nieskierowany graf \( G=(V,E) \) posiada cykl Eulera wtedy i
tylko wtedy, gdy \( G \) jest spójny oraz każdy wierzchołek w \( G \) ma parzysty stopień.

Idea algorytmu:
\begin{enumerate}
    \item Wybierz dowolny wierzchołek początkowy.
    \item W każym kroku idź niewykorzystaną dotąd krawędzią wychodzącą z wierzchołka, w którym się znajdujesz.
    \item Powtarzaj krok 2, dopóki to możliwe.
\end{enumerate}
Jeśli algorytm zatrzymuje się w wierzchołku bez wyjścia, musi to być wierzchołek początkowy, ponieważ stopień każdego wierzchołka jest parzysty.
Przedstawiona metoda pozwala na znalezienie kolejnych cykli, które po połączeniu dają cykl Eulera. Można skleić je w złożoności liniowej, więc cały algorytm ma złożoność \( \Theta(n+m) \). 