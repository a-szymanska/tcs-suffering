SSL to protokół, który umożliwia bezpieczną komunikację w Internecie w ramach HTTPS, chroni dane w warstwie transportowej poprzez ich zaszyfrowanie, pozwala na weryfikacje tożsamości i zapewnia integralność danych. zapobiega atakom man in the middle. Używa do tego podpisów cyfrowych i haszowania za pomocą SHA2.
\subsection{Wystawianie certyfikatów}
Certyfikaty SSL strony uzyskują od urzędów certyfikacji na pewien okres czasu, po tym jak sprawdzą one w jakiś sposób (np. przez to, że uruchomimy jakiś skrypt je pingujący z serwera, do którego jest podpięta domena), naszą tożsamość. Wystawiają także specjalne certyfikaty prywatnym instytucją, które umożliwiają im podpisywanie kolejnych domen, tworzy się w ten sposób łańcuch certyfikatów, gdyż każdy certyfikat, aby móc potwierdzić jego autentyczność musi także podać certyfikat wystawiającego.
\subsection{Inicjacja połączenia}
\begin{enumerate}
    \item W trakcie handshake'u, serwer wysyła swój certyfikat SSL, albo ich łańcuch i użytkownik decyduje czy im ufać czy nie. Wysyła także informacje o metodzie szyfrowania, czasem też prosi klienta o jego certyfikat.
    \item Klien generuje klucz sesji, wysyła go serwerowi szyfrując go kluczem publicznym serwera, dzięki temu tylko serwer z jego unikalnym kluczem prywatnym może odszyfrować klucz sesji.
    \item Wysyłane są komunikaty potwierdzające pomyślne wymianę kluczami od teraz całość interakcji jest szyfrowana tajnym kluczem sesji.
\end{enumerate}
\subsection{Moduł SSL w Pythonie}
Poniżej prezentuje kod przykładowego prostego serwera i klienta 
\subsubsection{Klient}
\begin{verbatim}
import socket
import ssl

hostname = 'example.com'
context = ssl.create_default_context()

with socket.create_connection((hostname, 443)) as sock:
    with context.wrap_socket(sock, server_hostname=hostname) as ssock:
        ssock.send(b"GET / HTTP/1.1\r\nHost: example.com\r\n\r\n")
\end{verbatim}
\subsubsection{Serwer}
\begin{verbatim}
import ssl
import socket

context = ssl.create_default_context(ssl.Purpose.CLIENT_AUTH)
context.load_cert_chain(certfile='cert.pem', keyfile='key.pem')

with socket.socket(socket.AF_INET, socket.SOCK_STREAM, 0) as sock:
    sock.bind(('localhost', 8443))
    sock.listen(5)
    with context.wrap_socket(sock, server_side=True) as ssock:
        conn, addr = ssock.accept()
        conn.send(b"Bezpieczne połączenie TLS!")
        conn.close()
\end{verbatim}
