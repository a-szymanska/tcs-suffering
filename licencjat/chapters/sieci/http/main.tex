\subsection{Nagłówki}
W nagłówkach zawieramy dodatkowe informacje, takie jak data, język, typ danych czy informacje o hoście, aktualnie typów nagłówków jest bardzo dużo, w HTTP/1.0 tylko 14.\\
\subsubsection{Nagłówki w HTTP/1.0}
\begin{multicols}{2}
\begin{itemize}
    \item \textbf{Date}
    \item \textbf{Pragma} - zależne od implementacji
    \item \textbf{Authorization} - hasło uwierzytelniające
    \item \textbf{From} - adres email proszącego o dane (archaizm)
    \item \textbf{If-Modified-Since} - prosi o przesłanie dokumenty tylko jeśli był zmodyfikowany od podanej daty, używany do cache'owania
    \item \textbf{Referer} - adres strony, z której było przekierowanie
    \item \textbf{Server} - identyfikuje serwer i użyte w nim oprogramowanie
    \item \textbf{WWW-Authenticate} - określa sposób w jaki ma zostać przeprowadzone uwierzytelnienie użytkownika
    \item \textbf{Allow} - określa metody http obsługiwane przez serwer
    \item \textbf{Content-Encoding} - podaje format kompresji treści
    \item \textbf{Content-Length} - długość w bajtach przesyłanej wiadomości, dla danych przesyłanych z serwera obowiązkowy
    \item \textbf{Content-Type} - w jakim formacie jest dokument (html, pdf i.t.d.)
    \item \textbf{Expires} - data, po której dokument jest nieaktualny, używany do cache'owania
    \item \textbf{Last-Modified} - data ostatniej modyfikacji, używany do cache'owania
\end{itemize}
\end{multicols}
\subsection{Statusy}
Na zapytanie dostajemy od serwera odpowiedź z kodem statusu i opcjonalnie z jakimś plikiem (jeśli się wszystko powiodło)\\
\begin{itemize}
  \item \textbf{1xx} oznaczają, że serwer otrzymał poprawny request, i jeszcze go nie prze-procesował 
  \item \textbf{2xx} Sukces, robię to co mi kazano
  \item \textbf{3xx} Przekierowania
  \item \textbf{4xx} Błąd po stronie klienta
  \item \textbf{5xx} Błąd po stronie serwera
\end{itemize}

\subsubsection{Przykładowe, i najczęściej używane statusy}
\begin{multicols}{2}
\begin{itemize}
    \item 200 OK
    \item 201 Created
    \item 202 Accepted
    \item 204 No Content
    \item 301 Moved Permanently
    \item 302 Moved Temporarily
    \item 304 Not Modified
    \item 400 Bad Request
    \item 401 Unauthorized
    \item 403 Forbidden
    \item 404 Not Found
    \item 418 I am a teapot
    \item 500 Internal Server Error
    \item 501 Not implemented
    \item 502 Bad Gateway
    \item 503 Service Unavailable
\end{itemize}
\end{multicols}
\subsection{Różnice TCP}
\begin{itemize}
  \item HTTP/1.0 \\
    Każde żądanie wymaga nowego połączenia TCP co skutkowało dużym obciążeniem serwera, za każdym razem trzeba robić uwierzytelnienia i.t.p. 
  \item HTTP/1.1 \\
    Domyślnie używa persistent connections – jedno połączenie TCP może obsłużyć wiele żądań i odpowiedzi. 
  \item HTTP/2.0 \\
    Wprowadza binarny format nagłówków oraz kompresuje wiadomość co zmniejsza zużycie pasma. Wprowadza także funkcje multiplexingu, można w jednym pakiecie odpowiedzieć na wiele żądań http. 
  \item HTTP/3.0 \\
    Nie używa już TCP tylko QUIC, który pozwala na szybsze nawiązywanie połączenia i wydajniejszy multiplexing.
\end{itemize}
