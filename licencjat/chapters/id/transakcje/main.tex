\textbf{Transakcja} to jedna lub wiele operacji czytania i pisania oraz porządkiem wykonania ich traktowanych jako całość.
Mogą zostać wykonane albo wszystkie z nich, albo żadna. Jeśli coś pójdzie nie tak, transakcja może zostać wycofana (\textit{abort / rollback}).

Celem nie jest oczywiście wykonywanie transakcji szeregowo, tylko współbieżnie na ile to możliwe. Żeby efektem nie było więcej problemów niż przyspieszenia, wymaga to jednak zachowania własności ACID:
\begin{itemize}
    \item \textbf{atomowość (atomicity)} - transakcja jest neipodzielna, czyli nie może być wykonana częściowo \\
    Transakcja może być wycofana, na przykład, gdy prowadzi do rozspójnienia lub zakleszczenia. Podczas transakcji może też wystąpić awaria.
    Rozwiązaniem może być logowanie zmian lub tworzenie kopii zmienianych stron plików i umożliwianie odczytu dopiero po zatwierdzeniu (shadow paging).
    
    \item \textbf{integralność (consistency)} - po zatwierdzeniu transakcji baza danych musi spełniać wszystkie warunki poprawności \\
    Czasami dozwolone jest, żeby pojedyncze operacje w transakcji tymczasowo naruszało warunki, ale po zakończeniu transakcji musi być przywrócony poprawny stan.
    
    \item \textbf{izolacja (isolation)} - wynik równoległego wykonania transakcji musi być szeregowalny \\
    Mówiąc prościej, dwie transakcje wykonywane w tym samym czasie nie mogą wpływać na siebie.
    Aby zapewnić izolację wykorzystuje się metody pesymistyczne, które nie pozwalają na zaistnienie problemu, i optymistyczne, które naprawiają sytuację po zaistnieniu problemu.
    
    \item \textbf{trwałość (durability)} - po pomyślnym zakończeniu transakcji jej efekty w bazie danych są trwałe \\
    Oznacza to, że wyniki zatwierdzonej transakcji nie zostaną utracone nawet w przypadku twardych awarii (brak zasilania, błąd oprogramowania).
    Dane mogą zniknąć z pamięci głónej, ale pozostaną zapisane na dysku. Sposobami zapewnienia trwałości jest zapisywanie zmian na dysku przed zakończenie transakcji  lub przechowywanie wystarczających informacji o nich, tak żeby można było stan sprzed awarii.
\end{itemize}