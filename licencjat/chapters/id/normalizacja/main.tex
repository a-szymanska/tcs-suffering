Normalizacja to technika projektowania bazy danych, której celem jest uniknięcie niepożądanych cech w schematach relacji. Analizując zależności pomiędzy atrybutami, można zidentyfikować te, w których potencjalnie wystąpi redundancja lub anomalia modyfikacji.
Takie schematy należy rozłożyć na mniejsze (dekompozycja), żeby zminimalizować niebezpieczeństwo rozspójnienia danych. Mówiąc inaczej, normalizacja polega na dostosowaniu schematu relacji, czyli struktury tabel tak, żeby spełniały założenia postaci normalnej.

\subsection*{Postaci normalne}
\begin{itemize}
    \item Pierwsza postać normalna (1NF) - każda wartość powinna być atomowa, nie występują powtórzenia
    \item Druga postać normalna (2NF) - 1NF + atrybut niekluczowy nie jest częściowo zależny od klucza
    \item Trzecia postać normalna (3NF) - 2NF + każdy atrybut jest zależny bezpośrednio (nie przechodnio) od każdego klucza
    \item Postać normalna Boyce'a Codda (BCNF) - 2NF + każdy atrybut, od którego zależny jest inny atrybut jest superkluczem relacji
    \item Czwarta postać normalna (4NF) - każdy atrybut, od którego w relacji wielowartościowej* zależny jest inny atrybut jest superkluczem relacji 
\end{itemize}
* Jeśli \( X,\;Y,\;Z \) to w sumie wszystkie atrybuty \( \mathcal{R} \) oraz \( Z \cap (X \cup Y) = \empty \), to \( \mathcal{R} \) spełnia zależność wielowartościową \( X \twoheadrightarrow Y \),
gdy dla dowolnych \( t_1,\; t_2 \) jeśli \( t_1[X] = t_2[X] \), to istnieje taka krotka \( t_3 \), że \( t_1[X, Y] = t_3[X, Y] \) oraz \( t_2[Z] = t_3[Z] \).


Zyski:
\begin{itemize}
    \item zmniejszenie ilości danych przy zachowaniu wszystkich informacji
    \item uniknięcie anomalii przy dodawaniu, aktualizacji, usuwaniu
    \item ułatwienie tworzenia indeksów, więc przyspieszenie wykonania niektórych zapytań
    \item umożliwienie szybszego zarządzania transakcjami
    \item bardziej efektywne składowanie tabel na dysku
\end{itemize}