Żeby zrozumieć strategie optymalizacji, warto krótko przypomnieć schemat przetwarzania przez SZBD zapytań SQL na kod wykonywalny.
\begin{enumerate}
    \item zapytanie SQL \\
    \downarrow tłumaczenie kanoniczne
    \item wyrażenie algebry relacyjnej \\
    \downarrow optymalizacja heurystyczna
    \item dostosowane wyrażenie \\
    \downarrow optymalizacja na podstawie kosztu
    \item plan fizyczny \\
    \downarrow dodanie instrukcji wykonania operatorów
    \item kod wykonywalny
\end{enumerate}

Poniżej kolejne fazy optymalizacji nieco bardziej szcezgółowo:
\subsection*{Optymalizacja heurystyczna}
Polega na iteracyjnej modyfikacji wyrażenia kanonicznego, tak żeby uzyskać jak najlepszy plan logiczny. Zmiany nie mogą doprowadzić do tego, żeby wynik wykonania wyrażenia różnił się od wyniku dla początkowego wyrażenia.
Stosuje się następujące heurystyki:
\begin{itemize}
    \item Wykonuj operację selekcji jak najwcześniej - zmniejsza to rozmiar relacji
    \item Jeśli po iloczynie kartezjańskim występuje selekcja, zastąp obie operacje przez złączenie
    \item Stosując łączność selekcji wykonaj ja najwcześniej najbardziej restrykcyjną selekcję
    \item Wykonuj operację rzutowania jak najwcześniej - zmniejsza to rozmiar relacji
    \item Każdą operację wykonuj tylko raz - jeśli jakieś wyrażenie powtarza się w drzewie, zapamiętaj jego wynik
\end{itemize}

\subsection*{Optymalizacja oparta na podstawie kosztów}
Polega na analizie kosztów planów zapytania. Optymalizator szacuje koszt dla pewnego
podzbioru potencjalnych planów i wybiera ten ,,najtańszy''.

Przybliżony koszt planu jest wyliczany jako:
\begin{itemize}
    \item szacunkowy koszt wykonania każdego operatora (zależy od rozmiarów argumentów wejściowych)
    \item szacunkowy rozmiar wyniku wykonania każdego operatora
\end{center}
Na potrzeby tych obliczeń system bazodanowy przechowuje statystyki dotyczące relacji oraz atrybutów.


\subsection*{Próbkowanie}
System bazy danych może pobrać próbki z tabel i na tej podstawie oszacować selektywność. Próbki są aktualizowane po większych zmianach w tabeli.