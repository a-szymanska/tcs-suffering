
\subsection{Podział urządzeń I/O}

\subsubsection{Ze względu na szybkość}
\begin{itemize}
    \item wolne - dostarczają/obsługują dane znacznie wolniej niż CPU
    \item średnio szybkie - porównywalnie szybkie (ale wciąż wolniejsze) do CPU
    \item szybkie -- dostarczają dane szybciej niż CPU
\end{itemize}
możliwa jest modyfikacja prędkości przez (sprzętowe) buforowanie

\subsubsection{Ze względu na obsługę gotowości}
\begin{itemize}
    \item \textbf{Polling (polled)} - wymagają sprawdzenia, proces cyklicznie sprawdza czy urządzenie jest gotowe do wykonania operacji, najprostsze ale mało efektywne, używane zwykle jak przesył jest w miarę ciągły, monitor ethernet
    \item \textbf{Komunikacja przez przerwania sprzętowe (Asynchronous)} - informują o gotowości (przez przerwania), skomplikowane w implementacji, możliwość wystąpienia błędów związanych z przerwaniami, ale efektywne, klawiatury tak działają
    \item \textbf{Sampling} - zawsze gotowe do wykonania operacji, nie trzeba nic sprawdzać, najprostszy model, małe opóźnienia, ale nie wszystkie urządzenia są w stanie zapewnić pełną gotowość, używane do krytycznych operacji typu komunikacja z RAM
\end{itemize}

\subsection{Obsługa urządzeń}
\subsubsection{Port mapped (PIO)}
Urządzenia są dostępne poprzez specjalne adresy portów I/O, oddzielne od przestrzeni adresowej pamięci. Znajdują się w przestrzeni portów, w której standardowo jest miejsce na 64k portów.

Są różne typu portów, read-only, write-only, read-write i dual (dual obsługują różne funkcje w zależności od operacji)

W \textbf{standardzie x86} mamy dedykowane instrukcje do komunikacji z portami \texttt{in} i \texttt{out}

Odczyt/zapis każdego słowa wymaga (zazwyczaj wielu) operacji procesora, więc mapowanie przez port \textit{możliwe jest zwykle tylko dla wolnych urządzeń}, jest możliwość używania pollingu

\subsection{Memory mapped (MMIO)}
Urządzenia są mapowane do tej samej przestrzeni adresowej co pamięć, komunikujemy się więc używając normalnych instrukcji zapisu odczytu przez \textbf{pipe}.

Używa tych samych magistrali adresów i danych, co pamięć operacyjna, co pozwala na jednolitą obsługę operacji pamięciowych oraz operacji I/O.

Operacje odczytu i zapisu wymagają użycia standardowych instrukcji procesora co jest niemożliwe dla szybkich urządzeń.

Konieczne jest zarządzanie sposobem, w jaki dane z urządzeń peryferyjnych są buforowane przez cache procesora, aby zapewnić spójność danych. Istnieją różne strategie zarządzania cache'm:
\begin{itemize}
    \item \textbf{Wyłączony cache}: Cache jest wyłączony dla przestrzeni adresowej MMIO, co zapewnia, że dane są zawsze bezpośrednio odczytywane z urządzenia.
    \item \textbf{Write-back}: Dane są najpierw zapisywane do cache, a następnie do pamięci RAM. Może to poprawić wydajność, ale wymaga ostrożnego zarządzania spójnością danych.
    \item \textbf{Write-through}: Dane są zapisywane bezpośrednio do pamięci RAM i opcjonalnie do cache. Zapewnia to spójność, ale może być mniej wydajne.
    \item \textbf{Write-combining}: Dane są grupowane w cache przed zapisaniem do pamięci, co może zwiększyć wydajność dla pewnych typów operacji.
\end{itemize}

Możemy w dodatku korzystać z pollingu.

\subsubsection{Direct memory access (DMA)}
Urządzenie komunikuje się bezpośrednio z RAMem bez udziału CPU, dzięki czemu można w ten sposób podpiąć szybkie urządzenia, a polling staje się absolutnie niemożliwy, konieczne są specjalne protokoły zarządzania szynami adresów i danych, tak żeby urządzenie nie zeżarło całego przesyłu przez szynę danych / całej uwagi RAMu.

\paragraph{Silniki DMA}
Silniki DMA to programowalne urządzenia do transferu danych między urządzeniami a RAM, albo pomiędzy urządzeniami, pozwalają na odciążenie CPU, można dzięki temu nadbudować memory mapped I/O.

Wadami jest to, że konkurują z procesorem o dostęp do szyny systemowej i to że są drogie i skomplikowane

\subsubsection{Rozwiązania hybrydowe}
Prawda jest jednak taka, że najczęściej w naturze spotykane są rozwiązania hybrydowe, przykładem jest konfiguracja w której:
\begin{itemize}
    \item kontrola stanu gotowości za pomocą PIO/MMIO
    \item Transfer danych za pomocą DMA
    \item wbudowany DMA engine, programowany za pomocą PIO/MMIO
    \item dostępny zarówno polling, jak i działanie asynchroniczne przez przerwania
\end{itemize}
