mmap pozwala odwzorować pik bezpośrednio do przestrzeni adresowej procesu, tak aby mieć dostęp do danych jak do zwykłej pamięci.
\begin{itemize}
  \item Wywołanie mmap – rejestracja odwzorowania w jądrze.
  \item Próba odczytu przez proces
  \item System zarządzania pamięcią sprawdza obecność strony, jeśli jej nie ma to do jądra idzie page fault. Jeśli jest to odczytuje mu dalej normalnie.
  \item Jądro analizuje dostęp do pliku i lokalizuje żądane dane 
  \item Dane są wczytywane z dysku do RAM do nowej strony
  \item Jądro aktualizuje tablice stron i wznawia proces 
  \item Po załadowaniu stron, kolejne odczyty już są szybkie. 
\end{itemize}






Dane są wczytywane z dysku do RAM.

Jądro aktualizuje tablicę stron i wznawia proces.

Kolejne odczyty korzystają z już załadowanych stron (szybko).
