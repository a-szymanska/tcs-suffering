\chapter{ASD1}

\section{\textcolor{pink}{Algorytmy zachłanne, wymagane własności problemu, przykład: kody Huffmana.}}

\section{\textcolor{pink}{Programowanie dynamiczne jako metoda konstrukcji algorytmu, technika spamiętywania, przykład: najdłuższy wspólny podciąg lub optymalne drzewo BST.}}

\section{\textcolor{pink}{Zrównoważone drzewa wyszukiwań binarnych na przykładzie drzew AVL lub drzew czerwono-czarnych.}}

\section{\textcolor{pink}{Algorytmy Dijkstry oraz Bellmana-Forda znajdowania najkrótszych ścieżek w grafie z ustalonym źródłem.}}

\section{\textcolor{pink}{Znajdowanie najkrótszych ścieżek dla wszystkich par wierzchołków grafu, algorytm Warshalla-Floyda lub algorytm Johnsona.}}

\section{\textcolor{pink}{Minimalne drzewa rozpinające w grafie, algorytmy Jarnika-Prima oraz Kruskala.}}

\section{\textcolor{pink}{Sieci przepływowe, metoda Forda-Fulkersona, algorytm Edmondsa-Karpa.}}

\section{\textcolor{pink}{Najliczniejsze skojarzenia w grafie dwudzielnym, metoda przepływów lub algorytm Hopcrofta-Karpa.}}