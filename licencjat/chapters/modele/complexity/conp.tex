\begin{definition}
    Problem \(  L \in  \conp \iff \complement{L} \in \np \).
\end{definition}

\begin{lemma}
    Problem \( L \) jest C-trudny wtedy i tylko wtedy, gdy \( \complement{L} \) jest  coC-trudny. 
\end{lemma}
\begin{proof}
    \( \implies \)
        
    Niech \( L' \in C\). Jako że \(L\) jest trudny, to \(L' <_{p} L\).  Wobec tego istnieje taka funkcja \(f\), obliczalna w czasie wielomianowym, że \( x \in L' \iff f(x) \in L\).
    
    Równoważnie, \( x \not\in L' \iff f(x) \not \in L\). Oznacza to, że \( x \in \complement{L'} \iff f(x) \in \complement{L}\). Każdy problem należący do \(coC\) redukuje się do \( \complement{L} \), czyli \(\complement{L}\) jest \(coC\)-trudny.

    \( \impliedby \)
        
    Niech \( L' \in coC\). Skoro \(L\) jest trudny, to \(L' <_{p} \complement{L}\). To oznacza, że istnieje taka funkcja \(f\), że \( x \in L' \iff f(x) \in \complement{L} \). 
        
    Równoważnie, \( x \not \in L' \iff f(x) \not \in \complement{L} \). Stąd wiemy, że \( x \in \complement{L'} \iff f(x) \in L \). Wobec tego, każdy problem z \(C\) redukuje się do \(L\), czyli \(L\) jest \(C\)-trudny. 
\end{proof}


Przykład problemu \(\conp\)-zupełnego:
\begin{definition}
    Problem \textsc{TAUTOLOGY} definiujemy jako problem, w którym:
    \begin{itemize}
        \item wejście: formuła rachunku zdań \( \varphi \)
        \item wyjście: Czy \( \varphi \) jest tautologią?
    \end{itemize}
\end{definition}

\begin{theorem}
    \( \textsc{TAUTOLOGY} \) jest \( \textsc{coNP} \)-zupełny.
\end{theorem}
\begin{proof}
    Pokażemy, że \( \overline{\textsc{TAUTOLOGY}} \) jest \( \textsc{NP} \)-zupełny.
    
    Mając formułę \( \varphi \), chcemy sprawdzić, czy nie tautologią jest, czyli czy istnieje wartościowanie \( v \) takie, że \( v(\varphi) = 0 \). 
    
    Nie jest trudno zauważyć, że \( v(\lnot \varphi) = 0 \iff v(\varphi) = 1 \).
    
    Wynika z tego, że:
    \begin{enumerate}
        \item \( \textsc{SAT} <_p \overline{\textsc{TAUTOLOGY}}\), bo dostając zapytanie, czy formuła \( \varphi \) jest spełnialna, możemy ją zanegować i sprawdzić czy po zanegowaniu \textit{nie} jest tautologią (jeśli nie ma świadków spełnialności, to formuła jest tautologią; wpp nie jest). Z tego mamy, że \(\overline{\textsc{TAUTOLOGY}}\) jest \(\np\)-trudny, bo można zredukować do niego inny problem \(\np\)-trudny.
        \item \(\overline{\textsc{TAUTOLOGY}} <_p \textsc{SAT} \), bo dostając zapytanie, czy formuła \( \varphi \) \textit{nie} jest tautologią, możemy ją zanegować i sprawdzić, czy jest spełnialna. Wynika z tego, że \(\overline{\textsc{TAUTOLOGY}}\) jest w \(\np\), bo możemy przeprowadzić wielomianową redukcję tego problemu do problemu z klasy \(\np\). 
    \end{enumerate}
    
    Skoro \(\overline{\textsc{TAUTOLOGY}}\) jest w \(\np\) oraz jest \(\np\)-trudny, to jest \(\np\)-zupełny. A skoro tak, to \(\textsc{TAUTOLOGY}\) musi być \(\conp\)-zupełny, co było do pokazania. 
\end{proof}