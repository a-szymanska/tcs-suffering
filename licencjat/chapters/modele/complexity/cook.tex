\begin{theorem}[Cook-Levin]
    Problem \( \textsc{SAT} \) jest \np-zupełny.
\end{theorem}
\begin{proof}
    Oczywiście \( \textsc{SAT} \in \np \), co udowodniliśmy wcześniej.
    
    Pozostaje pokazać, że \textsc{SAT} jest \np-trudny -- jeśli \( L \in \np \), to istnieje redukcja z \( L \) do \( \textsc{SAT} \).
    
    Skoro \( L \in \np \) to istnieje niedeterministyczna Maszyna Turinga \( M \), która go akceptuje. Intuicyjnie -- będziemy śledzić obliczenie prowadzone przez \( M \) na zadanym słowie wejściowym \( w \) i zapisywać je jako formułę logiczną.
    
    Maszyna \( M \) działa w czasie \( p(\abs{x}) \), zatem możemy wyobrazić sobie tablicę o wymiarach \( p(\abs{x}) \times p(\abs{x}) \), w której wiersze reprezentują kolejne konfiguracje maszyny.

    Formuła logiczna, która chcemy skonstruować, ma opisywać poprawne przejścia w konfiguracjach maszyny.
    Do tego potrzebne jest kodowanie za pomocą formuł boolowskichpozycji wartości na taśmie.
    Definiujemy zmienną \( x_{ija} \), która oznacza, że w \(i\)-tym wierszu (w \(i\)-tej w kolejności konfiguracji), \(j\)-tej kolumnie (\(j\)-tym symbolu w konfiguracji) znajduje się symbol \(a\). 
    
    Przy takiej interpretacji chcemy określić warunki na konfigurację startową i końcową w taki sposób, aby istnienie wartościowania formuły na 1 było równoważne istnieniu obliczenia akceptującego.
    
    \subsubsection{Wartości zmiennych są spójne}
        Nie mogą istnieć zmienne \(x_{ija} = 1 \land x_{ijb} = 1\), bo to by znaczyło że w jednym miejscu na taśmie są 2 symbole.
        Nie może też być tak, że \( \forall_a x_{ija} = 0\), bo każda komórka jest zawsze wypełniona (blank również jest symbolem).
    
        Żeby to zagwarantować, definiujemy formułę \(\varphi_1\): 
        \[ 
            \varphi_1 = \bigwedge_{\substack{1 \leq i \leq p(n) \\ 1 \leq j \leq p(n)}} A_{ij} \land B_{ij} , 
        \]
        gdzie \(A_{ij}\) odpowiada formule wymuszającej, żeby chociaż jeden symbol znalazł się na taśmie, \(B_{ij}\) odpowiada formule wymuszającej, żeby w jednym miejscu taśmy nie było dwóch lub więcej symboli. Definiujemy je następująco:  
        \[ 
            A_{ij} = \bigvee_{a \in \Gamma} x_{ija}
        \]        
        \[
            B_{ij} = \bigwedge_{\substack{a\in\Gamma \\ b\in\Gamma \\ a \not = b}} \neg(x_{ija} \land x_{ijb})
        \]
        Obie formuły są wielomianowe oraz rozmiar całej tablicy są wielomianowe względem wejścia. 
    
    \subsubsection{Symulacja rozpoczyna się poprawnie}
        Aby w ogóle rozpocząć symulację NTM, trzeba ,,zainicjalizować'' jej konfigurację startową: na pierwszym miejscu znajduje się stan startowy \(q_s\), dalej słowo \(w\), a potem blanki (tak by długość wiersza wynosiła \(p(|w|)\)). 
        
        Definiujemy formułę wymuszającą początkowe wartościowanie:
        \[
            \varphi_2 = a_{1,1, q_{s}} \land a_{1,2, w_1} \land a_{1,2, w_2} \land \dots \land a_{1, k, \blank} \land a_{1, k+1, \blank} \land \dots \land a_{1, p(|w|), \blank}
        \]
        Długość formuły \(\varphi_2\) jest wielomianowa względem \(w\). 
    
    \subsubsection{Formuła jest spełnialna, jeśli NTM przeszła do stanu akceptującego}
        Dla uniknięcia dziwnych przypadków zakładamy, że jeśli NTM wejdzie do stanu akceptującego, to pozostaje w nim.

        Żeby sprawdzić, czy NTM akceptuje, wystarczy sprawdzić, czy którykolwiek ze znaków w ostatnim wierszu jest stanem akceptującym. Definiujemy zatem formułę:
        \[ 
            \varphi_3 = \bigvee_{q_f \in F } \bigvee_{1 \leq j \leq p(|w|)} x_{p(|w|), j, q_f}
        \]
        Jeśli uda się znaleźć stan akceptujący, to słowo jest akceptowane a sprawa zamknięta. Jeśli nie, to dane obliczenie nie akceptuje słowa.
    
    \subsubsection{Przejścia są wykonywane poprawnie}
        Trzeba zaimplementować jakoś przejścia.
        
        Zauważmy, że przejście głowicy powoduje jedynie lokalną zmianę konfiguracji -- z konfiguracji na konfigurację zmieniają się co najwyżej 3 symbole (stan, znak na prawo od wcześniej lokacji głowicy oraz głowica może przesunąć się w lewo, zamieniając swoją kolejnosć z symbolem po lewej).

        Musimy zatem zakodować każde możliwe przejście z \( \sigma \). Przydad się formuła wymuszająca, żeby symbole na współrzędnych \( (i_1, j_1) \) oraz \((i_2, j_2)\) były sobie równe:
        \[ 
            E(i_1, j_1, i_2, j_2) = \bigwedge_{a \in \Gamma} ((x_{i_1, j_1, a} \land x_{i_2, j_2, a}) \lor (\neg x_{i_1, j_1, a} \land \neg x_{i_2, j_2, a}) )
        \]
        
        Definiujmy formułę wymuszającą przejście w prawo dla stanu \(q\) i znaku \(a\), na współrzędnych \(i, j\), którą nazwiemy \(T_r(q, a, i, j)\)\footnote{Osoby zaniepokojone użyciem implikacji uspokajam, że \(p \implies q \equiv \neg p \lor q\).}:
        \[
            \begin{split}
            T_r(&q,a, i, j)\\
                &= \bigvee_{(q', a', 1) \in \delta(q,a)} \pars{\pars{x_{i,j,q} \land x_{i, j + 1, a}} \implies \pars{E(i, j-1, i+1, j-1) \land x_{i+1, j, a'} \land x_{i+1, j+1, q'}}}
            \end{split}
        \]
        Jeśli pod współrzędnymi \(i,j\) jest stan \(q\), to jeśli mamy przesunąć się prawo, chcemy aby zmienna na lewo była niezmieniona, na jego miejsce wskoczyła zmodyfikowana zmienna (która wcześniej była po jego prawej), a on sam zmienił swój stan na nowy. 
        
        Analogicznie kodujemy przesunięcia w lewo:
        \[
            \begin{split}
                T_l(&q, a , i, j)\\
                    &= \bigvee_{(q', a', -1) \in \delta(q,a)} \pars{\pars{x_{i,j,q} \land x_{i, j + 1, a}} \implies \pars{x_{i+1, j-1, q'} \land E(i, j-1, i+1, j) \land x_{i + 1,j + 1,a'}}}
            \end{split}
        \]
    
        Definiujemy formułę odpowiedzialną za zebranie tego wszystkiego w całość, by zapewnić poprawność przejść:
        \[ 
            C = \bigwedge_{\substack{q \in Q \\ a \in \Gamma \\ 1 \leq i \leq p(|w|) \\ 1 \leq j \leq p(|w|)}} T_r(q,a,i,j) \lor T_l(q,a,i,j) 
        \]
        Formuła \(C\) jest wielomianowa względem rozmiaru wejścia, bo rozmiary \(Q\) i \(\Gamma\) to stałe, więc dla każdego pola w tablicy (których jest wielomianowo wiele) dodajemy stałą liczbę formuł mających stałe długości. 
        
        Trzeba jeszcze wymyślić sposób, żeby w sytuacji, gdzie wartości są poza tym lokalnym \textit{miejscem niebezpiecznym}, zostały przepisane. 
        
        Potrzebna jest do tego formuła logiczna sprawdzająca, czy coś \textbf{nie} jest stanem:
        \[
            NQ(i,j) = \bigwedge_{q \in Q} \neg x_{ijq}
        \]
        Symbol może być przepisany z konfiguracji na konfigurację poniżej wtedy i tylko wtedy, gdy nie jest stanem i nie ma stanu obok siebie (po lewej ani prawej).
        Jeśli jest stanem (lub sąsiaduje z jakimś stanem), to jego otoczeniem w konfiguracji poniżej zajmuje się formuła \(C\).
        
        Definiujemy więc formułę wymuszającą przepisanie symbolu z konfiguracji na konfigurację poniżej: 
        \[ 
            K(i,j) = (NQ(i, j-1) \land NQ(i,j) \land NQ(i, j+1)) \implies E(i, j, i+1, j)
        \]
        
        Rozszerzamy to na wszystkie możliwe współrzędne:
        \[
            D = \bigwedge_{\substack{1 \leq i \leq p(|w|) \\ 1 \leq j \leq p(|w|)}} K(i,j)
        \]
        Formuła \(D\) jest wielomianowa względem wejścia, bo \(K(i,j)\) ma stałą długość, a z kolei tych formuł jest wielomianowo wiele -- jedna na każdą komórkę tablicy.
        
        Koniec końców, otrzymujemy wyrażenie wymuszające poprawne przejścia: 
        \[ 
            \varphi_4 = C \land D
        \]
        Formuła \(\varphi_4\) również ma rozmiar wielomianowy, jako że jest koniunkcją dwóch formuł długości wielomianowej. 
        
        \subsubsection{Pozbieranie w całość}
        Zbieramy to wszystko do kupy i otrzymujemy wyrażenie:
        \[ 
            \varphi = \varphi_1 \land \varphi_2 \land \varphi_3 \land \varphi_4
        \]
        Jest ono spełnialne wtedy i tylko wtedy, gdy NTM akceptuje słowo \(w\), a jego konstrukcja jest wielomianowa. Tym samym dowiedliśmy tw. Cooka-Levina.
\end{proof}