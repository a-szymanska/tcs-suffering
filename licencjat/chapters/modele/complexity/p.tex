\begin{definition}
    Język \( L \) jest w klasie P (polynomial) jeśli istnieje wielomian \( p \)
    oraz Deterministyczna Maszyna Turinga działająca w czasie \( p \) taka, że \( L(M) = L \).
\end{definition}

\begin{theorem}
    Język pusty jest w \(\p\). 
\end{theorem}
\begin{proof}
    Maszyna rozpoznająca ten język od razu przechodzi do stanu odrzucającego, a tym samym działa w czasie stałym.
\end{proof}

\begin{theorem}
    \( \textsc{DNF-SAT} \in \p \)
\end{theorem}
\begin{proof}
    Jeśli problem \textsc{SAT} jest w postaci \textsc{DNF}, to można rozwiązać go w czasie wielomianowym.
    By to zrozumieć, warto zobaczyć przykład formuły w \textsc{DNF}:
    \[ 
        \overbrace{(x_1 \land \neg x_2 \land \neg x_3)}^{\text{term DNF}} \lor \overbrace{(x_1 \land x_4)}^{\text{term DNF}} \lor \overbrace{(\neg x_5)}^{\text{term DNF}}    
    \] 
    Żeby formuła w \textsc{DNF} była spełnialna, któryś z jej termów musi być spełnialny, czyli nie może zawierać zmiennej i jej zaprzeczenia.

    Ten warunek jest łatwy do zbadania w czasie wielomianowym od rozmiaru wejścia. Po kolei sprawdzamy go dla każdej klauzuli -- jeśli którakolwiek z nich jest spełnialna, to cała formuła jest spełnialna.
\end{proof}

\begin{theorem}
    \( \textsc{2-SAT} \in \p \)
\end{theorem}
\begin{proof}
    Dowód nawiazuje do algorytmu znanego z ASD, który opiera się na następującej obserwacji:
    \[ 
        p \lor q \equiv (\neg p \implies q) \land (\neg q \implies p)
    \]
    Tworzymy skierowany graf literałów \(G\) dla formuły \(\varphi\), którą zadanej na wejściu. Na przykład dla formuły:
    \[
        (x_1 \lor \neg x_2) \land (x_3 \lor \neg x_3)
    \]
    graf ma 6 wierzchołków: \(x_1, \neg x_1, x_2, \neg x_2, x_3, \neg x_3\).

    Po rozpisaniu formuły, korzystając z wcześniejszej obserwacji:
    \[
        (x_1 \lor \neg x_2) \land (x_3 \lor \neg x_3) \equiv (\neg x_1 \implies \neg x_2) \land (x_2 \implies x_1) \land (\neg x_3 \implies \neg x_3) \land (x_3 \implies x_3),
    \]
    strzałki implikacji zostają krawędziami w grafie.

    \begin{center}
        \begin {tikzpicture}[-latex, auto, node distance=4cm and 5cm, on grid, semithick, literal/.style={circle, draw, minimum width=1cm}]
        \node[literal] (X1) {\(x_1\)};
        \node[literal] (X2) [right=of X1] {\(x_2\)};
        \node[literal] (X3) [right=of X2] {\(x_3\)};
        \node[literal] (Y1) [below=of X1] {\(\neg x_1\)};
        \node[literal] (Y2) [below=of X2] {\(\neg x_2\)};
        \node[literal] (Y3) [below=of X3] {\(\neg x_3\)};

        \path (Y1) edge [bend left=25] (Y2);
        \path (X2) edge [bend left=25] (X1);
        \path (Y3) edge [loop left] (Y3);
        \path (X3) edge [loop left] (X3);
        \end{tikzpicture}
    \end{center}

    Wyznaczamy silnie spójne składowe w tym grafie. Dla każdej pary wierzchołków w obrębie jednej składowej istnieje ścieżka. W tym przypadku, oznacza to istnienie ciągu implikacji pomiędzy dowolną parą literałów.
    Zatem wszystkie literały w obrębie jednej silnie spójnej składowej są sobie \textbf{równoważne}, czyli muszą być wartościowane tak samo, żeby formuła była prawdziwa.

    \begin{lemma}
        Formuła \(\varphi\) jest spełnialna wtedy i tylko wtedy, gdy w żadnej silnie spójnej składowej \(G\) nie występuje zmienna oraz jej zaprzeczenie. 
    \end{lemma}
    \begin{proof}
        Jeśli w obrębie silnie spójnej składowej występuje zmienna wraz ze swoim zaprzeczeniem, to w oczywisty sposób \(\varphi\) nie jest spełnialna. 

        ZW sytuacji, gdy nie ma takiej silnie spójnej składowej, kompresujemy każdą silnie spójną składową do jednego wierzchołka. Tak otrzymany graf silnie spójnych składowych jest DAGiem.
        Można więc wykonać na nim sortowanie topologiczne. Następnie, przeglądając literały w odwrotnym porządku topologicznym, przypisujemy im wartości. Wtedy wszystkie literały w obrębie danej silnie spójnej składowej są wartościowane tak samo.
        Jeśli wierzchołek nie ma przypisanego wartościowania, to wartościujemy go na 1, a wierzchołek zawierający zanegowane zmienne występujące w tym wierzchołku wartościujemy na 0. 

        Dla dowolnej silnej spójnej składowej \(S\), jeżeli dostaje wartościowanie 1, to istnieje \textbf{dokładnie} jedna silnie spójna składowa która zostanie wartościowana na 0. Załóżmy nie wprost, że tak nie jest:
        \begin{enumerate}
            \item Jeśli więcej niż jedna składowa dostała wartościowanie 0, to oznacza, że występują takie literały \(a, b\), że \(a \iff b\), ale \(\neg a \notiff \neg b\), co jest sprzeczne z konstrukcją grafu \(G\). 
            \item Analogiczny argument stosujemy by wykazać, że składowa wartościowana na 0 nie była już tak wartościowana wcześniej podczas działania algorytmu. 
        \end{enumerate}

        W ten sposób otrzymujemy bijekcję między silnie spójnymi składowymi (składowe, które zawierają jakieś literały przechodzą na składowe, które zawierają dokładnie negacje tych literałów). 

        Algorytm przypisywania wartości działa poprawnie, to znaczy z wierzchołka o wartościowaniu 1 nie ma krawędzi do wierzchołka o wartościowaniu 0.
        Załóżmy nie wprost, że doszło do takiej sytuacji: mamy zatem literały \(x\) i \(y\) takie, że \( x \implies y\), ale wartość \(y\) została ustawiona przez algorytm na 0, a \(x\) na 1. Zauważamy, że \(\neg y\) musiało wcześniej dostać wartościowanie 1 oraz \( \neg y \implies \neg x\) z definicji grafu \(G\).
        Skoro teraz wartościujemy spójną składową \(x\), to jest ona później w porządku topologicznym niż składowa \( \neg y\) i wcześniej niż składowa \( \neg x\), więc nie może istnieć ścieżka z \( \neg y\) do \( \neg x\). Prowadzi to do sprzeczności.
    \end{proof}

    Podsumowując, żeby stwierdzić, czy formuła \textsc{2-SAT} jest spełnialna, wystarczy sprawdzić czy istnieje jakaś ,,wewnętrznie sprzeczna'' silnie spójna składowa w \(G\). Aby explicite otrzymać wartościowanie, można użyć opisanego algorytmu. Zarówno sprawdzenie jak i powyższy algorytm są wielomianowe. 
\end{proof}