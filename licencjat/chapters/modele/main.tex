\chapter{Modele Obliczeń} 

\section{\textcolor{pink}{Omów hierarchię Chomsky'ego języków, ilustrując klasy i zależności między nimi odpowiednimi przykładami}}
Jednym z zadań SZBD jest zapewnianie tego, żeby zmiany dokonane przez widoczną lub zatwierdzoną transakcję były trwałe, a efekty transakcji wycofanej lub przerwanyj przez awarię nie.

Jest parę typów awarii:
\begin{itemize}
    \item awaria systemu: awaria sprzętu, błędy oprogramowania, uszkodzenie dysku
    \item błędy logiczne (np. naruszenie ograniczeń integralnościowych)
    \item błędy systemowe (np. zakleszczenia)
\end{itemize}
Do poradzenia sobie z pierwszym rodzajem wystarczy replikacja bazy i rozproszenie danych. W przypadku pozostałych dwóch trzeba odtworzyć bazę po awarii.
Ważne jest, żeby zminimalizować czas potrzebny do tego, ponieważ w trakcie odtwarzania bazy serwer musi być wyłączony dla użytkowników.
Służące do tego algorytmy składają się z dwóch części:
\begin{enumerate}
    \item zapisywanie informacji potrzebnych do odtworzenia stanu bazy na bieżąco podczas zwykłych operacji
    \item czynności po awarii, prowadzące do przywrócenia spójności, atomowości i trwałości
\end{enumerate}

\subsection*{Strategie SZBD}
Brudne strony, czyli zmieniane po ostatnim zapisie na dysku, mogą pozostawać w buforze długo po zatwierdzeniu transakcji. Jeśli nastąpi awaria, to naniesione zmiany są tracone.
Żeby zadbać o poprawny stan bazy można wykonać operacje:
\begin{itemize}
    \item UNDO, wycofanie zmian - zapewnia atomowość
    \item REDO, powtórzenie transakcji - zapewnia trwałość.
\end{itemize}
Podjęte czynności zależą od strategii SZBD: \\
Czy niezatwierdzona transakcja może nadpisać wartość w pamięci trwałej? \\
\begin{itemize}
    \item strategia steal: TAK
    \item strategia no-steal: NIE
\end{itemize}
Czy wszystkie zmiany muszą być zapisane w pamięci trwałej przed zatwierdzeniem transakcji?
\begin{itemize}
    \item strategia force: TAK
    \item strategia no-force: NIE
\end{itemize}

Strategia \textbf{no-steal + force} pozwala uniknąć wykonywania sekwencji UNDO i REDO. Jednak jest mało efektywna i istnieje ryzyko samozakleszczenia transakcji, jeśli braknie miejsca w buforze.
Częściej wykorzystywana jest strategia \textbf{steal + no-force} opakowana przez \textbf{protokół WAL (write-ahead log)}.
Oprócz plików z danymi przechowywany jest dziennik z logami (dziennik), w którym zapisywane są wszystkie zmiany przed wprowadzeniem ich na dysku.
Zawiera on informacje wystarczające odzyskania zawartości bazy za pomocą operacji UNDO i REDO.
Żeby dzienniki nie rozrastały się w nieskończoność, tworzy się punkty kontrolne. W przypadku awarii powtarzane są tylko transakcja zatwierdzona po ostatnim punkcie kontrolnym, a niezatwierdzone są wycofywane.

\subsection{Algorytm ARIES}
Algorytm ARIES (Algorithm for Recovery and Isolation Exploiting Semantics) służy do odtwarzania transakcji, korzystając z dziennika WAL i blokad.
Składa się z trzech faz:
\begin{enumerate}
    \item Analiza: rekonstrukcja Tabeli Brudnych Stron (DPT) i Tabeli Transakcji (TT), wyznaczenie pierwszego wpisu o zmianach tworzących brudną stronę
    \item Faza REDO: powtórzenie wszystkich operacji od momentu powstania brudnej strony, przywrócenie stanu sprzed awarii
    \item Faza UNDO: wycofanie efektów niezatwierdzonych transakcji, dodanie wpisów kompensacyjnych do dziennika
\end{enumerate}

\section{\textcolor{pink}{Omów lemat o pompowaniu dla języków bezkontekstowych. Podaj przykład
języka bezkontekstowego. Podaj przykład języka, który nie jest bezkontekstowy}}

Przykład języka bezkontekstowego znajduje się w sekcji \ref{cfl_example}.

\subsection{Lemat o pompowaniu dla języków regularnych}
\label{regular-pumping}
\begin{lemma}[O pompowaniu dla języków regularnych]
    Jeśli \( L \) jest językiem regularnym to:
    \[
        \exists_{n > 0}\; \forall_{w : \abs{w} \geq n} \exists_{xyz = w}\; \left( \abs{xy} \leq n \land \abs{y} \geq 1 \land \left( \forall_{i \in \natural}\; xy^iz \in L \right) \right)
    \]
\end{lemma}
\begin{proof}
    Skoro \( L \) jest językiem regularnym to istnieje DFA A, który rozpoznaje L.
    
    Niech \( n = \abs{Q} \) i weźmy dowolne słowo \( w \) dla którego \( m = \abs{w} \geq n \).
    
    Skoro słowo jest akceptowane to istnieje ścieżka \( s = q_0, \dots q_m \in F \).
    
    Mamy więc \( m + 1 > n \) stanów, czyli jakiś stan musi się powtarzać. 
    Z takich stanów wybieramy stan \( q_i \), którego pierwsze powtórzenie jest najwcześniej. Niech \( q_j \) będzie drugim wystąpieniem stanu \( q_i \).
    Mamy zatem \( x = a_0\dots a_{i-1} \), \( y = a_i\dots a_{j-1} \), \( z = a_j \dots a_{m-1} \).
    
    Oczywiście \( \abs{xy} \leq n \), bo inaczej \( q_i \) nie byłby stanem, który powtarza się najwcześniej.
    
    Skoro  \( q_i = q_j \) to słowo \( y \) może wystąpić dowolną (również zerową) liczbę razy w akceptowanym słowie, czyli dla dowolnego \( i \in \natural \) \( xy^iz \in L \).
\end{proof}
Jest to warunek konieczny, ale \textbf{niewystarczający} by język był regularny. Istnieją języki które nie są regularne, a spełniają warunki lematu o pompowaniu. 

\subsection{Lemat o pompowaniu dla języków bezkontekstowych}
\begin{theorem}[O pompowaniu dla języków bezkontekstowych]
    Jeżeli język \(L\) nad \(\Sigma^*\) jest bezkontekstowy, to: 
    
    \( \exists_{n_0 \in \natural} \) \\
    \( \forall_{w \in L}\; \card{w} \geq n_0 \) \\
    \( \exists_{a, b, c, d, e \in \Sigma^*} \hspace{5pt} w = abcde \land |bcd| \leq n_0 \land |bd| \geq 1 \) \\
    \( \forall_{i \in \natural} \hspace{5pt} ab^{i}cd^{i}e \in L\)
\end{theorem}
\begin{proof}
    Chcemy, żeby w wywodzie słowa \( w \), na pewnej ścieżce powtórzył się nieterminal \( A \), czyli \( A \rightarrow_G^* c\) oraz \( A \rightarrow_G^* bcd \) Możemy wtedy wstawić pomopować \( bcd \) do  \( bbcdd \) (patrz: rysunek poniżej).
    \begin{figure}[H]
    \centering
    \resizebox{0.5\textwidth}{!}{
        \begin{circuitikz}
    \tikzstyle{every node}=[font=\huge]
    \draw [short] (48.75,8.25) -- (41.25,-1.75);
    \draw [short] (48.75,4.5) -- (43.75,-1.75);
    \draw [short] (48.75,8.25) -- (56.25,-1.75);
    \draw [short] (48.75,4.5) -- (53.75,-1.75);
    \draw [short] (48.75,2) -- (43.75,-4.25);
    \draw [short] (48.75,2) -- (53.75,-4.25);
    \draw [short] (53.75,-4.25) -- (43.75,-4.25);
    \draw [short] (45.75,-1.75) -- (41.25,-1.75);
    \draw [short] (51.75,-1.75) -- (56.25,-1.75);
    \draw [short] (48.75,-1.75) -- (46.5,-4.25);
    \draw [short] (48.75,-1.75) -- (51,-4.25);
    \node at (48.75,4.85) {\(A\)};
    \node at (48.75,2.35) {\(A\)};
    \draw [dashed] (48.75,8.25) -- (48.75,5.25);
    \draw [dashed] (48.75,4.5) -- (48.75,2.8);
    \draw [dashed] (48.75,-1.75) -- (48.75,2);
    \node at (42.5,-2.25) {\(a\)};
    \node at (44.75,-2.25) {\(b\)};
    \node at (45,-4.75) {\(b\)};
    \node at (48.75,-4.75) {\(c\)};
    \node at (52.25,-4.75) {\(d\)};
    \node at (52.75,-2.25) {\(d\)};
    \node at (55,-2.25) {\(e\)};
\end{circuitikz}
    }
    \end{figure}
    
    Niech \( n = \card{N} \) będzie liczbą dostępnych nieterminali oraz \( m \) długością najdłuższej produkcji.
    
    Wówczas, jeśli słowo \( w \) jest długości co najmniej \( n_0 = m^{n + 1} + 1 \), to drzewo wywodu musi mieć wysokość co najmniej \( n + 1 \), a zatem jakiś nieterminal powtarza się na jakiejś ścieżce.
    
    Aby zapewnić warunek \( \card{bcd} \leq n_0 \) z tezy, wybieramy ten nieterminal, którego niższe wystąpienie jest najwyżej ze wszystkich -- znajduje się na głębokości co najwyżej \( n + 1 \).
    
    Jeśli mamy pecha i dla takiego wyboru \( \card{bd} = 0 \) to znaczy że przeszliśmy ścieżką, która niczego nie produkuje. 
    Szukamy wtedy innego nieterminala, który spełnia warunek -- taki nieterminal musi istnieć, ponieważ słowo, które produkujemy, jest długie.
\end{proof}

\subsubsection{Zastosowanie lematu o pompowaniu dla języków bezkontekstowych}
\label{context-pumping}
\begin{theorem}
Język \( L = \set{a^nb^nc^n : n \in \natural} \) nie jest bezkontekstowy.
\end{theorem}

\begin{proof}
Pokażemy, że dla \(L\) nie zachodzi lemat o pompowaniu dla języków bezkontekstowych, czyli:

    \( \forall_{n_0 \in \natural} \) \\
    \( \exists_{w \in L}\; \card{w} \geq n_0 \) \\
    \( \forall_{a, b, c, d, e \in \Sigma^*} \hspace{5pt} w = abcde \land |bcd| \leq n_0 \land |bd| \geq 1 \) \\
    \( \exists_{i \in \natural} \hspace{5pt} ab^{i}cd^{i}e \not\in L\)

Dla określonego \(n_0\) wybieramy słowo \(w = a^{n_0}b^{n_0}c^{n_0}\) o długości \(3n_0\). Dla dowolnego podziału słowa \(w\), który spełnia warunki lematu, podsłowo \( |bcd| \) może zawierać dokładnie 1 lub dokładnie 2 litery ze zbioru liter \( \set{a,b,c} \). Ustalając \(i=0\), zmieniejszymy liczbę wystąpień jednej lub dwóch liter występujących w słowie \( w \), ale nie wszystkich trzech.
W ten sposób otrzymamy słowo, w którym jakaś litera występuje \( n_0 \) razy i jakaś występuje mniej niż \( n_0 \) razy. To słowo nie moze więc należeć do \(L\).
\end{proof}

\section{\textcolor{pink}{W jaki sposób języki regularne są charakteryzowane przez automaty skończone?
Nakreśl ideę dowodu}}
Jednym z zadań SZBD jest zapewnianie tego, żeby zmiany dokonane przez widoczną lub zatwierdzoną transakcję były trwałe, a efekty transakcji wycofanej lub przerwanyj przez awarię nie.

Jest parę typów awarii:
\begin{itemize}
    \item awaria systemu: awaria sprzętu, błędy oprogramowania, uszkodzenie dysku
    \item błędy logiczne (np. naruszenie ograniczeń integralnościowych)
    \item błędy systemowe (np. zakleszczenia)
\end{itemize}
Do poradzenia sobie z pierwszym rodzajem wystarczy replikacja bazy i rozproszenie danych. W przypadku pozostałych dwóch trzeba odtworzyć bazę po awarii.
Ważne jest, żeby zminimalizować czas potrzebny do tego, ponieważ w trakcie odtwarzania bazy serwer musi być wyłączony dla użytkowników.
Służące do tego algorytmy składają się z dwóch części:
\begin{enumerate}
    \item zapisywanie informacji potrzebnych do odtworzenia stanu bazy na bieżąco podczas zwykłych operacji
    \item czynności po awarii, prowadzące do przywrócenia spójności, atomowości i trwałości
\end{enumerate}

\subsection*{Strategie SZBD}
Brudne strony, czyli zmieniane po ostatnim zapisie na dysku, mogą pozostawać w buforze długo po zatwierdzeniu transakcji. Jeśli nastąpi awaria, to naniesione zmiany są tracone.
Żeby zadbać o poprawny stan bazy można wykonać operacje:
\begin{itemize}
    \item UNDO, wycofanie zmian - zapewnia atomowość
    \item REDO, powtórzenie transakcji - zapewnia trwałość.
\end{itemize}
Podjęte czynności zależą od strategii SZBD: \\
Czy niezatwierdzona transakcja może nadpisać wartość w pamięci trwałej? \\
\begin{itemize}
    \item strategia steal: TAK
    \item strategia no-steal: NIE
\end{itemize}
Czy wszystkie zmiany muszą być zapisane w pamięci trwałej przed zatwierdzeniem transakcji?
\begin{itemize}
    \item strategia force: TAK
    \item strategia no-force: NIE
\end{itemize}

Strategia \textbf{no-steal + force} pozwala uniknąć wykonywania sekwencji UNDO i REDO. Jednak jest mało efektywna i istnieje ryzyko samozakleszczenia transakcji, jeśli braknie miejsca w buforze.
Częściej wykorzystywana jest strategia \textbf{steal + no-force} opakowana przez \textbf{protokół WAL (write-ahead log)}.
Oprócz plików z danymi przechowywany jest dziennik z logami (dziennik), w którym zapisywane są wszystkie zmiany przed wprowadzeniem ich na dysku.
Zawiera on informacje wystarczające odzyskania zawartości bazy za pomocą operacji UNDO i REDO.
Żeby dzienniki nie rozrastały się w nieskończoność, tworzy się punkty kontrolne. W przypadku awarii powtarzane są tylko transakcja zatwierdzona po ostatnim punkcie kontrolnym, a niezatwierdzone są wycofywane.

\subsection{Algorytm ARIES}
Algorytm ARIES (Algorithm for Recovery and Isolation Exploiting Semantics) służy do odtwarzania transakcji, korzystając z dziennika WAL i blokad.
Składa się z trzech faz:
\begin{enumerate}
    \item Analiza: rekonstrukcja Tabeli Brudnych Stron (DPT) i Tabeli Transakcji (TT), wyznaczenie pierwszego wpisu o zmianach tworzących brudną stronę
    \item Faza REDO: powtórzenie wszystkich operacji od momentu powstania brudnej strony, przywrócenie stanu sprzed awarii
    \item Faza UNDO: wycofanie efektów niezatwierdzonych transakcji, dodanie wpisów kompensacyjnych do dziennika
\end{enumerate}

\section{\textcolor{pink}{Omów zamkniętość klasy języków regularnych na operacje na językach}}
Jednym z zadań SZBD jest zapewnianie tego, żeby zmiany dokonane przez widoczną lub zatwierdzoną transakcję były trwałe, a efekty transakcji wycofanej lub przerwanyj przez awarię nie.

Jest parę typów awarii:
\begin{itemize}
    \item awaria systemu: awaria sprzętu, błędy oprogramowania, uszkodzenie dysku
    \item błędy logiczne (np. naruszenie ograniczeń integralnościowych)
    \item błędy systemowe (np. zakleszczenia)
\end{itemize}
Do poradzenia sobie z pierwszym rodzajem wystarczy replikacja bazy i rozproszenie danych. W przypadku pozostałych dwóch trzeba odtworzyć bazę po awarii.
Ważne jest, żeby zminimalizować czas potrzebny do tego, ponieważ w trakcie odtwarzania bazy serwer musi być wyłączony dla użytkowników.
Służące do tego algorytmy składają się z dwóch części:
\begin{enumerate}
    \item zapisywanie informacji potrzebnych do odtworzenia stanu bazy na bieżąco podczas zwykłych operacji
    \item czynności po awarii, prowadzące do przywrócenia spójności, atomowości i trwałości
\end{enumerate}

\subsection*{Strategie SZBD}
Brudne strony, czyli zmieniane po ostatnim zapisie na dysku, mogą pozostawać w buforze długo po zatwierdzeniu transakcji. Jeśli nastąpi awaria, to naniesione zmiany są tracone.
Żeby zadbać o poprawny stan bazy można wykonać operacje:
\begin{itemize}
    \item UNDO, wycofanie zmian - zapewnia atomowość
    \item REDO, powtórzenie transakcji - zapewnia trwałość.
\end{itemize}
Podjęte czynności zależą od strategii SZBD: \\
Czy niezatwierdzona transakcja może nadpisać wartość w pamięci trwałej? \\
\begin{itemize}
    \item strategia steal: TAK
    \item strategia no-steal: NIE
\end{itemize}
Czy wszystkie zmiany muszą być zapisane w pamięci trwałej przed zatwierdzeniem transakcji?
\begin{itemize}
    \item strategia force: TAK
    \item strategia no-force: NIE
\end{itemize}

Strategia \textbf{no-steal + force} pozwala uniknąć wykonywania sekwencji UNDO i REDO. Jednak jest mało efektywna i istnieje ryzyko samozakleszczenia transakcji, jeśli braknie miejsca w buforze.
Częściej wykorzystywana jest strategia \textbf{steal + no-force} opakowana przez \textbf{protokół WAL (write-ahead log)}.
Oprócz plików z danymi przechowywany jest dziennik z logami (dziennik), w którym zapisywane są wszystkie zmiany przed wprowadzeniem ich na dysku.
Zawiera on informacje wystarczające odzyskania zawartości bazy za pomocą operacji UNDO i REDO.
Żeby dzienniki nie rozrastały się w nieskończoność, tworzy się punkty kontrolne. W przypadku awarii powtarzane są tylko transakcja zatwierdzona po ostatnim punkcie kontrolnym, a niezatwierdzone są wycofywane.

\subsection{Algorytm ARIES}
Algorytm ARIES (Algorithm for Recovery and Isolation Exploiting Semantics) służy do odtwarzania transakcji, korzystając z dziennika WAL i blokad.
Składa się z trzech faz:
\begin{enumerate}
    \item Analiza: rekonstrukcja Tabeli Brudnych Stron (DPT) i Tabeli Transakcji (TT), wyznaczenie pierwszego wpisu o zmianach tworzących brudną stronę
    \item Faza REDO: powtórzenie wszystkich operacji od momentu powstania brudnej strony, przywrócenie stanu sprzed awarii
    \item Faza UNDO: wycofanie efektów niezatwierdzonych transakcji, dodanie wpisów kompensacyjnych do dziennika
\end{enumerate}

\section{\textcolor{pink}{Omów zamkniętość klasy języków bezkontekstowych na operacje na językach}}
Jednym z zadań SZBD jest zapewnianie tego, żeby zmiany dokonane przez widoczną lub zatwierdzoną transakcję były trwałe, a efekty transakcji wycofanej lub przerwanyj przez awarię nie.

Jest parę typów awarii:
\begin{itemize}
    \item awaria systemu: awaria sprzętu, błędy oprogramowania, uszkodzenie dysku
    \item błędy logiczne (np. naruszenie ograniczeń integralnościowych)
    \item błędy systemowe (np. zakleszczenia)
\end{itemize}
Do poradzenia sobie z pierwszym rodzajem wystarczy replikacja bazy i rozproszenie danych. W przypadku pozostałych dwóch trzeba odtworzyć bazę po awarii.
Ważne jest, żeby zminimalizować czas potrzebny do tego, ponieważ w trakcie odtwarzania bazy serwer musi być wyłączony dla użytkowników.
Służące do tego algorytmy składają się z dwóch części:
\begin{enumerate}
    \item zapisywanie informacji potrzebnych do odtworzenia stanu bazy na bieżąco podczas zwykłych operacji
    \item czynności po awarii, prowadzące do przywrócenia spójności, atomowości i trwałości
\end{enumerate}

\subsection*{Strategie SZBD}
Brudne strony, czyli zmieniane po ostatnim zapisie na dysku, mogą pozostawać w buforze długo po zatwierdzeniu transakcji. Jeśli nastąpi awaria, to naniesione zmiany są tracone.
Żeby zadbać o poprawny stan bazy można wykonać operacje:
\begin{itemize}
    \item UNDO, wycofanie zmian - zapewnia atomowość
    \item REDO, powtórzenie transakcji - zapewnia trwałość.
\end{itemize}
Podjęte czynności zależą od strategii SZBD: \\
Czy niezatwierdzona transakcja może nadpisać wartość w pamięci trwałej? \\
\begin{itemize}
    \item strategia steal: TAK
    \item strategia no-steal: NIE
\end{itemize}
Czy wszystkie zmiany muszą być zapisane w pamięci trwałej przed zatwierdzeniem transakcji?
\begin{itemize}
    \item strategia force: TAK
    \item strategia no-force: NIE
\end{itemize}

Strategia \textbf{no-steal + force} pozwala uniknąć wykonywania sekwencji UNDO i REDO. Jednak jest mało efektywna i istnieje ryzyko samozakleszczenia transakcji, jeśli braknie miejsca w buforze.
Częściej wykorzystywana jest strategia \textbf{steal + no-force} opakowana przez \textbf{protokół WAL (write-ahead log)}.
Oprócz plików z danymi przechowywany jest dziennik z logami (dziennik), w którym zapisywane są wszystkie zmiany przed wprowadzeniem ich na dysku.
Zawiera on informacje wystarczające odzyskania zawartości bazy za pomocą operacji UNDO i REDO.
Żeby dzienniki nie rozrastały się w nieskończoność, tworzy się punkty kontrolne. W przypadku awarii powtarzane są tylko transakcja zatwierdzona po ostatnim punkcie kontrolnym, a niezatwierdzone są wycofywane.

\subsection{Algorytm ARIES}
Algorytm ARIES (Algorithm for Recovery and Isolation Exploiting Semantics) służy do odtwarzania transakcji, korzystając z dziennika WAL i blokad.
Składa się z trzech faz:
\begin{enumerate}
    \item Analiza: rekonstrukcja Tabeli Brudnych Stron (DPT) i Tabeli Transakcji (TT), wyznaczenie pierwszego wpisu o zmianach tworzących brudną stronę
    \item Faza REDO: powtórzenie wszystkich operacji od momentu powstania brudnej strony, przywrócenie stanu sprzed awarii
    \item Faza UNDO: wycofanie efektów niezatwierdzonych transakcji, dodanie wpisów kompensacyjnych do dziennika
\end{enumerate}

\section{\textcolor{pink}{Omów twierdzenie Myhilla-Nerode'a, podając ideę dowodu i związek z minimalizacją automatów skończonych}}
Jednym z zadań SZBD jest zapewnianie tego, żeby zmiany dokonane przez widoczną lub zatwierdzoną transakcję były trwałe, a efekty transakcji wycofanej lub przerwanyj przez awarię nie.

Jest parę typów awarii:
\begin{itemize}
    \item awaria systemu: awaria sprzętu, błędy oprogramowania, uszkodzenie dysku
    \item błędy logiczne (np. naruszenie ograniczeń integralnościowych)
    \item błędy systemowe (np. zakleszczenia)
\end{itemize}
Do poradzenia sobie z pierwszym rodzajem wystarczy replikacja bazy i rozproszenie danych. W przypadku pozostałych dwóch trzeba odtworzyć bazę po awarii.
Ważne jest, żeby zminimalizować czas potrzebny do tego, ponieważ w trakcie odtwarzania bazy serwer musi być wyłączony dla użytkowników.
Służące do tego algorytmy składają się z dwóch części:
\begin{enumerate}
    \item zapisywanie informacji potrzebnych do odtworzenia stanu bazy na bieżąco podczas zwykłych operacji
    \item czynności po awarii, prowadzące do przywrócenia spójności, atomowości i trwałości
\end{enumerate}

\subsection*{Strategie SZBD}
Brudne strony, czyli zmieniane po ostatnim zapisie na dysku, mogą pozostawać w buforze długo po zatwierdzeniu transakcji. Jeśli nastąpi awaria, to naniesione zmiany są tracone.
Żeby zadbać o poprawny stan bazy można wykonać operacje:
\begin{itemize}
    \item UNDO, wycofanie zmian - zapewnia atomowość
    \item REDO, powtórzenie transakcji - zapewnia trwałość.
\end{itemize}
Podjęte czynności zależą od strategii SZBD: \\
Czy niezatwierdzona transakcja może nadpisać wartość w pamięci trwałej? \\
\begin{itemize}
    \item strategia steal: TAK
    \item strategia no-steal: NIE
\end{itemize}
Czy wszystkie zmiany muszą być zapisane w pamięci trwałej przed zatwierdzeniem transakcji?
\begin{itemize}
    \item strategia force: TAK
    \item strategia no-force: NIE
\end{itemize}

Strategia \textbf{no-steal + force} pozwala uniknąć wykonywania sekwencji UNDO i REDO. Jednak jest mało efektywna i istnieje ryzyko samozakleszczenia transakcji, jeśli braknie miejsca w buforze.
Częściej wykorzystywana jest strategia \textbf{steal + no-force} opakowana przez \textbf{protokół WAL (write-ahead log)}.
Oprócz plików z danymi przechowywany jest dziennik z logami (dziennik), w którym zapisywane są wszystkie zmiany przed wprowadzeniem ich na dysku.
Zawiera on informacje wystarczające odzyskania zawartości bazy za pomocą operacji UNDO i REDO.
Żeby dzienniki nie rozrastały się w nieskończoność, tworzy się punkty kontrolne. W przypadku awarii powtarzane są tylko transakcja zatwierdzona po ostatnim punkcie kontrolnym, a niezatwierdzone są wycofywane.

\subsection{Algorytm ARIES}
Algorytm ARIES (Algorithm for Recovery and Isolation Exploiting Semantics) służy do odtwarzania transakcji, korzystając z dziennika WAL i blokad.
Składa się z trzech faz:
\begin{enumerate}
    \item Analiza: rekonstrukcja Tabeli Brudnych Stron (DPT) i Tabeli Transakcji (TT), wyznaczenie pierwszego wpisu o zmianach tworzących brudną stronę
    \item Faza REDO: powtórzenie wszystkich operacji od momentu powstania brudnej strony, przywrócenie stanu sprzed awarii
    \item Faza UNDO: wycofanie efektów niezatwierdzonych transakcji, dodanie wpisów kompensacyjnych do dziennika
\end{enumerate}

\section{\textcolor{pink}{Determinizm i niedeterminizm dla maszyn Turinga: omów oba modele i związek
między nimi}}
Jednym z zadań SZBD jest zapewnianie tego, żeby zmiany dokonane przez widoczną lub zatwierdzoną transakcję były trwałe, a efekty transakcji wycofanej lub przerwanyj przez awarię nie.

Jest parę typów awarii:
\begin{itemize}
    \item awaria systemu: awaria sprzętu, błędy oprogramowania, uszkodzenie dysku
    \item błędy logiczne (np. naruszenie ograniczeń integralnościowych)
    \item błędy systemowe (np. zakleszczenia)
\end{itemize}
Do poradzenia sobie z pierwszym rodzajem wystarczy replikacja bazy i rozproszenie danych. W przypadku pozostałych dwóch trzeba odtworzyć bazę po awarii.
Ważne jest, żeby zminimalizować czas potrzebny do tego, ponieważ w trakcie odtwarzania bazy serwer musi być wyłączony dla użytkowników.
Służące do tego algorytmy składają się z dwóch części:
\begin{enumerate}
    \item zapisywanie informacji potrzebnych do odtworzenia stanu bazy na bieżąco podczas zwykłych operacji
    \item czynności po awarii, prowadzące do przywrócenia spójności, atomowości i trwałości
\end{enumerate}

\subsection*{Strategie SZBD}
Brudne strony, czyli zmieniane po ostatnim zapisie na dysku, mogą pozostawać w buforze długo po zatwierdzeniu transakcji. Jeśli nastąpi awaria, to naniesione zmiany są tracone.
Żeby zadbać o poprawny stan bazy można wykonać operacje:
\begin{itemize}
    \item UNDO, wycofanie zmian - zapewnia atomowość
    \item REDO, powtórzenie transakcji - zapewnia trwałość.
\end{itemize}
Podjęte czynności zależą od strategii SZBD: \\
Czy niezatwierdzona transakcja może nadpisać wartość w pamięci trwałej? \\
\begin{itemize}
    \item strategia steal: TAK
    \item strategia no-steal: NIE
\end{itemize}
Czy wszystkie zmiany muszą być zapisane w pamięci trwałej przed zatwierdzeniem transakcji?
\begin{itemize}
    \item strategia force: TAK
    \item strategia no-force: NIE
\end{itemize}

Strategia \textbf{no-steal + force} pozwala uniknąć wykonywania sekwencji UNDO i REDO. Jednak jest mało efektywna i istnieje ryzyko samozakleszczenia transakcji, jeśli braknie miejsca w buforze.
Częściej wykorzystywana jest strategia \textbf{steal + no-force} opakowana przez \textbf{protokół WAL (write-ahead log)}.
Oprócz plików z danymi przechowywany jest dziennik z logami (dziennik), w którym zapisywane są wszystkie zmiany przed wprowadzeniem ich na dysku.
Zawiera on informacje wystarczające odzyskania zawartości bazy za pomocą operacji UNDO i REDO.
Żeby dzienniki nie rozrastały się w nieskończoność, tworzy się punkty kontrolne. W przypadku awarii powtarzane są tylko transakcja zatwierdzona po ostatnim punkcie kontrolnym, a niezatwierdzone są wycofywane.

\subsection{Algorytm ARIES}
Algorytm ARIES (Algorithm for Recovery and Isolation Exploiting Semantics) służy do odtwarzania transakcji, korzystając z dziennika WAL i blokad.
Składa się z trzech faz:
\begin{enumerate}
    \item Analiza: rekonstrukcja Tabeli Brudnych Stron (DPT) i Tabeli Transakcji (TT), wyznaczenie pierwszego wpisu o zmianach tworzących brudną stronę
    \item Faza REDO: powtórzenie wszystkich operacji od momentu powstania brudnej strony, przywrócenie stanu sprzed awarii
    \item Faza UNDO: wycofanie efektów niezatwierdzonych transakcji, dodanie wpisów kompensacyjnych do dziennika
\end{enumerate}


\section{\textcolor{pink}{Omów złożoność obliczeniową problemu stopu oraz jego dopełnienia}}
Jednym z zadań SZBD jest zapewnianie tego, żeby zmiany dokonane przez widoczną lub zatwierdzoną transakcję były trwałe, a efekty transakcji wycofanej lub przerwanyj przez awarię nie.

Jest parę typów awarii:
\begin{itemize}
    \item awaria systemu: awaria sprzętu, błędy oprogramowania, uszkodzenie dysku
    \item błędy logiczne (np. naruszenie ograniczeń integralnościowych)
    \item błędy systemowe (np. zakleszczenia)
\end{itemize}
Do poradzenia sobie z pierwszym rodzajem wystarczy replikacja bazy i rozproszenie danych. W przypadku pozostałych dwóch trzeba odtworzyć bazę po awarii.
Ważne jest, żeby zminimalizować czas potrzebny do tego, ponieważ w trakcie odtwarzania bazy serwer musi być wyłączony dla użytkowników.
Służące do tego algorytmy składają się z dwóch części:
\begin{enumerate}
    \item zapisywanie informacji potrzebnych do odtworzenia stanu bazy na bieżąco podczas zwykłych operacji
    \item czynności po awarii, prowadzące do przywrócenia spójności, atomowości i trwałości
\end{enumerate}

\subsection*{Strategie SZBD}
Brudne strony, czyli zmieniane po ostatnim zapisie na dysku, mogą pozostawać w buforze długo po zatwierdzeniu transakcji. Jeśli nastąpi awaria, to naniesione zmiany są tracone.
Żeby zadbać o poprawny stan bazy można wykonać operacje:
\begin{itemize}
    \item UNDO, wycofanie zmian - zapewnia atomowość
    \item REDO, powtórzenie transakcji - zapewnia trwałość.
\end{itemize}
Podjęte czynności zależą od strategii SZBD: \\
Czy niezatwierdzona transakcja może nadpisać wartość w pamięci trwałej? \\
\begin{itemize}
    \item strategia steal: TAK
    \item strategia no-steal: NIE
\end{itemize}
Czy wszystkie zmiany muszą być zapisane w pamięci trwałej przed zatwierdzeniem transakcji?
\begin{itemize}
    \item strategia force: TAK
    \item strategia no-force: NIE
\end{itemize}

Strategia \textbf{no-steal + force} pozwala uniknąć wykonywania sekwencji UNDO i REDO. Jednak jest mało efektywna i istnieje ryzyko samozakleszczenia transakcji, jeśli braknie miejsca w buforze.
Częściej wykorzystywana jest strategia \textbf{steal + no-force} opakowana przez \textbf{protokół WAL (write-ahead log)}.
Oprócz plików z danymi przechowywany jest dziennik z logami (dziennik), w którym zapisywane są wszystkie zmiany przed wprowadzeniem ich na dysku.
Zawiera on informacje wystarczające odzyskania zawartości bazy za pomocą operacji UNDO i REDO.
Żeby dzienniki nie rozrastały się w nieskończoność, tworzy się punkty kontrolne. W przypadku awarii powtarzane są tylko transakcja zatwierdzona po ostatnim punkcie kontrolnym, a niezatwierdzone są wycofywane.

\subsection{Algorytm ARIES}
Algorytm ARIES (Algorithm for Recovery and Isolation Exploiting Semantics) służy do odtwarzania transakcji, korzystając z dziennika WAL i blokad.
Składa się z trzech faz:
\begin{enumerate}
    \item Analiza: rekonstrukcja Tabeli Brudnych Stron (DPT) i Tabeli Transakcji (TT), wyznaczenie pierwszego wpisu o zmianach tworzących brudną stronę
    \item Faza REDO: powtórzenie wszystkich operacji od momentu powstania brudnej strony, przywrócenie stanu sprzed awarii
    \item Faza UNDO: wycofanie efektów niezatwierdzonych transakcji, dodanie wpisów kompensacyjnych do dziennika
\end{enumerate}

\section{\textcolor{pink}{Omów klasy złożoności: PTIME, NPTIME oraz coNPTIME. Podaj przykład
problemu, który jest w PTIME oraz przykłady jezyków zupełnych dla NPTIME i coNPTIME.
Nakreśl dowód twierdzenia Cooke'a}}
Jednym z zadań SZBD jest zapewnianie tego, żeby zmiany dokonane przez widoczną lub zatwierdzoną transakcję były trwałe, a efekty transakcji wycofanej lub przerwanyj przez awarię nie.

Jest parę typów awarii:
\begin{itemize}
    \item awaria systemu: awaria sprzętu, błędy oprogramowania, uszkodzenie dysku
    \item błędy logiczne (np. naruszenie ograniczeń integralnościowych)
    \item błędy systemowe (np. zakleszczenia)
\end{itemize}
Do poradzenia sobie z pierwszym rodzajem wystarczy replikacja bazy i rozproszenie danych. W przypadku pozostałych dwóch trzeba odtworzyć bazę po awarii.
Ważne jest, żeby zminimalizować czas potrzebny do tego, ponieważ w trakcie odtwarzania bazy serwer musi być wyłączony dla użytkowników.
Służące do tego algorytmy składają się z dwóch części:
\begin{enumerate}
    \item zapisywanie informacji potrzebnych do odtworzenia stanu bazy na bieżąco podczas zwykłych operacji
    \item czynności po awarii, prowadzące do przywrócenia spójności, atomowości i trwałości
\end{enumerate}

\subsection*{Strategie SZBD}
Brudne strony, czyli zmieniane po ostatnim zapisie na dysku, mogą pozostawać w buforze długo po zatwierdzeniu transakcji. Jeśli nastąpi awaria, to naniesione zmiany są tracone.
Żeby zadbać o poprawny stan bazy można wykonać operacje:
\begin{itemize}
    \item UNDO, wycofanie zmian - zapewnia atomowość
    \item REDO, powtórzenie transakcji - zapewnia trwałość.
\end{itemize}
Podjęte czynności zależą od strategii SZBD: \\
Czy niezatwierdzona transakcja może nadpisać wartość w pamięci trwałej? \\
\begin{itemize}
    \item strategia steal: TAK
    \item strategia no-steal: NIE
\end{itemize}
Czy wszystkie zmiany muszą być zapisane w pamięci trwałej przed zatwierdzeniem transakcji?
\begin{itemize}
    \item strategia force: TAK
    \item strategia no-force: NIE
\end{itemize}

Strategia \textbf{no-steal + force} pozwala uniknąć wykonywania sekwencji UNDO i REDO. Jednak jest mało efektywna i istnieje ryzyko samozakleszczenia transakcji, jeśli braknie miejsca w buforze.
Częściej wykorzystywana jest strategia \textbf{steal + no-force} opakowana przez \textbf{protokół WAL (write-ahead log)}.
Oprócz plików z danymi przechowywany jest dziennik z logami (dziennik), w którym zapisywane są wszystkie zmiany przed wprowadzeniem ich na dysku.
Zawiera on informacje wystarczające odzyskania zawartości bazy za pomocą operacji UNDO i REDO.
Żeby dzienniki nie rozrastały się w nieskończoność, tworzy się punkty kontrolne. W przypadku awarii powtarzane są tylko transakcja zatwierdzona po ostatnim punkcie kontrolnym, a niezatwierdzone są wycofywane.

\subsection{Algorytm ARIES}
Algorytm ARIES (Algorithm for Recovery and Isolation Exploiting Semantics) służy do odtwarzania transakcji, korzystając z dziennika WAL i blokad.
Składa się z trzech faz:
\begin{enumerate}
    \item Analiza: rekonstrukcja Tabeli Brudnych Stron (DPT) i Tabeli Transakcji (TT), wyznaczenie pierwszego wpisu o zmianach tworzących brudną stronę
    \item Faza REDO: powtórzenie wszystkich operacji od momentu powstania brudnej strony, przywrócenie stanu sprzed awarii
    \item Faza UNDO: wycofanie efektów niezatwierdzonych transakcji, dodanie wpisów kompensacyjnych do dziennika
\end{enumerate}