\chapter{Modele Obliczeń} 

\section{\textcolor{pink}{Omów hierarchię Chomsky'ego języków, ilustrując klasy i zależności między nimi odpowiednimi przykładami}}
\subsection{Deeply pipelined}
Nowoczesne procesory x86 posiadają głębokie potoki wykonawcze. Cykl przetwarzania instrukcji (pobranie, dekodowanie, wykonanie, zapis wyników) jest rozbity na wiele mniejszych etapów. Dzięki temu możliwe jest przetwarzanie wielu instrukcji równolegle, choć każda z nich znajduje się na innym etapie wykonania. Im głębszy potok, tym większy potencjalny zysk z wysokiego taktowania, ale też większe straty przy błędach przewidywania.
\subsection{Speculative execution}
Jeśli mamy skoki warunkowe to możemy próbować przewidywać czy skok nastąpi czy nie i na tej podstawie wykonywać instrukcje na zapas, czekając jedynie z fazą commit na faktyczne potwierdzenie czy skok następuje czy nie.
Jeśli zgadliśmy poprawnie to super – od razu commitujemy wynik. Natomiast jeśli nie zgadliśmy to musimy teraz wyrzucić cały pipeline na śmietnik i w efekcie dostajemy opóźnienie.
Robimy to zwykle automatem, który kiedy raz skoczyliśmy następnym razem wie, że też pewnie skoczymy.
\subsection{Out-of-order execution}
Procesory nie czekają na wykonanie instrukcji w kolejności ich występowania, analizują zależnośći i przestawiają instrukcje tak aby efektywniej wykorzystać zasoby. 
\subsection{Superscalar}
Procesor równolegle wykonuje wiele instrukcji w jednym cyklu zegara. Pozwala to na większą liczbę operacji niż by to wynikało z taktowania zegara. 
\subsection{Complex instruction set computer}
Procesory x86 mają bardzo wiele instukcji, które są często skomplikowane. Procesory typu ARM mają mało instrukcji i muszą często wywołać ich wiele, aby uzyskać tą samą funkcjonalność co x86. Skomplikowane instrukcje są optymalne, ale trudne w skalowaniu.


\section{\textcolor{pink}{Omów lemat o pompowaniu dla języków bezkontekstowych. Podaj przykład
języka bezkontekstowego. Podaj przykład języka, który nie jest bezkontekstowy}}

Przykład języka bezkontekstowego znajduje się w sekcji \ref{cfl_example}.

\subsection{Lemat o pompowaniu dla języków regularnych}
\label{regular-pumping}
\begin{lemma}[O pompowaniu dla języków regularnych]
    Jeśli \( L \) jest językiem regularnym to:
    \[
        \exists_{n > 0}\; \forall_{w : \abs{w} \geq n} \exists_{xyz = w}\; \left( \abs{xy} \leq n \land \abs{y} \geq 1 \land \left( \forall_{i \in \natural}\; xy^iz \in L \right) \right)
    \]
\end{lemma}
\begin{proof}
    Skoro \( L \) jest językiem regularnym to istnieje DFA A, który rozpoznaje L.
    
    Niech \( n = \abs{Q} \) i weźmy dowolne słowo \( w \) dla którego \( m = \abs{w} \geq n \).
    
    Skoro słowo jest akceptowane to istnieje ścieżka \( s = q_0, \dots q_m \in F \).
    
    Mamy więc \( m + 1 > n \) stanów, czyli jakiś stan musi się powtarzać. 
    Z takich stanów wybieramy stan \( q_i \), którego pierwsze powtórzenie jest najwcześniej. Niech \( q_j \) będzie drugim wystąpieniem stanu \( q_i \).
    Mamy zatem \( x = a_0\dots a_{i-1} \), \( y = a_i\dots a_{j-1} \), \( z = a_j \dots a_{m-1} \).
    
    Oczywiście \( \abs{xy} \leq n \), bo inaczej \( q_i \) nie byłby stanem, który powtarza się najwcześniej.
    
    Skoro  \( q_i = q_j \) to słowo \( y \) może wystąpić dowolną (również zerową) liczbę razy w akceptowanym słowie, czyli dla dowolnego \( i \in \natural \) \( xy^iz \in L \).
\end{proof}
Jest to warunek konieczny, ale \textbf{niewystarczający} by język był regularny. Istnieją języki które nie są regularne, a spełniają warunki lematu o pompowaniu. 

\subsection{Lemat o pompowaniu dla języków bezkontekstowych}
\begin{theorem}[O pompowaniu dla języków bezkontekstowych]
    Jeżeli język \(L\) nad \(\Sigma^*\) jest bezkontekstowy, to: 
    
    \( \exists_{n_0 \in \natural} \) \\
    \( \forall_{w \in L}\; \card{w} \geq n_0 \) \\
    \( \exists_{a, b, c, d, e \in \Sigma^*} \hspace{5pt} w = abcde \land |bcd| \leq n_0 \land |bd| \geq 1 \) \\
    \( \forall_{i \in \natural} \hspace{5pt} ab^{i}cd^{i}e \in L\)
\end{theorem}
\begin{proof}
    Chcemy, żeby w wywodzie słowa \( w \), na pewnej ścieżce powtórzył się nieterminal \( A \), czyli \( A \rightarrow_G^* c\) oraz \( A \rightarrow_G^* bcd \) Możemy wtedy wstawić pomopować \( bcd \) do  \( bbcdd \) (patrz: rysunek poniżej).
    \begin{figure}[H]
    \centering
    \resizebox{0.5\textwidth}{!}{
        \begin{circuitikz}
    \tikzstyle{every node}=[font=\huge]
    \draw [short] (48.75,8.25) -- (41.25,-1.75);
    \draw [short] (48.75,4.5) -- (43.75,-1.75);
    \draw [short] (48.75,8.25) -- (56.25,-1.75);
    \draw [short] (48.75,4.5) -- (53.75,-1.75);
    \draw [short] (48.75,2) -- (43.75,-4.25);
    \draw [short] (48.75,2) -- (53.75,-4.25);
    \draw [short] (53.75,-4.25) -- (43.75,-4.25);
    \draw [short] (45.75,-1.75) -- (41.25,-1.75);
    \draw [short] (51.75,-1.75) -- (56.25,-1.75);
    \draw [short] (48.75,-1.75) -- (46.5,-4.25);
    \draw [short] (48.75,-1.75) -- (51,-4.25);
    \node at (48.75,4.85) {\(A\)};
    \node at (48.75,2.35) {\(A\)};
    \draw [dashed] (48.75,8.25) -- (48.75,5.25);
    \draw [dashed] (48.75,4.5) -- (48.75,2.8);
    \draw [dashed] (48.75,-1.75) -- (48.75,2);
    \node at (42.5,-2.25) {\(a\)};
    \node at (44.75,-2.25) {\(b\)};
    \node at (45,-4.75) {\(b\)};
    \node at (48.75,-4.75) {\(c\)};
    \node at (52.25,-4.75) {\(d\)};
    \node at (52.75,-2.25) {\(d\)};
    \node at (55,-2.25) {\(e\)};
\end{circuitikz}
    }
    \end{figure}
    
    Niech \( n = \card{N} \) będzie liczbą dostępnych nieterminali oraz \( m \) długością najdłuższej produkcji.
    
    Wówczas, jeśli słowo \( w \) jest długości co najmniej \( n_0 = m^{n + 1} + 1 \), to drzewo wywodu musi mieć wysokość co najmniej \( n + 1 \), a zatem jakiś nieterminal powtarza się na jakiejś ścieżce.
    
    Aby zapewnić warunek \( \card{bcd} \leq n_0 \) z tezy, wybieramy ten nieterminal, którego niższe wystąpienie jest najwyżej ze wszystkich -- znajduje się na głębokości co najwyżej \( n + 1 \).
    
    Jeśli mamy pecha i dla takiego wyboru \( \card{bd} = 0 \) to znaczy że przeszliśmy ścieżką, która niczego nie produkuje. 
    Szukamy wtedy innego nieterminala, który spełnia warunek -- taki nieterminal musi istnieć, ponieważ słowo, które produkujemy, jest długie.
\end{proof}

\subsubsection{Zastosowanie lematu o pompowaniu dla języków bezkontekstowych}
\label{context-pumping}
\begin{theorem}
Język \( L = \set{a^nb^nc^n : n \in \natural} \) nie jest bezkontekstowy.
\end{theorem}

\begin{proof}
Pokażemy, że dla \(L\) nie zachodzi lemat o pompowaniu dla języków bezkontekstowych, czyli:

    \( \forall_{n_0 \in \natural} \) \\
    \( \exists_{w \in L}\; \card{w} \geq n_0 \) \\
    \( \forall_{a, b, c, d, e \in \Sigma^*} \hspace{5pt} w = abcde \land |bcd| \leq n_0 \land |bd| \geq 1 \) \\
    \( \exists_{i \in \natural} \hspace{5pt} ab^{i}cd^{i}e \not\in L\)

Dla określonego \(n_0\) wybieramy słowo \(w = a^{n_0}b^{n_0}c^{n_0}\) o długości \(3n_0\). Dla dowolnego podziału słowa \(w\), który spełnia warunki lematu, podsłowo \( |bcd| \) może zawierać dokładnie 1 lub dokładnie 2 litery ze zbioru liter \( \set{a,b,c} \). Ustalając \(i=0\), zmieniejszymy liczbę wystąpień jednej lub dwóch liter występujących w słowie \( w \), ale nie wszystkich trzech.
W ten sposób otrzymamy słowo, w którym jakaś litera występuje \( n_0 \) razy i jakaś występuje mniej niż \( n_0 \) razy. To słowo nie moze więc należeć do \(L\).
\end{proof}

\section{\textcolor{pink}{W jaki sposób języki regularne są charakteryzowane przez automaty skończone?
Nakreśl ideę dowodu}}
\subsection{Deeply pipelined}
Nowoczesne procesory x86 posiadają głębokie potoki wykonawcze. Cykl przetwarzania instrukcji (pobranie, dekodowanie, wykonanie, zapis wyników) jest rozbity na wiele mniejszych etapów. Dzięki temu możliwe jest przetwarzanie wielu instrukcji równolegle, choć każda z nich znajduje się na innym etapie wykonania. Im głębszy potok, tym większy potencjalny zysk z wysokiego taktowania, ale też większe straty przy błędach przewidywania.
\subsection{Speculative execution}
Jeśli mamy skoki warunkowe to możemy próbować przewidywać czy skok nastąpi czy nie i na tej podstawie wykonywać instrukcje na zapas, czekając jedynie z fazą commit na faktyczne potwierdzenie czy skok następuje czy nie.
Jeśli zgadliśmy poprawnie to super – od razu commitujemy wynik. Natomiast jeśli nie zgadliśmy to musimy teraz wyrzucić cały pipeline na śmietnik i w efekcie dostajemy opóźnienie.
Robimy to zwykle automatem, który kiedy raz skoczyliśmy następnym razem wie, że też pewnie skoczymy.
\subsection{Out-of-order execution}
Procesory nie czekają na wykonanie instrukcji w kolejności ich występowania, analizują zależnośći i przestawiają instrukcje tak aby efektywniej wykorzystać zasoby. 
\subsection{Superscalar}
Procesor równolegle wykonuje wiele instrukcji w jednym cyklu zegara. Pozwala to na większą liczbę operacji niż by to wynikało z taktowania zegara. 
\subsection{Complex instruction set computer}
Procesory x86 mają bardzo wiele instukcji, które są często skomplikowane. Procesory typu ARM mają mało instrukcji i muszą często wywołać ich wiele, aby uzyskać tą samą funkcjonalność co x86. Skomplikowane instrukcje są optymalne, ale trudne w skalowaniu.


\section{\textcolor{pink}{Omów zamkniętość klasy języków regularnych na operacje na językach}}
\subsection{Deeply pipelined}
Nowoczesne procesory x86 posiadają głębokie potoki wykonawcze. Cykl przetwarzania instrukcji (pobranie, dekodowanie, wykonanie, zapis wyników) jest rozbity na wiele mniejszych etapów. Dzięki temu możliwe jest przetwarzanie wielu instrukcji równolegle, choć każda z nich znajduje się na innym etapie wykonania. Im głębszy potok, tym większy potencjalny zysk z wysokiego taktowania, ale też większe straty przy błędach przewidywania.
\subsection{Speculative execution}
Jeśli mamy skoki warunkowe to możemy próbować przewidywać czy skok nastąpi czy nie i na tej podstawie wykonywać instrukcje na zapas, czekając jedynie z fazą commit na faktyczne potwierdzenie czy skok następuje czy nie.
Jeśli zgadliśmy poprawnie to super – od razu commitujemy wynik. Natomiast jeśli nie zgadliśmy to musimy teraz wyrzucić cały pipeline na śmietnik i w efekcie dostajemy opóźnienie.
Robimy to zwykle automatem, który kiedy raz skoczyliśmy następnym razem wie, że też pewnie skoczymy.
\subsection{Out-of-order execution}
Procesory nie czekają na wykonanie instrukcji w kolejności ich występowania, analizują zależnośći i przestawiają instrukcje tak aby efektywniej wykorzystać zasoby. 
\subsection{Superscalar}
Procesor równolegle wykonuje wiele instrukcji w jednym cyklu zegara. Pozwala to na większą liczbę operacji niż by to wynikało z taktowania zegara. 
\subsection{Complex instruction set computer}
Procesory x86 mają bardzo wiele instukcji, które są często skomplikowane. Procesory typu ARM mają mało instrukcji i muszą często wywołać ich wiele, aby uzyskać tą samą funkcjonalność co x86. Skomplikowane instrukcje są optymalne, ale trudne w skalowaniu.


\section{\textcolor{pink}{Omów zamkniętość klasy języków bezkontekstowych na operacje na językach}}
\subsection{Deeply pipelined}
Nowoczesne procesory x86 posiadają głębokie potoki wykonawcze. Cykl przetwarzania instrukcji (pobranie, dekodowanie, wykonanie, zapis wyników) jest rozbity na wiele mniejszych etapów. Dzięki temu możliwe jest przetwarzanie wielu instrukcji równolegle, choć każda z nich znajduje się na innym etapie wykonania. Im głębszy potok, tym większy potencjalny zysk z wysokiego taktowania, ale też większe straty przy błędach przewidywania.
\subsection{Speculative execution}
Jeśli mamy skoki warunkowe to możemy próbować przewidywać czy skok nastąpi czy nie i na tej podstawie wykonywać instrukcje na zapas, czekając jedynie z fazą commit na faktyczne potwierdzenie czy skok następuje czy nie.
Jeśli zgadliśmy poprawnie to super – od razu commitujemy wynik. Natomiast jeśli nie zgadliśmy to musimy teraz wyrzucić cały pipeline na śmietnik i w efekcie dostajemy opóźnienie.
Robimy to zwykle automatem, który kiedy raz skoczyliśmy następnym razem wie, że też pewnie skoczymy.
\subsection{Out-of-order execution}
Procesory nie czekają na wykonanie instrukcji w kolejności ich występowania, analizują zależnośći i przestawiają instrukcje tak aby efektywniej wykorzystać zasoby. 
\subsection{Superscalar}
Procesor równolegle wykonuje wiele instrukcji w jednym cyklu zegara. Pozwala to na większą liczbę operacji niż by to wynikało z taktowania zegara. 
\subsection{Complex instruction set computer}
Procesory x86 mają bardzo wiele instukcji, które są często skomplikowane. Procesory typu ARM mają mało instrukcji i muszą często wywołać ich wiele, aby uzyskać tą samą funkcjonalność co x86. Skomplikowane instrukcje są optymalne, ale trudne w skalowaniu.


\section{\textcolor{pink}{Omów twierdzenie Myhilla-Nerode'a, podając ideę dowodu i związek z minimalizacją automatów skończonych}}
\subsection{Deeply pipelined}
Nowoczesne procesory x86 posiadają głębokie potoki wykonawcze. Cykl przetwarzania instrukcji (pobranie, dekodowanie, wykonanie, zapis wyników) jest rozbity na wiele mniejszych etapów. Dzięki temu możliwe jest przetwarzanie wielu instrukcji równolegle, choć każda z nich znajduje się na innym etapie wykonania. Im głębszy potok, tym większy potencjalny zysk z wysokiego taktowania, ale też większe straty przy błędach przewidywania.
\subsection{Speculative execution}
Jeśli mamy skoki warunkowe to możemy próbować przewidywać czy skok nastąpi czy nie i na tej podstawie wykonywać instrukcje na zapas, czekając jedynie z fazą commit na faktyczne potwierdzenie czy skok następuje czy nie.
Jeśli zgadliśmy poprawnie to super – od razu commitujemy wynik. Natomiast jeśli nie zgadliśmy to musimy teraz wyrzucić cały pipeline na śmietnik i w efekcie dostajemy opóźnienie.
Robimy to zwykle automatem, który kiedy raz skoczyliśmy następnym razem wie, że też pewnie skoczymy.
\subsection{Out-of-order execution}
Procesory nie czekają na wykonanie instrukcji w kolejności ich występowania, analizują zależnośći i przestawiają instrukcje tak aby efektywniej wykorzystać zasoby. 
\subsection{Superscalar}
Procesor równolegle wykonuje wiele instrukcji w jednym cyklu zegara. Pozwala to na większą liczbę operacji niż by to wynikało z taktowania zegara. 
\subsection{Complex instruction set computer}
Procesory x86 mają bardzo wiele instukcji, które są często skomplikowane. Procesory typu ARM mają mało instrukcji i muszą często wywołać ich wiele, aby uzyskać tą samą funkcjonalność co x86. Skomplikowane instrukcje są optymalne, ale trudne w skalowaniu.


\section{\textcolor{pink}{Determinizm i niedeterminizm dla maszyn Turinga: omów oba modele i związek
między nimi}}
\subsection{Deeply pipelined}
Nowoczesne procesory x86 posiadają głębokie potoki wykonawcze. Cykl przetwarzania instrukcji (pobranie, dekodowanie, wykonanie, zapis wyników) jest rozbity na wiele mniejszych etapów. Dzięki temu możliwe jest przetwarzanie wielu instrukcji równolegle, choć każda z nich znajduje się na innym etapie wykonania. Im głębszy potok, tym większy potencjalny zysk z wysokiego taktowania, ale też większe straty przy błędach przewidywania.
\subsection{Speculative execution}
Jeśli mamy skoki warunkowe to możemy próbować przewidywać czy skok nastąpi czy nie i na tej podstawie wykonywać instrukcje na zapas, czekając jedynie z fazą commit na faktyczne potwierdzenie czy skok następuje czy nie.
Jeśli zgadliśmy poprawnie to super – od razu commitujemy wynik. Natomiast jeśli nie zgadliśmy to musimy teraz wyrzucić cały pipeline na śmietnik i w efekcie dostajemy opóźnienie.
Robimy to zwykle automatem, który kiedy raz skoczyliśmy następnym razem wie, że też pewnie skoczymy.
\subsection{Out-of-order execution}
Procesory nie czekają na wykonanie instrukcji w kolejności ich występowania, analizują zależnośći i przestawiają instrukcje tak aby efektywniej wykorzystać zasoby. 
\subsection{Superscalar}
Procesor równolegle wykonuje wiele instrukcji w jednym cyklu zegara. Pozwala to na większą liczbę operacji niż by to wynikało z taktowania zegara. 
\subsection{Complex instruction set computer}
Procesory x86 mają bardzo wiele instukcji, które są często skomplikowane. Procesory typu ARM mają mało instrukcji i muszą często wywołać ich wiele, aby uzyskać tą samą funkcjonalność co x86. Skomplikowane instrukcje są optymalne, ale trudne w skalowaniu.



\section{\textcolor{pink}{Omów złożoność obliczeniową problemu stopu oraz jego dopełnienia}}
\subsection{Deeply pipelined}
Nowoczesne procesory x86 posiadają głębokie potoki wykonawcze. Cykl przetwarzania instrukcji (pobranie, dekodowanie, wykonanie, zapis wyników) jest rozbity na wiele mniejszych etapów. Dzięki temu możliwe jest przetwarzanie wielu instrukcji równolegle, choć każda z nich znajduje się na innym etapie wykonania. Im głębszy potok, tym większy potencjalny zysk z wysokiego taktowania, ale też większe straty przy błędach przewidywania.
\subsection{Speculative execution}
Jeśli mamy skoki warunkowe to możemy próbować przewidywać czy skok nastąpi czy nie i na tej podstawie wykonywać instrukcje na zapas, czekając jedynie z fazą commit na faktyczne potwierdzenie czy skok następuje czy nie.
Jeśli zgadliśmy poprawnie to super – od razu commitujemy wynik. Natomiast jeśli nie zgadliśmy to musimy teraz wyrzucić cały pipeline na śmietnik i w efekcie dostajemy opóźnienie.
Robimy to zwykle automatem, który kiedy raz skoczyliśmy następnym razem wie, że też pewnie skoczymy.
\subsection{Out-of-order execution}
Procesory nie czekają na wykonanie instrukcji w kolejności ich występowania, analizują zależnośći i przestawiają instrukcje tak aby efektywniej wykorzystać zasoby. 
\subsection{Superscalar}
Procesor równolegle wykonuje wiele instrukcji w jednym cyklu zegara. Pozwala to na większą liczbę operacji niż by to wynikało z taktowania zegara. 
\subsection{Complex instruction set computer}
Procesory x86 mają bardzo wiele instukcji, które są często skomplikowane. Procesory typu ARM mają mało instrukcji i muszą często wywołać ich wiele, aby uzyskać tą samą funkcjonalność co x86. Skomplikowane instrukcje są optymalne, ale trudne w skalowaniu.


\section{\textcolor{pink}{Omów klasy złożoności: PTIME, NPTIME oraz coNPTIME. Podaj przykład
problemu, który jest w PTIME oraz przykłady jezyków zupełnych dla NPTIME i coNPTIME.
Nakreśl dowód twierdzenia Cooke'a}}
\subsection{Deeply pipelined}
Nowoczesne procesory x86 posiadają głębokie potoki wykonawcze. Cykl przetwarzania instrukcji (pobranie, dekodowanie, wykonanie, zapis wyników) jest rozbity na wiele mniejszych etapów. Dzięki temu możliwe jest przetwarzanie wielu instrukcji równolegle, choć każda z nich znajduje się na innym etapie wykonania. Im głębszy potok, tym większy potencjalny zysk z wysokiego taktowania, ale też większe straty przy błędach przewidywania.
\subsection{Speculative execution}
Jeśli mamy skoki warunkowe to możemy próbować przewidywać czy skok nastąpi czy nie i na tej podstawie wykonywać instrukcje na zapas, czekając jedynie z fazą commit na faktyczne potwierdzenie czy skok następuje czy nie.
Jeśli zgadliśmy poprawnie to super – od razu commitujemy wynik. Natomiast jeśli nie zgadliśmy to musimy teraz wyrzucić cały pipeline na śmietnik i w efekcie dostajemy opóźnienie.
Robimy to zwykle automatem, który kiedy raz skoczyliśmy następnym razem wie, że też pewnie skoczymy.
\subsection{Out-of-order execution}
Procesory nie czekają na wykonanie instrukcji w kolejności ich występowania, analizują zależnośći i przestawiają instrukcje tak aby efektywniej wykorzystać zasoby. 
\subsection{Superscalar}
Procesor równolegle wykonuje wiele instrukcji w jednym cyklu zegara. Pozwala to na większą liczbę operacji niż by to wynikało z taktowania zegara. 
\subsection{Complex instruction set computer}
Procesory x86 mają bardzo wiele instukcji, które są często skomplikowane. Procesory typu ARM mają mało instrukcji i muszą często wywołać ich wiele, aby uzyskać tą samą funkcjonalność co x86. Skomplikowane instrukcje są optymalne, ale trudne w skalowaniu.
