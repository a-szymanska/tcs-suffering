    Pokażemy, że dla dowolnego DFA \(A\) można skonstruować takie wyrażenie regularne \(\alpha\), że \( L(A) = L(\alpha)\). 
    
    Jeśli \( w \in L(A) \), to istnieje ścieżka \( s \rightarrow q_f \in F \) po stanach
    \[
        q_1, q_2, \dots, q_{k+1},
    \]
    qdzie \( s = q_1 \) oraz \( q_{k+1} \in F\).
    Definiujemy \( \alpha_{i, j}^k \) jako wyrażenie regularne, które reprezentuje wszystkie ścieżki prowadzące od \( q_i \) do \( q_j \), w której stany pośrednie należą do \( \set{q_1, \dots, q_k} \). Dla każdych \(i, j\):
    \[
        \alpha_{i, j}^0 = \set{a \in \Sigma : \delta(q_i, a) = q_j } \cup \beta_{i, j},
    \]
    gdzie \( \beta_{i, j} = \set{\eps} \), gdy \( i = j \) oraz \( \beta_{i, j} = \varnothing \) w przeciwnym przypadku.
    
    Dla dowolnych \(k, i, j \leq |Q|, k \geq 1 \) zachodzi:
    \[ 
        \alpha^{k}_{ij} = \alpha^{k-1}_{ij} + \alpha^{k-1}_{ik} (\alpha^{k-1}_{kk})^* \alpha^{k-1}_{kj} 
    \]
    
    Pierwszy składnik sumy \( \alpha^{k-1}_{ij} \) odpowiada ścieżkom ze stanu \(q_i\) do stanu \(q_j\), które nie przechodzą przez \(q_k\) ani stan o wyższym numerze.

    Drugi składnik sumy odpowiada ścieżkom, które przechodzą przez \(q_k\). W takiej sytuacji Ścieżka dzieli się na 3 fragmenty:
    \begin{enumerate}
        \item Przejście z \(q_i\) do \(q_k\) (pierwsza wizyta w  \(q_k\)) -- opisywane wyrażeniem \(\alpha^{k-1}_{ik}\)
        \item Dowolna liczba przejść, zaczynających i kończących się w \(q_k\), po stanach z \( \set{q_1, \dots, q_{k-1}} \) -- opisywane wyrażeniem \( (\alpha^{k-1}_{kk})^* \).
        \item Przejście z \(q_k\) do \(q_j\) (bez powrotów do \(q_k\)) -- opisywane wyrażeniem \( \alpha^{k-1}_{kj} \). 
    \end{enumerate}
    
    Na podstawie DFA można konstruować wyrażenia \( \alpha^{k}_{ij} \) dla kolejnych \(k\).
    Wyrażenie regularne, które opisuje ścieżkę akceptującą ma postać:
    \[
        \sum_{q_j \in F} \alpha^{k}_{ij},
    \]
    gdzie \(k\) to liczba stanów, a \(q_i\) to stan początkowy. Wyrażenie jest poprawnie zdefiniowane, co można udowodnić indukcyjnie.
