\begin{theorem}
    Niech \( L \subseteq \Sigma^* \).
    Następujące warunki są równoważne:
    \begin{enumerate}
        \item Istnieje DFA \( A_D \), dla którego \( L(A_D) = L \)
        \item Istnieje NFA \( A_N \), dla którego \( L(A_N) = L \) 
    \end{enumerate}
\end{theorem}
\begin{proof} 
    \begin{description}
        \item \( \implies \)
        
        Każdy DFA jest NFA, ponieważ każda funkcja jest relacją.
            
        \item \( \impliedby \)
        
        Niech \( A_N = (Q, \Sigma, \delta, S, F) \).
        Korzystamy z faktu, że \( \tilde \delta \) jest funkcją i konstruujemy DFA \(A_D\):
        \[ A_D = (\powerset(Q), \Sigma, \tilde \delta, S, \mathcal{F}), \]
        gdzie \( \mathcal{F} = \set{\beta \in \powerset(Q) \mid \beta \cap F \neq \varnothing} \)
        
        Tworzymy DFA, którego stanami są zbiory wszystkich stanów, w jakich NFA mogłoby się znaleźć po przejściu danego samego prefiksu słowa.
        Definicja \( \mathcal{F} \) jest intuicyjna, bo przejścia prowadzą do stanu z \( F \) wtedy i tylko wtedy, gdy jakies obliczenie NFA kończy się w stanie akceptującym.  
        
        Pozostaje pokazać, że \( \hat \delta_{A_N}(S, u) \cap F \neq \varnothing \iff \hat \delta_{A_D}(S, u) \in \mathcal{F} \),
        ale to wynika wprost definicji, ponieważ \( \hat \delta_{A_D} = \hat \delta_{A_N} \)
    \end{description}
\end{proof}

\begin{theorem}
    Niech \( L \subseteq \Sigma^* \).
    Następujące warunki są równoważne:
    \begin{enumerate}
            \item Istnieje DFA \( A_D \), dla którego \( L(A_D) = L \)
        \item Istnieje \(\eps\)-NFA \( A_N \), którego \( L(A_N) = L \) 
    \end{enumerate}
\end{theorem}
\begin{proof}
     \begin{description}
        \item \( \implies \)
        
        Każdy DFA jest \(\eps\)-NFA, ponieważ każda funkcja jest relacją.
            
        \item \( \impliedby \)
        
        Niech \(A_N\) będzie \(\eps\)-NFA takim, że: 
        \[ A_N = (Q, \Sigma, \delta, S, F) \]
        
        Tak jak w poprzednim przypadku, stany DFA przechowują informację o tym, w jakich stanach może znaleźć się \(\eps-NFA\).
        Żeby obsłużyć epsilonowe przejścia, trzeba chodzić po \(\eps\)-domknięciach z użyciem funkcji \( \Delta \). 
        
        Konstruujemy więc DFA:
        \[ A_D = (\powerset^{A_N, \eps}(Q), \Sigma, \Delta, S, \mathcal{F}), \]
        gdzie \( \mathcal{F} = \set{\beta \in \powerset^{A_N, \eps}(Q) \mid \beta \cap F \neq \varnothing} \)
    \end{description}
\end{proof}