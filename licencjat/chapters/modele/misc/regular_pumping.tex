\begin{lemma}[O pompowaniu dla języków regularnych]
    Jeśli \( L \) jest językiem regularnym to:
    \[
        \exists_{n > 0}\; \forall_{w : \abs{w} \geq n} \exists_{xyz = w}\; \left( \abs{xy} \leq n \land \abs{y} \geq 1 \land \left( \forall_{i \in \natural}\; xy^iz \in L \right) \right)
    \]
\end{lemma}
\begin{proof}
    Skoro \( L \) jest językiem regularnym to istnieje DFA A, który rozpoznaje L.
    
    Niech \( n = \abs{Q} \) i weźmy dowolne słowo \( w \) dla którego \( m = \abs{w} \geq n \).
    
    Skoro słowo jest akceptowane to istnieje ścieżka \( s = q_0, \dots q_m \in F \).
    
    Mamy więc \( m + 1 > n \) stanów, czyli jakiś stan musi się powtarzać. 
    Z takich stanów wybieramy stan \( q_i \), którego pierwsze powtórzenie jest najwcześniej. Niech \( q_j \) będzie drugim wystąpieniem stanu \( q_i \).
    Mamy zatem \( x = a_0\dots a_{i-1} \), \( y = a_i\dots a_{j-1} \), \( z = a_j \dots a_{m-1} \).
    
    Oczywiście \( \abs{xy} \leq n \), bo inaczej \( q_i \) nie byłby stanem, który powtarza się najwcześniej.
    
    Skoro  \( q_i = q_j \) to słowo \( y \) może wystąpić dowolną (również zerową) liczbę razy w akceptowanym słowie, czyli dla dowolnego \( i \in \natural \) \( xy^iz \in L \).
\end{proof}
Jest to warunek konieczny, ale \textbf{niewystarczający} by język był regularny. Istnieją języki które nie są regularne, a spełniają warunki lematu o pompowaniu. 