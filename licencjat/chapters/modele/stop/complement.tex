\begin{theorem}
\(\complement L_{HALT} \not \in \re\)
\end{theorem}

Powyższe twierdzenie wynika wprost z lematu:

\begin{lemma}
    Jeśli \( L \in RE \setminus R \), to \( \complement{L} \not \in RE\).
\end{lemma}
\begin{proof}
    Załóżmy, że \( L \in RE \setminus R \) oraz \( \complement{L} \in RE\). 
    Mamy więc maszynę \( M \) rozpoznającą \( L \) oraz \( N \) rozpoznającą \( \complement{L} \)
    
    Konstruujemy DMT \( M' \) która:
    \begin{enumerate}
        \item wczytuje wejście \( w \)
        \item powtarza, aż do akceptacji:
        \begin{enumerate}
            \item wykonuje jeden krok symulacji \( M \) na \( w \)
            \item wykonuje jeden krok symulacji \( N \) na \( w \)
        \end{enumerate}
        \item jeśli \( M \) zaakceptowała, wypisuje ,,TAK''
        \item jeśli \( N \) zaakceptowała, wypisuje ,,NIE''
    \end{enumerate}
    
    Oczywiście \( w \in L \lor w \in \complement{L} \), więc albo \( M \), albo \(N\) akceptuje \( w \).
    Stanie się to po skończonej liczbie kroków niezależnie od \( w \), zatem \( L(M') = M \) oraz \( M' \) ma własność stopu.
    
    W takim razie \( L \in (RE \setminus R) \cap R = \empty \).
\end{proof}