Niech \( M = (Q, \Sigma, \Gamma, \delta, q_0, \blank, F) \) będzie (deterministyczną) maszyną, którą chcemy symulować. Pokażemy, jak zakodować \( M \) nad alfabetem \( \Sigma_U = \set{0, 1} \).
\begin{itemize}
    \item \( Q = \set{q_1, \dots, q_n} \) -- kodujemy \( q_i \) jako ciąg \( i \) jedynek
    \item \( \blank \) -- kodujemy jako pojedynczą jedynkę
    \item \( \Sigma = \set{a_1, \dots, a_n} \) -- kodujemy \( a_i \) jako ciąg \( i + 1 \) jedynek
    \item \( \Gamma = \set{z_1, \dots z_n} \) -- kodujemy \( z_i \) jako ciąg \( i + \card{\Sigma + 1} \) jedynek, dla odróżnienia od \( \Sigma \)
    \item \( \delta = \set{((q_1, z_1), (q_2, z_2, \pm 1))} \)
    
    \( ((q_i, z_j), (q_k, z_l), 1) \) -- kodujemy jako
    \[
        \underbrace{1 \dots 1}_i
        0
        \underbrace{1 \dots 1}_j
        0
        \underbrace{1 \dots 1}_k
        0
        \underbrace{1 \dots 1}_l
        0
        1
    \]
    \( ((q_i, z_j), (q_k, z_l), -1) \) -- kodujemy jako
    \[
        \underbrace{1 \dots 1}_i
        0
        \underbrace{1 \dots 1}_j
        0
        \underbrace{1 \dots 1}_k
        0
        \underbrace{1 \dots 1}_l
        0
        11
    \]
    Dla funkcji \(\delta\) kodujemy wszystkie jej pary w dowolnej kolejności, oddzielając je dwoma zerami.
    \item \( F = \set{q_{i_1}, \dots q_{i_n}} \) -- kodujemy jako
    \[
        \underbrace{1 \dots 1}_{i_1} 0 \dots 0 \underbrace{1 \dots 1}_{i_n}
    \]
\end{itemize}

Maszynę \( M \) kodujemy, wpisując kodowania \( \delta, q_0, F \) oddzielone ciągiem \( 000 \).

Słowo \( w = a_{i_1} \dots a_{i_n} \) kodujemy jako
\[
    \underbrace{1 \dots 1}_{i_1} 0 \dots 0 \underbrace{1 \dots 1}_{i_n}
\]

Konfigurację \( X_1 \dots X_k q_i X_{k+1} \dots X_n \) kodujemy jako
\[
    0\underbrace{1 \dots 1}_{\text{kod } X_1} 0 \underbrace{1 \dots 1}_{\text{kod } X_k} 00 \underbrace{1 \dots 1}_{\text{kod } q_i} 00 \underbrace{1 \dots 1}_{\text{kod } X_{k + 1}} 0 \underbrace{1 \dots 1}_{\text{kod } X_n} 0
\]

Możemy założyć, że kodowanie \( M \) jest zapisane na taśmie wejściowej, po której nigdy nie piszemy, a konfiguracje \( M \) są zapisywane na osobnej taśmie.

Krok symulacji przebiega następująco:
\begin{enumerate}
    \item Zlokalizuj głowicę \( M \) (szukamy \( 00 \)).
    \item Znajdź w kodowaniu \( \delta \) wpis odpowiadający przejściu na \( q_i X_k \).
    \item Wykonaj zadane przejście -- zapisz nową konfigurację na roboczej taśmie i przepisz ją z powrotem na taśmę, na której operujemy.
\end{enumerate}