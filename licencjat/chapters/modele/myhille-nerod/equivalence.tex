\begin{definition}
    Stany \( q_1, q_2 \) automatu DFA A są \textbf{A-równoważne}, jeśli
    \[
        \forall_{w \in \Sigma^*}\; \hat \delta(q_1, w) \in F \iff \hat \delta(q_2, w) \in F
    \]
    i \textbf{A-rozróżnialne}, jeśli
    \[
        \exists_{w \in \Sigma^*}\; \hat \delta(q_1, w) \in F \iff \hat \delta(q_2, w) \notin F
    \]
\end{definition}

\begin{lemma}
    A-równoważnosć jest relacją równoważności.
\end{lemma}
\begin{proof}
    \begin{enumerate}
        \item Zwrotność: 
            \[
                \forall_{w \in \Sigma^*}\; \hat \delta(q_1, w) \in F \iff \hat \delta(q_1, w) \in F
            \]
        \item Symetryczność:
            \[ 
                \forall_{w \in \Sigma^*}\; \hat \delta(q_1, w) \in F \iff \hat \delta(q_2, w) \in F
            \]
            \[
                \equiv 
            \]
            \[
                \forall_{w \in \Sigma^*}\; \hat \delta(q_2, w) \in F \iff \hat \delta(q_1, w) \in F
            \]
            co wynika z przemienności równoważności.
        \item Przechodniość: \\
        Zakładając, że dla \(q_1\), \(q_2\) oraz \(q_3\) zachodzi:
        \[ 
             \forall_{w \in \Sigma^*}\; \hat \delta(q_1, w) \in F \iff \hat \delta(q_2, w) \in F
        \]
        oraz
        \[ 
             \forall_{w \in \Sigma^*} \hat \delta(q_2, w) \in F \iff \hat \delta(q_3, w) \in F
        \]
        otrzymujemy:
        \[ 
            \forall_{w \in \Sigma^*} \hat \delta(q_1, w) \in F \iff \hat \delta(q_2, w) \in F \iff \hat \delta(q_3, w) \in F,
        \]
        czyli \(q_1\) i \(q_3\) również są A-równoważne. 
    \end{enumerate}
\end{proof}
