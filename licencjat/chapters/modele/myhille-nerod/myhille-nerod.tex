\begin{definition}
    Relacja jest \(L\)-kongruentna dla pewnego języka \(L \in \powerset(\Sigma^*)\) wtedy i tylko wtedy, gdy:
    \begin{enumerate}
        \item \(\forall_{v \in \Sigma^*}\; x \equiv y \implies xv \equiv yv \)
        \item \( x \equiv y \implies (x \in L \iff y \in L)\)
    \end{enumerate}
\end{definition}


\begin{theorem}[Myhill--Nerode]
    Język \( L \) jest regularny wtedy i tylko wtedy, gdy istnieje relacja równoważności \( \equiv \subseteq \Sigma^* \times \Sigma^*\) o skończenie wielu klasach równoważności, która jest \(L\)-kongruentna.
\end{theorem}
\begin{proof} \( \)
    \begin{description}
        \item \( \implies \)
        
        Skoro \( L \) jest regularny, to istnieje DFA \( A \), na podstawie którego definiujemy
        \[
            x \equiv y \iff \hat \delta(s, x) = \hat \delta(s, y),
        \]
        gdzie \(s\) jest stanem startowym w \(A\).
        
        Należy dowieść, że:
        \begin{enumerate}
            \item Relacja ta jest \(L\)-kongruentna:
            \begin{enumerate}
                \item \(x \equiv y \iff \hat \delta(s, x) = \hat \delta(s, y) = q \text{\linebreak jednocześnie } \hat \delta(q, v) = \hat \delta(q, v) \implies xv \equiv yv\)
                \item \(x \equiv y \iff \hat \delta(s, x) = \hat \delta(s, y) = q \), więc jeśli \(q \in F\), to warunek jest spełniony, a jeśli \(q \not \in F\), to również. 
            \end{enumerate}
            \item Relacja ma skończenie wiele klas równoważności, bo funkcja \( \hat \delta \) może mapować dane słowo na maksymalnie \( |Q| \) stanów, a wszystkie słowa, które przechodzą na ten sam stan, są ze sobą równoważne. 
        \end{enumerate}
        
        \item \( \impliedby \)
        
        Konstruujemy automat, którego stanami są klasy równoważności słów wyznaczone przez relację \( \equiv \). Możemy tak zrobić, ponieważ klas równoważności jest skończenie wiele.
        Jako stany końcowe ustalamy te klasy równoważności, w których wszystkie elementy należą do języka (z racji że jest to L-kongruencja, wszystkie elementy klasy mogą albo należeć, albo nie należeć do języka).
        Stanem startowym jest klasa, do której należy \( \eps \). 
        
        Funkcję przejścia definiujemy w ten sposób, że dla każdej klasy dodajemy przejście do klasy, w której wylądują powstałe słowa po dodaniu jednej litery z alfabetu.
        Warto tutaj zauważyć, że po dodaniu litery do dowolnych słów z klasy, wszystkie powstałe w ten sposób słowa muszą znaleźć się w jednej klasie (z warunku pierwszego L-kongruencji). 
        
        Opisana procedura generowania przejść jest wykonywana dla każdej klasy i każdej litery.
        
        Pozostaje udowodnić, że powstały DFA akceptuje język, na którym zdefiniowana jest L-kongruencja. 
        
        Zauważamy, że \(\hat\delta(s, w) \) jest klasą równoważności, do której należy słowo \(w\). Dowód przebiega indukcyjnie po długości \(w\): gdy \(|w| = 0\), to \( w = \eps \) i z definicji \(q_s\) zawiera \(\eps\).
        
        W kroku indukcyjnym, jeśli mamy słowo \(w' = wa\), to z założenia indukcyjnego \(\hat\delta(w)\) jest klasą równoważności \(w\). Wtedy przejście po literze \(a\) z tej klasy prowadzi do klasy zawierającej \(wa\) z definicji funkcji \(\delta\). 
        
        Wynika z tego, że \(L(A) = L\), ponieważ jeśli \(w \in L\), to klasa równoważności \(w\) jest stanem akceptującym. Jeśli \(w \notin L\), to klasa równoważności \(w\) nie jest stanem akceptującym. To dowodzi poprawności konstrukcji.
    \end{description}
\end{proof}