Algorytm D służy do minimalizacji deterministycznych automatów skończonych.

\begin{itemize}
    \item Wejście: DFA \( A \)
    \item Wyjście: (trójkątna) macierz Boolowska \( M \), w której \( M_{i, j} = 1 \iff a_i, a_j \) są rozróżnialne
\end{itemize}

Początkowo \( M \) jest wypełniona zerami.
Jeśli \( a_i \in F \) oraz \( a_j \notin F \), to \( M_{i, j} = 1 \).
Dopóki macierz \( M \) się zmienia, to dla każdej pary stanów \( q_1, q_2 \) oraz litery \( a \), jeśli \( \hat \delta(q_1, a) = q_k, \hat \delta(q_2, a) = q_l \) oraz \( M_{k, l} = 1 \), ustawiamy \( M_{i, j} = 1 \).

Jeśli po zakończeniu algorytmu dla stanów \(q_i,\; q_j\) jest \(M_{i,j} = 0\), to \(q_i,\; q_j\) nie są rozróżnialne, a więc są A-równoważne. W ten sposób znajdujemy wszystkie stany, które są sobie równoważne (działa to trochę jak Floyd-Warshall).
Wszystkie A-równoważne stany można skompresować w jeden. Tak powstałe DFA jest minimalne pod względem liczby stanów, spośród wszystkich DFA rozpoznających ten sam język. 

\begin{theorem}
    Automat otrzymany w wyniku pracy algorytmu \(D\) jest najmniejszy spośród DFA rozpoznających dany język.
\end{theorem}
\begin{proof}
    Dla dowodu nie wprost załóżmy, że w wyniku minimalizacji automatu otrzymaliśmy automat \(A\), a istnieje DFA \(B\) taki, że \( L(A) = L(B) \) i \(B\) ma mniej stanów niż \(A\).
    Definiujemy DFA \(C\) takie, że jego stanem startowym jest \(s_B\), a \(Q_C = Q_A \cup Q_B\), funkcje przejść również sumujemy. Zauważmy, że dla automatu C:
    \[
        \forall_{q_A \in Q_A} \exists_{q_B \in Q_B}  \hspace{5pt} \textit{\(q_a\) i \(q_b\) są \(C\)-równoważne}
    \]
    Jako, że \(A\) i \(B\) akceptują te same języki, to \(s_A\) i \(s_B\) są \(C\)-równoważne. W takim razie dla każdego \(a \in \Sigma\), stany \(\delta(s_A, a)\) i \(\delta(s_B, a)\) również są \(C\)-równoważne. Stosując tę konstrukcję dalej, pokażemy, że dla każego stanu z \(A\) istnieje równoważny stan w \(B\). 
    
    Skoro z założenia nie wprost \(A\) ma więcej stanów niż \(B\), to istnieją takie stany \(q_1, q_2 \in A\), które są równoważne z jakimś stanem \(q_3 \in B\). Ponieważ jednak równoważnosć jest przechodnia, mamy że \(q_1\) i \(q_2\) są wzajemnie równoważne.
    To prowadzi do sprzeczności, bo wszystkie stany wzajemnie równoważne zostały złączone w jeden w Algorytmie D.
\end{proof}