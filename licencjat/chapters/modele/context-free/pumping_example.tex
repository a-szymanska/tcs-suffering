\begin{theorem}
Język \( L = \set{a^nb^nc^n : n \in \natural} \) nie jest bezkontekstowy.
\end{theorem}

\begin{proof}
Pokażemy, że dla \(L\) nie zachodzi lemat o pompowaniu dla języków bezkontekstowych, czyli:

    \( \forall_{n_0 \in \natural} \) \\
    \( \exists_{w \in L}\; \card{w} \geq n_0 \) \\
    \( \forall_{a, b, c, d, e \in \Sigma^*} \hspace{5pt} w = abcde \land |bcd| \leq n_0 \land |bd| \geq 1 \) \\
    \( \exists_{i \in \natural} \hspace{5pt} ab^{i}cd^{i}e \not\in L\)

Dla określonego \(n_0\) wybieramy słowo \(w = a^{n_0}b^{n_0}c^{n_0}\) o długości \(3n_0\). Dla dowolnego podziału słowa \(w\), który spełnia warunki lematu, podsłowo \( |bcd| \) może zawierać dokładnie 1 lub dokładnie 2 litery ze zbioru liter \( \set{a,b,c} \). Ustalając \(i=0\), zmieniejszymy liczbę wystąpień jednej lub dwóch liter występujących w słowie \( w \), ale nie wszystkich trzech.
W ten sposób otrzymamy słowo, w którym jakaś litera występuje \( n_0 \) razy i jakaś występuje mniej niż \( n_0 \) razy. To słowo nie moze więc należeć do \(L\).
\end{proof}