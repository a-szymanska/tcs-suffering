\begin{theorem}[O pompowaniu dla języków bezkontekstowych]
    Jeżeli język \(L\) nad \(\Sigma^*\) jest bezkontekstowy, to: 
    
    \( \exists_{n_0 \in \natural} \) \\
    \( \forall_{w \in L}\; \card{w} \geq n_0 \) \\
    \( \exists_{a, b, c, d, e \in \Sigma^*} \hspace{5pt} w = abcde \land |bcd| \leq n_0 \land |bd| \geq 1 \) \\
    \( \forall_{i \in \natural} \hspace{5pt} ab^{i}cd^{i}e \in L\)
\end{theorem}
\begin{proof}
    Chcemy, żeby w wywodzie słowa \( w \), na pewnej ścieżce powtórzył się nieterminal \( A \), czyli \( A \rightarrow_G^* c\) oraz \( A \rightarrow_G^* bcd \) Możemy wtedy wstawić pomopować \( bcd \) do  \( bbcdd \) (patrz: rysunek poniżej).
    \begin{figure}[H]
    \centering
    \resizebox{0.5\textwidth}{!}{
        \begin{circuitikz}
    \tikzstyle{every node}=[font=\huge]
    \draw [short] (48.75,8.25) -- (41.25,-1.75);
    \draw [short] (48.75,4.5) -- (43.75,-1.75);
    \draw [short] (48.75,8.25) -- (56.25,-1.75);
    \draw [short] (48.75,4.5) -- (53.75,-1.75);
    \draw [short] (48.75,2) -- (43.75,-4.25);
    \draw [short] (48.75,2) -- (53.75,-4.25);
    \draw [short] (53.75,-4.25) -- (43.75,-4.25);
    \draw [short] (45.75,-1.75) -- (41.25,-1.75);
    \draw [short] (51.75,-1.75) -- (56.25,-1.75);
    \draw [short] (48.75,-1.75) -- (46.5,-4.25);
    \draw [short] (48.75,-1.75) -- (51,-4.25);
    \node at (48.75,4.85) {\(A\)};
    \node at (48.75,2.35) {\(A\)};
    \draw [dashed] (48.75,8.25) -- (48.75,5.25);
    \draw [dashed] (48.75,4.5) -- (48.75,2.8);
    \draw [dashed] (48.75,-1.75) -- (48.75,2);
    \node at (42.5,-2.25) {\(a\)};
    \node at (44.75,-2.25) {\(b\)};
    \node at (45,-4.75) {\(b\)};
    \node at (48.75,-4.75) {\(c\)};
    \node at (52.25,-4.75) {\(d\)};
    \node at (52.75,-2.25) {\(d\)};
    \node at (55,-2.25) {\(e\)};
\end{circuitikz}
    }
    \end{figure}
    
    Niech \( n = \card{N} \) będzie liczbą dostępnych nieterminali oraz \( m \) długością najdłuższej produkcji.
    
    Wówczas, jeśli słowo \( w \) jest długości co najmniej \( n_0 = m^{n + 1} + 1 \), to drzewo wywodu musi mieć wysokość co najmniej \( n + 1 \), a zatem jakiś nieterminal powtarza się na jakiejś ścieżce.
    
    Aby zapewnić warunek \( \card{bcd} \leq n_0 \) z tezy, wybieramy ten nieterminal, którego niższe wystąpienie jest najwyżej ze wszystkich -- znajduje się na głębokości co najwyżej \( n + 1 \).
    
    Jeśli mamy pecha i dla takiego wyboru \( \card{bd} = 0 \) to znaczy że przeszliśmy ścieżką, która niczego nie produkuje. 
    Szukamy wtedy innego nieterminala, który spełnia warunek -- taki nieterminal musi istnieć, ponieważ słowo, które produkujemy, jest długie.
\end{proof}