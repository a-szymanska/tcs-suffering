W konstrukcji liczb rzeczywistych pojawia się pojęcie ciągów Cauchy'ego, znane dobrze z analizy matematycznej.
\begin{definition}[Ciąg Cauchy'ego]
    Ciagiem Cauchy'ego zbioru A nazywamy każdy taki ciag (czyli funkcję) \( a : \natural \rightarrow A \), który spełnia warunek (Cauchy'ego):
    \[
        \forall_{A \ni \eps > 0} \ \exists_{n_0 \in \natural} \ \forall_{p,k \in \natural} \ p,k > n_0 \implies \abs{a_p - a_k} < \eps
    \]
\end{definition}
Dodatkowo każdy ciąg Cauchy'ego jest ograniczony, czyli
\[
    \exists_{M > 0} \ \forall_{n \in \natural} \ \abs{a_n} < M
\]

\begin{definition}
    Na zbiorze ciągów Cauchy'ego \( X = \set{a : \natural \rightarrow \rational \mid a \text{ jest ciągiem Cauchy'ego }} \) określamy relację, w której \( a \simeq b \), gdy:
    \[
        \forall_{\eps > 0} \ \exists_{n_0 \in \natural} \ \forall_{n \in \natural} \ n > n_0 \implies \abs{a_n - b_n} < \eps
    \]
\end{definition}
\begin{lemma}
    Relacja \( \simeq \) jest relacją równoważności.
\end{lemma}
\begin{proof}

    Zwrotność i symetryczność relacji są oczywiste, pozostaje wykazać przechodniość. 
    
    Jeśli \( a \simeq b \) oraz \( b \simeq c \), to:
    \[
        \forall_{\eps > 0} \ \exists_{n_1 \in \natural} \ \forall_{n \in \natural} \ (n > n_1 \implies \abs{a_n - b_n} < \eps)
    \]
    \[
        \forall_{\eps > 0} \ \exists_{n_2 \in \natural} \ \forall_{n \in \natural} \ (n > n_2 \implies \abs{b_n - c_n} < \eps)
    \]
    Dla ustalonego \( \eps > 0 \) dobieramy liczby \( n_1, n_2 \leq \frac{\eps}{2} \). Dla \( n_1, n_2 < n \) otrzymujemy nierówności \( |a_n - b_n| < \eps / 2 \) i \( |b_n - c_n| < \eps / 2 \).
    Zatem dla \( n > \max(n_1, n_2) \) zachodzi \( |a_n - b_n| < \eps / 2 \) i \( |b_n - c_n| < \eps / 2 \). Korzystając z nierówności trójkąta, mamy:
    \[
        \forall_{\natural \ni n > n_1} \ \abs{a_n - b_n} < \frac{\eps}{2}
    \]
    \[
        \forall_{\natural \ni n > n_2} \ \abs{b_n - c_n} < \frac{\eps}{2}
    \]
    Połączenie tych nierówności daje
    \[
        \abs{a_n - c_n} \leq \abs{a_n - b_n} + \abs{b_n - c_n} < \frac{\eps}{2} + \frac{\eps}{2} = \eps,
    \]
    co dowodzi przechodniości relacji \( \simeq \).
\end{proof}
                        
Czas na definicję liczb rzeczywistych według George'a Cantora:
\begin{definition}[Zbiór liczb rzeczywistych]
    Zbiór liczb rzeczywistych \( \real \) definiujemy jako zbiór \( X/\simeq \).    
\end{definition}
Działania na tym zbiorze są określone następująco:
\begin{itemize}
    \item dodawanie: \( [a]_{\simeq} + [b]_{\simeq} = [a + b]_{\simeq} \)
    \item mnożenie: \( [a]_{\simeq} \cdot [b]_{\simeq} = [a \cdot b]_{\simeq} \)
\end{itemize}