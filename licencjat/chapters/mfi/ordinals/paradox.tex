\subsection{Paradoks liczb porządkowych}
\begin{theorem}[Antynomia Burali-Forti]
    Nie istnieje zbiór wszystkich liczb porządkowych.
\end{theorem}
\begin{proof}
    Dla dowodu nie wprost załóżmy, że \( X \) jest zbiorem wszystkich liczb porządkowych.
    Z~własności liczb porządkowych wiemy, że \( X \) jest dobrze uporządkowany przez inkluzję.
    Niech \( x, y \in X \) oraz \( x \neq y \). Wtedy \( x \subsetneq y \), czyli \( x \in y \) lub \( y \subsetneq x \), więc \( y \in x \), zatem \( X \) spełnia pierwszy z warunków bycia liczbą porządkową.

    Skoro \( x \in X \) jest liczbą porządkową, to \( x \subset X \), więc zbiór \( X \) spełnia oba warunki i jest liczbą porządkową. Musi więc być \( X \in X \) -- taki zbiór nie może istnieć, co daje sprzeczność z~założeniem.
\end{proof}