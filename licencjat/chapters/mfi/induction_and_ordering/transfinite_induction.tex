Na początek parę potrzebnych faktów, dzięki którym indukcja (pozaskończona) jest możliwa -- uogólniają to, co intuicyjnie działa na liczbach naturalnych.

\begin{definition}[Dobry porządek]
    Porządek \( (X, \leq) \) nazywamy dobrym, gdy w każdym niepustym podzbiórze \( X \) jest element najmniejszy.
\end{definition}

\begin{theorem}
    W dobrym porządku każdy element za wyjątkiem największego posiada następnik.
\end{theorem}
\begin{proof}
    Niech \( (X, \leq) \) będzie zbiorem dobrze uporządkowanym. Wybieramy \( x \in X \), który nie jest elementem największym.
    Definiujemy zbiór \( A = \set{y \in X : x < y} \), który jest niepusty, ponieważ \( x \) nie jest największy. Skoro \( X \) jest dobrze uporządkowany, to w \( A \) musi istnieć element najmniejszy, oznaczmy go przez \( y \).
    I to jest następnik \( x \). Przyjrzyjmy się potencjalnemu następnikowi \( z > x, \ z \in X \). Musi być tak, że \( z \in A \) oraz \( y \leq z \), bo \( y \) jest najmniejszy w \( A \).
    Więc \( y \) rzeczywiście jest następnikiem \( x \).
\end{proof}
Nie jest to oczywista własność -- na przykład w zbiorze liczb wymiernych żaden element nie ma następnika.

\begin{theorem}
    W dobrym porządku każdy przedział początkowy właściwy jest postaci \( \set{x \in X : x < x_0} \) dla pewnego \( x_0 \in X \).
\end{theorem}
\begin{proof}
    Niech \( A \) będzie właściwym przedziałem początkowym \( X \). Zbiór \( X \setminus A \) jest niepusty, więc możemy wziąć jego element najmniejszy \( x_0 \).

    Dla dowodu nie wprost załóżmy, że istnieje taki \( y \in A \), że \( x_0 \leq y \). Ponieważ \( A \) jest przedziałem początkowym, to \( x_0 \) również musiałby być elementem \( A \), co daje sprzeczność z tym, że \( x_0 \in X \setminus A \).

    Również nie wprost załóżmy, że istnieje \( y \in X \setminus A \) taki, że \( y < x_0 \). To jednak daje sprzeczność z tym, że \( x_0 \) jest elementem najmniejszym w \( X \setminus A \).

    To dowodzi, że \( A = \set{x \in X : x < x_0} \).
\end{proof}

\begin{definition}[Zasada indukcji]
    Niech \( (X, \leq) \) będzie liniowym porządkiem. W \( (X, \leq) \) obowiązuje Zasada Indukcji,
    jeśli dla dowolnego zbioru \( Z \) takiego że:
    \begin{enumerate}
        \item \( Z \subset X \)
        \item \( Z \neq \empty \)
        \item jeżeli \( \set{y \in X : y < x} \subset Z \), to \( x \in Z \)
    \end{enumerate}
    zachodzi \( Z = X \).
\end{definition}

\begin{theorem}
    W dobrym porządku obowiązuje zasada indukcji.
\end{theorem}
\begin{proof}
    Niech (X,≤)
    będzie dobrym porządkiem. Niech Z
    będzie dowolnym zbiorem takim, że:
   
   Z⊂X
   ,
   element najmniejszy X
    należy do Z
   ,
   dla dowolnego x∈X
    jeśli {y∈X:y<x}⊂Z
    to x∈Z
   .
   Pokażemy, że Z=X
   . Niech A=X∖Z
   . Dla dowodu niewprost przypuśćmy, że A≠∅
   . W takim przypadku w zbiorze A
    istnieje element najmniejszy a
   . Skoro a
    jest najmniejszy w A
   , to każdy element b∈X
   , dla którego b<a
    musi należeć do Z
    (nie może należeć do A
    więc należy do X∖A=Z
   ). Wtedy wiemy, że {b∈X:b<a}⊂Z
   , a więc z trzeciej własności zbioru Z
    otrzymujemy a∈Z
   , a więc dostaliśmy sprzeczność (bo a∈A∩Z
   , a te zbiory są rozłączne).
   
   Okazuje się, że dobre porządki są nawet bardziej związane z zasadą indukcji. Wyrazem tego jest poniższe twierdzenie.
\end{proof}

\begin{theorem}
    Każdy porządek liniowy, w którym istnieje element najmniejszy i obowiązuje zasada indukcji jest dobry.
\end{theorem}
\begin{proof}
    Niech (X,≤)
 będzie liniowym porządkiem, w którym istnieje element najmniejszy ⊥
 oraz obowiązuje zasada indukcji. Niech A⊂X
 będzie podzbiorem X
, w którym nie ma elementu najmniejszego. Zdefiniujmy zbiór Z
 jako zbiór tych elementów X
, które są mniejsze od wszystkich elementów z A
, czyli:

Z={z∈X:∀a∈Az<a}.

Zbiór Z
 jest niepusty, gdyż ⊥∈Z
 (⊥
 nie może należeć do A
, gdyż byłby najmniejszy). Pokażemy, że dla dowolnego x∈X
, jeśli {y∈X:y<x}⊂Z
, to x∈Z
. Przypuśćmy, że tak nie jest. Wtedy dla pewnego x0∈X
 mamy {y∈X:y<x0}⊂Z
 oraz x0∉Z
. Wynika stąd, że istnieje element a∈A
 taki, że a≤x0
, ponieważ jednak żaden element mniejszy od x0
 nie należy do A
, to a=x0
, a więc x0∈A
. Z tego samego powodu i z faktu, że porządek jest liniowy otrzymujemy, że element x0
 jest najmniejszy w A
, co jest sprzeczne z założeniem, że w A
 nie ma elementu najmniejszego. Wobec tego konieczne jest, aby x∈Z
.

Pokazaliśmy, że zbiór Z
 spełnia założenia zasady indukcji. Ponieważ zasada ta obowiązuje w (X,≤)
, to otrzymujemy Z=X
. Wynika stąd, że zbiór A
 musi być pusty. Wobec tego każdy niepusty podzbiór X
 ma element najmniejszy, a więc (X,≤)
 jest dobrym porządkiem.

Twierdzenie o definiowaniu przez indukcję udowodnione dla liczb naturalnych również ma swój odpowiednik dla dobrych porządków. Mówi ono, że jeśli wyspecyfikujemy sposób konstrukcji wartości funkcji na argumentach (x,b)
 na podstawie wartości x,b
 oraz wartości tej funkcji dla wszystkich (y,b)
 takich, że y<x
, to wyznaczymy jednoznacznie funkcję h
 odpowiadającą tej specyfikacji. Twierdzenie to, nazywane jest twierdzeniem o definiowaniu przez indukcję pozaskończoną, gdyż najważniejsze zastosowania ma właśnie dla zbiorów nieskończonych.
\end{proof}

Lemat
(X, ≤) jest porz¡dkiem dobrym oraz f : X → X jest iniekcj¡ rosn¡c¡
to x ≤ f (x).
Lemat
Je»eli (X, ≤) oraz (Y , ≤) s¡ dobrymi podobnymi porz¡dkami to
podobie«stwo jest jedyne.