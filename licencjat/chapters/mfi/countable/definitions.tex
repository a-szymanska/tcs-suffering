Krótkie przypomnienie, czym są zbiory skończone.
\begin{definition}[Zbiór skończony]
    Zbiór \( X \) nazywamy skończonym, gdy \( X \sim_{m} n \) dla pewnej liczby naturalnej \( n \). \\
    Zbiór \( X \) nazywamy nieskończonym, gdy \( X \) nie jest skończony.
\end{definition}
Wynikają z tego, że jeśli zbiór \( A \) jest skończony i \( B \subset A \), to \( B \) też jest skończony.

\begin{definition}[Zbiór przeliczalny]
    Zbiór \( X \) jest przeliczalny, gdy \( X \sim_{m} N_0 \) dla pewnego \( N_0 \subset \natural \).
\end{definition}
\begin{theorem}
    Zbiór \( X \) jest przeliczalny wtedy i tylko wtedy, gdy \( X \) jest skończony lub równoliczny z \( \natural \).
\end{theorem}
\begin{proof}
    Jeśli zbiór \( X \) jest skończony, czyli \( X \sim_{m} n \in \natural \) lub jest równoliczny z \( \natural \), to niewątpliwie jest równoliczny z jakimś podzbiorem \( \natural \), co z definicji oznacza, że jest przeliczalny.
    
    Załóżmy, że \( X \sim_{m} N_0 \subset \natural \). Jeśli zbiór \( N_0 \) jest skończony, to \( X \) też jest skończony z przechodniości relacji równoliczności (patrz: Podrozdział \ref{mfi:equinumerosity}).
    Jeśli \( N_0 \) jest nieskończony, to jest on równoliczny z \( \natural \), więc ponownie z przechodniości relacji równoliczności, \( X \) jest równoliczny z \( \natural \).
\end{proof}

\begin{lemma}
    \label{lemma:countable-properties}
    Własności zbiorów przeliczalnych:
    \begin{enumerate}
        \item Podzbiór zbioru przeliczalnego jest przeliczalny.
        \item Suma zbiorów przeliczalnych jest przeliczalna.
        \item Iloczyn kartezjański zbiorów przeliczalnych jest przeliczalny.
        \item Zbiór \( \natural^k \) dla \( k \geq 1 \) jest przeliczalny.
        \item Dla skończonej rodziny zbiorów przeliczalnych \( x \in \powerset(\powerset(X)) \) zbiór \( \prod x \) jest przeliczalny.
        \item Dla zbioru przeliczalnego \( X \) jego rozkład \( r \in \powerset(\powerset(X)) \) jest przeliczalny.
    \end{enumerate}
\end{lemma}