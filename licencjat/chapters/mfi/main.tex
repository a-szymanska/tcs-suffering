\chapter{\texorpdfstring{M\(\varphi\)}{MFI}}

% Ta kolejność jest jakaś losowa, ewidentnie nie ma tu dobrego porządku.
% Dlaczego definiowanie przez indukcję jest przed indukcją
% a domykanie relacji przed iloczynem kartezjańskim
% lol

\section*{Wprowadzenie}
\input{chapters/mfi/introduction}

\section{Definicje dodawania, mnożenia, potęgowania i~odejmowania w operaciu o~twierdzenie o~definiowaniu przez indukcję. Własności tych działań}
\subsection{Deeply pipelined}
Nowoczesne procesory x86 posiadają głębokie potoki wykonawcze. Cykl przetwarzania instrukcji (pobranie, dekodowanie, wykonanie, zapis wyników) jest rozbity na wiele mniejszych etapów. Dzięki temu możliwe jest przetwarzanie wielu instrukcji równolegle, choć każda z nich znajduje się na innym etapie wykonania. Im głębszy potok, tym większy potencjalny zysk z wysokiego taktowania, ale też większe straty przy błędach przewidywania.
\subsection{Speculative execution}
Jeśli mamy skoki warunkowe to możemy próbować przewidywać czy skok nastąpi czy nie i na tej podstawie wykonywać instrukcje na zapas, czekając jedynie z fazą commit na faktyczne potwierdzenie czy skok następuje czy nie.
Jeśli zgadliśmy poprawnie to super – od razu commitujemy wynik. Natomiast jeśli nie zgadliśmy to musimy teraz wyrzucić cały pipeline na śmietnik i w efekcie dostajemy opóźnienie.
Robimy to zwykle automatem, który kiedy raz skoczyliśmy następnym razem wie, że też pewnie skoczymy.
\subsection{Out-of-order execution}
Procesory nie czekają na wykonanie instrukcji w kolejności ich występowania, analizują zależnośći i przestawiają instrukcje tak aby efektywniej wykorzystać zasoby. 
\subsection{Superscalar}
Procesor równolegle wykonuje wiele instrukcji w jednym cyklu zegara. Pozwala to na większą liczbę operacji niż by to wynikało z taktowania zegara. 
\subsection{Complex instruction set computer}
Procesory x86 mają bardzo wiele instukcji, które są często skomplikowane. Procesory typu ARM mają mało instrukcji i muszą często wywołać ich wiele, aby uzyskać tą samą funkcjonalność co x86. Skomplikowane instrukcje są optymalne, ale trudne w skalowaniu.


\section{Domykanie relacji ze względu na różne własności. Podaj przykłady własności, na które istnieje i~nie istnieje domknięcie}
\subsection{Deeply pipelined}
Nowoczesne procesory x86 posiadają głębokie potoki wykonawcze. Cykl przetwarzania instrukcji (pobranie, dekodowanie, wykonanie, zapis wyników) jest rozbity na wiele mniejszych etapów. Dzięki temu możliwe jest przetwarzanie wielu instrukcji równolegle, choć każda z nich znajduje się na innym etapie wykonania. Im głębszy potok, tym większy potencjalny zysk z wysokiego taktowania, ale też większe straty przy błędach przewidywania.
\subsection{Speculative execution}
Jeśli mamy skoki warunkowe to możemy próbować przewidywać czy skok nastąpi czy nie i na tej podstawie wykonywać instrukcje na zapas, czekając jedynie z fazą commit na faktyczne potwierdzenie czy skok następuje czy nie.
Jeśli zgadliśmy poprawnie to super – od razu commitujemy wynik. Natomiast jeśli nie zgadliśmy to musimy teraz wyrzucić cały pipeline na śmietnik i w efekcie dostajemy opóźnienie.
Robimy to zwykle automatem, który kiedy raz skoczyliśmy następnym razem wie, że też pewnie skoczymy.
\subsection{Out-of-order execution}
Procesory nie czekają na wykonanie instrukcji w kolejności ich występowania, analizują zależnośći i przestawiają instrukcje tak aby efektywniej wykorzystać zasoby. 
\subsection{Superscalar}
Procesor równolegle wykonuje wiele instrukcji w jednym cyklu zegara. Pozwala to na większą liczbę operacji niż by to wynikało z taktowania zegara. 
\subsection{Complex instruction set computer}
Procesory x86 mają bardzo wiele instukcji, które są często skomplikowane. Procesory typu ARM mają mało instrukcji i muszą często wywołać ich wiele, aby uzyskać tą samą funkcjonalność co x86. Skomplikowane instrukcje są optymalne, ale trudne w skalowaniu.


\section{Zbiory przeliczalne i~ich przykłady}
\label{mfi:countable}
\subsection{Deeply pipelined}
Nowoczesne procesory x86 posiadają głębokie potoki wykonawcze. Cykl przetwarzania instrukcji (pobranie, dekodowanie, wykonanie, zapis wyników) jest rozbity na wiele mniejszych etapów. Dzięki temu możliwe jest przetwarzanie wielu instrukcji równolegle, choć każda z nich znajduje się na innym etapie wykonania. Im głębszy potok, tym większy potencjalny zysk z wysokiego taktowania, ale też większe straty przy błędach przewidywania.
\subsection{Speculative execution}
Jeśli mamy skoki warunkowe to możemy próbować przewidywać czy skok nastąpi czy nie i na tej podstawie wykonywać instrukcje na zapas, czekając jedynie z fazą commit na faktyczne potwierdzenie czy skok następuje czy nie.
Jeśli zgadliśmy poprawnie to super – od razu commitujemy wynik. Natomiast jeśli nie zgadliśmy to musimy teraz wyrzucić cały pipeline na śmietnik i w efekcie dostajemy opóźnienie.
Robimy to zwykle automatem, który kiedy raz skoczyliśmy następnym razem wie, że też pewnie skoczymy.
\subsection{Out-of-order execution}
Procesory nie czekają na wykonanie instrukcji w kolejności ich występowania, analizują zależnośći i przestawiają instrukcje tak aby efektywniej wykorzystać zasoby. 
\subsection{Superscalar}
Procesor równolegle wykonuje wiele instrukcji w jednym cyklu zegara. Pozwala to na większą liczbę operacji niż by to wynikało z taktowania zegara. 
\subsection{Complex instruction set computer}
Procesory x86 mają bardzo wiele instukcji, które są często skomplikowane. Procesory typu ARM mają mało instrukcji i muszą często wywołać ich wiele, aby uzyskać tą samą funkcjonalność co x86. Skomplikowane instrukcje są optymalne, ale trudne w skalowaniu.


\section{Konstrukcja Cantora liczb rzeczywistych. Porządek na liczbach rzeczywistych. Twierdzenie o~rozwinięciu liczby rzeczywistej w~szereg}
\subsection{Deeply pipelined}
Nowoczesne procesory x86 posiadają głębokie potoki wykonawcze. Cykl przetwarzania instrukcji (pobranie, dekodowanie, wykonanie, zapis wyników) jest rozbity na wiele mniejszych etapów. Dzięki temu możliwe jest przetwarzanie wielu instrukcji równolegle, choć każda z nich znajduje się na innym etapie wykonania. Im głębszy potok, tym większy potencjalny zysk z wysokiego taktowania, ale też większe straty przy błędach przewidywania.
\subsection{Speculative execution}
Jeśli mamy skoki warunkowe to możemy próbować przewidywać czy skok nastąpi czy nie i na tej podstawie wykonywać instrukcje na zapas, czekając jedynie z fazą commit na faktyczne potwierdzenie czy skok następuje czy nie.
Jeśli zgadliśmy poprawnie to super – od razu commitujemy wynik. Natomiast jeśli nie zgadliśmy to musimy teraz wyrzucić cały pipeline na śmietnik i w efekcie dostajemy opóźnienie.
Robimy to zwykle automatem, który kiedy raz skoczyliśmy następnym razem wie, że też pewnie skoczymy.
\subsection{Out-of-order execution}
Procesory nie czekają na wykonanie instrukcji w kolejności ich występowania, analizują zależnośći i przestawiają instrukcje tak aby efektywniej wykorzystać zasoby. 
\subsection{Superscalar}
Procesor równolegle wykonuje wiele instrukcji w jednym cyklu zegara. Pozwala to na większą liczbę operacji niż by to wynikało z taktowania zegara. 
\subsection{Complex instruction set computer}
Procesory x86 mają bardzo wiele instukcji, które są często skomplikowane. Procesory typu ARM mają mało instrukcji i muszą często wywołać ich wiele, aby uzyskać tą samą funkcjonalność co x86. Skomplikowane instrukcje są optymalne, ale trudne w skalowaniu.


\section{Iloczyn kartezjański i~jego własności. Pojęcia relacji, złożenia, relacji odwrotnej, własności tych pojęć}
\label{mfi:cartesian_and_relations}
\subsection{Deeply pipelined}
Nowoczesne procesory x86 posiadają głębokie potoki wykonawcze. Cykl przetwarzania instrukcji (pobranie, dekodowanie, wykonanie, zapis wyników) jest rozbity na wiele mniejszych etapów. Dzięki temu możliwe jest przetwarzanie wielu instrukcji równolegle, choć każda z nich znajduje się na innym etapie wykonania. Im głębszy potok, tym większy potencjalny zysk z wysokiego taktowania, ale też większe straty przy błędach przewidywania.
\subsection{Speculative execution}
Jeśli mamy skoki warunkowe to możemy próbować przewidywać czy skok nastąpi czy nie i na tej podstawie wykonywać instrukcje na zapas, czekając jedynie z fazą commit na faktyczne potwierdzenie czy skok następuje czy nie.
Jeśli zgadliśmy poprawnie to super – od razu commitujemy wynik. Natomiast jeśli nie zgadliśmy to musimy teraz wyrzucić cały pipeline na śmietnik i w efekcie dostajemy opóźnienie.
Robimy to zwykle automatem, który kiedy raz skoczyliśmy następnym razem wie, że też pewnie skoczymy.
\subsection{Out-of-order execution}
Procesory nie czekają na wykonanie instrukcji w kolejności ich występowania, analizują zależnośći i przestawiają instrukcje tak aby efektywniej wykorzystać zasoby. 
\subsection{Superscalar}
Procesor równolegle wykonuje wiele instrukcji w jednym cyklu zegara. Pozwala to na większą liczbę operacji niż by to wynikało z taktowania zegara. 
\subsection{Complex instruction set computer}
Procesory x86 mają bardzo wiele instukcji, które są często skomplikowane. Procesory typu ARM mają mało instrukcji i muszą często wywołać ich wiele, aby uzyskać tą samą funkcjonalność co x86. Skomplikowane instrukcje są optymalne, ale trudne w skalowaniu.


\section{Konstrukcja liczb naturalnych von Neumanna, twierdzenie o~indukcji. Własności liczb naturalnych}
\label{mfi:nat_and_induction}
\subsection{Deeply pipelined}
Nowoczesne procesory x86 posiadają głębokie potoki wykonawcze. Cykl przetwarzania instrukcji (pobranie, dekodowanie, wykonanie, zapis wyników) jest rozbity na wiele mniejszych etapów. Dzięki temu możliwe jest przetwarzanie wielu instrukcji równolegle, choć każda z nich znajduje się na innym etapie wykonania. Im głębszy potok, tym większy potencjalny zysk z wysokiego taktowania, ale też większe straty przy błędach przewidywania.
\subsection{Speculative execution}
Jeśli mamy skoki warunkowe to możemy próbować przewidywać czy skok nastąpi czy nie i na tej podstawie wykonywać instrukcje na zapas, czekając jedynie z fazą commit na faktyczne potwierdzenie czy skok następuje czy nie.
Jeśli zgadliśmy poprawnie to super – od razu commitujemy wynik. Natomiast jeśli nie zgadliśmy to musimy teraz wyrzucić cały pipeline na śmietnik i w efekcie dostajemy opóźnienie.
Robimy to zwykle automatem, który kiedy raz skoczyliśmy następnym razem wie, że też pewnie skoczymy.
\subsection{Out-of-order execution}
Procesory nie czekają na wykonanie instrukcji w kolejności ich występowania, analizują zależnośći i przestawiają instrukcje tak aby efektywniej wykorzystać zasoby. 
\subsection{Superscalar}
Procesor równolegle wykonuje wiele instrukcji w jednym cyklu zegara. Pozwala to na większą liczbę operacji niż by to wynikało z taktowania zegara. 
\subsection{Complex instruction set computer}
Procesory x86 mają bardzo wiele instukcji, które są często skomplikowane. Procesory typu ARM mają mało instrukcji i muszą często wywołać ich wiele, aby uzyskać tą samą funkcjonalność co x86. Skomplikowane instrukcje są optymalne, ale trudne w skalowaniu.


\section{Zasada minimum. Zasada maksimum. Twierdzenie o~definiowaniu przez indukcję}
\subsection{Deeply pipelined}
Nowoczesne procesory x86 posiadają głębokie potoki wykonawcze. Cykl przetwarzania instrukcji (pobranie, dekodowanie, wykonanie, zapis wyników) jest rozbity na wiele mniejszych etapów. Dzięki temu możliwe jest przetwarzanie wielu instrukcji równolegle, choć każda z nich znajduje się na innym etapie wykonania. Im głębszy potok, tym większy potencjalny zysk z wysokiego taktowania, ale też większe straty przy błędach przewidywania.
\subsection{Speculative execution}
Jeśli mamy skoki warunkowe to możemy próbować przewidywać czy skok nastąpi czy nie i na tej podstawie wykonywać instrukcje na zapas, czekając jedynie z fazą commit na faktyczne potwierdzenie czy skok następuje czy nie.
Jeśli zgadliśmy poprawnie to super – od razu commitujemy wynik. Natomiast jeśli nie zgadliśmy to musimy teraz wyrzucić cały pipeline na śmietnik i w efekcie dostajemy opóźnienie.
Robimy to zwykle automatem, który kiedy raz skoczyliśmy następnym razem wie, że też pewnie skoczymy.
\subsection{Out-of-order execution}
Procesory nie czekają na wykonanie instrukcji w kolejności ich występowania, analizują zależnośći i przestawiają instrukcje tak aby efektywniej wykorzystać zasoby. 
\subsection{Superscalar}
Procesor równolegle wykonuje wiele instrukcji w jednym cyklu zegara. Pozwala to na większą liczbę operacji niż by to wynikało z taktowania zegara. 
\subsection{Complex instruction set computer}
Procesory x86 mają bardzo wiele instukcji, które są często skomplikowane. Procesory typu ARM mają mało instrukcji i muszą często wywołać ich wiele, aby uzyskać tą samą funkcjonalność co x86. Skomplikowane instrukcje są optymalne, ale trudne w skalowaniu.


\section{\textcolor{pink}{Relacje równoważności i~podziały zbiorów. Relacja równoważności jako środek do definiowania pojęć abstrakcyjnych}}

\section{Twierdzenie Cantora-Bernsteina. Twierdzenie Cantora. Czy istnieje zbiór wszystkich zbiorów? Odpowiedź uzasadnij}
\label{mfi:cantor}
\subsection{Deeply pipelined}
Nowoczesne procesory x86 posiadają głębokie potoki wykonawcze. Cykl przetwarzania instrukcji (pobranie, dekodowanie, wykonanie, zapis wyników) jest rozbity na wiele mniejszych etapów. Dzięki temu możliwe jest przetwarzanie wielu instrukcji równolegle, choć każda z nich znajduje się na innym etapie wykonania. Im głębszy potok, tym większy potencjalny zysk z wysokiego taktowania, ale też większe straty przy błędach przewidywania.
\subsection{Speculative execution}
Jeśli mamy skoki warunkowe to możemy próbować przewidywać czy skok nastąpi czy nie i na tej podstawie wykonywać instrukcje na zapas, czekając jedynie z fazą commit na faktyczne potwierdzenie czy skok następuje czy nie.
Jeśli zgadliśmy poprawnie to super – od razu commitujemy wynik. Natomiast jeśli nie zgadliśmy to musimy teraz wyrzucić cały pipeline na śmietnik i w efekcie dostajemy opóźnienie.
Robimy to zwykle automatem, który kiedy raz skoczyliśmy następnym razem wie, że też pewnie skoczymy.
\subsection{Out-of-order execution}
Procesory nie czekają na wykonanie instrukcji w kolejności ich występowania, analizują zależnośći i przestawiają instrukcje tak aby efektywniej wykorzystać zasoby. 
\subsection{Superscalar}
Procesor równolegle wykonuje wiele instrukcji w jednym cyklu zegara. Pozwala to na większą liczbę operacji niż by to wynikało z taktowania zegara. 
\subsection{Complex instruction set computer}
Procesory x86 mają bardzo wiele instukcji, które są często skomplikowane. Procesory typu ARM mają mało instrukcji i muszą często wywołać ich wiele, aby uzyskać tą samą funkcjonalność co x86. Skomplikowane instrukcje są optymalne, ale trudne w skalowaniu.


\section{\textcolor{pink}{Ciągłość i~gęstość porządku. Zbiór liczb wymiernych a~zbiór liczb rzeczywistych}}

\section{\textcolor{pink}{Lemat Kuratowskiego-Zorna i~przykłady jego zastosowania}}

\section{Konstrukcja liczb całkowitych. Działania na liczbach całkowitych. Konstrukcja liczb wymiernych i~działania na nich}
\subsection{Deeply pipelined}
Nowoczesne procesory x86 posiadają głębokie potoki wykonawcze. Cykl przetwarzania instrukcji (pobranie, dekodowanie, wykonanie, zapis wyników) jest rozbity na wiele mniejszych etapów. Dzięki temu możliwe jest przetwarzanie wielu instrukcji równolegle, choć każda z nich znajduje się na innym etapie wykonania. Im głębszy potok, tym większy potencjalny zysk z wysokiego taktowania, ale też większe straty przy błędach przewidywania.
\subsection{Speculative execution}
Jeśli mamy skoki warunkowe to możemy próbować przewidywać czy skok nastąpi czy nie i na tej podstawie wykonywać instrukcje na zapas, czekając jedynie z fazą commit na faktyczne potwierdzenie czy skok następuje czy nie.
Jeśli zgadliśmy poprawnie to super – od razu commitujemy wynik. Natomiast jeśli nie zgadliśmy to musimy teraz wyrzucić cały pipeline na śmietnik i w efekcie dostajemy opóźnienie.
Robimy to zwykle automatem, który kiedy raz skoczyliśmy następnym razem wie, że też pewnie skoczymy.
\subsection{Out-of-order execution}
Procesory nie czekają na wykonanie instrukcji w kolejności ich występowania, analizują zależnośći i przestawiają instrukcje tak aby efektywniej wykorzystać zasoby. 
\subsection{Superscalar}
Procesor równolegle wykonuje wiele instrukcji w jednym cyklu zegara. Pozwala to na większą liczbę operacji niż by to wynikało z taktowania zegara. 
\subsection{Complex instruction set computer}
Procesory x86 mają bardzo wiele instukcji, które są często skomplikowane. Procesory typu ARM mają mało instrukcji i muszą często wywołać ich wiele, aby uzyskać tą samą funkcjonalność co x86. Skomplikowane instrukcje są optymalne, ale trudne w skalowaniu.


\section{Przykłady zbiorów nieprzeliczalnych}
\subsection{Deeply pipelined}
Nowoczesne procesory x86 posiadają głębokie potoki wykonawcze. Cykl przetwarzania instrukcji (pobranie, dekodowanie, wykonanie, zapis wyników) jest rozbity na wiele mniejszych etapów. Dzięki temu możliwe jest przetwarzanie wielu instrukcji równolegle, choć każda z nich znajduje się na innym etapie wykonania. Im głębszy potok, tym większy potencjalny zysk z wysokiego taktowania, ale też większe straty przy błędach przewidywania.
\subsection{Speculative execution}
Jeśli mamy skoki warunkowe to możemy próbować przewidywać czy skok nastąpi czy nie i na tej podstawie wykonywać instrukcje na zapas, czekając jedynie z fazą commit na faktyczne potwierdzenie czy skok następuje czy nie.
Jeśli zgadliśmy poprawnie to super – od razu commitujemy wynik. Natomiast jeśli nie zgadliśmy to musimy teraz wyrzucić cały pipeline na śmietnik i w efekcie dostajemy opóźnienie.
Robimy to zwykle automatem, który kiedy raz skoczyliśmy następnym razem wie, że też pewnie skoczymy.
\subsection{Out-of-order execution}
Procesory nie czekają na wykonanie instrukcji w kolejności ich występowania, analizują zależnośći i przestawiają instrukcje tak aby efektywniej wykorzystać zasoby. 
\subsection{Superscalar}
Procesor równolegle wykonuje wiele instrukcji w jednym cyklu zegara. Pozwala to na większą liczbę operacji niż by to wynikało z taktowania zegara. 
\subsection{Complex instruction set computer}
Procesory x86 mają bardzo wiele instukcji, które są często skomplikowane. Procesory typu ARM mają mało instrukcji i muszą często wywołać ich wiele, aby uzyskać tą samą funkcjonalność co x86. Skomplikowane instrukcje są optymalne, ale trudne w skalowaniu.


\section{\textcolor{pink}{Aksjomatyczne ujęcie teorii mnogości. Aksjomat wyboru.}}

\section{Twierdzenie Knastera-Tarskiego (dla zbiorów). Lemat Banacha}
\label{mfi:knaster_tarski_and_banach}
\subsection{Deeply pipelined}
Nowoczesne procesory x86 posiadają głębokie potoki wykonawcze. Cykl przetwarzania instrukcji (pobranie, dekodowanie, wykonanie, zapis wyników) jest rozbity na wiele mniejszych etapów. Dzięki temu możliwe jest przetwarzanie wielu instrukcji równolegle, choć każda z nich znajduje się na innym etapie wykonania. Im głębszy potok, tym większy potencjalny zysk z wysokiego taktowania, ale też większe straty przy błędach przewidywania.
\subsection{Speculative execution}
Jeśli mamy skoki warunkowe to możemy próbować przewidywać czy skok nastąpi czy nie i na tej podstawie wykonywać instrukcje na zapas, czekając jedynie z fazą commit na faktyczne potwierdzenie czy skok następuje czy nie.
Jeśli zgadliśmy poprawnie to super – od razu commitujemy wynik. Natomiast jeśli nie zgadliśmy to musimy teraz wyrzucić cały pipeline na śmietnik i w efekcie dostajemy opóźnienie.
Robimy to zwykle automatem, który kiedy raz skoczyliśmy następnym razem wie, że też pewnie skoczymy.
\subsection{Out-of-order execution}
Procesory nie czekają na wykonanie instrukcji w kolejności ich występowania, analizują zależnośći i przestawiają instrukcje tak aby efektywniej wykorzystać zasoby. 
\subsection{Superscalar}
Procesor równolegle wykonuje wiele instrukcji w jednym cyklu zegara. Pozwala to na większą liczbę operacji niż by to wynikało z taktowania zegara. 
\subsection{Complex instruction set computer}
Procesory x86 mają bardzo wiele instukcji, które są często skomplikowane. Procesory typu ARM mają mało instrukcji i muszą często wywołać ich wiele, aby uzyskać tą samą funkcjonalność co x86. Skomplikowane instrukcje są optymalne, ale trudne w skalowaniu.


\section{Równoliczność zbiorów na przykładach \texorpdfstring{\(\pars{A^B}^C \eqnum A^{B \times C}\)}{(A\^B)\^C ~ A\^(B x C)} oraz \texorpdfstring{\(\pars{A \times B}^C \eqnum A^C \times B^C\)}{(A x B)\^C ~ A\^C x B\^C}}
\label{mfi:equinumerosity}
\subsection{Deeply pipelined}
Nowoczesne procesory x86 posiadają głębokie potoki wykonawcze. Cykl przetwarzania instrukcji (pobranie, dekodowanie, wykonanie, zapis wyników) jest rozbity na wiele mniejszych etapów. Dzięki temu możliwe jest przetwarzanie wielu instrukcji równolegle, choć każda z nich znajduje się na innym etapie wykonania. Im głębszy potok, tym większy potencjalny zysk z wysokiego taktowania, ale też większe straty przy błędach przewidywania.
\subsection{Speculative execution}
Jeśli mamy skoki warunkowe to możemy próbować przewidywać czy skok nastąpi czy nie i na tej podstawie wykonywać instrukcje na zapas, czekając jedynie z fazą commit na faktyczne potwierdzenie czy skok następuje czy nie.
Jeśli zgadliśmy poprawnie to super – od razu commitujemy wynik. Natomiast jeśli nie zgadliśmy to musimy teraz wyrzucić cały pipeline na śmietnik i w efekcie dostajemy opóźnienie.
Robimy to zwykle automatem, który kiedy raz skoczyliśmy następnym razem wie, że też pewnie skoczymy.
\subsection{Out-of-order execution}
Procesory nie czekają na wykonanie instrukcji w kolejności ich występowania, analizują zależnośći i przestawiają instrukcje tak aby efektywniej wykorzystać zasoby. 
\subsection{Superscalar}
Procesor równolegle wykonuje wiele instrukcji w jednym cyklu zegara. Pozwala to na większą liczbę operacji niż by to wynikało z taktowania zegara. 
\subsection{Complex instruction set computer}
Procesory x86 mają bardzo wiele instukcji, które są często skomplikowane. Procesory typu ARM mają mało instrukcji i muszą często wywołać ich wiele, aby uzyskać tą samą funkcjonalność co x86. Skomplikowane instrukcje są optymalne, ale trudne w skalowaniu.


\section{\textcolor{pink}{Zasada indukcji pozaskończonej a~dobry porządek}}
\subsection{Deeply pipelined}
Nowoczesne procesory x86 posiadają głębokie potoki wykonawcze. Cykl przetwarzania instrukcji (pobranie, dekodowanie, wykonanie, zapis wyników) jest rozbity na wiele mniejszych etapów. Dzięki temu możliwe jest przetwarzanie wielu instrukcji równolegle, choć każda z nich znajduje się na innym etapie wykonania. Im głębszy potok, tym większy potencjalny zysk z wysokiego taktowania, ale też większe straty przy błędach przewidywania.
\subsection{Speculative execution}
Jeśli mamy skoki warunkowe to możemy próbować przewidywać czy skok nastąpi czy nie i na tej podstawie wykonywać instrukcje na zapas, czekając jedynie z fazą commit na faktyczne potwierdzenie czy skok następuje czy nie.
Jeśli zgadliśmy poprawnie to super – od razu commitujemy wynik. Natomiast jeśli nie zgadliśmy to musimy teraz wyrzucić cały pipeline na śmietnik i w efekcie dostajemy opóźnienie.
Robimy to zwykle automatem, który kiedy raz skoczyliśmy następnym razem wie, że też pewnie skoczymy.
\subsection{Out-of-order execution}
Procesory nie czekają na wykonanie instrukcji w kolejności ich występowania, analizują zależnośći i przestawiają instrukcje tak aby efektywniej wykorzystać zasoby. 
\subsection{Superscalar}
Procesor równolegle wykonuje wiele instrukcji w jednym cyklu zegara. Pozwala to na większą liczbę operacji niż by to wynikało z taktowania zegara. 
\subsection{Complex instruction set computer}
Procesory x86 mają bardzo wiele instukcji, które są często skomplikowane. Procesory typu ARM mają mało instrukcji i muszą często wywołać ich wiele, aby uzyskać tą samą funkcjonalność co x86. Skomplikowane instrukcje są optymalne, ale trudne w skalowaniu.


\section{Liczby porządkowe von Neumanna i~ich własności. Antynomia Burali-Forti}
\subsection{Deeply pipelined}
Nowoczesne procesory x86 posiadają głębokie potoki wykonawcze. Cykl przetwarzania instrukcji (pobranie, dekodowanie, wykonanie, zapis wyników) jest rozbity na wiele mniejszych etapów. Dzięki temu możliwe jest przetwarzanie wielu instrukcji równolegle, choć każda z nich znajduje się na innym etapie wykonania. Im głębszy potok, tym większy potencjalny zysk z wysokiego taktowania, ale też większe straty przy błędach przewidywania.
\subsection{Speculative execution}
Jeśli mamy skoki warunkowe to możemy próbować przewidywać czy skok nastąpi czy nie i na tej podstawie wykonywać instrukcje na zapas, czekając jedynie z fazą commit na faktyczne potwierdzenie czy skok następuje czy nie.
Jeśli zgadliśmy poprawnie to super – od razu commitujemy wynik. Natomiast jeśli nie zgadliśmy to musimy teraz wyrzucić cały pipeline na śmietnik i w efekcie dostajemy opóźnienie.
Robimy to zwykle automatem, który kiedy raz skoczyliśmy następnym razem wie, że też pewnie skoczymy.
\subsection{Out-of-order execution}
Procesory nie czekają na wykonanie instrukcji w kolejności ich występowania, analizują zależnośći i przestawiają instrukcje tak aby efektywniej wykorzystać zasoby. 
\subsection{Superscalar}
Procesor równolegle wykonuje wiele instrukcji w jednym cyklu zegara. Pozwala to na większą liczbę operacji niż by to wynikało z taktowania zegara. 
\subsection{Complex instruction set computer}
Procesory x86 mają bardzo wiele instukcji, które są często skomplikowane. Procesory typu ARM mają mało instrukcji i muszą często wywołać ich wiele, aby uzyskać tą samą funkcjonalność co x86. Skomplikowane instrukcje są optymalne, ale trudne w skalowaniu.
