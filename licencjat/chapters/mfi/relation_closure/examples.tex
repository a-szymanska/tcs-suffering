\subsection{Domykanie relacji ze względu własności}
Z definicji domknięcia łatwo wyprowadzić poniższe fakty dla relacji \( R \subset X^2 \):
\begin{itemize}
    \item Domknięciem zwrotnym relacji \( R \) jest \( R \cup 1_X \).
    \item Domknięciem symetrycznym relacji \( R \) jest \( R \cup R^{-1} \).
    \item Domknięciem przechodnim relacji \( R \) jest \( R^{*} = \bigcup\set{R^i \colon i > 0} \).
\end{itemize}
Zwrotność, symetryczność i przechodniość są więc przykładami własności, na które istnieją domknięcia.

Przykładem takiej, na które nie istnieją może być asymetryczność:
\[
    (a, b) \in R \implies (b, a) \notin R
\]
Jeśli relacja \( R \) nie jest asymetryczna, to istnieją pewne \( a, b \) takie, że \( (a, b), (b, a) \in R \).
Przywrócenie asymetryczności wymagałoby usunięcia jednej z par z \( S \), a wtedy \( R \not\subset S \), co jest niezgodne z definicją domknięcia.
Podobnie działa to w przypadku antysymetryczności:
\[
    (a, b) \in R \land (b, a) \in R \implies a = b,
\]
antyzwrotności i acykliczności.