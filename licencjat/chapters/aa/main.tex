\chapter{Analiza Algorytmów}

\section{\textcolor{pink}{Zastosowanie funkcji tworzących prawdopodobieństwa do analizy złożoności średniej (np. problem sekretarki lub quicksort).}}

\section{\textcolor{pink}{Zastosowanie zmiennych losowych wskaźnikowych w analizie algorytmów (np. permutacje losowe lub quicksort).}}

\section{\textcolor{pink}{Analiza amortyzowana, nietrywialny przykład zastosowania: szkic analizy drzew splay lub kopców Fibonacciego.}}

\section{\textcolor{pink}{Zaawansowane problemy haszowania: uniwersalne rodziny funkcji haszujących, haszowanie doskonałe.}}

\section{\textcolor{pink}{Twierdzenie o rekurencji uniwersalnej, trzy przykłady zastosowania w analizie algorytmów.}}

\section{\textcolor{pink}{Złożoność sortowania, dolne ograniczenia w przypadku pesymistycznym i średnim.}}

\section{\textcolor{pink}{Problem sumowania zbiorów rozłącznych, rozwiązanie drzewowe z kompresją ścieżek, szkic analizy i uzasadnienie wystąpienia logarytmu iterowanego.}}

\section{\textcolor{pink}{Randomizacja drzew poszukiwań binarnych: model permutacyjny, kopcodrzewa.}}