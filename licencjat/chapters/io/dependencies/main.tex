\begin{definition}[Wstrzykiwanie zależności]
To wzorzec projektowy polegający na usuwaniu bezpośrednich zależności pomiędzy obiektami. Ta technika polega na dostarczaniu potrzebnych obiektów (zależności) zamiast samodzielnego konstruowania ich w obiekcie, który ich potrzebuje.
\end{definition}
Można wyróżnić parę sposób realizacji wstrzykiwania zależności:
\begin{itemize}
    \singlespacing
    \item poprzez konstruktor - obiekt dostaje konstruktor klasy
    \item poprzez setter - klient udostępnia setter, który służy do wstrzyknięcia zależności
    \item poprzez interfejs - zależność udostępnia metodę, która wstrzykuje zależność do przekazanego do niej obiektu
\end{itemize}
Wstrzykiwanie zależności ułatwia modyfikację klas bez konieczności dostosowywania tych, które od nich zależą. Umożliwia również testowanie poszczególnych komponentów w odizolowaniu od innych, ponieważ można łatwiej ,,mockować'' zależności. Bez tego, testując klasę, konieczne byłoby również testowanie wszystkich obiektów, które tworzy.