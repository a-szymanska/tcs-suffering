\chapter{Programowanie Obiektowe}

\section{\textcolor{pink}{Co to jest kompozycja jako obiektowa technika programowania? Podaj przykłady kompozycji w dwóch obiektowych językach programowania.}}

\section{\textcolor{pink}{Co to jest dziedziczenie jako obiektowa technika programowania? Podaj przykłady dziedziczenia w dwóch obiektowych językach programowania.}}

\section{\textcolor{pink}{Co to jest polimorfizm jako obiektowa technika programowania? Podaj przykłady zastosowania polimorfizmu w dwóch obiektowych językach programowania.}}

\section{\textcolor{pink}{Co to są szablony? Wyjaśnij pojęcie i podaj przykład implementacji używającej szablonów w dowolnym języku obiektowym.}}

\section{\textcolor{pink}{Wybierz język obiektowy z rozwiniętą kontrolą dostępu do pól i metod obiektów. Opisz występujące w nim mechanizmy kontroli.}}

\section{\textcolor{pink}{Na czym polega reflection, RTTI? Zaprezentuj zastosowania tych mechanizmów w obiektowym języku programowania.}}

\section{\textcolor{pink}{Na przykładzie obiektowego języka programowania przedstaw strukturę klas służących do obsługi GUI. Opisz technikę sterowania kierowanego zdarzeniami.}}

\section{\textcolor{pink}{Opisz rodzaje klas dostępne w języku programowania Java. Czym są klasy abstrakcyjne? Czym różni się od interfejsów? Jakie ich zastosowania?}}

\section{\textcolor{pink}{Porównaj mechanizmy konstrukcji i niszczenia obiektów w Java i C++. Wskaż mocne i słabe punkty każdego z podejść.}}

\section{\textcolor{pink}{Przedstaw bibliotekę kontenerów dostępnych w Java z opisem zastosowań poszczególnych klas.}}