\chapter{Matematyka dyskretna} 
Krótkie przypomnienie, czym są zbiory skończone.
\begin{definition}[Zbiór skończony]
    Zbiór \( X \) nazywamy skończonym, gdy \( X \sim_{m} n \) dla pewnej liczby naturalnej \( n \). \\
    Zbiór \( X \) nazywamy nieskończonym, gdy \( X \) nie jest skończony.
\end{definition}
Wynikają z tego, że jeśli zbiór \( A \) jest skończony i \( B \subset A \), to \( B \) też jest skończony.

\begin{definition}[Zbiór przeliczalny]
    Zbiór \( X \) jest przeliczalny, gdy \( X \sim_{m} N_0 \) dla pewnego \( N_0 \subset \natural \).
\end{definition}
\begin{theorem}
    Zbiór \( X \) jest przeliczalny wtedy i tylko wtedy, gdy \( X \) jest skończony lub równoliczny z \( \natural \).
\end{theorem}
\begin{proof}
    Jeśli zbiór \( X \) jest skończony, czyli \( X \sim_{m} n \in \natural \) lub jest równoliczny z \( \natural \), to niewątpliwie jest równoliczny z jakimś podzbiorem \( \natural \), co z definicji oznacza, że jest przeliczalny.
    
    Załóżmy, że \( X \sim_{m} N_0 \subset \natural \). Jeśli zbiór \( N_0 \) jest skończony, to \( X \) też jest skończony z przechodniości relacji równoliczności (patrz: Podrozdział \ref{mfi:equinumerosity}).
    Jeśli \( N_0 \) jest nieskończony, to jest on równoliczny z \( \natural \), więc ponownie z przechodniości relacji równoliczności, \( X \) jest równoliczny z \( \natural \).
\end{proof}

\begin{lemma}
    \label{lemma:countable-properties}
    Własności zbiorów przeliczalnych:
    \begin{enumerate}
        \item Podzbiór zbioru przeliczalnego jest przeliczalny.
        \item Suma zbiorów przeliczalnych jest przeliczalna.
        \item Iloczyn kartezjański zbiorów przeliczalnych jest przeliczalny.
        \item Zbiór \( \natural^k \) dla \( k \geq 1 \) jest przeliczalny.
        \item Dla skończonej rodziny zbiorów przeliczalnych \( x \in \powerset(\powerset(X)) \) zbiór \( \prod x \) jest przeliczalny.
        \item Dla zbioru przeliczalnego \( X \) jego rozkład \( r \in \powerset(\powerset(X)) \) jest przeliczalny.
    \end{enumerate}
\end{lemma}

\section{\textcolor{pink}{Zasada włączeń i wyłączeń. Przykłady zastosowań}}
\subsection{Deeply pipelined}
Nowoczesne procesory x86 posiadają głębokie potoki wykonawcze. Cykl przetwarzania instrukcji (pobranie, dekodowanie, wykonanie, zapis wyników) jest rozbity na wiele mniejszych etapów. Dzięki temu możliwe jest przetwarzanie wielu instrukcji równolegle, choć każda z nich znajduje się na innym etapie wykonania. Im głębszy potok, tym większy potencjalny zysk z wysokiego taktowania, ale też większe straty przy błędach przewidywania.
\subsection{Speculative execution}
Jeśli mamy skoki warunkowe to możemy próbować przewidywać czy skok nastąpi czy nie i na tej podstawie wykonywać instrukcje na zapas, czekając jedynie z fazą commit na faktyczne potwierdzenie czy skok następuje czy nie.
Jeśli zgadliśmy poprawnie to super – od razu commitujemy wynik. Natomiast jeśli nie zgadliśmy to musimy teraz wyrzucić cały pipeline na śmietnik i w efekcie dostajemy opóźnienie.
Robimy to zwykle automatem, który kiedy raz skoczyliśmy następnym razem wie, że też pewnie skoczymy.
\subsection{Out-of-order execution}
Procesory nie czekają na wykonanie instrukcji w kolejności ich występowania, analizują zależnośći i przestawiają instrukcje tak aby efektywniej wykorzystać zasoby. 
\subsection{Superscalar}
Procesor równolegle wykonuje wiele instrukcji w jednym cyklu zegara. Pozwala to na większą liczbę operacji niż by to wynikało z taktowania zegara. 
\subsection{Complex instruction set computer}
Procesory x86 mają bardzo wiele instukcji, które są często skomplikowane. Procesory typu ARM mają mało instrukcji i muszą często wywołać ich wiele, aby uzyskać tą samą funkcjonalność co x86. Skomplikowane instrukcje są optymalne, ale trudne w skalowaniu.


\section{\textcolor{pink}{Porządki częściowe, Twierdzenia Dilwortha i Spernera}}
\subsection{Deeply pipelined}
Nowoczesne procesory x86 posiadają głębokie potoki wykonawcze. Cykl przetwarzania instrukcji (pobranie, dekodowanie, wykonanie, zapis wyników) jest rozbity na wiele mniejszych etapów. Dzięki temu możliwe jest przetwarzanie wielu instrukcji równolegle, choć każda z nich znajduje się na innym etapie wykonania. Im głębszy potok, tym większy potencjalny zysk z wysokiego taktowania, ale też większe straty przy błędach przewidywania.
\subsection{Speculative execution}
Jeśli mamy skoki warunkowe to możemy próbować przewidywać czy skok nastąpi czy nie i na tej podstawie wykonywać instrukcje na zapas, czekając jedynie z fazą commit na faktyczne potwierdzenie czy skok następuje czy nie.
Jeśli zgadliśmy poprawnie to super – od razu commitujemy wynik. Natomiast jeśli nie zgadliśmy to musimy teraz wyrzucić cały pipeline na śmietnik i w efekcie dostajemy opóźnienie.
Robimy to zwykle automatem, który kiedy raz skoczyliśmy następnym razem wie, że też pewnie skoczymy.
\subsection{Out-of-order execution}
Procesory nie czekają na wykonanie instrukcji w kolejności ich występowania, analizują zależnośći i przestawiają instrukcje tak aby efektywniej wykorzystać zasoby. 
\subsection{Superscalar}
Procesor równolegle wykonuje wiele instrukcji w jednym cyklu zegara. Pozwala to na większą liczbę operacji niż by to wynikało z taktowania zegara. 
\subsection{Complex instruction set computer}
Procesory x86 mają bardzo wiele instukcji, które są często skomplikowane. Procesory typu ARM mają mało instrukcji i muszą często wywołać ich wiele, aby uzyskać tą samą funkcjonalność co x86. Skomplikowane instrukcje są optymalne, ale trudne w skalowaniu.


\section{\textcolor{pink}{Twierdzenie Ramseya. Przykłady zastosowań}}
\subsection{Deeply pipelined}
Nowoczesne procesory x86 posiadają głębokie potoki wykonawcze. Cykl przetwarzania instrukcji (pobranie, dekodowanie, wykonanie, zapis wyników) jest rozbity na wiele mniejszych etapów. Dzięki temu możliwe jest przetwarzanie wielu instrukcji równolegle, choć każda z nich znajduje się na innym etapie wykonania. Im głębszy potok, tym większy potencjalny zysk z wysokiego taktowania, ale też większe straty przy błędach przewidywania.
\subsection{Speculative execution}
Jeśli mamy skoki warunkowe to możemy próbować przewidywać czy skok nastąpi czy nie i na tej podstawie wykonywać instrukcje na zapas, czekając jedynie z fazą commit na faktyczne potwierdzenie czy skok następuje czy nie.
Jeśli zgadliśmy poprawnie to super – od razu commitujemy wynik. Natomiast jeśli nie zgadliśmy to musimy teraz wyrzucić cały pipeline na śmietnik i w efekcie dostajemy opóźnienie.
Robimy to zwykle automatem, który kiedy raz skoczyliśmy następnym razem wie, że też pewnie skoczymy.
\subsection{Out-of-order execution}
Procesory nie czekają na wykonanie instrukcji w kolejności ich występowania, analizują zależnośći i przestawiają instrukcje tak aby efektywniej wykorzystać zasoby. 
\subsection{Superscalar}
Procesor równolegle wykonuje wiele instrukcji w jednym cyklu zegara. Pozwala to na większą liczbę operacji niż by to wynikało z taktowania zegara. 
\subsection{Complex instruction set computer}
Procesory x86 mają bardzo wiele instukcji, które są często skomplikowane. Procesory typu ARM mają mało instrukcji i muszą często wywołać ich wiele, aby uzyskać tą samą funkcjonalność co x86. Skomplikowane instrukcje są optymalne, ale trudne w skalowaniu.


\section{\textcolor{pink}{Funkcje tworzące. Wyznaczanie liczb Fibonacciego za pomocą funkcji tworzących}}
\subsection{Deeply pipelined}
Nowoczesne procesory x86 posiadają głębokie potoki wykonawcze. Cykl przetwarzania instrukcji (pobranie, dekodowanie, wykonanie, zapis wyników) jest rozbity na wiele mniejszych etapów. Dzięki temu możliwe jest przetwarzanie wielu instrukcji równolegle, choć każda z nich znajduje się na innym etapie wykonania. Im głębszy potok, tym większy potencjalny zysk z wysokiego taktowania, ale też większe straty przy błędach przewidywania.
\subsection{Speculative execution}
Jeśli mamy skoki warunkowe to możemy próbować przewidywać czy skok nastąpi czy nie i na tej podstawie wykonywać instrukcje na zapas, czekając jedynie z fazą commit na faktyczne potwierdzenie czy skok następuje czy nie.
Jeśli zgadliśmy poprawnie to super – od razu commitujemy wynik. Natomiast jeśli nie zgadliśmy to musimy teraz wyrzucić cały pipeline na śmietnik i w efekcie dostajemy opóźnienie.
Robimy to zwykle automatem, który kiedy raz skoczyliśmy następnym razem wie, że też pewnie skoczymy.
\subsection{Out-of-order execution}
Procesory nie czekają na wykonanie instrukcji w kolejności ich występowania, analizują zależnośći i przestawiają instrukcje tak aby efektywniej wykorzystać zasoby. 
\subsection{Superscalar}
Procesor równolegle wykonuje wiele instrukcji w jednym cyklu zegara. Pozwala to na większą liczbę operacji niż by to wynikało z taktowania zegara. 
\subsection{Complex instruction set computer}
Procesory x86 mają bardzo wiele instukcji, które są często skomplikowane. Procesory typu ARM mają mało instrukcji i muszą często wywołać ich wiele, aby uzyskać tą samą funkcjonalność co x86. Skomplikowane instrukcje są optymalne, ale trudne w skalowaniu.


\section{\textcolor{pink}{Skojarzenia w grafach dwudzielnych. Twierdzenie Halla}}
\subsection{Deeply pipelined}
Nowoczesne procesory x86 posiadają głębokie potoki wykonawcze. Cykl przetwarzania instrukcji (pobranie, dekodowanie, wykonanie, zapis wyników) jest rozbity na wiele mniejszych etapów. Dzięki temu możliwe jest przetwarzanie wielu instrukcji równolegle, choć każda z nich znajduje się na innym etapie wykonania. Im głębszy potok, tym większy potencjalny zysk z wysokiego taktowania, ale też większe straty przy błędach przewidywania.
\subsection{Speculative execution}
Jeśli mamy skoki warunkowe to możemy próbować przewidywać czy skok nastąpi czy nie i na tej podstawie wykonywać instrukcje na zapas, czekając jedynie z fazą commit na faktyczne potwierdzenie czy skok następuje czy nie.
Jeśli zgadliśmy poprawnie to super – od razu commitujemy wynik. Natomiast jeśli nie zgadliśmy to musimy teraz wyrzucić cały pipeline na śmietnik i w efekcie dostajemy opóźnienie.
Robimy to zwykle automatem, który kiedy raz skoczyliśmy następnym razem wie, że też pewnie skoczymy.
\subsection{Out-of-order execution}
Procesory nie czekają na wykonanie instrukcji w kolejności ich występowania, analizują zależnośći i przestawiają instrukcje tak aby efektywniej wykorzystać zasoby. 
\subsection{Superscalar}
Procesor równolegle wykonuje wiele instrukcji w jednym cyklu zegara. Pozwala to na większą liczbę operacji niż by to wynikało z taktowania zegara. 
\subsection{Complex instruction set computer}
Procesory x86 mają bardzo wiele instukcji, które są często skomplikowane. Procesory typu ARM mają mało instrukcji i muszą często wywołać ich wiele, aby uzyskać tą samą funkcjonalność co x86. Skomplikowane instrukcje są optymalne, ale trudne w skalowaniu.


\section{\textcolor{pink}{Kolorowanie grafów, twierdzenia Brooksa}}
\input{chapters/dyskretna/colours/brooks/brooks}

\section{\textcolor{pink}{Liczba chromatyczna a liczba kolorująca grafów}}
\subsection{Deeply pipelined}
Nowoczesne procesory x86 posiadają głębokie potoki wykonawcze. Cykl przetwarzania instrukcji (pobranie, dekodowanie, wykonanie, zapis wyników) jest rozbity na wiele mniejszych etapów. Dzięki temu możliwe jest przetwarzanie wielu instrukcji równolegle, choć każda z nich znajduje się na innym etapie wykonania. Im głębszy potok, tym większy potencjalny zysk z wysokiego taktowania, ale też większe straty przy błędach przewidywania.
\subsection{Speculative execution}
Jeśli mamy skoki warunkowe to możemy próbować przewidywać czy skok nastąpi czy nie i na tej podstawie wykonywać instrukcje na zapas, czekając jedynie z fazą commit na faktyczne potwierdzenie czy skok następuje czy nie.
Jeśli zgadliśmy poprawnie to super – od razu commitujemy wynik. Natomiast jeśli nie zgadliśmy to musimy teraz wyrzucić cały pipeline na śmietnik i w efekcie dostajemy opóźnienie.
Robimy to zwykle automatem, który kiedy raz skoczyliśmy następnym razem wie, że też pewnie skoczymy.
\subsection{Out-of-order execution}
Procesory nie czekają na wykonanie instrukcji w kolejności ich występowania, analizują zależnośći i przestawiają instrukcje tak aby efektywniej wykorzystać zasoby. 
\subsection{Superscalar}
Procesor równolegle wykonuje wiele instrukcji w jednym cyklu zegara. Pozwala to na większą liczbę operacji niż by to wynikało z taktowania zegara. 
\subsection{Complex instruction set computer}
Procesory x86 mają bardzo wiele instukcji, które są często skomplikowane. Procesory typu ARM mają mało instrukcji i muszą często wywołać ich wiele, aby uzyskać tą samą funkcjonalność co x86. Skomplikowane instrukcje są optymalne, ale trudne w skalowaniu.


\section{\textcolor{pink}{Kolorowania krawędziowe grafów. Twierdzenie Vizinga}}
\subsection{Deeply pipelined}
Nowoczesne procesory x86 posiadają głębokie potoki wykonawcze. Cykl przetwarzania instrukcji (pobranie, dekodowanie, wykonanie, zapis wyników) jest rozbity na wiele mniejszych etapów. Dzięki temu możliwe jest przetwarzanie wielu instrukcji równolegle, choć każda z nich znajduje się na innym etapie wykonania. Im głębszy potok, tym większy potencjalny zysk z wysokiego taktowania, ale też większe straty przy błędach przewidywania.
\subsection{Speculative execution}
Jeśli mamy skoki warunkowe to możemy próbować przewidywać czy skok nastąpi czy nie i na tej podstawie wykonywać instrukcje na zapas, czekając jedynie z fazą commit na faktyczne potwierdzenie czy skok następuje czy nie.
Jeśli zgadliśmy poprawnie to super – od razu commitujemy wynik. Natomiast jeśli nie zgadliśmy to musimy teraz wyrzucić cały pipeline na śmietnik i w efekcie dostajemy opóźnienie.
Robimy to zwykle automatem, który kiedy raz skoczyliśmy następnym razem wie, że też pewnie skoczymy.
\subsection{Out-of-order execution}
Procesory nie czekają na wykonanie instrukcji w kolejności ich występowania, analizują zależnośći i przestawiają instrukcje tak aby efektywniej wykorzystać zasoby. 
\subsection{Superscalar}
Procesor równolegle wykonuje wiele instrukcji w jednym cyklu zegara. Pozwala to na większą liczbę operacji niż by to wynikało z taktowania zegara. 
\subsection{Complex instruction set computer}
Procesory x86 mają bardzo wiele instukcji, które są często skomplikowane. Procesory typu ARM mają mało instrukcji i muszą często wywołać ich wiele, aby uzyskać tą samą funkcjonalność co x86. Skomplikowane instrukcje są optymalne, ale trudne w skalowaniu.


\section{\textcolor{pink}{Przepływy w sieciach. Twierdzenie o maksymalnym przepływie i minimalnym}
przekroju}
\subsection{Deeply pipelined}
Nowoczesne procesory x86 posiadają głębokie potoki wykonawcze. Cykl przetwarzania instrukcji (pobranie, dekodowanie, wykonanie, zapis wyników) jest rozbity na wiele mniejszych etapów. Dzięki temu możliwe jest przetwarzanie wielu instrukcji równolegle, choć każda z nich znajduje się na innym etapie wykonania. Im głębszy potok, tym większy potencjalny zysk z wysokiego taktowania, ale też większe straty przy błędach przewidywania.
\subsection{Speculative execution}
Jeśli mamy skoki warunkowe to możemy próbować przewidywać czy skok nastąpi czy nie i na tej podstawie wykonywać instrukcje na zapas, czekając jedynie z fazą commit na faktyczne potwierdzenie czy skok następuje czy nie.
Jeśli zgadliśmy poprawnie to super – od razu commitujemy wynik. Natomiast jeśli nie zgadliśmy to musimy teraz wyrzucić cały pipeline na śmietnik i w efekcie dostajemy opóźnienie.
Robimy to zwykle automatem, który kiedy raz skoczyliśmy następnym razem wie, że też pewnie skoczymy.
\subsection{Out-of-order execution}
Procesory nie czekają na wykonanie instrukcji w kolejności ich występowania, analizują zależnośći i przestawiają instrukcje tak aby efektywniej wykorzystać zasoby. 
\subsection{Superscalar}
Procesor równolegle wykonuje wiele instrukcji w jednym cyklu zegara. Pozwala to na większą liczbę operacji niż by to wynikało z taktowania zegara. 
\subsection{Complex instruction set computer}
Procesory x86 mają bardzo wiele instukcji, które są często skomplikowane. Procesory typu ARM mają mało instrukcji i muszą często wywołać ich wiele, aby uzyskać tą samą funkcjonalność co x86. Skomplikowane instrukcje są optymalne, ale trudne w skalowaniu.
