\chapter{Matematyka dyskretna} 
\begin{definition}
    Mówimy, że maszyna M (deterministyczna lub nie) \textbf{działa w czasie} \( T : \natural \rightarrow \natural \), jeśli dla każdej konfiguracji startowej \( q_0 w \) każde obliczenie jest akceptujące lub odrzucające i ma długość co najwyżej \( T(\abs{w}) \)
\end{definition}

\begin{definition}
    Mówimy, że funkcja \( f \) jest \textbf{redukcją wielomianową} (Karpa) jeśli istnieje wielomian \( p \) oraz Maszyna Turinga obliczająca \( f \) w czasie \( p \). 
    
    Jeśli istnieje redukcja wielomianowa z języka \( L_1 \) do języka \( L_2 \), zapisujemy to jako \( L_1 \leq_p L_2 \)
\end{definition}

\begin{definition}
    Dla klasy problemów \( C \subseteq R \) problem \( L \) jest \textbf{C-trudny}, jeśli \( \forall_{L' \in C}\; L' \leq_p L \)
\end{definition}
\begin{definition}
    Dla klasy problemów \( C \subseteq R \) problem \( L \) jest \textbf{C-zupełny}, jeśli jest \(C\)-trudny i \( L \in C \)
\end{definition}

\begin{lemma}
    Jeśli \( L_1 \) jest C-trudny i \( L_1 \leq_p L_2 \) to \( L_2 \) też jest C-trudny. 
\end{lemma}
\begin{proof}
    Skoro \( L_1 \) jest C-trudny, to dla każdego \( L' \in C \) mamy redukcję \( f \) z \( L' \) do \( L_1 \).
    Mamy też redukcję \( g \) z \( L_1 \) do \( L_2 \).
    
    Składając \( g \circ f \) dostajemy redukcję z \( L' \)  do \( L_2 \) co dowodzi, że \( L_2 \) jest trudniejszy niż dowolny \( L' \in C \).
\end{proof}


\section{\textcolor{pink}{Zasada włączeń i wyłączeń. Przykłady zastosowań}}
Jednym z zadań SZBD jest zapewnianie tego, żeby zmiany dokonane przez widoczną lub zatwierdzoną transakcję były trwałe, a efekty transakcji wycofanej lub przerwanyj przez awarię nie.

Jest parę typów awarii:
\begin{itemize}
    \item awaria systemu: awaria sprzętu, błędy oprogramowania, uszkodzenie dysku
    \item błędy logiczne (np. naruszenie ograniczeń integralnościowych)
    \item błędy systemowe (np. zakleszczenia)
\end{itemize}
Do poradzenia sobie z pierwszym rodzajem wystarczy replikacja bazy i rozproszenie danych. W przypadku pozostałych dwóch trzeba odtworzyć bazę po awarii.
Ważne jest, żeby zminimalizować czas potrzebny do tego, ponieważ w trakcie odtwarzania bazy serwer musi być wyłączony dla użytkowników.
Służące do tego algorytmy składają się z dwóch części:
\begin{enumerate}
    \item zapisywanie informacji potrzebnych do odtworzenia stanu bazy na bieżąco podczas zwykłych operacji
    \item czynności po awarii, prowadzące do przywrócenia spójności, atomowości i trwałości
\end{enumerate}

\subsection*{Strategie SZBD}
Brudne strony, czyli zmieniane po ostatnim zapisie na dysku, mogą pozostawać w buforze długo po zatwierdzeniu transakcji. Jeśli nastąpi awaria, to naniesione zmiany są tracone.
Żeby zadbać o poprawny stan bazy można wykonać operacje:
\begin{itemize}
    \item UNDO, wycofanie zmian - zapewnia atomowość
    \item REDO, powtórzenie transakcji - zapewnia trwałość.
\end{itemize}
Podjęte czynności zależą od strategii SZBD: \\
Czy niezatwierdzona transakcja może nadpisać wartość w pamięci trwałej? \\
\begin{itemize}
    \item strategia steal: TAK
    \item strategia no-steal: NIE
\end{itemize}
Czy wszystkie zmiany muszą być zapisane w pamięci trwałej przed zatwierdzeniem transakcji?
\begin{itemize}
    \item strategia force: TAK
    \item strategia no-force: NIE
\end{itemize}

Strategia \textbf{no-steal + force} pozwala uniknąć wykonywania sekwencji UNDO i REDO. Jednak jest mało efektywna i istnieje ryzyko samozakleszczenia transakcji, jeśli braknie miejsca w buforze.
Częściej wykorzystywana jest strategia \textbf{steal + no-force} opakowana przez \textbf{protokół WAL (write-ahead log)}.
Oprócz plików z danymi przechowywany jest dziennik z logami (dziennik), w którym zapisywane są wszystkie zmiany przed wprowadzeniem ich na dysku.
Zawiera on informacje wystarczające odzyskania zawartości bazy za pomocą operacji UNDO i REDO.
Żeby dzienniki nie rozrastały się w nieskończoność, tworzy się punkty kontrolne. W przypadku awarii powtarzane są tylko transakcja zatwierdzona po ostatnim punkcie kontrolnym, a niezatwierdzone są wycofywane.

\subsection{Algorytm ARIES}
Algorytm ARIES (Algorithm for Recovery and Isolation Exploiting Semantics) służy do odtwarzania transakcji, korzystając z dziennika WAL i blokad.
Składa się z trzech faz:
\begin{enumerate}
    \item Analiza: rekonstrukcja Tabeli Brudnych Stron (DPT) i Tabeli Transakcji (TT), wyznaczenie pierwszego wpisu o zmianach tworzących brudną stronę
    \item Faza REDO: powtórzenie wszystkich operacji od momentu powstania brudnej strony, przywrócenie stanu sprzed awarii
    \item Faza UNDO: wycofanie efektów niezatwierdzonych transakcji, dodanie wpisów kompensacyjnych do dziennika
\end{enumerate}

\section{\textcolor{pink}{Porządki częściowe, Twierdzenia Dilwortha i Spernera}}
Jednym z zadań SZBD jest zapewnianie tego, żeby zmiany dokonane przez widoczną lub zatwierdzoną transakcję były trwałe, a efekty transakcji wycofanej lub przerwanyj przez awarię nie.

Jest parę typów awarii:
\begin{itemize}
    \item awaria systemu: awaria sprzętu, błędy oprogramowania, uszkodzenie dysku
    \item błędy logiczne (np. naruszenie ograniczeń integralnościowych)
    \item błędy systemowe (np. zakleszczenia)
\end{itemize}
Do poradzenia sobie z pierwszym rodzajem wystarczy replikacja bazy i rozproszenie danych. W przypadku pozostałych dwóch trzeba odtworzyć bazę po awarii.
Ważne jest, żeby zminimalizować czas potrzebny do tego, ponieważ w trakcie odtwarzania bazy serwer musi być wyłączony dla użytkowników.
Służące do tego algorytmy składają się z dwóch części:
\begin{enumerate}
    \item zapisywanie informacji potrzebnych do odtworzenia stanu bazy na bieżąco podczas zwykłych operacji
    \item czynności po awarii, prowadzące do przywrócenia spójności, atomowości i trwałości
\end{enumerate}

\subsection*{Strategie SZBD}
Brudne strony, czyli zmieniane po ostatnim zapisie na dysku, mogą pozostawać w buforze długo po zatwierdzeniu transakcji. Jeśli nastąpi awaria, to naniesione zmiany są tracone.
Żeby zadbać o poprawny stan bazy można wykonać operacje:
\begin{itemize}
    \item UNDO, wycofanie zmian - zapewnia atomowość
    \item REDO, powtórzenie transakcji - zapewnia trwałość.
\end{itemize}
Podjęte czynności zależą od strategii SZBD: \\
Czy niezatwierdzona transakcja może nadpisać wartość w pamięci trwałej? \\
\begin{itemize}
    \item strategia steal: TAK
    \item strategia no-steal: NIE
\end{itemize}
Czy wszystkie zmiany muszą być zapisane w pamięci trwałej przed zatwierdzeniem transakcji?
\begin{itemize}
    \item strategia force: TAK
    \item strategia no-force: NIE
\end{itemize}

Strategia \textbf{no-steal + force} pozwala uniknąć wykonywania sekwencji UNDO i REDO. Jednak jest mało efektywna i istnieje ryzyko samozakleszczenia transakcji, jeśli braknie miejsca w buforze.
Częściej wykorzystywana jest strategia \textbf{steal + no-force} opakowana przez \textbf{protokół WAL (write-ahead log)}.
Oprócz plików z danymi przechowywany jest dziennik z logami (dziennik), w którym zapisywane są wszystkie zmiany przed wprowadzeniem ich na dysku.
Zawiera on informacje wystarczające odzyskania zawartości bazy za pomocą operacji UNDO i REDO.
Żeby dzienniki nie rozrastały się w nieskończoność, tworzy się punkty kontrolne. W przypadku awarii powtarzane są tylko transakcja zatwierdzona po ostatnim punkcie kontrolnym, a niezatwierdzone są wycofywane.

\subsection{Algorytm ARIES}
Algorytm ARIES (Algorithm for Recovery and Isolation Exploiting Semantics) służy do odtwarzania transakcji, korzystając z dziennika WAL i blokad.
Składa się z trzech faz:
\begin{enumerate}
    \item Analiza: rekonstrukcja Tabeli Brudnych Stron (DPT) i Tabeli Transakcji (TT), wyznaczenie pierwszego wpisu o zmianach tworzących brudną stronę
    \item Faza REDO: powtórzenie wszystkich operacji od momentu powstania brudnej strony, przywrócenie stanu sprzed awarii
    \item Faza UNDO: wycofanie efektów niezatwierdzonych transakcji, dodanie wpisów kompensacyjnych do dziennika
\end{enumerate}

\section{\textcolor{pink}{Twierdzenie Ramseya. Przykłady zastosowań}}
Jednym z zadań SZBD jest zapewnianie tego, żeby zmiany dokonane przez widoczną lub zatwierdzoną transakcję były trwałe, a efekty transakcji wycofanej lub przerwanyj przez awarię nie.

Jest parę typów awarii:
\begin{itemize}
    \item awaria systemu: awaria sprzętu, błędy oprogramowania, uszkodzenie dysku
    \item błędy logiczne (np. naruszenie ograniczeń integralnościowych)
    \item błędy systemowe (np. zakleszczenia)
\end{itemize}
Do poradzenia sobie z pierwszym rodzajem wystarczy replikacja bazy i rozproszenie danych. W przypadku pozostałych dwóch trzeba odtworzyć bazę po awarii.
Ważne jest, żeby zminimalizować czas potrzebny do tego, ponieważ w trakcie odtwarzania bazy serwer musi być wyłączony dla użytkowników.
Służące do tego algorytmy składają się z dwóch części:
\begin{enumerate}
    \item zapisywanie informacji potrzebnych do odtworzenia stanu bazy na bieżąco podczas zwykłych operacji
    \item czynności po awarii, prowadzące do przywrócenia spójności, atomowości i trwałości
\end{enumerate}

\subsection*{Strategie SZBD}
Brudne strony, czyli zmieniane po ostatnim zapisie na dysku, mogą pozostawać w buforze długo po zatwierdzeniu transakcji. Jeśli nastąpi awaria, to naniesione zmiany są tracone.
Żeby zadbać o poprawny stan bazy można wykonać operacje:
\begin{itemize}
    \item UNDO, wycofanie zmian - zapewnia atomowość
    \item REDO, powtórzenie transakcji - zapewnia trwałość.
\end{itemize}
Podjęte czynności zależą od strategii SZBD: \\
Czy niezatwierdzona transakcja może nadpisać wartość w pamięci trwałej? \\
\begin{itemize}
    \item strategia steal: TAK
    \item strategia no-steal: NIE
\end{itemize}
Czy wszystkie zmiany muszą być zapisane w pamięci trwałej przed zatwierdzeniem transakcji?
\begin{itemize}
    \item strategia force: TAK
    \item strategia no-force: NIE
\end{itemize}

Strategia \textbf{no-steal + force} pozwala uniknąć wykonywania sekwencji UNDO i REDO. Jednak jest mało efektywna i istnieje ryzyko samozakleszczenia transakcji, jeśli braknie miejsca w buforze.
Częściej wykorzystywana jest strategia \textbf{steal + no-force} opakowana przez \textbf{protokół WAL (write-ahead log)}.
Oprócz plików z danymi przechowywany jest dziennik z logami (dziennik), w którym zapisywane są wszystkie zmiany przed wprowadzeniem ich na dysku.
Zawiera on informacje wystarczające odzyskania zawartości bazy za pomocą operacji UNDO i REDO.
Żeby dzienniki nie rozrastały się w nieskończoność, tworzy się punkty kontrolne. W przypadku awarii powtarzane są tylko transakcja zatwierdzona po ostatnim punkcie kontrolnym, a niezatwierdzone są wycofywane.

\subsection{Algorytm ARIES}
Algorytm ARIES (Algorithm for Recovery and Isolation Exploiting Semantics) służy do odtwarzania transakcji, korzystając z dziennika WAL i blokad.
Składa się z trzech faz:
\begin{enumerate}
    \item Analiza: rekonstrukcja Tabeli Brudnych Stron (DPT) i Tabeli Transakcji (TT), wyznaczenie pierwszego wpisu o zmianach tworzących brudną stronę
    \item Faza REDO: powtórzenie wszystkich operacji od momentu powstania brudnej strony, przywrócenie stanu sprzed awarii
    \item Faza UNDO: wycofanie efektów niezatwierdzonych transakcji, dodanie wpisów kompensacyjnych do dziennika
\end{enumerate}

\section{\textcolor{pink}{Funkcje tworzące. Wyznaczanie liczb Fibonacciego za pomocą funkcji tworzących}}
Jednym z zadań SZBD jest zapewnianie tego, żeby zmiany dokonane przez widoczną lub zatwierdzoną transakcję były trwałe, a efekty transakcji wycofanej lub przerwanyj przez awarię nie.

Jest parę typów awarii:
\begin{itemize}
    \item awaria systemu: awaria sprzętu, błędy oprogramowania, uszkodzenie dysku
    \item błędy logiczne (np. naruszenie ograniczeń integralnościowych)
    \item błędy systemowe (np. zakleszczenia)
\end{itemize}
Do poradzenia sobie z pierwszym rodzajem wystarczy replikacja bazy i rozproszenie danych. W przypadku pozostałych dwóch trzeba odtworzyć bazę po awarii.
Ważne jest, żeby zminimalizować czas potrzebny do tego, ponieważ w trakcie odtwarzania bazy serwer musi być wyłączony dla użytkowników.
Służące do tego algorytmy składają się z dwóch części:
\begin{enumerate}
    \item zapisywanie informacji potrzebnych do odtworzenia stanu bazy na bieżąco podczas zwykłych operacji
    \item czynności po awarii, prowadzące do przywrócenia spójności, atomowości i trwałości
\end{enumerate}

\subsection*{Strategie SZBD}
Brudne strony, czyli zmieniane po ostatnim zapisie na dysku, mogą pozostawać w buforze długo po zatwierdzeniu transakcji. Jeśli nastąpi awaria, to naniesione zmiany są tracone.
Żeby zadbać o poprawny stan bazy można wykonać operacje:
\begin{itemize}
    \item UNDO, wycofanie zmian - zapewnia atomowość
    \item REDO, powtórzenie transakcji - zapewnia trwałość.
\end{itemize}
Podjęte czynności zależą od strategii SZBD: \\
Czy niezatwierdzona transakcja może nadpisać wartość w pamięci trwałej? \\
\begin{itemize}
    \item strategia steal: TAK
    \item strategia no-steal: NIE
\end{itemize}
Czy wszystkie zmiany muszą być zapisane w pamięci trwałej przed zatwierdzeniem transakcji?
\begin{itemize}
    \item strategia force: TAK
    \item strategia no-force: NIE
\end{itemize}

Strategia \textbf{no-steal + force} pozwala uniknąć wykonywania sekwencji UNDO i REDO. Jednak jest mało efektywna i istnieje ryzyko samozakleszczenia transakcji, jeśli braknie miejsca w buforze.
Częściej wykorzystywana jest strategia \textbf{steal + no-force} opakowana przez \textbf{protokół WAL (write-ahead log)}.
Oprócz plików z danymi przechowywany jest dziennik z logami (dziennik), w którym zapisywane są wszystkie zmiany przed wprowadzeniem ich na dysku.
Zawiera on informacje wystarczające odzyskania zawartości bazy za pomocą operacji UNDO i REDO.
Żeby dzienniki nie rozrastały się w nieskończoność, tworzy się punkty kontrolne. W przypadku awarii powtarzane są tylko transakcja zatwierdzona po ostatnim punkcie kontrolnym, a niezatwierdzone są wycofywane.

\subsection{Algorytm ARIES}
Algorytm ARIES (Algorithm for Recovery and Isolation Exploiting Semantics) służy do odtwarzania transakcji, korzystając z dziennika WAL i blokad.
Składa się z trzech faz:
\begin{enumerate}
    \item Analiza: rekonstrukcja Tabeli Brudnych Stron (DPT) i Tabeli Transakcji (TT), wyznaczenie pierwszego wpisu o zmianach tworzących brudną stronę
    \item Faza REDO: powtórzenie wszystkich operacji od momentu powstania brudnej strony, przywrócenie stanu sprzed awarii
    \item Faza UNDO: wycofanie efektów niezatwierdzonych transakcji, dodanie wpisów kompensacyjnych do dziennika
\end{enumerate}

\section{\textcolor{pink}{Skojarzenia w grafach dwudzielnych. Twierdzenie Halla}}
Jednym z zadań SZBD jest zapewnianie tego, żeby zmiany dokonane przez widoczną lub zatwierdzoną transakcję były trwałe, a efekty transakcji wycofanej lub przerwanyj przez awarię nie.

Jest parę typów awarii:
\begin{itemize}
    \item awaria systemu: awaria sprzętu, błędy oprogramowania, uszkodzenie dysku
    \item błędy logiczne (np. naruszenie ograniczeń integralnościowych)
    \item błędy systemowe (np. zakleszczenia)
\end{itemize}
Do poradzenia sobie z pierwszym rodzajem wystarczy replikacja bazy i rozproszenie danych. W przypadku pozostałych dwóch trzeba odtworzyć bazę po awarii.
Ważne jest, żeby zminimalizować czas potrzebny do tego, ponieważ w trakcie odtwarzania bazy serwer musi być wyłączony dla użytkowników.
Służące do tego algorytmy składają się z dwóch części:
\begin{enumerate}
    \item zapisywanie informacji potrzebnych do odtworzenia stanu bazy na bieżąco podczas zwykłych operacji
    \item czynności po awarii, prowadzące do przywrócenia spójności, atomowości i trwałości
\end{enumerate}

\subsection*{Strategie SZBD}
Brudne strony, czyli zmieniane po ostatnim zapisie na dysku, mogą pozostawać w buforze długo po zatwierdzeniu transakcji. Jeśli nastąpi awaria, to naniesione zmiany są tracone.
Żeby zadbać o poprawny stan bazy można wykonać operacje:
\begin{itemize}
    \item UNDO, wycofanie zmian - zapewnia atomowość
    \item REDO, powtórzenie transakcji - zapewnia trwałość.
\end{itemize}
Podjęte czynności zależą od strategii SZBD: \\
Czy niezatwierdzona transakcja może nadpisać wartość w pamięci trwałej? \\
\begin{itemize}
    \item strategia steal: TAK
    \item strategia no-steal: NIE
\end{itemize}
Czy wszystkie zmiany muszą być zapisane w pamięci trwałej przed zatwierdzeniem transakcji?
\begin{itemize}
    \item strategia force: TAK
    \item strategia no-force: NIE
\end{itemize}

Strategia \textbf{no-steal + force} pozwala uniknąć wykonywania sekwencji UNDO i REDO. Jednak jest mało efektywna i istnieje ryzyko samozakleszczenia transakcji, jeśli braknie miejsca w buforze.
Częściej wykorzystywana jest strategia \textbf{steal + no-force} opakowana przez \textbf{protokół WAL (write-ahead log)}.
Oprócz plików z danymi przechowywany jest dziennik z logami (dziennik), w którym zapisywane są wszystkie zmiany przed wprowadzeniem ich na dysku.
Zawiera on informacje wystarczające odzyskania zawartości bazy za pomocą operacji UNDO i REDO.
Żeby dzienniki nie rozrastały się w nieskończoność, tworzy się punkty kontrolne. W przypadku awarii powtarzane są tylko transakcja zatwierdzona po ostatnim punkcie kontrolnym, a niezatwierdzone są wycofywane.

\subsection{Algorytm ARIES}
Algorytm ARIES (Algorithm for Recovery and Isolation Exploiting Semantics) służy do odtwarzania transakcji, korzystając z dziennika WAL i blokad.
Składa się z trzech faz:
\begin{enumerate}
    \item Analiza: rekonstrukcja Tabeli Brudnych Stron (DPT) i Tabeli Transakcji (TT), wyznaczenie pierwszego wpisu o zmianach tworzących brudną stronę
    \item Faza REDO: powtórzenie wszystkich operacji od momentu powstania brudnej strony, przywrócenie stanu sprzed awarii
    \item Faza UNDO: wycofanie efektów niezatwierdzonych transakcji, dodanie wpisów kompensacyjnych do dziennika
\end{enumerate}

\section{\textcolor{pink}{Kolorowanie grafów, twierdzenia Brooksa}}
\input{chapters/dyskretna/colours/brooks/brooks}

\section{\textcolor{pink}{Liczba chromatyczna a liczba kolorująca grafów}}
Jednym z zadań SZBD jest zapewnianie tego, żeby zmiany dokonane przez widoczną lub zatwierdzoną transakcję były trwałe, a efekty transakcji wycofanej lub przerwanyj przez awarię nie.

Jest parę typów awarii:
\begin{itemize}
    \item awaria systemu: awaria sprzętu, błędy oprogramowania, uszkodzenie dysku
    \item błędy logiczne (np. naruszenie ograniczeń integralnościowych)
    \item błędy systemowe (np. zakleszczenia)
\end{itemize}
Do poradzenia sobie z pierwszym rodzajem wystarczy replikacja bazy i rozproszenie danych. W przypadku pozostałych dwóch trzeba odtworzyć bazę po awarii.
Ważne jest, żeby zminimalizować czas potrzebny do tego, ponieważ w trakcie odtwarzania bazy serwer musi być wyłączony dla użytkowników.
Służące do tego algorytmy składają się z dwóch części:
\begin{enumerate}
    \item zapisywanie informacji potrzebnych do odtworzenia stanu bazy na bieżąco podczas zwykłych operacji
    \item czynności po awarii, prowadzące do przywrócenia spójności, atomowości i trwałości
\end{enumerate}

\subsection*{Strategie SZBD}
Brudne strony, czyli zmieniane po ostatnim zapisie na dysku, mogą pozostawać w buforze długo po zatwierdzeniu transakcji. Jeśli nastąpi awaria, to naniesione zmiany są tracone.
Żeby zadbać o poprawny stan bazy można wykonać operacje:
\begin{itemize}
    \item UNDO, wycofanie zmian - zapewnia atomowość
    \item REDO, powtórzenie transakcji - zapewnia trwałość.
\end{itemize}
Podjęte czynności zależą od strategii SZBD: \\
Czy niezatwierdzona transakcja może nadpisać wartość w pamięci trwałej? \\
\begin{itemize}
    \item strategia steal: TAK
    \item strategia no-steal: NIE
\end{itemize}
Czy wszystkie zmiany muszą być zapisane w pamięci trwałej przed zatwierdzeniem transakcji?
\begin{itemize}
    \item strategia force: TAK
    \item strategia no-force: NIE
\end{itemize}

Strategia \textbf{no-steal + force} pozwala uniknąć wykonywania sekwencji UNDO i REDO. Jednak jest mało efektywna i istnieje ryzyko samozakleszczenia transakcji, jeśli braknie miejsca w buforze.
Częściej wykorzystywana jest strategia \textbf{steal + no-force} opakowana przez \textbf{protokół WAL (write-ahead log)}.
Oprócz plików z danymi przechowywany jest dziennik z logami (dziennik), w którym zapisywane są wszystkie zmiany przed wprowadzeniem ich na dysku.
Zawiera on informacje wystarczające odzyskania zawartości bazy za pomocą operacji UNDO i REDO.
Żeby dzienniki nie rozrastały się w nieskończoność, tworzy się punkty kontrolne. W przypadku awarii powtarzane są tylko transakcja zatwierdzona po ostatnim punkcie kontrolnym, a niezatwierdzone są wycofywane.

\subsection{Algorytm ARIES}
Algorytm ARIES (Algorithm for Recovery and Isolation Exploiting Semantics) służy do odtwarzania transakcji, korzystając z dziennika WAL i blokad.
Składa się z trzech faz:
\begin{enumerate}
    \item Analiza: rekonstrukcja Tabeli Brudnych Stron (DPT) i Tabeli Transakcji (TT), wyznaczenie pierwszego wpisu o zmianach tworzących brudną stronę
    \item Faza REDO: powtórzenie wszystkich operacji od momentu powstania brudnej strony, przywrócenie stanu sprzed awarii
    \item Faza UNDO: wycofanie efektów niezatwierdzonych transakcji, dodanie wpisów kompensacyjnych do dziennika
\end{enumerate}

\section{\textcolor{pink}{Kolorowania krawędziowe grafów. Twierdzenie Vizinga}}
Jednym z zadań SZBD jest zapewnianie tego, żeby zmiany dokonane przez widoczną lub zatwierdzoną transakcję były trwałe, a efekty transakcji wycofanej lub przerwanyj przez awarię nie.

Jest parę typów awarii:
\begin{itemize}
    \item awaria systemu: awaria sprzętu, błędy oprogramowania, uszkodzenie dysku
    \item błędy logiczne (np. naruszenie ograniczeń integralnościowych)
    \item błędy systemowe (np. zakleszczenia)
\end{itemize}
Do poradzenia sobie z pierwszym rodzajem wystarczy replikacja bazy i rozproszenie danych. W przypadku pozostałych dwóch trzeba odtworzyć bazę po awarii.
Ważne jest, żeby zminimalizować czas potrzebny do tego, ponieważ w trakcie odtwarzania bazy serwer musi być wyłączony dla użytkowników.
Służące do tego algorytmy składają się z dwóch części:
\begin{enumerate}
    \item zapisywanie informacji potrzebnych do odtworzenia stanu bazy na bieżąco podczas zwykłych operacji
    \item czynności po awarii, prowadzące do przywrócenia spójności, atomowości i trwałości
\end{enumerate}

\subsection*{Strategie SZBD}
Brudne strony, czyli zmieniane po ostatnim zapisie na dysku, mogą pozostawać w buforze długo po zatwierdzeniu transakcji. Jeśli nastąpi awaria, to naniesione zmiany są tracone.
Żeby zadbać o poprawny stan bazy można wykonać operacje:
\begin{itemize}
    \item UNDO, wycofanie zmian - zapewnia atomowość
    \item REDO, powtórzenie transakcji - zapewnia trwałość.
\end{itemize}
Podjęte czynności zależą od strategii SZBD: \\
Czy niezatwierdzona transakcja może nadpisać wartość w pamięci trwałej? \\
\begin{itemize}
    \item strategia steal: TAK
    \item strategia no-steal: NIE
\end{itemize}
Czy wszystkie zmiany muszą być zapisane w pamięci trwałej przed zatwierdzeniem transakcji?
\begin{itemize}
    \item strategia force: TAK
    \item strategia no-force: NIE
\end{itemize}

Strategia \textbf{no-steal + force} pozwala uniknąć wykonywania sekwencji UNDO i REDO. Jednak jest mało efektywna i istnieje ryzyko samozakleszczenia transakcji, jeśli braknie miejsca w buforze.
Częściej wykorzystywana jest strategia \textbf{steal + no-force} opakowana przez \textbf{protokół WAL (write-ahead log)}.
Oprócz plików z danymi przechowywany jest dziennik z logami (dziennik), w którym zapisywane są wszystkie zmiany przed wprowadzeniem ich na dysku.
Zawiera on informacje wystarczające odzyskania zawartości bazy za pomocą operacji UNDO i REDO.
Żeby dzienniki nie rozrastały się w nieskończoność, tworzy się punkty kontrolne. W przypadku awarii powtarzane są tylko transakcja zatwierdzona po ostatnim punkcie kontrolnym, a niezatwierdzone są wycofywane.

\subsection{Algorytm ARIES}
Algorytm ARIES (Algorithm for Recovery and Isolation Exploiting Semantics) służy do odtwarzania transakcji, korzystając z dziennika WAL i blokad.
Składa się z trzech faz:
\begin{enumerate}
    \item Analiza: rekonstrukcja Tabeli Brudnych Stron (DPT) i Tabeli Transakcji (TT), wyznaczenie pierwszego wpisu o zmianach tworzących brudną stronę
    \item Faza REDO: powtórzenie wszystkich operacji od momentu powstania brudnej strony, przywrócenie stanu sprzed awarii
    \item Faza UNDO: wycofanie efektów niezatwierdzonych transakcji, dodanie wpisów kompensacyjnych do dziennika
\end{enumerate}

\section{\textcolor{pink}{Przepływy w sieciach. Twierdzenie o maksymalnym przepływie i minimalnym}
przekroju}
Jednym z zadań SZBD jest zapewnianie tego, żeby zmiany dokonane przez widoczną lub zatwierdzoną transakcję były trwałe, a efekty transakcji wycofanej lub przerwanyj przez awarię nie.

Jest parę typów awarii:
\begin{itemize}
    \item awaria systemu: awaria sprzętu, błędy oprogramowania, uszkodzenie dysku
    \item błędy logiczne (np. naruszenie ograniczeń integralnościowych)
    \item błędy systemowe (np. zakleszczenia)
\end{itemize}
Do poradzenia sobie z pierwszym rodzajem wystarczy replikacja bazy i rozproszenie danych. W przypadku pozostałych dwóch trzeba odtworzyć bazę po awarii.
Ważne jest, żeby zminimalizować czas potrzebny do tego, ponieważ w trakcie odtwarzania bazy serwer musi być wyłączony dla użytkowników.
Służące do tego algorytmy składają się z dwóch części:
\begin{enumerate}
    \item zapisywanie informacji potrzebnych do odtworzenia stanu bazy na bieżąco podczas zwykłych operacji
    \item czynności po awarii, prowadzące do przywrócenia spójności, atomowości i trwałości
\end{enumerate}

\subsection*{Strategie SZBD}
Brudne strony, czyli zmieniane po ostatnim zapisie na dysku, mogą pozostawać w buforze długo po zatwierdzeniu transakcji. Jeśli nastąpi awaria, to naniesione zmiany są tracone.
Żeby zadbać o poprawny stan bazy można wykonać operacje:
\begin{itemize}
    \item UNDO, wycofanie zmian - zapewnia atomowość
    \item REDO, powtórzenie transakcji - zapewnia trwałość.
\end{itemize}
Podjęte czynności zależą od strategii SZBD: \\
Czy niezatwierdzona transakcja może nadpisać wartość w pamięci trwałej? \\
\begin{itemize}
    \item strategia steal: TAK
    \item strategia no-steal: NIE
\end{itemize}
Czy wszystkie zmiany muszą być zapisane w pamięci trwałej przed zatwierdzeniem transakcji?
\begin{itemize}
    \item strategia force: TAK
    \item strategia no-force: NIE
\end{itemize}

Strategia \textbf{no-steal + force} pozwala uniknąć wykonywania sekwencji UNDO i REDO. Jednak jest mało efektywna i istnieje ryzyko samozakleszczenia transakcji, jeśli braknie miejsca w buforze.
Częściej wykorzystywana jest strategia \textbf{steal + no-force} opakowana przez \textbf{protokół WAL (write-ahead log)}.
Oprócz plików z danymi przechowywany jest dziennik z logami (dziennik), w którym zapisywane są wszystkie zmiany przed wprowadzeniem ich na dysku.
Zawiera on informacje wystarczające odzyskania zawartości bazy za pomocą operacji UNDO i REDO.
Żeby dzienniki nie rozrastały się w nieskończoność, tworzy się punkty kontrolne. W przypadku awarii powtarzane są tylko transakcja zatwierdzona po ostatnim punkcie kontrolnym, a niezatwierdzone są wycofywane.

\subsection{Algorytm ARIES}
Algorytm ARIES (Algorithm for Recovery and Isolation Exploiting Semantics) służy do odtwarzania transakcji, korzystając z dziennika WAL i blokad.
Składa się z trzech faz:
\begin{enumerate}
    \item Analiza: rekonstrukcja Tabeli Brudnych Stron (DPT) i Tabeli Transakcji (TT), wyznaczenie pierwszego wpisu o zmianach tworzących brudną stronę
    \item Faza REDO: powtórzenie wszystkich operacji od momentu powstania brudnej strony, przywrócenie stanu sprzed awarii
    \item Faza UNDO: wycofanie efektów niezatwierdzonych transakcji, dodanie wpisów kompensacyjnych do dziennika
\end{enumerate}