\textbf{Spooling} (simultaneous peripheral operations on-line) - polega na nie używaniu danego zasobu bezpośrednio, tylko na przekazywaniu zadań dla tego zasobu do specjalnego procesu (\textbf{spooler}), który kolejkuje i wykonuje te zadania. Ten mechanizm polega na tym, że urządzenie nie udostępnia biblioteki procedur, za pomocą których można się z nim komunikować, a zamiast tego zlecenia obsługiwane są przez specjalny proces działający w tle (\textbf{daemon}) oraz dedykowany folder (niekoniecznie w systemie pilków) - \textbf{spooling directory}. Procesy użytkownika umieszczają dane w tym folderze (w kolejności FIFO), a daemon decyduje, w jaki sposób zlecić ich wykonanie urządzeniu.

Typowym przykładem jest drukarka, która działa zbyt wolno, żeby dany program używał jej bezpośrednio. Dokumenty są więc przekazywane do spoolera, który po kolei je drukuje. Dzięki temu programy mogą wykonywać inne czynności w trakcie drukowania.

Należy używać spoolingu w przypadku powolnych zasobów wyjściowych. Na pewno nie nadaje się do obsługi zasobów wejściowych i interaktywnych, w przypadku których programy muszą czekać na dane wejściowe.