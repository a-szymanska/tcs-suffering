W przeciwieństwie do kernela, zwykłe procesy nie używają bezpośredniego, fizycznego adresowania pamięci, a korzystają z witrualnej przestrzeni adresowej, za pomocą której adresy w instrukcjach są tłumaczone na adresy fizyczne przez podjednostkę CPU zwaną MMU (Memory Management Unit). Każdy proces ma swoją własną odizolowaną przestrzeń adresową mapowaną do przestrzeni fizycznej.

\subsection{Segmentacja}
Segmentacja polega na podziale wirtualnej przestrzeni adresowej na segmenty ze względu na zastosowania, np. segmenty na kod, stos lub dane. Programy używające segmentacji muszą do każdego adresu dołączyć informację, do którego segmentu należy. Mapowanie segmentu to (tak oględnie): offset pamięci fizycznej + adres.

Segmentacja jest obecnie uznawana ze przestarzały mechanizm zastąpiony przez stronicowanie - w architekturze \texttt{amd64} segmentacja nie jest w ogóle wspierana. Prawie wszystkie rozwiązania, które zapewnia segmentacja można dość łatwo (przynajmniej częściowo) zapewnić używając stronicowania. Na przykład, jeśli potrzebujemy kilku rozszerzalnych liniowych przestrzeni adresowych, alokujemy bloki pamięci dla tych przestrzeni bardzo daleko od siebie w 64-bitowej wirtualnej przestrzeni adresowej.

\subsection{Stronicowanie}
Stronicowanie jest prostszym mechanizmem, w którym nie ma podziału na segmenty i jest tylko jedna wirtualna przestrzeń adresowa. Pamięć fizyczna jest dzielona na strony, zazwyczaj o rozmiarze kilku kB, które są mapowane do niektórych stron w pamięci wirtualnej. Wobec tego danemu procesowi wydaje się, że jest więcej dostępnej pamięci niż fizycznie obecnej. Programy nie są świadome stronicowania i nie muszą dołączać do adresów dodatkowych informacji.

\subsection{Segmentacja + Stronicowanie}

Można również połączyć mechanizmy segmentacji i stronicowania -- segmentacja nie tłumaczy wtedy bezpośrednio na adres fizyczny, tylko na adres, który jest w drugiej kolejności tłumaczony przez stronicowanie.

Ponieważ wirtualna przestrzeń adresowa jest bardzo duża, mapowanie stron jest zapisywane za pomocą tabel na kilku poziomach. Wirtualny adres jest dzielony na części, z których każda określa numer wpisu w tabeli na odpowiednim poziomie lub numer bajtu w obrębie strony. Dla każdego adresu wirtualnego MMU odczytuje po kolei wpisy w odpowiednich tabelach i w ten sposób oblicza adres fizyczny. Żeby przyspieszyć tłumaczenie, MMU może cache'ować numery stron do TLB (Translation Lookaside Buffer).

Każdy proces ma swoją własną strukturę tabel stron, więc przy zmianie kontekstu zmieniany jest również specjalny rejestr przechowujący adres tej struktury. Zmiana rejestru wymaga też unieważnienia TLB, co wpływa niekorzystnie na wydajność.

W Linuxie strony są mapowane na żądanie w następujący sposób:
\begin{itemize}
    \item Stronicowanie jest używane do zaimplementowania pamięci wirtualnej. Każda strona pamięci może być też zapisana na dysku w pliku swap lub partycji swap.
    \item Współdzielenie pamięci pomiędzy procesami jest zaimplementowane z wykorzystaniem stronicowania. Kod bibliotek współdzielonych jest ładowany do pamięci raz, a potem współdzielony pomiędzy wszystkimi procesami używając stronicowania
    \item Stronicowanie w Linuxie jest używane do wielu optymalizacji. Na przykład, gdy proces jest forkowany, to tylko jego tablica stron jest kopiowana, a nie cała pamięć.
    \item Gdy proces próbuje uzyskać dostęp do adresu, który nie jest zmapowany w tabeli stron, to procesor generuje błąd stronicowania (page fault), który powoduje przekazanie sterowania kernelowi. Kernel sprawdza, czy adres, do którego proces chciał uzyskać dostęp, jest poprawny.
    \item Jeśli adres nie jest poprawny, to kernel wysyła do procesu sygnał o błędzie, który domyślnie zatrzymuje proces.
    \item Jeśli adres jest poprawny, to kernel uzupełnia odpowiednie informacje w tabeli stron i oddaje sterowanie procesowi w taki sposób, żeby ponownie wykonał tę instrukcję, która wcześniej spowodowała błąd stronicowania.
\end{itemize}