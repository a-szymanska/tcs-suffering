\chapter{Systemy operacyjne}
\label{sysopy}
\section{\textcolor{pink}{Opisz mechanizmy komunikacji międzyprocesowej standardu POSIX.}}
Jednym z zadań SZBD jest zapewnianie tego, żeby zmiany dokonane przez widoczną lub zatwierdzoną transakcję były trwałe, a efekty transakcji wycofanej lub przerwanyj przez awarię nie.

Jest parę typów awarii:
\begin{itemize}
    \item awaria systemu: awaria sprzętu, błędy oprogramowania, uszkodzenie dysku
    \item błędy logiczne (np. naruszenie ograniczeń integralnościowych)
    \item błędy systemowe (np. zakleszczenia)
\end{itemize}
Do poradzenia sobie z pierwszym rodzajem wystarczy replikacja bazy i rozproszenie danych. W przypadku pozostałych dwóch trzeba odtworzyć bazę po awarii.
Ważne jest, żeby zminimalizować czas potrzebny do tego, ponieważ w trakcie odtwarzania bazy serwer musi być wyłączony dla użytkowników.
Służące do tego algorytmy składają się z dwóch części:
\begin{enumerate}
    \item zapisywanie informacji potrzebnych do odtworzenia stanu bazy na bieżąco podczas zwykłych operacji
    \item czynności po awarii, prowadzące do przywrócenia spójności, atomowości i trwałości
\end{enumerate}

\subsection*{Strategie SZBD}
Brudne strony, czyli zmieniane po ostatnim zapisie na dysku, mogą pozostawać w buforze długo po zatwierdzeniu transakcji. Jeśli nastąpi awaria, to naniesione zmiany są tracone.
Żeby zadbać o poprawny stan bazy można wykonać operacje:
\begin{itemize}
    \item UNDO, wycofanie zmian - zapewnia atomowość
    \item REDO, powtórzenie transakcji - zapewnia trwałość.
\end{itemize}
Podjęte czynności zależą od strategii SZBD: \\
Czy niezatwierdzona transakcja może nadpisać wartość w pamięci trwałej? \\
\begin{itemize}
    \item strategia steal: TAK
    \item strategia no-steal: NIE
\end{itemize}
Czy wszystkie zmiany muszą być zapisane w pamięci trwałej przed zatwierdzeniem transakcji?
\begin{itemize}
    \item strategia force: TAK
    \item strategia no-force: NIE
\end{itemize}

Strategia \textbf{no-steal + force} pozwala uniknąć wykonywania sekwencji UNDO i REDO. Jednak jest mało efektywna i istnieje ryzyko samozakleszczenia transakcji, jeśli braknie miejsca w buforze.
Częściej wykorzystywana jest strategia \textbf{steal + no-force} opakowana przez \textbf{protokół WAL (write-ahead log)}.
Oprócz plików z danymi przechowywany jest dziennik z logami (dziennik), w którym zapisywane są wszystkie zmiany przed wprowadzeniem ich na dysku.
Zawiera on informacje wystarczające odzyskania zawartości bazy za pomocą operacji UNDO i REDO.
Żeby dzienniki nie rozrastały się w nieskończoność, tworzy się punkty kontrolne. W przypadku awarii powtarzane są tylko transakcja zatwierdzona po ostatnim punkcie kontrolnym, a niezatwierdzone są wycofywane.

\subsection{Algorytm ARIES}
Algorytm ARIES (Algorithm for Recovery and Isolation Exploiting Semantics) służy do odtwarzania transakcji, korzystając z dziennika WAL i blokad.
Składa się z trzech faz:
\begin{enumerate}
    \item Analiza: rekonstrukcja Tabeli Brudnych Stron (DPT) i Tabeli Transakcji (TT), wyznaczenie pierwszego wpisu o zmianach tworzących brudną stronę
    \item Faza REDO: powtórzenie wszystkich operacji od momentu powstania brudnej strony, przywrócenie stanu sprzed awarii
    \item Faza UNDO: wycofanie efektów niezatwierdzonych transakcji, dodanie wpisów kompensacyjnych do dziennika
\end{enumerate}

\section{\textcolor{pink}{Porównaj mechanizmy szeregowania zadań na przykładzie serwera przetwarzającego zadania w trybie wsadowym, oraz systemu interaktywnego.}}
Jednym z zadań SZBD jest zapewnianie tego, żeby zmiany dokonane przez widoczną lub zatwierdzoną transakcję były trwałe, a efekty transakcji wycofanej lub przerwanyj przez awarię nie.

Jest parę typów awarii:
\begin{itemize}
    \item awaria systemu: awaria sprzętu, błędy oprogramowania, uszkodzenie dysku
    \item błędy logiczne (np. naruszenie ograniczeń integralnościowych)
    \item błędy systemowe (np. zakleszczenia)
\end{itemize}
Do poradzenia sobie z pierwszym rodzajem wystarczy replikacja bazy i rozproszenie danych. W przypadku pozostałych dwóch trzeba odtworzyć bazę po awarii.
Ważne jest, żeby zminimalizować czas potrzebny do tego, ponieważ w trakcie odtwarzania bazy serwer musi być wyłączony dla użytkowników.
Służące do tego algorytmy składają się z dwóch części:
\begin{enumerate}
    \item zapisywanie informacji potrzebnych do odtworzenia stanu bazy na bieżąco podczas zwykłych operacji
    \item czynności po awarii, prowadzące do przywrócenia spójności, atomowości i trwałości
\end{enumerate}

\subsection*{Strategie SZBD}
Brudne strony, czyli zmieniane po ostatnim zapisie na dysku, mogą pozostawać w buforze długo po zatwierdzeniu transakcji. Jeśli nastąpi awaria, to naniesione zmiany są tracone.
Żeby zadbać o poprawny stan bazy można wykonać operacje:
\begin{itemize}
    \item UNDO, wycofanie zmian - zapewnia atomowość
    \item REDO, powtórzenie transakcji - zapewnia trwałość.
\end{itemize}
Podjęte czynności zależą od strategii SZBD: \\
Czy niezatwierdzona transakcja może nadpisać wartość w pamięci trwałej? \\
\begin{itemize}
    \item strategia steal: TAK
    \item strategia no-steal: NIE
\end{itemize}
Czy wszystkie zmiany muszą być zapisane w pamięci trwałej przed zatwierdzeniem transakcji?
\begin{itemize}
    \item strategia force: TAK
    \item strategia no-force: NIE
\end{itemize}

Strategia \textbf{no-steal + force} pozwala uniknąć wykonywania sekwencji UNDO i REDO. Jednak jest mało efektywna i istnieje ryzyko samozakleszczenia transakcji, jeśli braknie miejsca w buforze.
Częściej wykorzystywana jest strategia \textbf{steal + no-force} opakowana przez \textbf{protokół WAL (write-ahead log)}.
Oprócz plików z danymi przechowywany jest dziennik z logami (dziennik), w którym zapisywane są wszystkie zmiany przed wprowadzeniem ich na dysku.
Zawiera on informacje wystarczające odzyskania zawartości bazy za pomocą operacji UNDO i REDO.
Żeby dzienniki nie rozrastały się w nieskończoność, tworzy się punkty kontrolne. W przypadku awarii powtarzane są tylko transakcja zatwierdzona po ostatnim punkcie kontrolnym, a niezatwierdzone są wycofywane.

\subsection{Algorytm ARIES}
Algorytm ARIES (Algorithm for Recovery and Isolation Exploiting Semantics) służy do odtwarzania transakcji, korzystając z dziennika WAL i blokad.
Składa się z trzech faz:
\begin{enumerate}
    \item Analiza: rekonstrukcja Tabeli Brudnych Stron (DPT) i Tabeli Transakcji (TT), wyznaczenie pierwszego wpisu o zmianach tworzących brudną stronę
    \item Faza REDO: powtórzenie wszystkich operacji od momentu powstania brudnej strony, przywrócenie stanu sprzed awarii
    \item Faza UNDO: wycofanie efektów niezatwierdzonych transakcji, dodanie wpisów kompensacyjnych do dziennika
\end{enumerate}

\section{\textcolor{pink}{Wyjaśnij pojęcie deadlock. Opisz metody wykrywania i zapobiegania powstawaniu deadlocku w kontekście współdzielonych zasobów.}}
Jednym z zadań SZBD jest zapewnianie tego, żeby zmiany dokonane przez widoczną lub zatwierdzoną transakcję były trwałe, a efekty transakcji wycofanej lub przerwanyj przez awarię nie.

Jest parę typów awarii:
\begin{itemize}
    \item awaria systemu: awaria sprzętu, błędy oprogramowania, uszkodzenie dysku
    \item błędy logiczne (np. naruszenie ograniczeń integralnościowych)
    \item błędy systemowe (np. zakleszczenia)
\end{itemize}
Do poradzenia sobie z pierwszym rodzajem wystarczy replikacja bazy i rozproszenie danych. W przypadku pozostałych dwóch trzeba odtworzyć bazę po awarii.
Ważne jest, żeby zminimalizować czas potrzebny do tego, ponieważ w trakcie odtwarzania bazy serwer musi być wyłączony dla użytkowników.
Służące do tego algorytmy składają się z dwóch części:
\begin{enumerate}
    \item zapisywanie informacji potrzebnych do odtworzenia stanu bazy na bieżąco podczas zwykłych operacji
    \item czynności po awarii, prowadzące do przywrócenia spójności, atomowości i trwałości
\end{enumerate}

\subsection*{Strategie SZBD}
Brudne strony, czyli zmieniane po ostatnim zapisie na dysku, mogą pozostawać w buforze długo po zatwierdzeniu transakcji. Jeśli nastąpi awaria, to naniesione zmiany są tracone.
Żeby zadbać o poprawny stan bazy można wykonać operacje:
\begin{itemize}
    \item UNDO, wycofanie zmian - zapewnia atomowość
    \item REDO, powtórzenie transakcji - zapewnia trwałość.
\end{itemize}
Podjęte czynności zależą od strategii SZBD: \\
Czy niezatwierdzona transakcja może nadpisać wartość w pamięci trwałej? \\
\begin{itemize}
    \item strategia steal: TAK
    \item strategia no-steal: NIE
\end{itemize}
Czy wszystkie zmiany muszą być zapisane w pamięci trwałej przed zatwierdzeniem transakcji?
\begin{itemize}
    \item strategia force: TAK
    \item strategia no-force: NIE
\end{itemize}

Strategia \textbf{no-steal + force} pozwala uniknąć wykonywania sekwencji UNDO i REDO. Jednak jest mało efektywna i istnieje ryzyko samozakleszczenia transakcji, jeśli braknie miejsca w buforze.
Częściej wykorzystywana jest strategia \textbf{steal + no-force} opakowana przez \textbf{protokół WAL (write-ahead log)}.
Oprócz plików z danymi przechowywany jest dziennik z logami (dziennik), w którym zapisywane są wszystkie zmiany przed wprowadzeniem ich na dysku.
Zawiera on informacje wystarczające odzyskania zawartości bazy za pomocą operacji UNDO i REDO.
Żeby dzienniki nie rozrastały się w nieskończoność, tworzy się punkty kontrolne. W przypadku awarii powtarzane są tylko transakcja zatwierdzona po ostatnim punkcie kontrolnym, a niezatwierdzone są wycofywane.

\subsection{Algorytm ARIES}
Algorytm ARIES (Algorithm for Recovery and Isolation Exploiting Semantics) służy do odtwarzania transakcji, korzystając z dziennika WAL i blokad.
Składa się z trzech faz:
\begin{enumerate}
    \item Analiza: rekonstrukcja Tabeli Brudnych Stron (DPT) i Tabeli Transakcji (TT), wyznaczenie pierwszego wpisu o zmianach tworzących brudną stronę
    \item Faza REDO: powtórzenie wszystkich operacji od momentu powstania brudnej strony, przywrócenie stanu sprzed awarii
    \item Faza UNDO: wycofanie efektów niezatwierdzonych transakcji, dodanie wpisów kompensacyjnych do dziennika
\end{enumerate}

\section{\textcolor{pink}{Wyjaśnij mechanizm spooling, podaj przykłady kiedy należy oraz kiedy nie da się używać spoolingu jako mechanizmu racjonalizującego dostęp do zasobu.}}
Jednym z zadań SZBD jest zapewnianie tego, żeby zmiany dokonane przez widoczną lub zatwierdzoną transakcję były trwałe, a efekty transakcji wycofanej lub przerwanyj przez awarię nie.

Jest parę typów awarii:
\begin{itemize}
    \item awaria systemu: awaria sprzętu, błędy oprogramowania, uszkodzenie dysku
    \item błędy logiczne (np. naruszenie ograniczeń integralnościowych)
    \item błędy systemowe (np. zakleszczenia)
\end{itemize}
Do poradzenia sobie z pierwszym rodzajem wystarczy replikacja bazy i rozproszenie danych. W przypadku pozostałych dwóch trzeba odtworzyć bazę po awarii.
Ważne jest, żeby zminimalizować czas potrzebny do tego, ponieważ w trakcie odtwarzania bazy serwer musi być wyłączony dla użytkowników.
Służące do tego algorytmy składają się z dwóch części:
\begin{enumerate}
    \item zapisywanie informacji potrzebnych do odtworzenia stanu bazy na bieżąco podczas zwykłych operacji
    \item czynności po awarii, prowadzące do przywrócenia spójności, atomowości i trwałości
\end{enumerate}

\subsection*{Strategie SZBD}
Brudne strony, czyli zmieniane po ostatnim zapisie na dysku, mogą pozostawać w buforze długo po zatwierdzeniu transakcji. Jeśli nastąpi awaria, to naniesione zmiany są tracone.
Żeby zadbać o poprawny stan bazy można wykonać operacje:
\begin{itemize}
    \item UNDO, wycofanie zmian - zapewnia atomowość
    \item REDO, powtórzenie transakcji - zapewnia trwałość.
\end{itemize}
Podjęte czynności zależą od strategii SZBD: \\
Czy niezatwierdzona transakcja może nadpisać wartość w pamięci trwałej? \\
\begin{itemize}
    \item strategia steal: TAK
    \item strategia no-steal: NIE
\end{itemize}
Czy wszystkie zmiany muszą być zapisane w pamięci trwałej przed zatwierdzeniem transakcji?
\begin{itemize}
    \item strategia force: TAK
    \item strategia no-force: NIE
\end{itemize}

Strategia \textbf{no-steal + force} pozwala uniknąć wykonywania sekwencji UNDO i REDO. Jednak jest mało efektywna i istnieje ryzyko samozakleszczenia transakcji, jeśli braknie miejsca w buforze.
Częściej wykorzystywana jest strategia \textbf{steal + no-force} opakowana przez \textbf{protokół WAL (write-ahead log)}.
Oprócz plików z danymi przechowywany jest dziennik z logami (dziennik), w którym zapisywane są wszystkie zmiany przed wprowadzeniem ich na dysku.
Zawiera on informacje wystarczające odzyskania zawartości bazy za pomocą operacji UNDO i REDO.
Żeby dzienniki nie rozrastały się w nieskończoność, tworzy się punkty kontrolne. W przypadku awarii powtarzane są tylko transakcja zatwierdzona po ostatnim punkcie kontrolnym, a niezatwierdzone są wycofywane.

\subsection{Algorytm ARIES}
Algorytm ARIES (Algorithm for Recovery and Isolation Exploiting Semantics) służy do odtwarzania transakcji, korzystając z dziennika WAL i blokad.
Składa się z trzech faz:
\begin{enumerate}
    \item Analiza: rekonstrukcja Tabeli Brudnych Stron (DPT) i Tabeli Transakcji (TT), wyznaczenie pierwszego wpisu o zmianach tworzących brudną stronę
    \item Faza REDO: powtórzenie wszystkich operacji od momentu powstania brudnej strony, przywrócenie stanu sprzed awarii
    \item Faza UNDO: wycofanie efektów niezatwierdzonych transakcji, dodanie wpisów kompensacyjnych do dziennika
\end{enumerate}

\section{\textcolor{pink}{Segmentacja i stronicowanie - porównaj mechanizmy. Opisz jak te mechanizmy są wykorzystywane na przykładzie wybranego systemu operacyjnego.}}
Jednym z zadań SZBD jest zapewnianie tego, żeby zmiany dokonane przez widoczną lub zatwierdzoną transakcję były trwałe, a efekty transakcji wycofanej lub przerwanyj przez awarię nie.

Jest parę typów awarii:
\begin{itemize}
    \item awaria systemu: awaria sprzętu, błędy oprogramowania, uszkodzenie dysku
    \item błędy logiczne (np. naruszenie ograniczeń integralnościowych)
    \item błędy systemowe (np. zakleszczenia)
\end{itemize}
Do poradzenia sobie z pierwszym rodzajem wystarczy replikacja bazy i rozproszenie danych. W przypadku pozostałych dwóch trzeba odtworzyć bazę po awarii.
Ważne jest, żeby zminimalizować czas potrzebny do tego, ponieważ w trakcie odtwarzania bazy serwer musi być wyłączony dla użytkowników.
Służące do tego algorytmy składają się z dwóch części:
\begin{enumerate}
    \item zapisywanie informacji potrzebnych do odtworzenia stanu bazy na bieżąco podczas zwykłych operacji
    \item czynności po awarii, prowadzące do przywrócenia spójności, atomowości i trwałości
\end{enumerate}

\subsection*{Strategie SZBD}
Brudne strony, czyli zmieniane po ostatnim zapisie na dysku, mogą pozostawać w buforze długo po zatwierdzeniu transakcji. Jeśli nastąpi awaria, to naniesione zmiany są tracone.
Żeby zadbać o poprawny stan bazy można wykonać operacje:
\begin{itemize}
    \item UNDO, wycofanie zmian - zapewnia atomowość
    \item REDO, powtórzenie transakcji - zapewnia trwałość.
\end{itemize}
Podjęte czynności zależą od strategii SZBD: \\
Czy niezatwierdzona transakcja może nadpisać wartość w pamięci trwałej? \\
\begin{itemize}
    \item strategia steal: TAK
    \item strategia no-steal: NIE
\end{itemize}
Czy wszystkie zmiany muszą być zapisane w pamięci trwałej przed zatwierdzeniem transakcji?
\begin{itemize}
    \item strategia force: TAK
    \item strategia no-force: NIE
\end{itemize}

Strategia \textbf{no-steal + force} pozwala uniknąć wykonywania sekwencji UNDO i REDO. Jednak jest mało efektywna i istnieje ryzyko samozakleszczenia transakcji, jeśli braknie miejsca w buforze.
Częściej wykorzystywana jest strategia \textbf{steal + no-force} opakowana przez \textbf{protokół WAL (write-ahead log)}.
Oprócz plików z danymi przechowywany jest dziennik z logami (dziennik), w którym zapisywane są wszystkie zmiany przed wprowadzeniem ich na dysku.
Zawiera on informacje wystarczające odzyskania zawartości bazy za pomocą operacji UNDO i REDO.
Żeby dzienniki nie rozrastały się w nieskończoność, tworzy się punkty kontrolne. W przypadku awarii powtarzane są tylko transakcja zatwierdzona po ostatnim punkcie kontrolnym, a niezatwierdzone są wycofywane.

\subsection{Algorytm ARIES}
Algorytm ARIES (Algorithm for Recovery and Isolation Exploiting Semantics) służy do odtwarzania transakcji, korzystając z dziennika WAL i blokad.
Składa się z trzech faz:
\begin{enumerate}
    \item Analiza: rekonstrukcja Tabeli Brudnych Stron (DPT) i Tabeli Transakcji (TT), wyznaczenie pierwszego wpisu o zmianach tworzących brudną stronę
    \item Faza REDO: powtórzenie wszystkich operacji od momentu powstania brudnej strony, przywrócenie stanu sprzed awarii
    \item Faza UNDO: wycofanie efektów niezatwierdzonych transakcji, dodanie wpisów kompensacyjnych do dziennika
\end{enumerate}

\section{\textcolor{pink}{Porównaj monolityczną architekturę systemu operacyjnego z architekturą opartą na mikro jądrze.}}
Jednym z zadań SZBD jest zapewnianie tego, żeby zmiany dokonane przez widoczną lub zatwierdzoną transakcję były trwałe, a efekty transakcji wycofanej lub przerwanyj przez awarię nie.

Jest parę typów awarii:
\begin{itemize}
    \item awaria systemu: awaria sprzętu, błędy oprogramowania, uszkodzenie dysku
    \item błędy logiczne (np. naruszenie ograniczeń integralnościowych)
    \item błędy systemowe (np. zakleszczenia)
\end{itemize}
Do poradzenia sobie z pierwszym rodzajem wystarczy replikacja bazy i rozproszenie danych. W przypadku pozostałych dwóch trzeba odtworzyć bazę po awarii.
Ważne jest, żeby zminimalizować czas potrzebny do tego, ponieważ w trakcie odtwarzania bazy serwer musi być wyłączony dla użytkowników.
Służące do tego algorytmy składają się z dwóch części:
\begin{enumerate}
    \item zapisywanie informacji potrzebnych do odtworzenia stanu bazy na bieżąco podczas zwykłych operacji
    \item czynności po awarii, prowadzące do przywrócenia spójności, atomowości i trwałości
\end{enumerate}

\subsection*{Strategie SZBD}
Brudne strony, czyli zmieniane po ostatnim zapisie na dysku, mogą pozostawać w buforze długo po zatwierdzeniu transakcji. Jeśli nastąpi awaria, to naniesione zmiany są tracone.
Żeby zadbać o poprawny stan bazy można wykonać operacje:
\begin{itemize}
    \item UNDO, wycofanie zmian - zapewnia atomowość
    \item REDO, powtórzenie transakcji - zapewnia trwałość.
\end{itemize}
Podjęte czynności zależą od strategii SZBD: \\
Czy niezatwierdzona transakcja może nadpisać wartość w pamięci trwałej? \\
\begin{itemize}
    \item strategia steal: TAK
    \item strategia no-steal: NIE
\end{itemize}
Czy wszystkie zmiany muszą być zapisane w pamięci trwałej przed zatwierdzeniem transakcji?
\begin{itemize}
    \item strategia force: TAK
    \item strategia no-force: NIE
\end{itemize}

Strategia \textbf{no-steal + force} pozwala uniknąć wykonywania sekwencji UNDO i REDO. Jednak jest mało efektywna i istnieje ryzyko samozakleszczenia transakcji, jeśli braknie miejsca w buforze.
Częściej wykorzystywana jest strategia \textbf{steal + no-force} opakowana przez \textbf{protokół WAL (write-ahead log)}.
Oprócz plików z danymi przechowywany jest dziennik z logami (dziennik), w którym zapisywane są wszystkie zmiany przed wprowadzeniem ich na dysku.
Zawiera on informacje wystarczające odzyskania zawartości bazy za pomocą operacji UNDO i REDO.
Żeby dzienniki nie rozrastały się w nieskończoność, tworzy się punkty kontrolne. W przypadku awarii powtarzane są tylko transakcja zatwierdzona po ostatnim punkcie kontrolnym, a niezatwierdzone są wycofywane.

\subsection{Algorytm ARIES}
Algorytm ARIES (Algorithm for Recovery and Isolation Exploiting Semantics) służy do odtwarzania transakcji, korzystając z dziennika WAL i blokad.
Składa się z trzech faz:
\begin{enumerate}
    \item Analiza: rekonstrukcja Tabeli Brudnych Stron (DPT) i Tabeli Transakcji (TT), wyznaczenie pierwszego wpisu o zmianach tworzących brudną stronę
    \item Faza REDO: powtórzenie wszystkich operacji od momentu powstania brudnej strony, przywrócenie stanu sprzed awarii
    \item Faza UNDO: wycofanie efektów niezatwierdzonych transakcji, dodanie wpisów kompensacyjnych do dziennika
\end{enumerate}

\section{\textcolor{pink}{Przedstaw mechanizm współdzielenia bibliotek programistycznych (w systemie Linux), uwzględniając odpowiednie metody adresowania}}
Jednym z zadań SZBD jest zapewnianie tego, żeby zmiany dokonane przez widoczną lub zatwierdzoną transakcję były trwałe, a efekty transakcji wycofanej lub przerwanyj przez awarię nie.

Jest parę typów awarii:
\begin{itemize}
    \item awaria systemu: awaria sprzętu, błędy oprogramowania, uszkodzenie dysku
    \item błędy logiczne (np. naruszenie ograniczeń integralnościowych)
    \item błędy systemowe (np. zakleszczenia)
\end{itemize}
Do poradzenia sobie z pierwszym rodzajem wystarczy replikacja bazy i rozproszenie danych. W przypadku pozostałych dwóch trzeba odtworzyć bazę po awarii.
Ważne jest, żeby zminimalizować czas potrzebny do tego, ponieważ w trakcie odtwarzania bazy serwer musi być wyłączony dla użytkowników.
Służące do tego algorytmy składają się z dwóch części:
\begin{enumerate}
    \item zapisywanie informacji potrzebnych do odtworzenia stanu bazy na bieżąco podczas zwykłych operacji
    \item czynności po awarii, prowadzące do przywrócenia spójności, atomowości i trwałości
\end{enumerate}

\subsection*{Strategie SZBD}
Brudne strony, czyli zmieniane po ostatnim zapisie na dysku, mogą pozostawać w buforze długo po zatwierdzeniu transakcji. Jeśli nastąpi awaria, to naniesione zmiany są tracone.
Żeby zadbać o poprawny stan bazy można wykonać operacje:
\begin{itemize}
    \item UNDO, wycofanie zmian - zapewnia atomowość
    \item REDO, powtórzenie transakcji - zapewnia trwałość.
\end{itemize}
Podjęte czynności zależą od strategii SZBD: \\
Czy niezatwierdzona transakcja może nadpisać wartość w pamięci trwałej? \\
\begin{itemize}
    \item strategia steal: TAK
    \item strategia no-steal: NIE
\end{itemize}
Czy wszystkie zmiany muszą być zapisane w pamięci trwałej przed zatwierdzeniem transakcji?
\begin{itemize}
    \item strategia force: TAK
    \item strategia no-force: NIE
\end{itemize}

Strategia \textbf{no-steal + force} pozwala uniknąć wykonywania sekwencji UNDO i REDO. Jednak jest mało efektywna i istnieje ryzyko samozakleszczenia transakcji, jeśli braknie miejsca w buforze.
Częściej wykorzystywana jest strategia \textbf{steal + no-force} opakowana przez \textbf{protokół WAL (write-ahead log)}.
Oprócz plików z danymi przechowywany jest dziennik z logami (dziennik), w którym zapisywane są wszystkie zmiany przed wprowadzeniem ich na dysku.
Zawiera on informacje wystarczające odzyskania zawartości bazy za pomocą operacji UNDO i REDO.
Żeby dzienniki nie rozrastały się w nieskończoność, tworzy się punkty kontrolne. W przypadku awarii powtarzane są tylko transakcja zatwierdzona po ostatnim punkcie kontrolnym, a niezatwierdzone są wycofywane.

\subsection{Algorytm ARIES}
Algorytm ARIES (Algorithm for Recovery and Isolation Exploiting Semantics) służy do odtwarzania transakcji, korzystając z dziennika WAL i blokad.
Składa się z trzech faz:
\begin{enumerate}
    \item Analiza: rekonstrukcja Tabeli Brudnych Stron (DPT) i Tabeli Transakcji (TT), wyznaczenie pierwszego wpisu o zmianach tworzących brudną stronę
    \item Faza REDO: powtórzenie wszystkich operacji od momentu powstania brudnej strony, przywrócenie stanu sprzed awarii
    \item Faza UNDO: wycofanie efektów niezatwierdzonych transakcji, dodanie wpisów kompensacyjnych do dziennika
\end{enumerate}

\section{\textcolor{pink}{Na przykładzie problemu ucztujących filozofów przedyskutuj pojęcia poprawności pod względem bezpieczeństwa i żywotności. Zaproponuj rozwiązanie spełniające oba te warunki.}}
Jednym z zadań SZBD jest zapewnianie tego, żeby zmiany dokonane przez widoczną lub zatwierdzoną transakcję były trwałe, a efekty transakcji wycofanej lub przerwanyj przez awarię nie.

Jest parę typów awarii:
\begin{itemize}
    \item awaria systemu: awaria sprzętu, błędy oprogramowania, uszkodzenie dysku
    \item błędy logiczne (np. naruszenie ograniczeń integralnościowych)
    \item błędy systemowe (np. zakleszczenia)
\end{itemize}
Do poradzenia sobie z pierwszym rodzajem wystarczy replikacja bazy i rozproszenie danych. W przypadku pozostałych dwóch trzeba odtworzyć bazę po awarii.
Ważne jest, żeby zminimalizować czas potrzebny do tego, ponieważ w trakcie odtwarzania bazy serwer musi być wyłączony dla użytkowników.
Służące do tego algorytmy składają się z dwóch części:
\begin{enumerate}
    \item zapisywanie informacji potrzebnych do odtworzenia stanu bazy na bieżąco podczas zwykłych operacji
    \item czynności po awarii, prowadzące do przywrócenia spójności, atomowości i trwałości
\end{enumerate}

\subsection*{Strategie SZBD}
Brudne strony, czyli zmieniane po ostatnim zapisie na dysku, mogą pozostawać w buforze długo po zatwierdzeniu transakcji. Jeśli nastąpi awaria, to naniesione zmiany są tracone.
Żeby zadbać o poprawny stan bazy można wykonać operacje:
\begin{itemize}
    \item UNDO, wycofanie zmian - zapewnia atomowość
    \item REDO, powtórzenie transakcji - zapewnia trwałość.
\end{itemize}
Podjęte czynności zależą od strategii SZBD: \\
Czy niezatwierdzona transakcja może nadpisać wartość w pamięci trwałej? \\
\begin{itemize}
    \item strategia steal: TAK
    \item strategia no-steal: NIE
\end{itemize}
Czy wszystkie zmiany muszą być zapisane w pamięci trwałej przed zatwierdzeniem transakcji?
\begin{itemize}
    \item strategia force: TAK
    \item strategia no-force: NIE
\end{itemize}

Strategia \textbf{no-steal + force} pozwala uniknąć wykonywania sekwencji UNDO i REDO. Jednak jest mało efektywna i istnieje ryzyko samozakleszczenia transakcji, jeśli braknie miejsca w buforze.
Częściej wykorzystywana jest strategia \textbf{steal + no-force} opakowana przez \textbf{protokół WAL (write-ahead log)}.
Oprócz plików z danymi przechowywany jest dziennik z logami (dziennik), w którym zapisywane są wszystkie zmiany przed wprowadzeniem ich na dysku.
Zawiera on informacje wystarczające odzyskania zawartości bazy za pomocą operacji UNDO i REDO.
Żeby dzienniki nie rozrastały się w nieskończoność, tworzy się punkty kontrolne. W przypadku awarii powtarzane są tylko transakcja zatwierdzona po ostatnim punkcie kontrolnym, a niezatwierdzone są wycofywane.

\subsection{Algorytm ARIES}
Algorytm ARIES (Algorithm for Recovery and Isolation Exploiting Semantics) służy do odtwarzania transakcji, korzystając z dziennika WAL i blokad.
Składa się z trzech faz:
\begin{enumerate}
    \item Analiza: rekonstrukcja Tabeli Brudnych Stron (DPT) i Tabeli Transakcji (TT), wyznaczenie pierwszego wpisu o zmianach tworzących brudną stronę
    \item Faza REDO: powtórzenie wszystkich operacji od momentu powstania brudnej strony, przywrócenie stanu sprzed awarii
    \item Faza UNDO: wycofanie efektów niezatwierdzonych transakcji, dodanie wpisów kompensacyjnych do dziennika
\end{enumerate}