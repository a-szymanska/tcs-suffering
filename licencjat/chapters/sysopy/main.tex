\chapter{Systemy operacyjne}
\label{sysopy}
\section{\textcolor{pink}{Opisz mechanizmy komunikacji międzyprocesowej standardu POSIX.}}
\subsection{Deeply pipelined}
Nowoczesne procesory x86 posiadają głębokie potoki wykonawcze. Cykl przetwarzania instrukcji (pobranie, dekodowanie, wykonanie, zapis wyników) jest rozbity na wiele mniejszych etapów. Dzięki temu możliwe jest przetwarzanie wielu instrukcji równolegle, choć każda z nich znajduje się na innym etapie wykonania. Im głębszy potok, tym większy potencjalny zysk z wysokiego taktowania, ale też większe straty przy błędach przewidywania.
\subsection{Speculative execution}
Jeśli mamy skoki warunkowe to możemy próbować przewidywać czy skok nastąpi czy nie i na tej podstawie wykonywać instrukcje na zapas, czekając jedynie z fazą commit na faktyczne potwierdzenie czy skok następuje czy nie.
Jeśli zgadliśmy poprawnie to super – od razu commitujemy wynik. Natomiast jeśli nie zgadliśmy to musimy teraz wyrzucić cały pipeline na śmietnik i w efekcie dostajemy opóźnienie.
Robimy to zwykle automatem, który kiedy raz skoczyliśmy następnym razem wie, że też pewnie skoczymy.
\subsection{Out-of-order execution}
Procesory nie czekają na wykonanie instrukcji w kolejności ich występowania, analizują zależnośći i przestawiają instrukcje tak aby efektywniej wykorzystać zasoby. 
\subsection{Superscalar}
Procesor równolegle wykonuje wiele instrukcji w jednym cyklu zegara. Pozwala to na większą liczbę operacji niż by to wynikało z taktowania zegara. 
\subsection{Complex instruction set computer}
Procesory x86 mają bardzo wiele instukcji, które są często skomplikowane. Procesory typu ARM mają mało instrukcji i muszą często wywołać ich wiele, aby uzyskać tą samą funkcjonalność co x86. Skomplikowane instrukcje są optymalne, ale trudne w skalowaniu.


\section{\textcolor{pink}{Porównaj mechanizmy szeregowania zadań na przykładzie serwera przetwarzającego zadania w trybie wsadowym, oraz systemu interaktywnego.}}
\subsection{Deeply pipelined}
Nowoczesne procesory x86 posiadają głębokie potoki wykonawcze. Cykl przetwarzania instrukcji (pobranie, dekodowanie, wykonanie, zapis wyników) jest rozbity na wiele mniejszych etapów. Dzięki temu możliwe jest przetwarzanie wielu instrukcji równolegle, choć każda z nich znajduje się na innym etapie wykonania. Im głębszy potok, tym większy potencjalny zysk z wysokiego taktowania, ale też większe straty przy błędach przewidywania.
\subsection{Speculative execution}
Jeśli mamy skoki warunkowe to możemy próbować przewidywać czy skok nastąpi czy nie i na tej podstawie wykonywać instrukcje na zapas, czekając jedynie z fazą commit na faktyczne potwierdzenie czy skok następuje czy nie.
Jeśli zgadliśmy poprawnie to super – od razu commitujemy wynik. Natomiast jeśli nie zgadliśmy to musimy teraz wyrzucić cały pipeline na śmietnik i w efekcie dostajemy opóźnienie.
Robimy to zwykle automatem, który kiedy raz skoczyliśmy następnym razem wie, że też pewnie skoczymy.
\subsection{Out-of-order execution}
Procesory nie czekają na wykonanie instrukcji w kolejności ich występowania, analizują zależnośći i przestawiają instrukcje tak aby efektywniej wykorzystać zasoby. 
\subsection{Superscalar}
Procesor równolegle wykonuje wiele instrukcji w jednym cyklu zegara. Pozwala to na większą liczbę operacji niż by to wynikało z taktowania zegara. 
\subsection{Complex instruction set computer}
Procesory x86 mają bardzo wiele instukcji, które są często skomplikowane. Procesory typu ARM mają mało instrukcji i muszą często wywołać ich wiele, aby uzyskać tą samą funkcjonalność co x86. Skomplikowane instrukcje są optymalne, ale trudne w skalowaniu.


\section{\textcolor{pink}{Wyjaśnij pojęcie deadlock. Opisz metody wykrywania i zapobiegania powstawaniu deadlocku w kontekście współdzielonych zasobów.}}
\subsection{Deeply pipelined}
Nowoczesne procesory x86 posiadają głębokie potoki wykonawcze. Cykl przetwarzania instrukcji (pobranie, dekodowanie, wykonanie, zapis wyników) jest rozbity na wiele mniejszych etapów. Dzięki temu możliwe jest przetwarzanie wielu instrukcji równolegle, choć każda z nich znajduje się na innym etapie wykonania. Im głębszy potok, tym większy potencjalny zysk z wysokiego taktowania, ale też większe straty przy błędach przewidywania.
\subsection{Speculative execution}
Jeśli mamy skoki warunkowe to możemy próbować przewidywać czy skok nastąpi czy nie i na tej podstawie wykonywać instrukcje na zapas, czekając jedynie z fazą commit na faktyczne potwierdzenie czy skok następuje czy nie.
Jeśli zgadliśmy poprawnie to super – od razu commitujemy wynik. Natomiast jeśli nie zgadliśmy to musimy teraz wyrzucić cały pipeline na śmietnik i w efekcie dostajemy opóźnienie.
Robimy to zwykle automatem, który kiedy raz skoczyliśmy następnym razem wie, że też pewnie skoczymy.
\subsection{Out-of-order execution}
Procesory nie czekają na wykonanie instrukcji w kolejności ich występowania, analizują zależnośći i przestawiają instrukcje tak aby efektywniej wykorzystać zasoby. 
\subsection{Superscalar}
Procesor równolegle wykonuje wiele instrukcji w jednym cyklu zegara. Pozwala to na większą liczbę operacji niż by to wynikało z taktowania zegara. 
\subsection{Complex instruction set computer}
Procesory x86 mają bardzo wiele instukcji, które są często skomplikowane. Procesory typu ARM mają mało instrukcji i muszą często wywołać ich wiele, aby uzyskać tą samą funkcjonalność co x86. Skomplikowane instrukcje są optymalne, ale trudne w skalowaniu.


\section{\textcolor{pink}{Wyjaśnij mechanizm spooling, podaj przykłady kiedy należy oraz kiedy nie da się używać spoolingu jako mechanizmu racjonalizującego dostęp do zasobu.}}
\subsection{Deeply pipelined}
Nowoczesne procesory x86 posiadają głębokie potoki wykonawcze. Cykl przetwarzania instrukcji (pobranie, dekodowanie, wykonanie, zapis wyników) jest rozbity na wiele mniejszych etapów. Dzięki temu możliwe jest przetwarzanie wielu instrukcji równolegle, choć każda z nich znajduje się na innym etapie wykonania. Im głębszy potok, tym większy potencjalny zysk z wysokiego taktowania, ale też większe straty przy błędach przewidywania.
\subsection{Speculative execution}
Jeśli mamy skoki warunkowe to możemy próbować przewidywać czy skok nastąpi czy nie i na tej podstawie wykonywać instrukcje na zapas, czekając jedynie z fazą commit na faktyczne potwierdzenie czy skok następuje czy nie.
Jeśli zgadliśmy poprawnie to super – od razu commitujemy wynik. Natomiast jeśli nie zgadliśmy to musimy teraz wyrzucić cały pipeline na śmietnik i w efekcie dostajemy opóźnienie.
Robimy to zwykle automatem, który kiedy raz skoczyliśmy następnym razem wie, że też pewnie skoczymy.
\subsection{Out-of-order execution}
Procesory nie czekają na wykonanie instrukcji w kolejności ich występowania, analizują zależnośći i przestawiają instrukcje tak aby efektywniej wykorzystać zasoby. 
\subsection{Superscalar}
Procesor równolegle wykonuje wiele instrukcji w jednym cyklu zegara. Pozwala to na większą liczbę operacji niż by to wynikało z taktowania zegara. 
\subsection{Complex instruction set computer}
Procesory x86 mają bardzo wiele instukcji, które są często skomplikowane. Procesory typu ARM mają mało instrukcji i muszą często wywołać ich wiele, aby uzyskać tą samą funkcjonalność co x86. Skomplikowane instrukcje są optymalne, ale trudne w skalowaniu.


\section{\textcolor{pink}{Segmentacja i stronicowanie - porównaj mechanizmy. Opisz jak te mechanizmy są wykorzystywane na przykładzie wybranego systemu operacyjnego.}}
\subsection{Deeply pipelined}
Nowoczesne procesory x86 posiadają głębokie potoki wykonawcze. Cykl przetwarzania instrukcji (pobranie, dekodowanie, wykonanie, zapis wyników) jest rozbity na wiele mniejszych etapów. Dzięki temu możliwe jest przetwarzanie wielu instrukcji równolegle, choć każda z nich znajduje się na innym etapie wykonania. Im głębszy potok, tym większy potencjalny zysk z wysokiego taktowania, ale też większe straty przy błędach przewidywania.
\subsection{Speculative execution}
Jeśli mamy skoki warunkowe to możemy próbować przewidywać czy skok nastąpi czy nie i na tej podstawie wykonywać instrukcje na zapas, czekając jedynie z fazą commit na faktyczne potwierdzenie czy skok następuje czy nie.
Jeśli zgadliśmy poprawnie to super – od razu commitujemy wynik. Natomiast jeśli nie zgadliśmy to musimy teraz wyrzucić cały pipeline na śmietnik i w efekcie dostajemy opóźnienie.
Robimy to zwykle automatem, który kiedy raz skoczyliśmy następnym razem wie, że też pewnie skoczymy.
\subsection{Out-of-order execution}
Procesory nie czekają na wykonanie instrukcji w kolejności ich występowania, analizują zależnośći i przestawiają instrukcje tak aby efektywniej wykorzystać zasoby. 
\subsection{Superscalar}
Procesor równolegle wykonuje wiele instrukcji w jednym cyklu zegara. Pozwala to na większą liczbę operacji niż by to wynikało z taktowania zegara. 
\subsection{Complex instruction set computer}
Procesory x86 mają bardzo wiele instukcji, które są często skomplikowane. Procesory typu ARM mają mało instrukcji i muszą często wywołać ich wiele, aby uzyskać tą samą funkcjonalność co x86. Skomplikowane instrukcje są optymalne, ale trudne w skalowaniu.


\section{\textcolor{pink}{Porównaj monolityczną architekturę systemu operacyjnego z architekturą opartą na mikro jądrze.}}
\subsection{Deeply pipelined}
Nowoczesne procesory x86 posiadają głębokie potoki wykonawcze. Cykl przetwarzania instrukcji (pobranie, dekodowanie, wykonanie, zapis wyników) jest rozbity na wiele mniejszych etapów. Dzięki temu możliwe jest przetwarzanie wielu instrukcji równolegle, choć każda z nich znajduje się na innym etapie wykonania. Im głębszy potok, tym większy potencjalny zysk z wysokiego taktowania, ale też większe straty przy błędach przewidywania.
\subsection{Speculative execution}
Jeśli mamy skoki warunkowe to możemy próbować przewidywać czy skok nastąpi czy nie i na tej podstawie wykonywać instrukcje na zapas, czekając jedynie z fazą commit na faktyczne potwierdzenie czy skok następuje czy nie.
Jeśli zgadliśmy poprawnie to super – od razu commitujemy wynik. Natomiast jeśli nie zgadliśmy to musimy teraz wyrzucić cały pipeline na śmietnik i w efekcie dostajemy opóźnienie.
Robimy to zwykle automatem, który kiedy raz skoczyliśmy następnym razem wie, że też pewnie skoczymy.
\subsection{Out-of-order execution}
Procesory nie czekają na wykonanie instrukcji w kolejności ich występowania, analizują zależnośći i przestawiają instrukcje tak aby efektywniej wykorzystać zasoby. 
\subsection{Superscalar}
Procesor równolegle wykonuje wiele instrukcji w jednym cyklu zegara. Pozwala to na większą liczbę operacji niż by to wynikało z taktowania zegara. 
\subsection{Complex instruction set computer}
Procesory x86 mają bardzo wiele instukcji, które są często skomplikowane. Procesory typu ARM mają mało instrukcji i muszą często wywołać ich wiele, aby uzyskać tą samą funkcjonalność co x86. Skomplikowane instrukcje są optymalne, ale trudne w skalowaniu.


\section{\textcolor{pink}{Przedstaw mechanizm współdzielenia bibliotek programistycznych (w systemie Linux), uwzględniając odpowiednie metody adresowania}}
\subsection{Deeply pipelined}
Nowoczesne procesory x86 posiadają głębokie potoki wykonawcze. Cykl przetwarzania instrukcji (pobranie, dekodowanie, wykonanie, zapis wyników) jest rozbity na wiele mniejszych etapów. Dzięki temu możliwe jest przetwarzanie wielu instrukcji równolegle, choć każda z nich znajduje się na innym etapie wykonania. Im głębszy potok, tym większy potencjalny zysk z wysokiego taktowania, ale też większe straty przy błędach przewidywania.
\subsection{Speculative execution}
Jeśli mamy skoki warunkowe to możemy próbować przewidywać czy skok nastąpi czy nie i na tej podstawie wykonywać instrukcje na zapas, czekając jedynie z fazą commit na faktyczne potwierdzenie czy skok następuje czy nie.
Jeśli zgadliśmy poprawnie to super – od razu commitujemy wynik. Natomiast jeśli nie zgadliśmy to musimy teraz wyrzucić cały pipeline na śmietnik i w efekcie dostajemy opóźnienie.
Robimy to zwykle automatem, który kiedy raz skoczyliśmy następnym razem wie, że też pewnie skoczymy.
\subsection{Out-of-order execution}
Procesory nie czekają na wykonanie instrukcji w kolejności ich występowania, analizują zależnośći i przestawiają instrukcje tak aby efektywniej wykorzystać zasoby. 
\subsection{Superscalar}
Procesor równolegle wykonuje wiele instrukcji w jednym cyklu zegara. Pozwala to na większą liczbę operacji niż by to wynikało z taktowania zegara. 
\subsection{Complex instruction set computer}
Procesory x86 mają bardzo wiele instukcji, które są często skomplikowane. Procesory typu ARM mają mało instrukcji i muszą często wywołać ich wiele, aby uzyskać tą samą funkcjonalność co x86. Skomplikowane instrukcje są optymalne, ale trudne w skalowaniu.


\section{\textcolor{pink}{Na przykładzie problemu ucztujących filozofów przedyskutuj pojęcia poprawności pod względem bezpieczeństwa i żywotności. Zaproponuj rozwiązanie spełniające oba te warunki.}}
\subsection{Deeply pipelined}
Nowoczesne procesory x86 posiadają głębokie potoki wykonawcze. Cykl przetwarzania instrukcji (pobranie, dekodowanie, wykonanie, zapis wyników) jest rozbity na wiele mniejszych etapów. Dzięki temu możliwe jest przetwarzanie wielu instrukcji równolegle, choć każda z nich znajduje się na innym etapie wykonania. Im głębszy potok, tym większy potencjalny zysk z wysokiego taktowania, ale też większe straty przy błędach przewidywania.
\subsection{Speculative execution}
Jeśli mamy skoki warunkowe to możemy próbować przewidywać czy skok nastąpi czy nie i na tej podstawie wykonywać instrukcje na zapas, czekając jedynie z fazą commit na faktyczne potwierdzenie czy skok następuje czy nie.
Jeśli zgadliśmy poprawnie to super – od razu commitujemy wynik. Natomiast jeśli nie zgadliśmy to musimy teraz wyrzucić cały pipeline na śmietnik i w efekcie dostajemy opóźnienie.
Robimy to zwykle automatem, który kiedy raz skoczyliśmy następnym razem wie, że też pewnie skoczymy.
\subsection{Out-of-order execution}
Procesory nie czekają na wykonanie instrukcji w kolejności ich występowania, analizują zależnośći i przestawiają instrukcje tak aby efektywniej wykorzystać zasoby. 
\subsection{Superscalar}
Procesor równolegle wykonuje wiele instrukcji w jednym cyklu zegara. Pozwala to na większą liczbę operacji niż by to wynikało z taktowania zegara. 
\subsection{Complex instruction set computer}
Procesory x86 mają bardzo wiele instukcji, które są często skomplikowane. Procesory typu ARM mają mało instrukcji i muszą często wywołać ich wiele, aby uzyskać tą samą funkcjonalność co x86. Skomplikowane instrukcje są optymalne, ale trudne w skalowaniu.
