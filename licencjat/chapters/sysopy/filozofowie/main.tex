W tym problemie pewna liczba filozofów siedzi przy okrągłym stole, każdy z nich na zmianę albo myśli, albo je. Pomiędzy ucztującymi na stole znajdują się widelce, po jednym pomiędzy każdą parą filzofów. Aby jeść, filozof musi najpierw podnieść dwa widelce. Każdy widelec może być w danym momencie podniesiony lub używany przez co najwyżej jednego filozofa. Problem polega na zaprojektowaniu takiej strategii postępowania dla każdego filozofa (algorytmu rozproszonego), żeby mógł on w nieskończoność na zmianę myśleć i jeść.

\begin{figure}[H]  
    \centering
    \includegraphics[width=10cm]{chapters/sysopy/filozofowie/An_illustration_of_the_dining_philosophers_problem.png}
\caption{bdesham, User:Belbury, CC BY 3.0, \href{https://creativecommons.org/licenses/by/3.0}{Wikimedia Commons; Ilustracja problemu filozofów}}
\end{figure}

Algorytm jest niepoprawny pod względem \textbf{bezpieczeństwa}, jeśli może doprowadzić do deadlocka, czyli na przykład gdy w tym samym momencie każdy filozof podniesie lewy widelec i zacznie czekać na dostępność prawego widelca. 

Algorytm jest niepoprawny pod względem \textbf{żywotności}, jeśli może dojść do sytuacji, w której jakiś filozof nigdy nie doczeka się obu widelców. Rozwiązanie polegające na numeracji widelców i podnoszeniu ich tylko w odpowiedniej kolejności rozwiązuje problem deadlocków, ale nie gwarantuje żywotności, ponieważ widelec o najmniejszym numerze jest podnoszony w pierwszej kolejności przez dwóch filozofów.

Rozwiązanie:
\begin{itemize}
\item Każdy widelec jest w każdym momencie przez kogoś trzymany.
\item Początkowy przydział widelców nie jest symetryczny, czyli nie jest tak, że każdy trzyma prawy widelec lub każdy trzyma lewy widelec.
\item Każdy filozof po zakończeniu jedzenia przekazuje równocześnie oba widelce do swoich sąsiadów
\end{itemize}

Utrzymujemy niezmiennik, że przydział widelców nie jest symetryczny. Przydział nie jest symetryczny wtedy i tylko wtedy, gdy przynajmniej jeden filozof trzyma oba widelce. Po zjedzeniu, filozof oddaje równocześnie oba widelce, więc nowy przydział też nie jest symetryczny.

Skoro przydział nigdy nie jest symetryczny, to zawsze przynajmniej jeden filozof może jeść, więc deadlock jest niemożliwy.

Załóżmy nie wprost, że któryś filozof X nigdy nie doczeka się widelca od sąsiada Y. Zatem Y trzyma widelec po stronie X i sam nigdy nie doczeka się widelca z drugiej strony, bo gdyby się doczekał, to w końcu by zjadł i oddał widelec sąsiadowi X. Kontynuując w ten sposób można pokazać indukcyjnie, że przydział jest symetryczny, co jest sprzeczne z niezmiennikiem. Zatem każdy filozof kiedyś doczeka się obu widelców.